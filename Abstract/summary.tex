%\documentclass[hyperpdf,bindnopdf]{hepthesis}
\documentclass{article}
\usepackage{amsmath}
\usepackage{mathpazo}
\usepackage{lmodern}
\usepackage{setspace}
\onehalfspacing

\begin{document}

\title{Calorimetry at a Future Linear Collider}
\author{Steven Green}
\date{}
\maketitle
\thispagestyle{empty}

This thesis describes the optimisation of the calorimeter design for collider experiments at the future Compact LInear Collider (CLIC) and the International Linear Collider (ILC).  The detector design for these experiments is built around high-granularity Particle Flow Calorimetry that, in contrast to traditional calorimetry, uses the energy measurements for charged particles from the tracking detectors.  This can only be realised if calorimetric energy deposits from charged particles can be separated from those of neutral particles.  This is made possible with fine granularity calorimeters and sophisticated pattern recognition software, which is provided by the PandoraPFA algorithm.  This thesis presents results on Particle Flow calorimetry performance for a number of detector configurations.  To obtain these results a new calibration procedure was developed and applied to the detector simulation and reconstruction to ensure optimal performance was achieved for each detector configuration considered.

This thesis also describes the development of a software compensation technique that significantly improves the intrinsic energy resolution of a Particle Flow Calorimetry detector. This technique is implemented within the PandoraPFA framework and demonstrates the gains that can be made by fully exploiting the information provided by the fine granularity calorimeters envisaged at a future linear collider.

A study of the sensitivity of the CLIC experiment to anomalous gauge couplings that effect vector boson scattering processes is presented. These anomalous couplings provide insight into possible beyond standard model physics. This study, which utilises the excellent jet energy resolution from Particle Flow Calorimetry, was performed at centre-of-mass energies of 1.4 TeV and 3 TeV with integrated luminosities of 1.5$\text{ab}^{-1}$ and 2$\text{ab}^{-1}$ respectively. The precision achievable at CLIC is shown to be approximately one to two orders of magnitude better than that currently offered by the LHC.

In addition, a study into various technology options for the CLIC vertex detector is described.

\end{document}

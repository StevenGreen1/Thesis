\chapter{Reconstruction Chain}
\label{chap:reconstructionchain}

\chapterquote{There, sir! that is the perfection of vessels!}
{Jules Verne, 1828--1905}

\section{Reconstruction Chain}

\section{Event Generation, Simulation and Reconstruction}

The jet fragmentation and hadronisation for the Z $\rightarrow$ uds events used for determining the metric for detector performance was controlled using PYTHIA \cite{Sjostrand:2006za} that had been tuned using data from LEP \cite{Alexander:1995bk}.  Single particle spatially isotropic samples of $K_{L}^{0}$, $\gamma$ and $\mu^{-}$ were produced for the calibration of each detector model.  A simple c++ script was written to generate the relevant HEPEvt common blocks for these samples. 

Detector model simulation was performed using MOKKA \cite{MoradeFreitas:2002kj}, a GEANT4 \cite{Agostinelli:2002hh} wrapper providing detailed geometric descriptions of detector concepts for the linear collider.  Event reconstruction was performed using MARLIN \cite{Gaede:2006pj}, a c++ framework designed for reconstruction at the linear collider.  PandoraPFA \cite{arXiv:0907.3577, arXiv:1209.4039} was used to apply Particle Flow Calorimetry in the reconstruction, the full details of which can be found in chapter PANDORA CHAPTER.

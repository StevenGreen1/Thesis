\chapter{The Sensitivity of CLIC to Anomalous Gauge Couplings through Vector Boson Scattering}
\label{chap:PhysicsAnalysis}

%% Restart the numbering to make sure that this is definitely page #1!
\pagenumbering{arabic}

\chapterquote{Kids, you tried your best, and you failed miserably.  The lesson is, never try.}%
{Homer Simpson}

\section{Background}
A process that will show sensitivity to the $\alpha_{4}$ and $\alpha_{5}$ anomalous gauge couplings in the CLIC experiemnt is vector boson scattering.  There are several channels that will be affected by these anomalos couplings at CLIC and these are summarised in figures \ref{fig:vbsw}, \ref{fig:vbsz}, \ref{fig:vbswz} and \ref{fig:vbszw} where $q = \text{u, d, s, b, c}$ and $l = \text{e, } \mu \text{, } \tau \text{, } \nu_{e} \text{, } \nu_{\nu} \text{, } \nu_{\tau}$.

\begin{figure}
\begin{tikzpicture}[]
\begin{feynman}
\vertex (a1);
\vertex[above left=2cm of a1] (a2);
\vertex[above right=1cm and 2cm of a1] (a3) {\(W^{\pm},Z\)};
\vertex[below left=2cm of a1] (a4);
\vertex[below right=1cm and 2cm of a1] (a5) {\(W^{\mp},Z\)};
\vertex[above left=1cm of a2] (i1) {\(e^{-}\)};
\vertex[below left=1cm of a4] (i2) {\(e^{+}\)};
\vertex[above right=1cm and 3cm of a3] (i3) {\(q,l\)};
\vertex[below right=1cm and 3cm of a3] (i4) {\(\bar{q},\bar{l}\)};
\vertex[above right=1cm and 3cm of a5] (i5) {\(q,l\)};
\vertex[below right=1cm and 3cm of a5] (i6) {\(\bar{q},\bar{l}\)};
\vertex[above=1cm of a3] (v1) {\(\nu_{e}\)};
\vertex[below=1cm of a5] (v2) {\(\bar{\nu_{e}}\)};
\diagram* {
   (a1) -- [boson, edge label'=\(W^{-}\)] (a2) 
   (a1) -- [boson] (a3) 
   (a1) -- [boson, edge label=\(W^{+}\)] (a4) 
   (a1) -- [boson] (a5) 
   (i1) -- [fermion] (a2) -- [fermion] (v1)
   (v2) -- [fermion] (a4) -- [fermion] (i2)
   (i4) -- [fermion] (a3) -- [fermion] (i3)
   (i6) -- [fermion] (a5) -- [fermion] (i5)
};
\end{feynman}
\end{tikzpicture}
\caption[Feynman diagram of vector boson scattering at CLIC involving radiation of W bosons.]{Feynman diagram of vector boson scattering at CLIC involving radiation of W bosons.}
\label{fig:vbsw}
\end{figure}

\begin{figure}
\begin{tikzpicture}[]
\begin{feynman}
\vertex (a1);
\vertex[above left=2cm of a1] (a2);
\vertex[above right=1cm and 2cm of a1] (a3) {\(W^{\pm},Z\)};
\vertex[below left=2cm of a1] (a4);
\vertex[below right=1cm and 2cm of a1] (a5) {\(W^{\mp},Z\)};
\vertex[above left=1cm of a2] (i1) {\(e^{-}\)};
\vertex[below left=1cm of a4] (i2) {\(e^{+}\)};
\vertex[above right=1cm and 3cm of a3] (i3) {\(q,l\)};
\vertex[below right=1cm and 3cm of a3] (i4) {\(\bar{q},\bar{l}\)};
\vertex[above right=1cm and 3cm of a5] (i5) {\(q,l\)};
\vertex[below right=1cm and 3cm of a5] (i6) {\(\bar{q},\bar{l}\)};
\vertex[above=1cm of a3] (v1) {\(e^{-}\)};
\vertex[below=1cm of a5] (v2) {\(e^{+}\)};
\diagram* {
   (a1) -- [boson, edge label'=\(Z\)] (a2)
   (a1) -- [boson] (a3)
   (a1) -- [boson, edge label=\(Z\)] (a4)
   (a1) -- [boson] (a5)
   (i1) -- [fermion] (a2) -- [fermion] (v1)
   (v2) -- [fermion] (a4) -- [fermion] (i2)
   (i4) -- [fermion] (a3) -- [fermion] (i3)
   (i6) -- [fermion] (a5) -- [fermion] (i5)
};
\end{feynman}
\end{tikzpicture}
\caption[Feynman diagram of vector boson scattering at CLIC involving radiation of Z bosons.]{Feynman diagram of vector boson scattering at CLIC involving radiation of Z bosons.}
\label{fig:vbsz}
\end{figure}

\begin{figure}
\begin{tikzpicture}[]
\begin{feynman}
\vertex (a1);
\vertex[above left=2cm of a1] (a2);
\vertex[above right=1cm and 2cm of a1] (a3) {\(W^{-}\)};
\vertex[below left=2cm of a1] (a4);
\vertex[below right=1cm and 2cm of a1] (a5) {\(Z\)};
\vertex[above left=1cm of a2] (i1) {\(e^{-}\)};
\vertex[below left=1cm of a4] (i2) {\(e^{+}\)};
\vertex[above right=1cm and 3cm of a3] (i3) {\(q,l\)};
\vertex[below right=1cm and 3cm of a3] (i4) {\(\bar{q},\bar{l}\)};
\vertex[above right=1cm and 3cm of a5] (i5) {\(q,l\)};
\vertex[below right=1cm and 3cm of a5] (i6) {\(\bar{q},\bar{l}\)};
\vertex[above=1cm of a3] (v1) {\(\nu_{e}\)};
\vertex[below=1cm of a5] (v2) {\(e^{+}\)};
\diagram* {
   (a1) -- [boson, edge label'=\(W^{-}\)] (a2)
   (a1) -- [boson] (a3)
   (a1) -- [boson, edge label=\(Z\)] (a4)
   (a1) -- [boson] (a5)
   (i1) -- [fermion] (a2) -- [fermion] (v1)
   (v2) -- [fermion] (a4) -- [fermion] (i2)
   (i4) -- [fermion] (a3) -- [fermion] (i3)
   (i6) -- [fermion] (a5) -- [fermion] (i5)
};
\end{feynman}
\end{tikzpicture}
\caption[Feynman diagram of vector boson scattering at CLIC involving radiation of one Z and one W boson.]{Feynman diagram of vector boson scattering at CLIC involving radiation of one Z and one W boson.}
\label{fig:vbswz}
\end{figure}

\begin{figure}
\begin{tikzpicture}[]
\begin{feynman}
\vertex (a1);
\vertex[above left=2cm of a1] (a2);
\vertex[above right=1cm and 2cm of a1] (a3) {\(Z\)};
\vertex[below left=2cm of a1] (a4);
\vertex[below right=1cm and 2cm of a1] (a5) {\(W^{+}\)};
\vertex[above left=1cm of a2] (i1) {\(e^{-}\)};
\vertex[below left=1cm of a4] (i2) {\(e^{+}\)};
\vertex[above right=1cm and 3cm of a3] (i3) {\(q,l\)};
\vertex[below right=1cm and 3cm of a3] (i4) {\(\bar{q},\bar{l}\)};
\vertex[above right=1cm and 3cm of a5] (i5) {\(q,l\)};
\vertex[below right=1cm and 3cm of a5] (i6) {\(\bar{q},\bar{l}\)};
\vertex[above=1cm of a3] (v1) {\(e^{-}\)};
\vertex[below=1cm of a5] (v2) {\(\bar{\nu_{e}}\)};
\diagram* {
   (a1) -- [boson, edge label'=\(Z\)] (a2)
   (a1) -- [boson] (a3)
   (a1) -- [boson, edge label=\(W^{+}\)] (a4)
   (a1) -- [boson] (a5)
   (i1) -- [fermion] (a2) -- [fermion] (v1)
   (v2) -- [fermion] (a4) -- [fermion] (i2)
   (i4) -- [fermion] (a3) -- [fermion] (i3)
   (i6) -- [fermion] (a5) -- [fermion] (i5)
};
\end{feynman}
\end{tikzpicture}
\caption[Feynman diagram of vector boson scattering at CLIC involving radiation of one Z and one W boson.]{Feynman diagram of vector boson scattering at CLIC involving radiation of one Z and one W boson.}
\label{fig:vbszw}
\end{figure}

To determine whether an event is sensitive to $\alpha_{4}$ and $\alpha_{5}$ it will be necessary to determine whether the visible final states have been produced from the decay of W and Z bosons.  A key descriminator in this procedure will be the invariant mass of the W and Z candidates.  In light of this the hadronic decays of the W and Z bosons are only considered as the leptonic decays may contain neutrinos.

As the W and Z bosons in vector boson scattering are intermediate states in the Feynman diagrams, they will not be directly observed in the detector and will instead contribute to processes with the final states containing possible decay products of the bosons \nu{\nu}qqqq, l{\nu}qqqq and llqqqq.  In theory all processes will be affected by non zero $\alpha_{4}$ and $\alpha_{5}$, however, the effects may be extremely small as they contribute to very high order expansions of the Hamiltonian.  Event generation software does not calculate the expansions of the Hamiltonian to all orders, but instead tuncates the expansion to leave the dominant terms.  In the case of anomalous couplings this corresponds to certain final state cross sections being invariant to changes in $\alpha_{4}$ and $\alpha_{5}$.  

\section{Event Generation}

The event generation software used by the CLIC experiment is Whizard. 

To find out which states show sensitivity to the anomalous couplings two cross section calculations were made using different values of $\alpha_{4}$ and $\alpha_{5}$ for relevant processes involving the hadronic decays of the W and Z bosons from vector boson scattering, which can be found in table \ref{table:xstest}.  In the standard model the values of $\alpha_{4}$ and $\alpha_{5}$ are zero.  The only final states showing sensitivty to the anomalous couplings are \nu{\nu}qqqq, l{\nu}qqqq and llqqqq, which correspond to the final states from the Feynman diagrams shown above (\ref{fig:vbsw}, \ref{fig:vbsz}, \ref{fig:vbswz} and \ref{fig:vbszw}).  As this analysis focuses on the hadronic decays of the bosons involved in vector boson scattering the final states involving leptonic decays of the bosons e.g. \nu{\nu}llqq were not included in this cross check.  These leptonic dominated final states were also removed from the background samples used in this study as isolated lepton finding would largely veto all such events from selection.

The sensitvity of an individual event to the anomalous gauge couplings is determined through an event weight.  This weight corresponds to the ratio using non-zero $\alpha_{4}$ and $\alpha_{5}$ and using zero $\alpha_{4}$ and $\alpha_{5}$ of the square of the matrix element used in the cross section calculation.  This reweighting procedure has many advantages over the alternative procedure of generating new samples with fixed $\alpha_{4}$ and $\alpha_{5}$ most notably the absence of systematic errors that may appear in new event generation.  

The cross check shows that the most sensitive channel to the anomalous gauge couplings is the \nu{\nu}qqqq indicating that the best sensitivity measurement should focus upon this channel, which is the aim of this analysis.  

The CLIC experiment has a repository of simulated and reconstructed samples that can be used for physics analyses, however, for the relevant final states there is no way to calculate the event weights for these samples.  Therefore, new samples for which reweighting is possible were created and processed through the CLIC reconstruction chain.  New samples were created only for the \nu \nu qqqq final state as the l{\nu}qqqq and llqqqq final states have a significantly lower sensitivity.  As will be shown in subsequent chapters, the application of an isolated lepton finder in the selection processor will largely veto the l{\nu}qqqq and llqqqq final states, therefore, the absence of weight information for these final states will not significantly affect the sensitivity measurement based on the \nu{\nu}qqqq final state.

\begin{table}[h]
\begin{tabular}{*4l}    
\toprule
Final State & Cross Section [fb] & Cross Section [fb] & Percentage Change [\%] & CLIC Cross Section\\
 & ($\alpha_{4} = 0.00$,  $\alpha_{5} = 0.00$) & ($\alpha_{4} = 0.05$,  $\alpha_{5} = 0.05$) & &\\
\midrule
ee $\rightarrow$ \nu \nu qqqq  &2.08E+01       &3.46E+01 & +66.3 & 24.7\\
ee $\rightarrow$ l \nu qqqq  &1.12E+02       &1.13E+02 & +0.9 & 115.3\\
ee $\rightarrow$ \nu \nu qqqq  &5.97E+01       &6.86E+01 & +14.9 & 71.7\\
\bottomrule
\hline
\end{tabular}
\caption[Cross section for selected processes for given value of $\alpha_{4}$ and $\alpha_{5}$.]{Cross section for selected processes for given value of $\alpha_{4}$ and $\alpha_{5}$.  Channels considered where there were no changes to the cross section measurment when varying $\alpha_{4}$ and $\alpha_{5}$ were ee $\rightarrow$ qqqq, ee $\rightarrow$ \nu \nu qq, ee $\rightarrow$ l \nu qq, ee $\rightarrow$ llqq, ee $\rightarrow$ qq, e$\gamma_{BS}$ $\rightarrow$ qqqqe, $\gamma_{BS}$ e $\rightarrow$ qqqqe, e$\gamma_{EPA}$ $\rightarrow$ qqqqe, $\gamma_{EPA}$ e $\rightarrow$ qqqqe, $\gamma_{BS}$ $\rightarrow$ qqqq\nu, $\gamma_{BS}$ e $\rightarrow$ qqqq\nu, e$\gamma_{EPA}$ $\rightarrow$ qqqq\nu, $\gamma_{EPA}$ e $\rightarrow$ qqqq\nu, $\gamma_{BS}\gamma_{BS}$ $\rightarrow$ qqqq, $\gamma_{BS}\gamma_{EPA}$ $\rightarrow$ qqqq, $\gamma_{EPA}\gamma_{BS}$ $\rightarrow$ qqqq and $\gamma_{EPA}\gamma_{EPA}$ $\rightarrow$ qqqq}
\label{table:xstest}
\end{table}

\section{Validation Of New Samples}

An ideantical setup to that used for the official CLIC sample was used for the event generation and reconstructin.  Several reconstructed level distributions were compared to the official CLIC samples to ensure the samples were behaving appropriately.  These are shown below.


\section{Reconstruction}



\section{Analysis Processor and Jet Pairing}

\section{Event Selection}

\section{Fit}

\chapter{The sensitivity of CLIC to anomalous gauge couplings through vector boson scattering}
\label{chap:PhysicsAnalysis}

\chapterquote{Kids, you tried your best, and you failed miserably.  The lesson is, never try.}%
{Homer Simpson}

%========================================================================================
%========================================================================================

\section{Motivation}
Vector boson scattering is the interaction of the form $\text{VV} \rightarrow \text{VV}$ where V is any of the electroweak gauge bosons $\text{W}^{+}$, $\text{W}^{-}$, Z or $\gamma$.  This is an interesting process to look at because it gives a detailed understanding of how the standard model Higgs is able to unitarise the otherwise unbounded cross section for longitudinal gauge boson scattering.  Vector boson scattering also provides insights into beyond standard model physics that impacts the electroweak sector through the probing of anomalous triple and quartic gauge couplings.  Presented in this section is an analysis into the sensitivity of CLIC to two of these anomalous quartic gauge couplings through the vector boson scattering process.

Triple and quartic gauge couplings lead to interactions of the form $\text{V} \rightarrow \text{VV}$ and $\text{VV} \rightarrow \text{VV}$ respectively.  In the standard model there are five permissible  vertices, shown in figure \ref{fig:smtripleandquarticvertices}, which arise from the kinematic term $\mathcal{L}_{kin} = -\frac{1}{4}B_{\mu\nu}B^{\mu\nu} - \frac{1}{4}W_{\mu\nu}W^{\mu\nu}$.

Anomalous triple and quartic gauge couplings are introduced as parameters in effective field theory models (EFTs).  These anomalous couplings can either modify the standard model triple and quartic gauge boson couplings or introduce new triple and quartic couplings that were previously forbidden.  

EFT models work under the assumption than new physics exists at an energy scale, $\Lambda$ that is much higher than the energy scales currently accessible to modern day particle physics experiments.  Furthermore, in limit $\Lambda \rightarrow \infty$ the standard model should be reproduced as the new physics becomes kinematically inaccessible.  Such theories are also model independent, giving them a wide span in the search for new physics.

A classic example of an EFT theory is the Fermi theory for beta decay.  The weak interaction occurring when a neutron decays into a proton, electron and anti-neutrino can be, at energies much below the mass of the W boson, can be treated as a four-point vertex with quartic coupling strength $\text{G}_{F}$, the Fermi Coupling constant.  (Feynamn Diagram if keep)

This analysis examines the anomalous quartic gauge couplings $\alpha_{4}$ and $\alpha_{5}$, which are introduced as part of an EFT described in chapter THEORY CHAPTER.  They appear in the Lagrangian through the following terms 

\begin{equation}
\alpha_{4}[\text{Tr}(V^{\mu}V_{\mu})]^{2} \text{ and } \alpha_{5}\text{Tr}(V^{\mu}V_{\nu})] \text{Tr}(V^{\nu}V_{\mu})]
\end{equation}

where $V_{\mu}$ corresponds, in a carefully chosen gauge, to a linear combination of the massive gauge bosons $\text{W}^{+}$, $\text{W}^{-}$ and Z.  These terms affect the coupling constants for the standard model vertices $\text{W}^{+}\text{W}^{-} \rightarrow \text{W}^{+}\text{W}^{-}$ and $\text{W}^{+}\text{W}^{-} \rightarrow \text{Z}\text{Z}$ as well as introducing the new vertex $\text{Z}\text{Z} \rightarrow \text{Z}\text{Z}$.  Vector boson scattering was an appropriate process to consider for a sensitivity study into the anomalous gauge couplings $\alpha_{4}$ and $\alpha_{5}$ as quartic gauge boson self-interaction vertices will be present in the dominant channels for such interactions (FEYNMAN DIAGRAM). 

As CLIC is purposefully deigned for high precision measurements it is ideal for a study into vector boson scattering.  The application of Particle Flow Calorimetry with fine granularity calorimeters gives CLIC excellent jet energy resolution, which allows it to clearly characterise the multi-jet final states.  When considering the invariant mass of these paired up jets, the nominal jet energy resolution for CLIC allows for accurate separation of W and Z bosons, which will be invaluable for event selection.  This precision also helps CLIC to characterise final states containing missing energy in the form of neutrinos.  The cross sections for the relevant processes are sufficiently large at the proposed running energies for CLIC to give a large sample size for this analysis.  Finally, this study offers the potential to give results several orders of magnitude better than the complementary studies at the LHC due to the reduction in hadronic backgrounds and the high $\sqrt{s}$ for the interaction offered by colliding leptons instead of protons.  All of the above reasons make a strong case for performing this analysis at CLIC.  

This study focuses on determining the sensitivity of CLIC to the anomalous gauge couplings based solely upon the vector boson scattering processes where the outgoing bosons decay purely hadronically.  This decision was made as the hadronic channels are the dominant decay modes of the W and the Z boson, with branching fractions of the order of 70\% for both (REFERENCE PDG), and given CLIC has excellent jet energy resolution .  Therefore, the signal final states in this analysis are thus: $\nu\nu\text{qqqq}$, $\text{l}\nu\text{qqqq}$ and llqqqq.  Feynman diagrams involving vector boson scattering and containing these final states are shown in figure \ref{fig:vectorbosonscatteringclic}.

\iffalse
% Feynam diagrams of triple and quartic vertices in the standard model.
\begin{figure}
  \begin{tikzpicture}[]
  \begin{feynman}
    \vertex (a1);
    \vertex[left=1cm of a1] (a2) {\(Z/$\gamma$\)};
    \vertex[above right=1cm of a1] (a3) {\(W^{+}\)};
    \vertex[below right=1cm of a1] (a4) {\(W^{-}\)};
    \diagram* {
       (a1) -- [boson] (a2) 
       (a1) -- [boson] (a3) 
       (a1) -- [boson] (a4) 
    };
  \end{feynman}
  \end{tikzpicture}%
  \begin{tikzpicture}[]
  \begin{feynman}
    \vertex (a1);
    \vertex[above left=1cm of a1] (a2) {\(W^{+}\)};
    \vertex[above right=1cm of a1] (a3) {\(W^{-}\)};
    \vertex[below left=1cm of a1] (a4) {\(W^{+}\)};
    \vertex[below right=1cm of a1] (a5) {\(W^{-}\)};
    \diagram* {
       (a1) -- [boson] (a2) 
       (a1) -- [boson] (a3) 
       (a1) -- [boson] (a4) 
       (a1) -- [boson] (a5) 
    };
  \end{feynman}
  \end{tikzpicture}%
  
  \begin{tikzpicture}[]
  \begin{feynman}
    \vertex (a1);
    \vertex[above left=1cm of a1] (a2) {\(W^{+}\)};
    \vertex[above right=1cm of a1] (a3) {\(W^{-}\)};
    \vertex[below left=1cm of a1] (a4) {\(Z\)};
    \vertex[below right=1cm of a1] (a5) {\(Z\)};
    \diagram* {
       (a1) -- [boson] (a2) 
       (a1) -- [boson] (a3) 
       (a1) -- [boson] (a4) 
       (a1) -- [boson] (a5) 
    };
  \end{feynman}
  \end{tikzpicture}%
  \begin{tikzpicture}[]
  \begin{feynman}
    \vertex (a1);
    \vertex[above left=1cm of a1] (a2) {\(W^{+}\)};
    \vertex[above right=1cm of a1] (a3) {\(W^{-}\)};
    \vertex[below left=1cm of a1] (a4) {\($\gamma$\)};
    \vertex[below right=1cm of a1] (a5) {\(Z\)};
    \diagram* {
       (a1) -- [boson] (a2) 
       (a1) -- [boson] (a3) 
       (a1) -- [boson] (a4) 
       (a1) -- [boson] (a5) 
    };
  \end{feynman}
  \end{tikzpicture}%
  
  \begin{tikzpicture}[]
  \begin{feynman}
    \vertex (a1);
    \vertex[above left=1cm of a1] (a2) {\(W^{+}\)};
    \vertex[above right=1cm of a1] (a3) {\(W^{-}\)};
    \vertex[below left=1cm of a1] (a4) {\($\gamma$\)};
    \vertex[below right=1cm of a1] (a5) {\($\gamma$\)};
    \diagram* {
       (a1) -- [boson] (a2) 
       (a1) -- [boson] (a3) 
       (a1) -- [boson] (a4) 
       (a1) -- [boson] (a5) 
    };
  \end{feynman}
  \end{tikzpicture}
  \caption[Gauge boson self-coupling vertices in the standard model.]{Gauge boson self-coupling vertices in the standard model.}
  \label{fig:smtripleandquarticvertices}
\end{figure}

\fi

% Feynam diagrams of sensitive processes 

\begin{figure}
\begin{tikzpicture}[]
\begin{feynman}
\vertex (a1);
\vertex[above left=2cm of a1] (a2);
\vertex[above right=1cm and 2cm of a1] (a3) {\(W^{\pm},Z\)};
\vertex[below left=2cm of a1] (a4);
\vertex[below right=1cm and 2cm of a1] (a5) {\(W^{\mp},Z\)};
\vertex[above left=1cm of a2] (i1) {\(e^{-}\)};
\vertex[below left=1cm of a4] (i2) {\(e^{+}\)};
\vertex[above right=1cm and 3cm of a3] (i3) {\(q,l\)};
\vertex[below right=1cm and 3cm of a3] (i4) {\(\bar{q},\bar{l}\)};
\vertex[above right=1cm and 3cm of a5] (i5) {\(q,l\)};
\vertex[below right=1cm and 3cm of a5] (i6) {\(\bar{q},\bar{l}\)};
\vertex[above=1cm of a3] (v1) {\(\nu_{e}\)};
\vertex[below=1cm of a5] (v2) {\(\bar{\nu_{e}}\)};
\diagram* {
   (a1) -- [boson, edge label'=\(W^{-}\)] (a2) 
   (a1) -- [boson] (a3) 
   (a1) -- [boson, edge label=\(W^{+}\)] (a4) 
   (a1) -- [boson] (a5) 
   (i1) -- [fermion] (a2) -- [fermion] (v1)
   (v2) -- [fermion] (a4) -- [fermion] (i2)
   (i4) -- [fermion] (a3) -- [fermion] (i3)
   (i6) -- [fermion] (a5) -- [fermion] (i5)
};
\end{feynman}
\end{tikzpicture}%
\begin{tikzpicture}[]
\begin{feynman}
\vertex (a1);
\vertex[above left=2cm of a1] (a2);
\vertex[above right=1cm and 2cm of a1] (a3) {\(W^{\pm},Z\)};
\vertex[below left=2cm of a1] (a4);
\vertex[below right=1cm and 2cm of a1] (a5) {\(W^{\mp},Z\)};
\vertex[above left=1cm of a2] (i1) {\(e^{-}\)};
\vertex[below left=1cm of a4] (i2) {\(e^{+}\)};
\vertex[above right=1cm and 3cm of a3] (i3) {\(q,l\)};
\vertex[below right=1cm and 3cm of a3] (i4) {\(\bar{q},\bar{l}\)};
\vertex[above right=1cm and 3cm of a5] (i5) {\(q,l\)};
\vertex[below right=1cm and 3cm of a5] (i6) {\(\bar{q},\bar{l}\)};
\vertex[above=1cm of a3] (v1) {\(e^{-}\)};
\vertex[below=1cm of a5] (v2) {\(e^{+}\)};
\diagram* {
   (a1) -- [boson, edge label'=\(Z\)] (a2)
   (a1) -- [boson] (a3)
   (a1) -- [boson, edge label=\(Z\)] (a4)
   (a1) -- [boson] (a5)
   (i1) -- [fermion] (a2) -- [fermion] (v1)
   (v2) -- [fermion] (a4) -- [fermion] (i2)
   (i4) -- [fermion] (a3) -- [fermion] (i3)
   (i6) -- [fermion] (a5) -- [fermion] (i5)
};
\end{feynman}
\end{tikzpicture}
\begin{tikzpicture}[]
\begin{feynman}
\vertex (a1);
\vertex[above left=2cm of a1] (a2);
\vertex[above right=1cm and 2cm of a1] (a3) {\(W^{-}\)};
\vertex[below left=2cm of a1] (a4);
\vertex[below right=1cm and 2cm of a1] (a5) {\(Z\)};
\vertex[above left=1cm of a2] (i1) {\(e^{-}\)};
\vertex[below left=1cm of a4] (i2) {\(e^{+}\)};
\vertex[above right=1cm and 3cm of a3] (i3) {\(q,l\)};
\vertex[below right=1cm and 3cm of a3] (i4) {\(\bar{q},\bar{l}\)};
\vertex[above right=1cm and 3cm of a5] (i5) {\(q,l\)};
\vertex[below right=1cm and 3cm of a5] (i6) {\(\bar{q},\bar{l}\)};
\vertex[above=1cm of a3] (v1) {\(\nu_{e}\)};
\vertex[below=1cm of a5] (v2) {\(e^{+}\)};
\diagram* {
   (a1) -- [boson, edge label'=\(W^{-}\)] (a2)
   (a1) -- [boson] (a3)
   (a1) -- [boson, edge label=\(Z\)] (a4)
   (a1) -- [boson] (a5)
   (i1) -- [fermion] (a2) -- [fermion] (v1)
   (v2) -- [fermion] (a4) -- [fermion] (i2)
   (i4) -- [fermion] (a3) -- [fermion] (i3)
   (i6) -- [fermion] (a5) -- [fermion] (i5)
};
\end{feynman}
\end{tikzpicture}%
\begin{tikzpicture}[]
\begin{feynman}
\vertex (a1);
\vertex[above left=2cm of a1] (a2);
\vertex[above right=1cm and 2cm of a1] (a3) {\(Z\)};
\vertex[below left=2cm of a1] (a4);
\vertex[below right=1cm and 2cm of a1] (a5) {\(W^{+}\)};
\vertex[above left=1cm of a2] (i1) {\(e^{-}\)};
\vertex[below left=1cm of a4] (i2) {\(e^{+}\)};
\vertex[above right=1cm and 3cm of a3] (i3) {\(q,l\)};
\vertex[below right=1cm and 3cm of a3] (i4) {\(\bar{q},\bar{l}\)};
\vertex[above right=1cm and 3cm of a5] (i5) {\(q,l\)};
\vertex[below right=1cm and 3cm of a5] (i6) {\(\bar{q},\bar{l}\)};
\vertex[above=1cm of a3] (v1) {\(e^{-}\)};
\vertex[below=1cm of a5] (v2) {\(\bar{\nu_{e}}\)};
\diagram* {
   (a1) -- [boson, edge label'=\(Z\)] (a2)
   (a1) -- [boson] (a3)
   (a1) -- [boson, edge label=\(W^{+}\)] (a4)
   (a1) -- [boson] (a5)
   (i1) -- [fermion] (a2) -- [fermion] (v1)
   (v2) -- [fermion] (a4) -- [fermion] (i2)
   (i4) -- [fermion] (a3) -- [fermion] (i3)
   (i6) -- [fermion] (a5) -- [fermion] (i5)
};
\end{feynman}
\end{tikzpicture}
\caption[Feynman diagram of vector boson scattering at CLIC.]{Feynman diagram of vector boson scattering at CLIC.  $q = \text{u, d, s, b, c}$ and $l = \text{e, } \mu \text{, } \tau \text{, } \nu_{e} \text{, } \nu_{\nu} \text{, } \nu_{\tau}$}
\label{fig:vectorbosonscatteringclic}
\end{figure}


%========================================================================================
%========================================================================================

\section{Event Generation, Simulation and Reconstruction}
\label{sec:eventgenerationandbackgrounds}

\textcolor{red}{Check this doesn't sound too close to Higgs paper.}

Events were generated for this analysis using the Whizard \cite{0708.4233, hep-ph/0102195} 1.95 program.  Due to the presence of beamstrahlung photons in the CLIC beam events were generated from collisions of $\text{e}^{+}\text{e}^{-}$, $\text{e}^{+}\gamma$, $\gamma\text{e}^{-}$ and $\gamma\gamma$.  The energy spectra used for all particles involved in these collisions accounted for the effects of radiation in the form of beamstrahlung photons and the intrinsic energy spread of the CLIC beam.  Furthermore, events involving the interaction between the electromagnetic field of the beam particles involving quasi-real photon mediators with low momenta, described by the Weizsacker-Williams approximation or the Equivalent Photon Approximation (EPA), were generated using Whizard and included in this analysis.  Fragmentation and hadronisation was implemented using PYTHIA 6.4 \cite{Sjostrand:2006za}, which was tuned for OPAL $\text{e}^{+}\text{e}^{-}$ collision data recorded at LEP (see \cite{Linssen:2012hp} for details).  The decays of tau leptons was simulated using Tauola \cite{Was:2000st}.  The full list of events simulated for this analysis, along with their standard model cross section at 1.4 TeV can be found in table \ref{table:crosssection1400GeV}.  The samples generated comprise all final states that would be relevant, either as signal or background processes, for an analysis involving the purely hadronic decay channels involved in a vector boson scattering process.  In full they are:

\begin{itemize}
\item Vector boson scattering signal final states that are expected to show sensitivity to the anomalous couplings: $\text{e}^{+}\text{e}^{-} \rightarrow \nu\nu\text{qqqq}$, $\text{e}^{+}\text{e}^{-} \rightarrow \text{l}\nu\text{qqqq}$ and $\text{e}^{+}\text{e}^{-} \rightarrow \text{llqqqq}$
\item Four jet final states arising from $\text{e}^{+}\text{e}^{-}$ interactions: $\text{e}^{+}\text{e}^{-} \rightarrow \text{qqqq}$.
\item Two jet final states arising from $\text{e}^{+}\text{e}^{-}$ interactions: $\text{e}^{+}\text{e}^{-} \rightarrow \nu{\nu}\text{qq}$, $\text{e}^{+}\text{e}^{-} \rightarrow \text{l}\nu\text{qq}$, $\text{e}^{+}\text{e}^{-} \rightarrow \text{llqq}$ and $\text{e}^{+}\text{e}^{-} \rightarrow \text{qq}$.
\item Four jet final states arising from the interactions of either $\text{e}^{+}$ or $\text{e}^{-}$ with a beamstrahlung photon: $\gamma_{\text{BS}}\text{e}^{-} \rightarrow \text{qqqq}\text{e}^{-}$, $\text{e}^{+}\gamma_{\text{BS}} \rightarrow \text{qqqq}\text{e}^{+}$, $\gamma_{\text{BS}}\text{e}^{-} \rightarrow \text{qqqq}\nu$ and $\text{e}^{+}\gamma_{\text{BS}} \rightarrow \text{qqqq}\nu$.
\item Four jet final states arising from the interactions of either $\text{e}^{+}$ or $\text{e}^{-}$ with the electromagnetic field of the opposing beam particle.  These cross sections are calculated using the EPA approximation, which represents the electromagnetic field of the opposing beam particle as a series of photons, so the final states appear as interactions of $\text{e}^{+}$ or $\text{e}^{-}$ with photons: $\gamma_{\text{EPA}}\text{e}^{-} \rightarrow \text{qqqq}\text{e}^{-}$, $\text{e}^{+}\gamma_{\text{EPA}} \rightarrow \text{qqqq}\text{e}^{+}$, $\gamma_{\text{EPA}}\text{e}^{-} \rightarrow \text{qqqq}\nu$ and $\text{e}^{+}\gamma_{\text{EPA}} \rightarrow \text{qqqq}\nu$.
\item Four jet final states arising from the interaction of the electromagnetic fields of opposing beam particles using the EPA approximation: $\gamma_{\text{EPA}}\gamma_{\text{EPA}} \rightarrow \text{qqqq}$.
\item Four jet final states arising from the interaction of the electromagnetic field of either $\text{e}^{+}$ or $\text{e}^{-}$ using the EPA approximation with a beamstrahlung photon: $\gamma_{\text{EPA}}\gamma_{\text{BS}} \rightarrow \text{qqqq}$ or $\gamma_{\text{BS}}\gamma_{\text{EPA}} \rightarrow \text{qqqq}$.
\item Four jet final states arising from the interaction of two beamstrahlung photons: $\gamma_{\text{BS}}\gamma_{\text{BS}} \rightarrow \text{qqqq}$.
\end{itemize}

Note: In the above list q = u, d, s, c and b, l = e, $\mu$, $\tau$ and $\nu$ = $\nu_{e}$, $\nu_{\mu}$ and $\nu_{\tau}$.

The samples used in this analysis were simulated with the CLID\_ILD detector model \cite{arXiv:1006.3396}.  The simulation was performed in MOKKA \cite{MoradeFreitas:2002kj}, a GEANT4 \cite{Agostinelli:2002hh} wrapper providing detailed geometric descriptions of detector concepts for the linear collider.  Events were reconstructed using MARLIN \cite{Gaede:2006pj}, a c++ framework designed for reconstruction at the linear collider.  PandoraPFA \cite{arXiv:0907.3577, arXiv:1209.4039} was used to apply Particle Flow Calorimetry in the reconstruction, the full details of which can be found in chapter PANDORA CHAPTER.
 
The CLIC\_ILD is a variant of the ILD detector described in section REFERENCE.  The only significant difference between the models is that CLIC\_ILD has a 60 layer scintillator-tugsten HCal in comparison to the 48 layers found in the default ILD detector.  The thicknesses of the layers in the HCal models are identical, so the extra layers correspond to an increase in the total thickness of the HCal.  This is needed to compensate for the effects of leakage at the higher energies seen by the CLIC experiment in comparison to the ILC. 

\begin{table}[h!]
\centering
\begin{tabular}{ l r }
\hline
Final State & Cross Section 1.4 TeV [fb] \\ 
\hline
$\text{e}^{+}\text{e}^{-} \rightarrow \nu{\nu}\text{qqqq}$ & 24.7 \\
$\text{e}^{+}\text{e}^{-} \rightarrow \text{l}\nu\text{qqqq}$ & 110.4\\
$\text{e}^{+}\text{e}^{-} \rightarrow \text{llqqqq}$ & 62.1\\
$\text{e}^{+}\text{e}^{-} \rightarrow \text{qqqq}$ & 1245.1\\
$\text{e}^{+}\text{e}^{-} \rightarrow \nu{\nu}\text{qq}$ & 787.7\\
$\text{e}^{+}\text{e}^{-} \rightarrow \text{l}\nu\text{qq}$ & 4309.7\\
$\text{e}^{+}\text{e}^{-} \rightarrow \text{llqq}$ & 2725.8\\
$\text{e}^{+}\text{e}^{-} \rightarrow \text{qq}$ & 4009.5\\
$\gamma_{\text{EPA}}\text{e}^{-} \rightarrow \text{qqqq}\text{e}^{-}$ & 287.1\\
$\gamma_{\text{BS}}\text{e}^{-} \rightarrow \text{qqqq}\text{e}^{-}$ & 1160.7\\
$\text{e}^{+}\gamma_{\text{EPA}} \rightarrow \text{qqqq}\text{e}^{+}$ & 286.9\\
$\text{e}^{+}\gamma_{\text{BS}} \rightarrow \text{qqqq}\text{e}^{+}$ & 1156.3\\
$\gamma_{\text{EPA}}\text{e}^{-} \rightarrow \text{qqqq}\nu$ & 32.6\\
$\gamma_{\text{BS}}\text{e}^{-} \rightarrow \text{qqqq}\nu$ & 136.9\\
$\text{e}^{+}\gamma_{\text{EPA}} \rightarrow \text{qqqq}\nu$ & 32.6\\
$\text{e}^{+}\gamma_{\text{BS}} \rightarrow \text{qqqq}\nu$ & 136.4\\
$\gamma_{\text{EPA}}\gamma_{\text{EPA}} \rightarrow \text{qqqq}$ & 753.0\\
$\gamma_{\text{EPA}}\gamma_{\text{BS}} \rightarrow \text{qqqq}$ & 4034.8\\
$\gamma_{\text{BS}}\gamma_{\text{EPA}} \rightarrow \text{qqqq}$ & 4018.7\\
$\gamma_{\text{BS}}\gamma_{\text{BS}} \rightarrow \text{qqqq}$ & 21406.2\\
\hline
\end{tabular}
\caption[Cross sections of signal and background processes at 1.4 TeV]{Cross sections of signal and background processes at 1.4 TeV. In the above table q $\in$ u, $\bar{\text{u}}$, d, $\bar{\text{d}}$, s, $\bar{\text{s}}$, c, $\bar{\text{c}}$, b or $\bar{\text{b}}$ while l $\in$ $\text{e}^{\pm}$, $\mu^{\pm}$ or $\tau^{\pm}$ and $\nu$ $\in$ $\nu_{e}$, $\nu_{\mu}$ and $\nu_{\tau}$.  The subscript EPA or BS for the incoming photons indicate whether the photon is generated from the equivalent photon approximation or beamstrahlung.}
\label{table:crosssection1400GeV}
\end{table}

%========================================================================================
%========================================================================================

\section{Modelling of Anomalous Gauge Couplings}
\label{sec:modellingofanomalouscouplings}
It was necessary when generating samples that are sensitive to the anomalous gauge couplings $\alpha_{4}$ and $\alpha_{5}$ to use Whizard version 1.97, instead of the previously quoted version 1.95.  This change was required as version 1.97 contained a unitarisation scheme that ensured cross sections for processes involving longitudinal gauge boson scattering were bound at the energies considered i.e. the TeV scale.  

The sensitivity of an individual event to the anomalous gauge couplings is determined through an event weight. This weight is given by the ratio of the squares of the matrix element used in the cross section calculation, one matrix element using non-zero values of $\alpha_{4}$ and $\alpha_{5}$ and the other matrix element using the standard model values of $\alpha_{4}$ and $\alpha_{5}$, i.e. 0.  The weight varies as a function of $\alpha_{4}$ and $\alpha_{5}$ as well as varying on an event by event basis as the kinematics of the final state changes.  Examples of the event weights as a function of $\alpha_{4}$ and $\alpha_{5}$ for selected events is shown in figure \ref{fig:eventweights1400raw} for 1.4 TeV \nu{\nu}qqqq final state events.

\begin{figure}
\centering
\subfloat[]{\label{fig:weight1}\includegraphics[width=0.5\textwidth]{PhysicsAnalysis/Plots/EventWeights/1400GeV/EventWeightsForEvent100001009_1400GeV_SPFOs_kt_0p70_10Bins_Start_0_End_10_1400GeV_Raw.pdf}}
\subfloat[]{\label{fig:weight2}\includegraphics[width=0.5\textwidth]{PhysicsAnalysis/Plots/EventWeights/1400GeV/EventWeightsForEvent100001014_1400GeV_SPFOs_kt_0p70_10Bins_Start_0_End_10_1400GeV_Raw.pdf}} \hfill
\subfloat[]{\label{fig:weight3}\includegraphics[width=0.5\textwidth]{PhysicsAnalysis/Plots/EventWeights/1400GeV/EventWeightsForEvent100001044_1400GeV_SPFOs_kt_0p70_10Bins_Start_0_End_10_1400GeV_Raw.pdf}}
\subfloat[]{\label{fig:weight4}\includegraphics[width=0.5\textwidth]{PhysicsAnalysis/Plots/EventWeights/1400GeV/EventWeightsForEvent100001051_1400GeV_SPFOs_kt_0p70_10Bins_Start_0_End_10_1400GeV_Raw.pdf}}
\caption[Event weights from Whizard for 1.4TeV \nu{\nu}qqqq final state events.]{A selection of plots showing how the event weight changes when varying the anomalous couplings $\alpha_{4}$ and $\alpha_{5}$ for 1.4TeV \nu{\nu}qqqq final state events. - Yellow is hard to see.}
\label{fig:eventweights1400raw}
\end{figure}

This reweighting procedure has many advantages over the alternative of generating new samples with fixed $\alpha_{4}$ and $\alpha_{5}$, notably the absence of systematic errors arising from new event generation, simulation and reconstruction.  Only final states showing a sensitivity to $\alpha_{4}$ and $\alpha_{5}$ require reweighting.  To determine those states a comparison was made between the cross section using the standard model values of $\alpha_{4}$ and $\alpha_{5}$, i.e. 0, and the same calculation using non-zero values of these couplings at 1.4 TeV.  This comparison was performed on all of the generated samples listed in table \ref{table:crosssection1400GeV} and the results for samples showing sensitivity to the couplings can be found in table \ref{table:crosssectionsensitivity1400}

The cross sections were found to differ when using non-zero values for the anomalous couplings in comparison to the standard model prediction only for the vector boson scattering signal final states $\nu\nu\text{qqqq}$, $\text{l}\nu\text{qqqq}$ and llqqqq.  In reality, non-zero anomalous couplings would change the cross sections of all processes considered; however, the sensitivity would only arise from high order terms in the Lagrangian.  Such terms would not be dominant in determining the cross section and so are omitted from the generator making certain final states appear invariant to changes in the anomalous couplings.

\begin{table}[h!]
\centering
\begin{tabular}{ l r r r r }
\hline
Final State & Cross Section [fb] & Cross Section [fb] & Percentage & CLIC Cross Section \\ 
& ($\alpha_{4} = \alpha_{5} = 0.00$) & ($\alpha_{4} = \alpha_{5} = 0.05$) & Change[\%] & [fb] \\ 
\hline
$\text{e}^{+}\text{e}^{-} \rightarrow \nu{\nu}\text{qqqq}$ & 20.8 & 34.6 & +66.3 & 24.7 \\
$\text{e}^{+}\text{e}^{-} \rightarrow \text{l}{\nu}\text{qqqq}$ & 112 & 113 & +0.9 & 115.3 \\
$\text{e}^{+}\text{e}^{-} \rightarrow \text{llqqqq}$ & 59.7 & 68.6 & +14.9 & 62.1 \\
\hline
\end{tabular}
\caption{Cross section for selected processes for given value of $\alpha_{4}$ and $\alpha_{5}$ at 1.4 TeV.}
\label{table:crosssectionsensitivity1400}
\end{table}

The cross section calculations show that the most sensitive final state to the anomalous gauge couplings is \nu{\nu}qqqq; therefore, this analysis will focus entirely upon this final state.  Furthermore, as the l{\nu}qqqq final state has a much reduced sensitivity in comparison to the \nu{\nu}qqqq state and as the llqqqq can be easily vetoed from the analysis, as will be shown in subsequent chapters, it is only necessary to consider the sensitivity of the \nu{\nu}qqqq final state.  For the aforementioned reasons the l{\nu}qqqq and llqqqq final states will be treated as backgrounds that are invariant to changes in the anomalous couplings $\alpha_{4}$ and $\alpha_{5}$.  

In order to determine the anomalous gauge coupling sensitive event weights it was necessary to use the anomalous gauge coupling model in Whizard, which enforces a unit CKM matrix.  In the context of vector boson scattering and the \nu{\nu}qqqq final state, which is the only final state requiring reweighting, this restricts the decays of the $\text{W}^{-}$ boson to d$\bar{\text{u}}$ and s$\bar{\text{c}}$, the $\text{W}^{+}$ boson to u$\bar{\text{d}}$ and c$\bar{\text{s}}$ and the Z boson to u$\bar{\text{u}}$ , d$\bar{\text{d}}$, s$\bar{\text{s}}$, c$\bar{\text{c}}$ and b$\bar{\text{b}}$.  In comparison, the non-unit CKM matrix allows for extra decay modes, however, this was found to have a negligible effect on the samples when comparing several reconstructed level distributions.  Furthermore, flavour tagging of jets was not used in this analysis as it offered negligible gains when performing event selection.  

%========================================================================================
%========================================================================================

\section{Data Analysis}
\label{sec:dataanalysis}
The focus of this section is to describe the post reconstruction procedure applied to the signal and background events, described in \ref{sec:crosssectioncheck}, to extract the relevant information needed for this sensitivity study. 

%========================================================================================

\subsection{Jet Finding} 
\label{sec:jetpairing}
After the reconstruction two further processors are applied to remove reconstructed particle flow objects (PFOs) that originate from beam related backgrounds, described in section CLIC BEAM CHAPER, from the event.  The first processor is the CLICTrackSelection, designed to veto poorly reconstructed tracks and to reject tracks where the time of arrival at the calorimeter between the helix fit to the track and to a straight line of flight differ by 50ns.  The latter would indicate the tracked particle does not create the calorimetric energy deposits that it has been associated to.  The second processor is the CLICPfoSelector, which applies cuts to the $p_{\text{T}}$ and timing information of the PFOs.  These cuts vary as a function of position in the detector and the reconstructed particle type in an attempt to target the regions of the detector where background, primarily low $p_{\text{T}}$ $\gamma\gamma \rightarrow \text{Hadrons}$ events, are more prominent.  Three configurations of the CLICPfoSelector have been developed for the CLIC environment and were considered for this analysis.  They are, in order of increasing background rejection, the Loose, Default and Tight selections. The full details of each can be found here \cite{arXiv:1209.4039}.

After the application of the CLICTrackSelection and CLICPfoSelector the MarlinFastJet processor, a wrapper for the FastJet \cite{Cacciari:2011ma} processor, was used to cluster the events into four jets.  These jets are then paired up to form two candidate bosons.  This pairing is performed on the assumption that the correct pairing is achieved when the difference between the invariant masses is a minima.  In the case of the signal final state, $\nu\nu$qqqq, it is assumed that the four jets and two candidate bosons map onto the four quarks and the two outgoing bosons involved in the vector boson scattering process.  The jet clustering was done using the longitudinally invariant $k_{t}$ jet algorithm in exclusive mode.  In contrast to the inclusive mode, the exclusive mode allows the user to request a fixed number of jets in the output from MarlinFastJet.  The longitudinally invariant $k_{t}$ algorithm proceeds as follows:

\begin{itemize}
\item For each pair of particles, i and j, the $k_{t}$ distance, $d_{ij}$, and beam distance, $d_{iB} = p_{t}^{2}$, was calculated.
\begin{equation}
d_{ij} = \text{min}(p_{ti}^{2}, p_{tj}^{2}){\Delta}R^{2}_{ij}/R^{2}
\end{equation}
where ${\Delta}R^{2}_{ij} = (y_{i} - y_{j})^2 + (\phi_{i} - \phi_{j})^2$, $p_{t}$ is the transverse momentum of the particle with respect to the beam axis, $y_{i}$ is the rapidity of particle i and $\phi_{i}$ is the azimuthal angle of particle i. $R$ is a configurable parameter that typically is of the order of 1.
\item The minimum distance, $d_\text{min}$, of all the $k_{t}$ and beam distances was found.  If the minimum occured for a $k_{t}$ distance, particles i and j were merged, summing their 4-momenta.  If the beam distance was the minima, particle i was declared to be part of the "beam" jet.  The particle was removed from the list of particles and not included in the final jet output.
\item This was repeated until the desired number of jets was created.  Alternatively, in inclusive mode this would be repeated until no particles are left in the event.
\end{itemize}

Two other clustering algorithms were considered, but, as figure \ref{fig:invariantmassalgoveto} shows, were found to be inappropriate for the experimental conditions at CLIC.  These alternative algorithm choices are applied in the same manor as the longitudinally invariant $k_{t}$ algorithm, however, they differ in the definition of the $k_{t}$ distance, $d_{ij}$, and the beam distance, $d_{iB}$.

The first alternative jet algorithm considered was the $k_{t}$ algorithm for $\text{e}^{+}\text{e}^{-}$ colliders (or Durham algorithm) where $d_{ij} = 2\text{min}(E_{i}^{2}, E_{j}^{2})(1-cos\theta_{ij})$ and $d_{iB}$ is not used.  $\theta_{ij}$ is the opening angle of particles i and j meaning that in the collinear limit $d_{ij}$ corresponds to the relative transverse momenta of the particles.  The major failure of this algorithm when applied to CLIC is the absence of $d_{iB}$, which leads to many beam related background particles being associated to jets.  As figure \ref{fig:invariantmassalgoveto} shows, the invariant mass of the paired jets, which should peak around the W and Z boson masses, is much larger than expected, due to the presence of the beam related backgrounds in the jets.  Also this algorithm is not invariant to boosts along the beam direction making it inappropriate to use at CLIC as the beam induced backgrounds modify the nominal collision kinematics.  

The second alternative jet algorithm considered was the Cambridge-Aachen jet algorithm where $d_{ij} = {\Delta}R_{ij}^{2}/R^2$ and $d_{iB} = 1$.  This algorithm performed poorly as neither accounts for the transverse momentum nor energy of the particles being clustered. In essence, this is a cone clustering algorithm with a cone radius defined through ${\Delta}R_{ij} = R$, which even for large R was found to discard too much energy in the event to be useful for this analysis.  This can be seen in figure \ref{fig:invariantmassalgoveto} as the invariant mass of the paired jets is much lower than expected.  This algorithm is useful for events with highly boosted jets, but at CLIC the jets are too disperse for this algorithm to be successfully applied.

\begin{figure}
\centering
\subfloat[]{\label{fig:invariantmassalgoveto1400GeV}\includegraphics[width=0.5\textwidth]{PhysicsAnalysis/Plots/SimpleInvMassPlot/InvariantMassesAlgorithmVeto.pdf}}
\caption[Reconstructed invariant masses for different choices of jet algorithm for 1.4 TeV \nu{\nu}qqqq events.]{Reconstructed masses for different choices of jet algorithm for 1.4 TeV \nu{\nu}qqqq events. These masses arise by forcing the reconstructed events into 4 jets and then pairing up the jets into pairs such that the reconstructed invariant masses of the pairs are closest to each other. These samples should be dominated by vector boson scattering involving pairs of W bosons and so it is expected that a peak at the W boson true mass should be observed. As this does not occur for the Cambridge-Aachen algorithm or the ee\_kt algorithm they were deemed unsuitable for this analysis at both 1.4. In the case of the kt algorithm and the ee\_kt algorithm an R parameter of 0.7 was used.}
\label{fig:invariantmassalgoveto}
\end{figure}

%========================================================================================

\subsubsection{Optimal Jet Finding Algorithm}
\label{sec:optimaljetalgorithm}
Optimisation of the jet algorithm configuration was performed on the choice of PFO selection as well as the value of the R parameter used in the longitudinally invariant $k_{t}$ algorithm.   The optimal configuration for the jet algorithm at 1.4 TeV was found to use default selected PFOs and an R parameter of 0.9.

This procedure involved performing the sensitivity study, described in section \ref{sec:fitting}, using solely the {\nu}{\nu}qqqq signal final state.  This procedure leads to the construction of a $\chi^{2}$ surface from which confidence contours can be extracted in the $\alpha_{4}$ and $\alpha_{5}$ space.  The $\chi^{2}$ surface for the optimal jet configuration at 1.4 TeV using the {\nu}{\nu}qqqq signal final state is shown in figure \ref{fig:chi2jetalgoideal1400GeV}.  This methodology ensured that the optimisation was done with respect to the physics of interest without having to perform the jet reconstruction for the large number of background events multiple times.  

Confidence limits on the individual parameters $\alpha_{4}$ and $\alpha_{5}$ were determined by setting the corresponding coupling term to zero and examining the now one dimensional $\chi^{2}$ distribution.  A fourth order polynomial was fitted to the minima of this distribution and the one sigma confidence limit defined using $\Delta\chi^{2}$ of 1.  $\Delta\chi^{2}$ is defined as the change in $\chi^{2}$ with respect to the minima in the $\chi^{2}$ surface.  Note that for the two dimensional $\chi^{2}$ surface a one sigma confidence limit is given by a $\Delta\chi^{2}$ of 2.28 due to the additional degree of freedom in the fit.  The one dimensional $\chi^{2}$ distribution for $\alpha_{4}$ and $\alpha_{5}$, assuming $\alpha_{5} = 0$ and $\alpha_{4} = 0$ respectively, for the optimal jet configuration at 1.4 TeV using the {\nu}{\nu}qqqq signal final state is shown in figures \ref{fig:a4chi2jetalgoideal1400GeV} and \ref{fig:a5chi2jetalgoideal1400GeV}.  Using these distributions the one sigma confidence limits on $\alpha_{4}$ are -0.0038 to 0.0047 and on $\alpha_{5}$ are -0.0027 to 0.0030.

\begin{figure}
\centering
\subfloat[$\chi^{2}$ sensitivity contours in $\alpha_{4}$ and $\alpha_{5}$ space.]{\label{fig:chi2jetalgoideal1400GeV}\includegraphics[width=0.5\textwidth]{PhysicsAnalysis/Plots/Chi2ContoursOptimisation/1400GeV/KtSPFOsR0p90.pdf}}\hfill
\subfloat[$\chi^{2}$ as a function of $\alpha_{4}$ assuming $\alpha_{5} = 0$.]{\label{fig:a4chi2jetalgoideal1400GeV}\includegraphics[width=0.5\textwidth]{PhysicsAnalysis/Plots/Chi2ContoursOptimisation/1400GeV/KtSPFOsR0p90_alpha4Optimal.pdf}}
\subfloat[$\chi^{2}$ as a function of $\alpha_{5}$ assuming $\alpha_{4} = 0$.]{\label{fig:a5chi2jetalgoideal1400GeV}\includegraphics[width=0.5\textwidth]{PhysicsAnalysis/Plots/Chi2ContoursOptimisation/1400GeV/KtSPFOsR0p90_alpha5Optimal.pdf}}
\caption[$\chi^{2}$ sensitivity distributions for the $\text{qqqq}\nu\nu$ final state arising from a fit to $\text{cos}\theta^{*}_{\text{Jets}}$ at 1.4 TeV for the optimal jet reconstruction parameters.]{$\chi^{2}$ sensitivity distributions for the $\text{qqqq}\nu\nu$ final state arising from a fit to $\text{cos}\theta^{*}_{\text{Jets}}$ at 1.4 TeV for the optimal jet reconstruction parameters. - Make the blue stand out more.} 
\label{fig:allchi2jetalgoideal1400GeV}
\end{figure}

%========================================================================================

\subsection{Lepton Finding} 
\label{sec:isolatedleptonfinding}
An isolated lepton finder was included in the analysis chain in an attempt to reject background events containing leptons.  The isolated lepton finder attempts to find whether a PFO is an electron or muon based on the calorimetric energy deposits.  Cuts are then placed on the tracks associated to any PFOs, initially tagged as electrons or muons, to determine whether the tracks originate from the impact point.  If the track cuts deem the PFO to have originated from the impact point, isolation cuts restricting the energy in a cone surrounding these PFO are applied to ensure the particles does not belong to a jet.  If a PFO passes all of these criteria then it is counted as an isolated lepton.  The efficiency of the lepton finder is summarised in table \ref{table:efficiencyleptonfinding}.  

\begin{table}[h!]
\centering
\begin{tabular}{ l r }
\hline
Final State & $\epsilon_{\text{Lepton Finding}}$ \\ 
\hline
$\text{e}^{+}\text{e}^{-} \rightarrow \nu{\nu}\text{qqqq}$ & 99.7 \\
$\text{e}^{+}\text{e}^{-} \rightarrow \text{l}\nu\text{qqqq}$ & 48.9 \\
\hline
\end{tabular}
\caption[The efficiency of isolated lepton finding at 1.4 TeV for the {\nu}{\nu}qqqq and l{\nu}qqqq final states.]{The efficiency of isolated lepton finding at 1.4 TeV for the {\nu}{\nu}qqqq and l{\nu}qqqq final states.  Efficiency here is defined as the fraction of events where no isolated leptons were found.}
\label{table:efficiencyleptonfinding}
\end{table}

%========================================================================================

\subsection{Discriminant Variables} 
\label{sec:analysisprocessor}
The next stage of the analysis involved the calculation of a number of event-based variables that were found to be useful for this analysis.  The variables that were calculated are as follows:

\begin{itemize}
\item \textbf{Particle level} variables:

\begin{itemize}
\item Number of PFOs.
\item Particle type and energy of the highest energy PFO.
\item Energy of the highest energy electron.
\item Cosine of the polar angle of the highest energy track.
\end{itemize}

\item \textbf{Candidate boson} variables:

\begin{itemize}
\item Energy of candidate bosons.
\item Invariant mass of the bosons.
\item Acolinearity of the boson pair, which is defined as 180 degrees minus the opening angle of the pair of bosons in the rest frame of the detector.
\item Acolinearities of the jets forming each of the candidate bosons.
\item $\text{Cos}(\theta^{*}_{Bosons})$.  Cosine of the opening of the bosons in the rest frame of the boson pair.
\item $\text{Cos}(\theta^{*}_{Jets})$.  Cosine of the opening of the jets forming each candidate in the rest frame of the candidate boson.
\end{itemize}

\item \textbf{Event based} variables:  

\begin{itemize}
\item The invariant mass of the visible system.
\item The vector sum of the transverse momentum of all PFOs in the event. 
\item The cosine of the polar angle of the missing 3-momentum assuming a collision at the nominal centre of mass energy.
\item Principle thrust $T$, defined through the following equation
\begin{equation}
T = \text{max}_{\bar{\textbf{n}}} (\frac{\Sigma_{i} \textbf{p}_{i}.\bar{\textbf{n}}}{\Sigma_{i} |\textbf{p}_{i}|^{2}})
\end{equation}
Where $p_{i}$ are the components of the momenta of PFO i in the rest frame of the detector, $\bar{\textbf{n}}$ is unit vector and the sum $\Sigma_{i}$ runs over all particles in the event.
\item Sphericity, defined through the sphericity tensor $S^{ab}$:
\begin{equation}
S^{ab} = \frac{\Sigma_{i}p^{\alpha}_{i}p^{\alpha}_{j}}{\Sigma_{i,\alpha=1,2,3}|p^{\alpha}_{i|^{2}}}
\end{equation}
Where $p_{i}$ are the components of the momenta of PFO i in the rest frame of the detector and the sum $\Sigma_{i}$ runs over all particles in the event.  Sphericity is defined as $\text{S} = \frac{3}{2}(\lambda_{2} + \lambda_{3})$, where $\lambda_{i}$ are the eigenvalues of the sphericity tensor defined such $\lambda_{1} \geq \lambda_{2} \geq \lambda_{3}$.  This provides a measure of how spherical the reconstructed event topology is with isotropic events having $S \approx 1$, while two jet events have $S \approx 0$.
\item Aplanarity defined as $\frac{3}{2} \lambda_{3}$ where $\lambda_{3}$ is an eigenvalue of the sphericity tensor.  This provides a measure of whether an event is linear or planar in shape.
\end{itemize}

\item \textbf{Jet clustering parameter} variables, $y_{ij}$ where $i = 1,2,3,4,5,6$ and $j=i+1$.  These are the smallest $k_{t}$ distance found when combining $j$ jets into $i$ jets.  Note: -log10 yij cut on.

\end{itemize}

%========================================================================================
%========================================================================================

\section{Event Selection}
\label{sec:eventselection}
As described in section \ref{sec:modellingofanomalouscouplings} the signal final state in this analysis is the \nu{\nu}qqqq final state, while the backgrounds consist of all 2 and 4 jet final states that could be confused for the signal state in the reconstruction.  A complete list of signal and background final states used for this analysis, alongside their standard model cross sections, can be found in table \ref{table:crosssection1400GeV}.  In an attempt to isolate signal from background, an event selection procedure consisting of a set of preselection cuts followed by the application of a multivariate analysis (MVA) was applied to this data set and the full details of those are given in the following section.

%========================================================================================

\subsection{Pre-Selection}
\label{sec:preselection1400GeV}
A refined selection of the \nu{\nu}qqqq signal final state is achieved using MVA.  However, to ensure efficiency in the training and application of that MVA a number of simple preselection cuts were developed that veto obvious background final states prior to the application of the MVA.  These cuts were developed such that as much background as possible would be rejected, while retaining enough signal to make the analysis viable.  Preselection cuts were applied to the transverse momentum, invariant mass of the visible system and the number of isolated leptons. The raw distributions of these variables is shown in figure \ref{fig:preselection1400} and based on these distributions the following cuts were applied:

\begin{itemize}
\item Transverse momentum > 100 GeV. This cut is effective due to the presence of missing energy in the form of neutrinos in the signal final state.
\item Visible mass of the system > 200 GeV.  This vetoes $\nu\nu\text{qq}$ and $\nu\text{lqq}$ final states as these states would be dominated by the channels involving a single W or Z boson propagator decaying hadronically, as shown in figure \ref{fig:backgroundresonancefd}.  Therefore the invariant mass of the visible system for these states peaks around the W and Z mass.  This is not the case for the \nu{\nu}qqqq signal final state, which typically has a much larger visible mass.
\item Number of isolated leptons = 0. This cut vetoes events with leptons in the final state.
\end{itemize}

The impact of these preselection cuts can be found in table FINAL SUMMARY TABLE REF \ref{}.

\iffalse
\begin{figure}
\centering
\subfloat[Transverse momentum of system.]{\label{fig:preselection1400_1}\includegraphics[width=0.5\textwidth]{PhysicsAnalysis/Plots/PreSelection/1400GeV/TransverseMomentum.pdf}}\hfill
\subfloat[Invariant mass of the visible system.]{\label{fig:preselection1400_2}\includegraphics[width=0.5\textwidth]{PhysicsAnalysis/Plots/PreSelection/1400GeV/InvariantMassSystem.pdf}}
\subfloat[Number of isolated leptons.]{\label{fig:preselection1400_3}\includegraphics[width=0.5\textwidth]{PhysicsAnalysis/Plots/PreSelection/1400GeV/NumberOfIsolatedLeptons.pdf}}
\caption[Distribution of variables cut on in the preselection at 1.4 TeV.]{Distribution of variables cut on in the preselection at 1.4 TeV.}
\label{fig:preselection1400}
\end{figure}

\begin{figure}
\begin{tikzpicture}[]
\begin{feynman}
\vertex (a1);
\vertex[above left=2cm of a1] (a2);
\vertex[right=2cm of a1] (a3) {\(Z\)};
\vertex[below left=2cm of a1] (a4);
\vertex[above left=1cm of a2] (i1) {\(e^{-}\)};
\vertex[below left=1cm of a4] (i2) {\(e^{+}\)};
\vertex[above right=1cm and 3cm of a3] (i3) {\(q\)};
\vertex[below right=1cm and 3cm of a3] (i4) {\(\bar{q}\)};
\vertex[above=2cm of a3] (v1) {\(\nu_{e}\)};
\vertex[below=2cm of a3] (v2) {\(\bar{\nu_{e}}\)};
\diagram* {
   (a1) -- [boson, edge label'=\(W^{-}\)] (a2)
   (a1) -- [boson] (a3)
   (a1) -- [boson, edge label=\(W^{+}\)] (a4)
   (i1) -- [fermion] (a2) -- [fermion] (v1)
   (v2) -- [fermion] (a4) -- [fermion] (i2)
   (i4) -- [fermion] (a3) -- [fermion] (i3)
};
\end{feynman}
\end{tikzpicture}

\begin{tikzpicture}[]
\begin{feynman}
\vertex (a1);
\vertex[above left=2cm of a1] (a2);
\vertex[right=2cm of a1] (a3) {\(W^{+}\)};
\vertex[below left=2cm of a1] (a4);
\vertex[above left=1cm of a2] (i1) {\(e^{-}\)};
\vertex[below left=1cm of a4] (i2) {\(e^{+}\)};
\vertex[above right=1cm and 3cm of a3] (i3) {\(q\)};
\vertex[below right=1cm and 3cm of a3] (i4) {\(\bar{q}\)};
\vertex[above=2cm of a3] (v1) {\(e^{-}\)};
\vertex[below=2cm of a3] (v2) {\(\bar{\nu_{e}}\)};
\diagram* {
   (a1) -- [boson, edge label'=\(Z\)] (a2)
   (a1) -- [boson] (a3)
   (a1) -- [boson, edge label=\(W^{+}\)] (a4)
   (i1) -- [fermion] (a2) -- [fermion] (v1)
   (v2) -- [fermion] (a4) -- [fermion] (i2)
   (i4) -- [fermion] (a3) -- [fermion] (i3)
};
\end{feynman}
\end{tikzpicture}

\begin{tikzpicture}[]
\begin{feynman}
\vertex (a1);
\vertex[above left=2cm of a1] (a2);
\vertex[right=2cm of a1] (a3) {\(W^{-}\)};
\vertex[below left=2cm of a1] (a4);
\vertex[above left=1cm of a2] (i1) {\(e^{-}\)};
\vertex[below left=1cm of a4] (i2) {\(e^{+}\)};
\vertex[above right=1cm and 3cm of a3] (i3) {\(q\)};
\vertex[below right=1cm and 3cm of a3] (i4) {\(\bar{q}\)};
\vertex[above=2cm of a3] (v1)  {\(\nu_{e}\)};
\vertex[below=2cm of a3] (v2) {\(e^{+}\)};
\diagram* {
   (a1) -- [boson, edge label'=\(W^{-}\)] (a2)
   (a1) -- [boson] (a3)
   (a1) -- [boson, edge label=\(Z\)] (a4)
   (i1) -- [fermion] (a2) -- [fermion] (v1)
   (v2) -- [fermion] (a4) -- [fermion] (i2)
   (i4) -- [fermion] (a3) -- [fermion] (i3)
};
\end{feynman}
\end{tikzpicture}

\caption[Feynman diagram dominant in the $\nu\nu\text{qq}$ and $\nu\text{lqq}$ final states at CLIC.]{Feynman diagram dominant in the $\nu\nu\text{qq}$ and $\nu\text{lqq}$ final states at CLIC.  $q = \text{u, d, s, b, c}$}
\label{fig:backgroundresonancefd}
\end{figure}
\fi

%========================================================================================

\subsection{Multivariate analysis}
\label{sec:mva1400GeV}
Having established the preselection cuts a MVA was applied, using the TMVA toolkit \cite{Hocker:2007ht}, to refine the event selection.  The signal and background final state samples were halved; one half sample was used to train the MVA and the remaining half sample was used in the subsequent analysis.  The halving of the signal and background sample had minimal impact on the analysis as all event numbers were normalised to the correct luminosity for CLIC running at 1.4 TeV and the sample size was sufficiently large.

The following variables were used for training of the MVA:

\begin{itemize}
\item Number of PFOs.
\item Particle type and energy of the highest energy PFO.
\item Energy of the highest energy electron.
\item Cosine of the polar angle of the highest energy track.
\item Energy and invariant mass of the candidate bosons.
\item Acolinearity of the boson pair.
\item Acolinearities of the jets forming each of the candidate bosons.
\item The vector sum of the transverse momentum of all PFOs in the event. 
\item The cosine of the polar angle of the missing 3-momentum assuming a collision at the nominal centre of mass energy.
\item Principle thrust.
\item Sphericity and aplanarity.
\item The jet clustering parameter $y_{ij}$ where $i = 1,2,3,4,5,6$ and $j=i+1$.
\end{itemize}

A variety of MVA options were considered and it was found that the optimal algorithm was the boosted decision tree (BDT) as shown by figure \ref{ig:mvaalternatives1400GeV}.  

\begin{figure}
\centering
\includegraphics[width=0.75\textwidth]{PhysicsAnalysis/Plots/MVAPlots/1400GeV/ThesisPlotMVAAlternatives1400GeV.pdf}
\caption[Background rejection as a function of signal efficiency for a variety of MVA options at 1.4 TeV.]{Background rejection as a function of signal efficiency for a variety of MVA options at 1.4 TeV.} 
\label{fig:mvaalternatives1400GeV}
\end{figure}

The BDT was further optimised by varying the number of trees used, the depth of the trees and the number of cuts applied and an optimal significance, S/$\sqrt(\text{S + B})$, of 53.6 was obtained.  

%========================================================================================

\subsection{Event Selection Summary}
\label{sec:eventselsummary1400GeV}
The event selection is summarised using the distribution of the invariant mass of the candidate bosons, which for the signal final state should peak around the W and Z masses.  This distribution is shown in figure \ref{fig:synbosonmass1400GeVMVAimpact} with no event selection, with the preselection cuts and with both preselections cuts and MVA applied.  The event selection is also summarised using the efficiencies that are shown in table \ref{table:selectionsummary1400GeV}.

As expected the dominant background processes after the MVA is applied are those that will look identical to the visible signal process i.e. qqqq and missing energy.  Two smaller sources of background that pass the MVA exists, those where two jets and missing energy are confused as four jets and missing energy and those where a lepton is not properly reconstructed and the events look like four jets and missing energy.  

\begin{figure}
\centering
\subfloat[No cuts applied.]{\label{fig:nocutssynbosonmass1400GeVMVAimpact}\includegraphics[width=0.5\textwidth]{PhysicsAnalysis/Plots/PostMVASelection/1400GeV/InvariantMassSynBosons_1400GeV_No_Cuts_StackPlot.pdf}}\hfill
\subfloat[Preselection cuts applied.]{\label{fig:nocutssynbosonmass1400GeVMVAimpact}\includegraphics[width=0.5\textwidth]{PhysicsAnalysis/Plots/PostMVASelection/1400GeV/InvariantMassSynBosons_1400GeV_Pt_gt100GeV_MVis_gt200GeV_NIsoLep_eq0_Cuts_StackPlot.pdf}}
\subfloat[Preselection cuts and MVA applied.]{\label{fig:postmvasynbosonmass1400GeVMVAimpact}\includegraphics[width=0.5\textwidth]{PhysicsAnalysis/Plots/PostMVASelection/1400GeV/InvariantMassSynBosons_1400GeV_PostPreSelection_PostMVA_Cuts_StackPlot.pdf}} 
\caption[Impact of preselection and MVA on the reconstructed invariant mass of the bosons arising from jet pairing at 1.4 TeV.]{Impact of preselection and MVA on the reconstructed invariant mass of the bosons arising from jet pairing at 1.4 TeV.}
\label{fig:synbosonmass1400GeVMVAimpact}
\end{figure}

\begin{table}[h!]
\centering
\begin{tabular}{ l r r r }
\hline
Final State & $\epsilon_{\text{presel}}$ & $\epsilon_{\text{BDT}}$ & $N_{\text{BDT}}$ \\ 
\hline
$\text{e}^{+}\text{e}^{-} \rightarrow \nu{\nu}\text{qqqq}$ & 56.7\% & 39.9\% & 14,770 \\
$\text{e}^{+}\text{e}^{-} \rightarrow \text{l}\nu\text{qqqq}$ & 25.7\% & 3.7\% & 6,159 \\
$\text{e}^{+}\text{e}^{-} \rightarrow \text{llqqqq}$ & 0.7\% & 0.1\% & 80 \\
$\text{e}^{+}\text{e}^{-} \rightarrow \text{qqqq}$ & 8.8\% & 0.1\% & 1,264 \\
$\text{e}^{+}\text{e}^{-} \rightarrow \nu{\nu}\text{qq}$ & 4.3\% & 0.3\% & 3,286 \\
$\text{e}^{+}\text{e}^{-} \rightarrow \text{l}\nu\text{qq}$ & 8.8\% & 0.1\% & 6,262 \\
$\text{e}^{+}\text{e}^{-} \rightarrow \text{llqq}$ & 0.1\% & - & 234 \\
$\text{e}^{+}\text{e}^{-} \rightarrow \text{qq}$ & 0.6\% & - & 1,016 \\
$\gamma_{\text{EPA}}\text{e}^{-} \rightarrow \text{qqqq}\text{e}^{-}$ & 0.2\% & - & 20 \\
$\gamma_{\text{BS}}\text{e}^{-} \rightarrow \text{qqqq}\text{e}^{-}$ & 0.1\% & - & 42 \\
$\text{e}^{+}\gamma_{\text{EPA}} \rightarrow \text{qqqq}\text{e}^{+}$ & 0.3\% & - & 19 \\
$\text{e}^{+}\gamma_{\text{BS}} \rightarrow \text{qqqq}\text{e}^{+}$ & - & - & 44 \\
$\gamma_{\text{EPA}}\text{e}^{-} \rightarrow \text{qqqq}\nu$ & 18.0\% & 7.3\% & 3,552 \\
$\gamma_{\text{BS}}\text{e}^{-} \rightarrow \text{qqqq}\nu$ & 23.2\% & 12.0\% & 18,540 \\
$\text{e}^{+}\gamma_{\text{EPA}} \rightarrow \text{qqqq}\nu$ & 18.2\% & 7.5\% & 3,652 \\
$\text{e}^{+}\gamma_{\text{BS}} \rightarrow \text{qqqq}\nu$ & 23.4\% & 12.2\% & 18,770 \\
$\gamma_{\text{EPA}}\gamma_{\text{EPA}} \rightarrow \text{qqqq}$ & 0.2\% & - & 68 \\
$\gamma_{\text{EPA}}\gamma_{\text{BS}} \rightarrow \text{qqqq}$ & 0.1\% & - & 55 \\
$\gamma_{\text{BS}}\gamma_{\text{EPA}} \rightarrow \text{qqqq}$ & - & - & 0 \\
$\gamma_{\text{BS}}\gamma_{\text{BS}} \rightarrow \text{qqqq}$ & - & - & 0 \\
\hline
\end{tabular}
\caption[Selection summary at 1.4TeV.]{Selection summary at 1.4TeV.   The subscript EPA or BS for the incoming photons indicate whether the photon is generated from the equivalent photon approximation or beamstrahlung.  Cells omitting the efficiency indicate an efficiency of less than 0.1\%.}
\label{table:selectionsummary1400GeV}
\end{table}

%========================================================================================
%========================================================================================

\section{Effect of Anomalous Coupling/Fitting Methodology}
\label{sec:fitting}
This section describes the procedure used for constructing the $\chi^{2}$ surface and the subsequent confidence contours used to determine the sensitivity of CLIC to the anomalous gauge couplings $\alpha_{4}$ and $\alpha_{5}$.

\subsection{Sensitive Distribution}
The sensitivity of CLIC to the anomalous gauge couplings is determined through the use of a $\chi^{2}$ fit to the distribution of $\text{cos}\theta^{*}_{Jets}$.  For a given event, the jet clustering and pairing proceeds as described in section \ref{sec:analysis} and leads to the event being clustered into four jets, which are then paired up to give two candidate bosons.  $\theta^{*}_{Jets}$ is defined as the opening angle of the jets in the rest frame of these candidate bosons.  The distribution of $\text{cos}\theta^{*}_{Jets}$ proved to be highly sensitive to the anomalous gauge couplings as shown in figure \ref{fig:costhetastarjets}.

The distribution of $\text{cos}\theta^{*}_{Bosons}$ was also considered for this sensitivity study, however, it proved to be less sensitive than $\text{cos}\theta^{*}_{Jets}$.  $\theta^{*}_{Bosons}$ is defined as the opening angle between the two candidate bosons in the rest frame of the candidate boson pair.  The reduced sensitivity can be seen when comparing figures \ref{fig:costhetastarjets} and \ref{fig:costhetastarbosons}.  Furthermore, it was found that the $\chi^{2}$ distribution formed from the two dimensional distribution of $\text{cos}\theta^{*}_{Jets}$ against $\text{cos}\theta^{*}_{Bosons}$ did not significantly benefit the sensitivity in comparison using the one dimensional distribution of $\text{cos}\theta^{*}_{Jets}$ and therefore was not considered for this analysis.

\begin{figure}
\subfloat[1.4 TeV Events]{\label{fig:costhetastarjets1400GeV} \includegraphics[width=0.5\textwidth]{PhysicsAnalysis/Plots/SensitiveDistributions/CosThetaStarSynJets_SPFOs_kt_0p70_1400GeV.pdf}}
\subfloat[3 TeV Events]{\label{fig:costhetastarjets3000GeV} \includegraphics[width=0.5\textwidth]{PhysicsAnalysis/Plots/SensitiveDistributions/CosThetaStarSynJets_SPFOs_kt_0p70_3000GeV.pdf}}
\caption[Sensitivity of $\text{cos}\theta^{8}_{Jets}$ to the anomalous gauge couplings $\alpha_{4}$ and $\alpha_{5}$ at 1.4 and 3 TeV.]{Sensitivity of $\text{cos}\theta^{*}_{Jets}$ to anomalous couplings at 1.4 and 3 TeV. The jet algorithm used for this example was the longitudinally invariant kt algorithm with an R parameter of 0.7. This sample corresponds to pure signal of hadronic decays in vector boson scattering i.e. \nu{\nu}qqqq.}
\label{fig:costhetastarjets}
\end{figure}

\begin{figure}
\subfloat[1.4 TeV Events]{\label{fig:costhetastarbosons1400GeV} \includegraphics[width=0.5\textwidth]{PhysicsAnalysis/Plots/SensitiveDistributions/CosThetaStarSynBosons_SPFOs_kt_0p70_1400GeV.pdf}}
\subfloat[3 TeV Events]{\label{fig:costhetastarbosons3000GeV} \includegraphics[width=0.5\textwidth]{PhysicsAnalysis/Plots/SensitiveDistributions/CosThetaStarSynBosons_SPFOs_kt_0p70_3000GeV.pdf}}
\caption[Sensitivity of $\text{cos}\theta^{8}_{Bosons}$ to the anomalous gauge couplings $\alpha_{4}$ and $\alpha_{5}$ at 1.4 and 3 TeV.]{Sensitivity of $\text{cos}\theta^{*}_{Bosons}$ to anomalous couplings at 1.4 and 3 TeV. The jet algorithm used for this example was the longitudinally invariant $k_{t}$ algorithm with an R parameter of 0.7. This sample corresponds to pure signal of hadronic decays in vector boson scattering i.e. \nu{\nu}qqqq. - Make legend bigger!}
\label{fig:costhetastarbosons}
\end{figure}

%========================================================================================

\subsection{$\chi^{2}$ Surface Definition}
\label{sec:chi2surfacedefinition}
The $\chi^{2}$ surface is defined through the following equation:

\begin{equation}
\chi^{2} = \Sigma_{i} \frac{(O_{i} - E_{i})^{2}}{E_{i}}
\end{equation}

Where $O_{i}$ is the observed, $\alpha_{4} = \alpha_{5} = 0$, and $E_{i}$ the expected, $\alpha_{4} \neq 0$ and $\alpha_{5} \neq 0$, bin content for bin i in the distribution of $\text{cos}\theta^{*}_{Jets}$.  $\Sigma_{i}$ is the sum over the bins of the $\text{cos}\theta^{*}_{Jets}$ distribution.  The distribution of $\text{cos}\theta^{*}_{Jets}$ was binned in a histograms containing 10 bins ranging from 0 to 1, as shown in figure \ref{fig:costhetastarjets}.  This binning was selected to maximise the sensitivity of the distribution, while minimising the effect of large bin by bin fluctuations arising from individual events with large event weights.

Confidence limits, which describe the sensitivity of CLIC to the anomalous gauge couplings, were found by examining the $\chi^{2}$ surface in $\alpha_{4}$ and $\alpha_{5}$ space.  Deviations about the minima of this surface, which by construction occurs at $\alpha_{4} = \alpha_{5} = 0$, yield confidence limits that indicate the probability of observing a particular value of $\alpha_{4}$ and $\alpha_{5}$ based on the $\text{cos}\theta^{*}_{Jets}$ distribution.  The confidence limits used in subsequent sections, 68\%, 90\% and 99\%, are defined using fixed deviations from the minima of $\chi^{2}$ contours of 2.28, 4.61 and 9.21 respectively.  These numbers arise from the integral of the two dimensional $\chi^{2}$ function.

It proved useful to consider the sensitivities to the individual parameters $\alpha_{4}$ and $\alpha_{5}$ independently.  This was done by projecting out the $\alpha_{4} = 0$ or $\alpha_{5} = 0$ one dimensional $\chi^{2}$ distribution from the two dimensional $\chi^{2}$ discussed in section \ref{sec:optimaljetalgorithm}.  It was then possible to extract the sensitivity to an individual parameters using confidence limits arising from the integral of the one dimensional $\chi^{2}$ function i.e. 68\% confidence limit occurs for $\chi^{2} = 0.989$.  In subsequent chapters these are the sensitivities quoted for individual anomalous gauge coupling parameters. 

%========================================================================================

\subsection{Event Weight Interpolation Scheme}
\label{sec:eventweightsinterpolation}
As described in section \ref{sec:eventweights}, event weights are used to determine the sensitivity of CLIC to the anomalous gauge couplings.  These event weights are extracted on an event by event basis for the signal final state $\nu\nu\text{qqqq}$ from the generator software Whizard.  To achieve a smooth $\chi^{2}$ distribution a fine sampling of the $\text{cos}\theta^{*}_{Jets}$ distribution in the $\alpha_{4}$ and $\alpha_{5}$ space is needed.  However, as extracting the event weights is highly CPU intensive, it is unfeasible to produce a finely sampled grid of event weights on an event by event basis by calling the generator.  To resolve this issue, an interpolation scheme was applied to determine the event weights within a sampled region of the $\alpha_{4}$ and $\alpha_{5}$ space.  This allows for an infinite sampling of the $\text{cos}\theta^{*}_{Jets}$ distribution in the space of $\alpha_{4}$ and $\alpha_{5}$ within the sampled region, without having to call the generator an infinite number of times.

A bicubic interpolation scheme, cubic interpolation along two dimensions, was applied to the event weights that were extracted from the generator.  This procedure is best illustrated by showing the interpolated surface superimposed with the raw event weights from the generator, which is shown for several $\nu\nu\text{qqqq}$ events at 1.4 TeV in figure \ref{fig:eventweights1400interpolated}.  This interpolation scheme produces a smooth and continuous surface that is sufficiently accurate for the fitting procedure applied in this analysis.  

For reference, at 1.4 TeV event weights were produced from the generator, Whizard, by stepping along $\alpha_{4}$ and $\alpha_{5}$ in steps of 0.01 ranging from -0.07 to 0.07, as shown in figure \ref{fig:eventweights1400raw}.  These range proved to be sufficient for the contours of interest for the CLIC sensitivity analysis at these energies.

\begin{figure}
\centering
\subfloat[]{\label{fig:weight1}\includegraphics[width=0.5\textwidth]{PhysicsAnalysis/Plots/EventWeights/1400GeV/EventWeightsForEvent100001009_1400GeV_SPFOs_kt_0p70_10Bins_Start_0_End_10_1400GeV_Interpolated.pdf}}
\subfloat[]{\label{fig:weight2}\includegraphics[width=0.5\textwidth]{PhysicsAnalysis/Plots/EventWeights/1400GeV/EventWeightsForEvent100001014_1400GeV_SPFOs_kt_0p70_10Bins_Start_0_End_10_1400GeV_Interpolated.pdf}} \hfill
\subfloat[]{\label{fig:weight3}\includegraphics[width=0.5\textwidth]{PhysicsAnalysis/Plots/EventWeights/1400GeV/EventWeightsForEvent100001044_1400GeV_SPFOs_kt_0p70_10Bins_Start_0_End_10_1400GeV_Interpolated.pdf}}
\subfloat[]{\label{fig:weight4}\includegraphics[width=0.5\textwidth]{PhysicsAnalysis/Plots/EventWeights/1400GeV/EventWeightsForEvent100001051_1400GeV_SPFOs_kt_0p70_10Bins_Start_0_End_10_1400GeV_Interpolated.pdf}}
\caption[Event weights from Whizard for 1.4TeV \nu{\nu}qqqq final state events with interpolated surface.]{A selection of plots showing how the event weight changes when varying the anomalous couplings $\alpha_{4}$ and $\alpha_{5}$ for 1.4TeV \nu{\nu}qqqq final state events.  The hollow circles show the event weight produced from the generator while the surface shown is found using bicubic interpolation between those points.}
\label{fig:eventweights1400interpolated}
\end{figure}

%========================================================================================
%========================================================================================

\section{Results}
The sensitivity of the CLIC experiment to the anomalous gauge couplings $\alpha_{4}$ and $\alpha_{5}$ at 1.4 TeV is shown in figure \ref{fig:finalresult1400GeV}.  This result shows the sensitivity after the application of preselection and MVA described in sections \ref{sec:preselection1400GeV} and \ref{sec:mva1400GeV} purposed to remove the included background channels, described in section \ref{sec:eventgenerationandbackgrounds}.  These contours yield the one $\sigma$ confidence limit on the measurement of $\alpha_{4}$ to the range -0.00831, 0.0130 and similarly for the measurement of $\alpha_{5}$ to the range -0.00606, 0.00904.

\begin{figure}
\centering
\subfloat[$\chi^{2}$ sensitivity contours in $\alpha_{4}$ and $\alpha_{5}$ space.]{\label{fig:finalresult1400GeV}\includegraphics[width=0.5\textwidth]{PhysicsAnalysis/Plots/FinalResult/1400GeV/Final.pdf}}\hfill
\subfloat[$\chi^{2}$ as a function of $\alpha_{4}$ assuming $\alpha_{5} = 0$.]{\label{fig:a4finalresult1400GeV}\includegraphics[width=0.5\textwidth]{PhysicsAnalysis/Plots/FinalResult/1400GeV/Final_alpha4.pdf}}
\subfloat[$\chi^{2}$ as a function of $\alpha_{5}$ assuming $\alpha_{4} = 0$.]{\label{fig:a5finalresult1400GeV}\includegraphics[width=0.5\textwidth]{PhysicsAnalysis/Plots/FinalResult/1400GeV/Final_alpha5.pdf}}
\caption[$\chi^{2}$ sensitivity distributions at 1.4 TeV arising from a fit to $\text{cos}\theta^{*}_{\text{Jets}}$.  Results include the effect of backgrounds after the application of preselection and MVA.]{$\chi^{2}$ sensitivity distributions at 1.4 TeV arising from a fit to $\text{cos}\theta^{*}_{\text{Jets}}$.  Results include the effect of backgrounds after the application of preselection and MVA.}
\label{fig:allfinalresult1400GeV}
\end{figure}

%========================================================================================

\subsection{Systematic Uncertainties}
A source of systematic error in this experiment is the uncertainty on the cross sections for the signal and background final states.  Based on the selection efficiencies given in table \ref{table:selectionsummary1400GeV}, the $\chi^{2}$ fit procedure is applied on a distribution that primarily consists of the background final states $\text{qqqq}\nu$ arising from the interaction of $\text{e}^{+}$ and $\text{e}^{+}$ with beamstrahlung photons.  Therefore, uncertainties in the cross section for these backgrounds should be considered.  A detailed study of the accuracy of the relevant cross section calculations has yet to be performed for CLIC and so a wide spectrum in the uncertainty of these cross sections is considered here.   

This is systematic is included in the $\chi^{2}$ through the use of a nuisance parameter, whereby the cross section for $\gamma_{\text{BS}}\text{e}^{-} \rightarrow \text{qqqq}\nu$ and $\text{e}^{+}\gamma_{\text{BS}} \rightarrow \text{qqqq}\nu$ are allowed to fluctuate.  Assuming the cross sections are fluctuated by a factor r, the magnitude of the fluctuation is moderated by an additional penalty term in the $\chi^{2}$ as follows:

\begin{equation}
\chi^{2}(r) = \sum_{i} \frac{(O_{i} - E_{i}(r))^{2}}{E_{i}(r)} + \frac{(r-1)^{2}}{\sigma_{r}^{2}} 
\end{equation}

where $O_{i}$ is the observed, $\alpha_{4} = \alpha_{5} = 0$, bin content for bin i in the distribution of $\text{cos}\theta^{*}_{Jets}$ with no background fluctuations.  $E_{i}(r)$ is the expected, $\alpha_{4} \neq 0$ and $\alpha_{5} \neq 0$, bin content for bin i in the distribution of $\text{cos}\theta^{*}_{Jets}$ with the $\gamma_{\text{BS}}\text{e}^{-} \rightarrow \text{qqqq}\nu$ and $\text{e}^{+}\gamma_{\text{BS}} \rightarrow \text{qqqq}\nu$ background cross sections fluctuated by the factor $r$.  $\sum_{i}$ is the sum over the bins of the $\text{cos}\theta^{*}_{Jets}$ distribution and $\sigma_{r}$ is the width of the distribution of $r$, which indicates the uncertainty on the measurement of the fluctuations and hence the background cross sections.

Once again the $\chi^{2}$ surface is constructed in the space of $\alpha_{4}$ and $\alpha_{5}$ by minimising $\chi^{2}(r)$ at each point.  The 68\% confidence contour is shown with the inclusion of this nuisance parameter for various values of $\sigma_{r}$ in figure \ref{fig:nuisance1400GeV}.   

Minimal changes in sensitivity are observed when allowing the backgrounds to fluctuate beyond the 5\% level as the increased luminosity, observed due to the presence of non-zero anomalous gauge couplings, is counteracted by the fluctuations in the backgrounds.  In this case shape information in the $\text{cos}\theta^{*}_{Jets}$ distribution is the only discriminator for the fit.  Below the 5\% systematic uncertainty level the backgrounds cannot fluctuate enough to overcome the extra luminosity due to the presence of the penalty term. 

Based on these contours it is clear that knowledge of the cross section for the $\gamma_{\text{BS}}\text{e}^{-} \rightarrow \text{qqqq}\nu$ and $\text{e}^{+}\gamma_{\text{BS}} \rightarrow \text{qqqq}\nu$ backgrounds to sub-percent level is highly desirable for this analysis.  

\begin{figure}
\centering
\includegraphics[width=0.75\textwidth]{PhysicsAnalysis/Plots/NuisanceFit/1400GeV/Nuisance.pdf}
\caption[68\% sensitivity contour including systematic errors, of varying magnitudes, in dominant background cross sections.]{68\% sensitivity contour including systematic errors, of varying magnitudes, in dominant background cross sections.}
\label{fig:nuisance1400GeV}
\end{figure}

%========================================================================================
%========================================================================================

\section{Sensitivity at 3 TeV}

The anomalous gauge coupling sensitivity study described in this chapter was reproduced for CLIC operating at 3 TeV.  The procedure for the 3 TeV analysis largely mirrors that of the 1.4 TeV analysis, therefore in this section only the differences between the analyses are highlighted.  

The signal and background final states for the 3 TeV analysis were identical to those used for the 1.4 TeV analysis as described in section \ref{sec:eventgenerationandbackgrounds}.  The cross sections at 3 TeV for those signal and background final states can be found in table \ref{table:crosssection3000GeV}.  The data analysis and event selection procedures used at 3 TeV mirrored those used at 1.4 TeV.  Detailed descriptions of both can be found in sections \ref{sec:dataanalysis} and \ref{sec:eventselection} respectively.  

Jet finding was performed using the longitudinally invariant $k_{t}$ algorithm as described in section \label{sec:jetpairing}.  Optimisation of the jet algorithm configuration, which uses pure signal only as described in section \ref{sec:optimaljetalgorithm}, found the optimal configuration at 3 TeV to be tight selected PFOs and an R parameter of 1.1.  The sensitivity contours and the one dimensional $\chi^{2}$ distributions for $\alpha_{4}$ and $\alpha_{5}$, assuming $\alpha_{5} = 0$ and $\alpha_{4} = 0$ respectively, for the optimal jet configuration at 3 TeV using the {\nu}{\nu}qqqq signal final state are shown in figure \ref{fig:allchi2jetalgoideal3000GeV}.  Using these distributions the one sigma confidence limits on $\alpha_{4}$ is -0.00047 to 0.00048 and on $\alpha_{5}$ is -0.00036 to 0.00034.

The event selection for the 3 TeV analysis is summarised in table \ref{table:selectionsummary3000GeV}.

\begin{table}[h!]
\centering
\begin{tabular}{ l r r }
\hline
Final State & Cross Section 3 TeV [fb]  \\ 
\hline
$\text{e}^{+}\text{e}^{-} \rightarrow \nu{\nu}\text{qqqq}$ & 71.5 \\
$\text{e}^{+}\text{e}^{-} \rightarrow \text{l}\nu\text{qqqq}$ & 106.6 \\
$\text{e}^{+}\text{e}^{-} \rightarrow \text{llqqqq}$ & 169.3 \\
$\text{e}^{+}\text{e}^{-} \rightarrow \text{qqqq}$ & 546.5 \\
$\text{e}^{+}\text{e}^{-} \rightarrow \nu{\nu}\text{qq}$ & 1317.5 \\
$\text{e}^{+}\text{e}^{-} \rightarrow \text{l}\nu\text{qq}$ & 5560.9 \\
$\text{e}^{+}\text{e}^{-} \rightarrow \text{llqq}$ & 3319.6 \\
$\text{e}^{+}\text{e}^{-} \rightarrow \text{qq}$ & 2948.9 \\
$\gamma_{\text{EPA}}\text{e}^{-} \rightarrow \text{qqqq}\text{e}^{-}$ & 287.8 \\
$\gamma_{\text{BS}}\text{e}^{-} \rightarrow \text{qqqq}\text{e}^{-}$ & 1268.6 \\
$\text{e}^{+}\gamma_{\text{EPA}} \rightarrow \text{qqqq}\text{e}^{+}$ & 287.8 \\
$\text{e}^{+}\gamma_{\text{BS}} \rightarrow \text{qqqq}\text{e}^{+}$ & 1267.3 \\
$\gamma_{\text{EPA}}\text{e}^{-} \rightarrow \text{qqqq}\nu$ & 54.2 \\
$\gamma_{\text{BS}}\text{e}^{-} \rightarrow \text{qqqq}\nu$ & 262.5 \\
$\text{e}^{+}\gamma_{\text{EPA}} \rightarrow \text{qqqq}\nu$ & 54.2 \\
$\text{e}^{+}\gamma_{\text{BS}} \rightarrow \text{qqqq}\nu$ & 262.3 \\
$\gamma_{\text{EPA}}\gamma_{\text{EPA}} \rightarrow \text{qqqq}$ & 402.7 \\
$\gamma_{\text{EPA}}\gamma_{\text{BS}} \rightarrow \text{qqqq}$ & 2423.1 \\
$\gamma_{\text{BS}}\gamma_{\text{EPA}} \rightarrow \text{qqqq}$ & 2420.6 \\
$\gamma_{\text{BS}}\gamma_{\text{BS}} \rightarrow \text{qqqq}$ & 13050.3 \\
\hline
\end{tabular}
\caption[Cross sections of signal and background processes at 3 TeV]{Cross sections of signal and background processes at 3 TeV. In the above table q $\in$ u, $\bar{\text{u}}$, d, $\bar{\text{d}}$, s, $\bar{\text{s}}$, c, $\bar{\text{c}}$, b or $\bar{\text{b}}$ while l $\in$ $\text{e}^{\pm}$, $\mu^{\pm}$ or $\tau^{\pm}$ and $\nu$ $\in$ $\nu_{e}$, $\nu_{\mu}$ and $\nu_{\tau}$.  The subscript EPA or BS for the incoming photons indicate whether the photon is generated from the equivalent photon approximation or beamstrahlung.}
\label{table:crosssection3000GeV}
\end{table}

\begin{figure}
\centering
\subfloat[$\chi^{2}$ sensitivity contours in $\alpha_{4}$ and $\alpha_{5}$ space.]{\label{fig:chi2jetalgoideal3000GeV}\includegraphics[width=0.5\textwidth]{PhysicsAnalysis/Plots/Chi2ContoursOptimisation/3000GeV/KtTPFOsR1p10.pdf}}\hfill
\subfloat[$\chi^{2}$ as a function of $\alpha_{4}$ assuming $\alpha_{5} = 0$.]{\label{fig:a4chi2jetalgoideal3000GeV}\includegraphics[width=0.5\textwidth]{PhysicsAnalysis/Plots/Chi2ContoursOptimisation/3000GeV/KtTPFOsR1p10_alpha4Optimal.pdf}}
\subfloat[$\chi^{2}$ as a function of $\alpha_{5}$ assuming $\alpha_{4} = 0$.]{\label{fig:a5chi2jetalgoideal3000GeV}\includegraphics[width=0.5\textwidth]{PhysicsAnalysis/Plots/Chi2ContoursOptimisation/3000GeV/KtTPFOsR1p10_alpha5Optimal.pdf}}
\caption[$\chi^{2}$ sensitivity distributions for the $\text{qqqq}\nu\nu$ final state arising from a fit to $\text{cos}\theta^{*}_{\text{Jets}}$ at 3 TeV for the optimal jet reconstruction parameters.]{$\chi^{2}$ sensitivity distributions for the $\text{qqqq}\nu\nu$ final state arising from a fit to $\text{cos}\theta^{*}_{\text{Jets}}$ at 3 TeV for the optimal jet reconstruction parameters.} 
\label{fig:allchi2jetalgoideal3000GeV}
\end{figure}

\begin{table}[h!]
\centering
\begin{tabular}{ l r r r }
\hline
Final State & $\epsilon_{\text{presel}}$ & $\epsilon_{\text{BDT}}$ & $N_{\text{BDT}}$ \\ 
\hline
$\text{e}^{+}\text{e}^{-} \rightarrow \nu{\nu}\text{qqqq}$ & 69.4\% & 45.3\% & 64,750 \\
$\text{e}^{+}\text{e}^{-} \rightarrow \text{l}\nu\text{qqqq}$ & 38.9\% & 10.9\% & 23,310 \\
$\text{e}^{+}\text{e}^{-} \rightarrow \text{llqqqq}$ & 7.0\% & 0.7\% & 2,409 \\
$\text{e}^{+}\text{e}^{-} \rightarrow \text{qqqq}$ & 3.4\% & 0.3\% & 3,069 \\
$\text{e}^{+}\text{e}^{-} \rightarrow \nu{\nu}\text{qq}$ & 14.4\% & 0.7\% & 19,040 \\
$\text{e}^{+}\text{e}^{-} \rightarrow \text{l}\nu\text{qq}$ & 16.5\% & 0.3\% & 27,910 \\
$\text{e}^{+}\text{e}^{-} \rightarrow \text{llqq}$ & 0.8\% & - & 786 \\
$\text{e}^{+}\text{e}^{-} \rightarrow \text{qq}$ & 1.1\% & - & 1,335 \\
$\gamma_{\text{EPA}}\text{e}^{-} \rightarrow \text{qqqq}\text{e}^{-}$ & 5.8\% & 0.5\% & 2,860 \\
$\gamma_{\text{BS}}\text{e}^{-} \rightarrow \text{qqqq}\text{e}^{-}$ & 4.1\% & 0.4\% & 8,352 \\
$\text{e}^{+}\gamma_{\text{EPA}} \rightarrow \text{qqqq}\text{e}^{+}$ & 5.9\% & 0.5\% & 3,063 \\
$\text{e}^{+}\gamma_{\text{BS}} \rightarrow \text{qqqq}\text{e}^{+}$ & 4.2\% & 0.4\% & 8,090 \\
$\gamma_{\text{EPA}}\text{e}^{-} \rightarrow \text{qqqq}\nu$ & 42.8\% & 16.6\% & 17,950 \\
$\gamma_{\text{BS}}\text{e}^{-} \rightarrow \text{qqqq}\nu$ & 51.6\% & 26.0\% & 108,000 \\
$\text{e}^{+}\gamma_{\text{EPA}} \rightarrow \text{qqqq}\nu$ & 43.1\% & 16.6\% & 17,980 \\
$\text{e}^{+}\gamma_{\text{BS}} \rightarrow \text{qqqq}\nu$ & 52.3\% & 26.5\% & 109,700 \\
$\gamma_{\text{EPA}}\gamma_{\text{EPA}} \rightarrow \text{qqqq}$ & 4.7\% & 0.4\% & 3,058 \\
$\gamma_{\text{EPA}}\gamma_{\text{BS}} \rightarrow \text{qqqq}$ & 3.0\% & 0.3\% & 9,812 \\
$\gamma_{\text{BS}}\gamma_{\text{EPA}} \rightarrow \text{qqqq}$ & 3.1\% & 0.2\% & 8,880 \\
$\gamma_{\text{BS}}\gamma_{\text{BS}} \rightarrow \text{qqqq}$ & 0.6\% & - & 2,213 \\
\hline
\end{tabular}
\caption[Selection summary at 3 TeV.]{Selection summary at 3 TeV.   The subscript EPA or BS for the incoming photons indicate whether the photon is generated from the equivalent photon approximation or beamstrahlung.  Cells omitting the efficiency indicate an efficiency of less than 0.1\%.}
\label{table:selectionsummary3000GeV}
\end{table}

Due to the increased sensitivity of the signal sample at 3 TeV, the stepping along $\alpha_{4}$ and $\alpha_{5}$ to extract the event weights from the generator was much finer than was used for the 1.4 TeV analysis.  At 3 TeV event weights were taken from the generator in steps of 0.0025 ranging from 0.0045 to -0.0045.  Bicubic interpolation was again used to make a continuous surface for the event weights.  These event weight surfaces were then used to construct the $\text{cos}\theta^{*}_{Jets}$ distribution and the $\chi^{2}$ surface used to determine the reported sensitivities.

The sensitivity of the CLIC experiment to the anomalous gauge couplings $\alpha_{4}$ and $\alpha_{5}$ at 3 TeV is shown in figure \ref{fig:finalresult3000GeV}.  This result shows the sensitivity after the application of preselection and MVA described in sections \ref{sec:preselection1400GeV} and \ref{sec:mva1400GeV} purposed to remove the included background channels, described in section \ref{sec:eventgenerationandbackgrounds}.  These contours yield the one $\sigma$ confidence limit on the measurement of $\alpha_{4}$ to the range -FILL IN, FILL IN and similarly for the measurement of $\alpha_{5}$ the range is -FILL IN, FILL IN.

\iffalse
\begin{figure}
\centering
\subfloat[$\chi^{2}$ sensitivity contours in $\alpha_{4}$ and $\alpha_{5}$ space.]{\label{fig:finalresult3000GeV}\includegraphics[width=0.5\textwidth]{PhysicsAnalysis/Plots/FinalResult/3000GeV/Final.pdf}}\hfill
\subfloat[$\chi^{2}$ as a function of $\alpha_{4}$ assuming $\alpha_{5} = 0$.]{\label{fig:a4finalresult3000GeV}\includegraphics[width=0.5\textwidth]{PhysicsAnalysis/Plots/FinalResult/3000GeV/Final_alpha4.pdf}}
\subfloat[$\chi^{2}$ as a function of $\alpha_{5}$ assuming $\alpha_{4} = 0$.]{\label{fig:a5finalresult3000GeV}\includegraphics[width=0.5\textwidth]{PhysicsAnalysis/Plots/FinalResult/3000GeV/Final_alpha5.pdf}}
\caption[$\chi^{2}$ sensitivity distributions at 3 TeV arising from a fit to $\text{cos}\theta^{*}_{\text{Jets}}$.  Results include the effect of backgrounds after the application of preselection and MVA.]{$\chi^{2}$ sensitivity distributions at 3 TeV arising from a fit to $\text{cos}\theta^{*}_{\text{Jets}}$.  Results include the effect of backgrounds after the application of preselection and MVA.}
\label{fig:allfinalresult3000GeV}
\end{figure}
\fi

%========================================================================================







\iffalse
$\text{e}^{+}\text{e}^{-} \rightarrow \nu{\nu}\text{qqqq}$
$\text{e}^{+}\text{e}^{-} \rightarrow \text{l}\nu\text{qqqq}$
$\text{e}^{+}\text{e}^{-} \rightarrow \text{llqqqq}$
$\text{e}^{+}\text{e}^{-} \rightarrow \text{qqqq}$
$\text{e}^{+}\text{e}^{-} \rightarrow \nu{\nu}\text{qq}$
$\text{e}^{+}\text{e}^{-} \rightarrow \text{l}\nu\text{qq}$
$\text{e}^{+}\text{e}^{-} \rightarrow \text{llqq}$
$\text{e}^{+}\text{e}^{-} \rightarrow \text{qq}$
$\gamma_{\text{EPA}}\text{e}^{-} \rightarrow \text{qqqq}\text{e}^{-}$
$\gamma_{\text{BS}}\text{e}^{-} \rightarrow \text{qqqq}\text{e}^{-}$
$\text{e}^{+}\gamma_{\text{EPA}} \rightarrow \text{qqqq}\text{e}^{+}$
$\text{e}^{+}\gamma_{\text{BS}} \rightarrow \text{qqqq}\text{e}^{+}$
$\gamma_{\text{EPA}}\text{e}^{-} \rightarrow \text{qqqq}\nu$
$\gamma_{\text{BS}}\text{e}^{-} \rightarrow \text{qqqq}\nu$
$\text{e}^{+}\gamma_{\text{EPA}} \rightarrow \text{qqqq}\nu$
$\text{e}^{+}\gamma_{\text{BS}} \rightarrow \text{qqqq}\nu$
$\gamma_{\text{EPA}}\gamma_{\text{EPA}} \rightarrow \text{qqqq}$
$\gamma_{\text{EPA}}\gamma_{\text{BS}} \rightarrow \text{qqqq}$
$\gamma_{\text{BS}}\gamma_{\text{EPA}} \rightarrow \text{qqqq}$
$\gamma_{\text{BS}}\gamma_{\text{BS}} \rightarrow \text{qqqq}$
\fi




\chapter{The Sensitivity of CLIC to Anomalous Gauge Couplings through Vector Boson Scattering}
\label{chap:PhysicsAnalysis}

\chapterquote{Kids, you tried your best, and you failed miserably.  The lesson is, never try.}%
{Homer Simpson}

\section{Motivation}
Vector boson scattering is the interaction of the form $\text{VV} \rightarrow \text{VV}$ where V is any of the electroweak gauge bosons $\text{W}^{+}$, $\text{W}^{-}$, Z or $\gamma$.  This is an interesting process to look at because it provides insights into beyond standard model physics that impacts the electroweak sector through the probing of anomalous triple and quartic gauge couplings.  Vector boson scattering also gives a detailed understanding of how the Higgs of the standard model is able to unitarise the otherwise unbounded cross section for longitudinal gauge boson scattering.  

The anomalous triple and quartic gauge couplings are introduced as parameters in an effective field theory....




A potential source of beyond the standard model physics that could be studied at the CLIC experiment is that of the anomalous gauge couplings $\alpha_{4}$ and $\alpha_{5}$.  The theoretical basis for these couplings is described in section THEORY REF.  CLIC would show sensitivity to these couplings through vector boson scattering processes that are summarised in figures \ref{fig:vbsw}, \ref{fig:vbsz}, \ref{fig:vbswz} and \ref{fig:vbszw}.  An analysis of the sensitivity of CLIC to this processes is presented in the following chapter.

\iffalse

\begin{figure}
\begin{tikzpicture}[]
\begin{feynman}
\vertex (a1);
\vertex[above left=2cm of a1] (a2);
\vertex[above right=1cm and 2cm of a1] (a3) {\(W^{\pm},Z\)};
\vertex[below left=2cm of a1] (a4);
\vertex[below right=1cm and 2cm of a1] (a5) {\(W^{\mp},Z\)};
\vertex[above left=1cm of a2] (i1) {\(e^{-}\)};
\vertex[below left=1cm of a4] (i2) {\(e^{+}\)};
\vertex[above right=1cm and 3cm of a3] (i3) {\(q,l\)};
\vertex[below right=1cm and 3cm of a3] (i4) {\(\bar{q},\bar{l}\)};
\vertex[above right=1cm and 3cm of a5] (i5) {\(q,l\)};
\vertex[below right=1cm and 3cm of a5] (i6) {\(\bar{q},\bar{l}\)};
\vertex[above=1cm of a3] (v1) {\(\nu_{e}\)};
\vertex[below=1cm of a5] (v2) {\(\bar{\nu_{e}}\)};
\diagram* {
   (a1) -- [boson, edge label'=\(W^{-}\)] (a2) 
   (a1) -- [boson] (a3) 
   (a1) -- [boson, edge label=\(W^{+}\)] (a4) 
   (a1) -- [boson] (a5) 
   (i1) -- [fermion] (a2) -- [fermion] (v1)
   (v2) -- [fermion] (a4) -- [fermion] (i2)
   (i4) -- [fermion] (a3) -- [fermion] (i3)
   (i6) -- [fermion] (a5) -- [fermion] (i5)
};
\end{feynman}
\end{tikzpicture}
\caption[Feynman diagram of vector boson scattering at CLIC involving radiation of W bosons.]{Feynman diagram of vector boson scattering at CLIC involving radiation of W bosons.  $q = \text{u, d, s, b, c}$ and $l = \text{e, } \mu \text{, } \tau \text{, } \nu_{e} \text{, } \nu_{\nu} \text{, } \nu_{\tau}$}
\label{fig:vbsw}
\end{figure}

\begin{figure}
\begin{tikzpicture}[]
\begin{feynman}
\vertex (a1);
\vertex[above left=2cm of a1] (a2);
\vertex[above right=1cm and 2cm of a1] (a3) {\(W^{\pm},Z\)};
\vertex[below left=2cm of a1] (a4);
\vertex[below right=1cm and 2cm of a1] (a5) {\(W^{\mp},Z\)};
\vertex[above left=1cm of a2] (i1) {\(e^{-}\)};
\vertex[below left=1cm of a4] (i2) {\(e^{+}\)};
\vertex[above right=1cm and 3cm of a3] (i3) {\(q,l\)};
\vertex[below right=1cm and 3cm of a3] (i4) {\(\bar{q},\bar{l}\)};
\vertex[above right=1cm and 3cm of a5] (i5) {\(q,l\)};
\vertex[below right=1cm and 3cm of a5] (i6) {\(\bar{q},\bar{l}\)};
\vertex[above=1cm of a3] (v1) {\(e^{-}\)};
\vertex[below=1cm of a5] (v2) {\(e^{+}\)};
\diagram* {
   (a1) -- [boson, edge label'=\(Z\)] (a2)
   (a1) -- [boson] (a3)
   (a1) -- [boson, edge label=\(Z\)] (a4)
   (a1) -- [boson] (a5)
   (i1) -- [fermion] (a2) -- [fermion] (v1)
   (v2) -- [fermion] (a4) -- [fermion] (i2)
   (i4) -- [fermion] (a3) -- [fermion] (i3)
   (i6) -- [fermion] (a5) -- [fermion] (i5)
};
\end{feynman}
\end{tikzpicture}
\caption[Feynman diagram of vector boson scattering at CLIC involving radiation of Z bosons.]{Feynman diagram of vector boson scattering at CLIC involving radiation of Z bosons.  $q = \text{u, d, s, b, c}$ and $l = \text{e, } \mu \text{, } \tau \text{, } \nu_{e} \text{, } \nu_{\nu} \text{, } \nu_{\tau}$}
\label{fig:vbsz}
\end{figure}

\begin{figure}
\begin{tikzpicture}[]
\begin{feynman}
\vertex (a1);
\vertex[above left=2cm of a1] (a2);
\vertex[above right=1cm and 2cm of a1] (a3) {\(W^{-}\)};
\vertex[below left=2cm of a1] (a4);
\vertex[below right=1cm and 2cm of a1] (a5) {\(Z\)};
\vertex[above left=1cm of a2] (i1) {\(e^{-}\)};
\vertex[below left=1cm of a4] (i2) {\(e^{+}\)};
\vertex[above right=1cm and 3cm of a3] (i3) {\(q,l\)};
\vertex[below right=1cm and 3cm of a3] (i4) {\(\bar{q},\bar{l}\)};
\vertex[above right=1cm and 3cm of a5] (i5) {\(q,l\)};
\vertex[below right=1cm and 3cm of a5] (i6) {\(\bar{q},\bar{l}\)};
\vertex[above=1cm of a3] (v1) {\(\nu_{e}\)};
\vertex[below=1cm of a5] (v2) {\(e^{+}\)};
\diagram* {
   (a1) -- [boson, edge label'=\(W^{-}\)] (a2)
   (a1) -- [boson] (a3)
   (a1) -- [boson, edge label=\(Z\)] (a4)
   (a1) -- [boson] (a5)
   (i1) -- [fermion] (a2) -- [fermion] (v1)
   (v2) -- [fermion] (a4) -- [fermion] (i2)
   (i4) -- [fermion] (a3) -- [fermion] (i3)
   (i6) -- [fermion] (a5) -- [fermion] (i5)
};
\end{feynman}
\end{tikzpicture}
\caption[Feynman diagram of vector boson scattering at CLIC involving radiation of Z bosons.]{Feynman diagram of vector boson scattering at CLIC involving radiation of a W and Z boson.  $q = \text{u, d, s, b, c}$ and $l = \text{e, } \mu \text{, } \tau \text{, } \nu_{e} \text{, } \nu_{\nu} \text{, } \nu_{\tau}$}
\label{fig:vbswz}
\end{figure}

\begin{figure}
\begin{tikzpicture}[]
\begin{feynman}
\vertex (a1);
\vertex[above left=2cm of a1] (a2);
\vertex[above right=1cm and 2cm of a1] (a3) {\(Z\)};
\vertex[below left=2cm of a1] (a4);
\vertex[below right=1cm and 2cm of a1] (a5) {\(W^{+}\)};
\vertex[above left=1cm of a2] (i1) {\(e^{-}\)};
\vertex[below left=1cm of a4] (i2) {\(e^{+}\)};
\vertex[above right=1cm and 3cm of a3] (i3) {\(q,l\)};
\vertex[below right=1cm and 3cm of a3] (i4) {\(\bar{q},\bar{l}\)};
\vertex[above right=1cm and 3cm of a5] (i5) {\(q,l\)};
\vertex[below right=1cm and 3cm of a5] (i6) {\(\bar{q},\bar{l}\)};
\vertex[above=1cm of a3] (v1) {\(e^{-}\)};
\vertex[below=1cm of a5] (v2) {\(\bar{\nu_{e}}\)};
\diagram* {
   (a1) -- [boson, edge label'=\(Z\)] (a2)
   (a1) -- [boson] (a3)
   (a1) -- [boson, edge label=\(W^{+}\)] (a4)
   (a1) -- [boson] (a5)
   (i1) -- [fermion] (a2) -- [fermion] (v1)
   (v2) -- [fermion] (a4) -- [fermion] (i2)
   (i4) -- [fermion] (a3) -- [fermion] (i3)
   (i6) -- [fermion] (a5) -- [fermion] (i5)
};
\end{feynman}
\end{tikzpicture}
\caption[Feynman diagram of vector boson scattering at CLIC involving radiation of Z bosons.]{Feynman diagram of vector boson scattering at CLIC involving radiation of a W and Z boson.  $q = \text{u, d, s, b, c}$ and $l = \text{e, } \mu \text{, } \tau \text{, } \nu_{e} \text{, } \nu_{\nu} \text{, } \nu_{\tau}$}
\label{fig:vbszw}
\end{figure}

\fi

\section{Event Generation, Simulation and Reconstruction}
\label{sec:eventgenerationandbackgrounds}

The event generation software used by the CLIC experiment is Whizard \cite{0708.4233, hep-ph/0102195}.  Whizard version 1.97 was used for generating the new samples, while version 1.95 is used for the official CLIC samples. It was recommended by the Whizard authors to use version 1.97 as it contains a unitarisation scheme that ensures the probabilities remain physical up to high energies when considering the effect of anomalous gauge couplings.  

The hadronic channels are the dominant decay modes of the W and the Z boson, with branching fractions of the order of 70\% for both (REFERENCE PDG), and as the vector boson scattering is the desired signal channel, the focus of this analysis will be upon the hadronic decays of the W and Z.  The vector boson scattering dominated signal final states containing hadronic decay products for the bosons are $\nu\nu\text{qqqq}$, $\text{l}\nu\text{qqqq}$ and llqqqq.  

For all samples considered in this analysis, the CLID\_ILD detector \cite{arXiv:1006.3396} was used.  The detector was simulated using MOKKA \cite{MoradeFreitas:2002kj}, a GEANT4 \cite{Agostinelli:2002hh} wrapper providing detailed geometric descriptions of detector concepts for the linear collider.  Events were reconstructed using MARLIN \cite{Gaede:2006pj}, a c++ framework designed for reconstruction at the linear collider.  PandoraPFA \cite{arXiv:0907.3577, arXiv:1209.4039} is used to apply particle flow calorimetry in this reconstruction.
 
The CLIC\_ILD is a variant of the ILD detector described in section REFERENCE.  The only significant difference between the modles is that CLIC\_ILD has a 60 layer scintillator-tugsten HCal in comparison to the 48 layers found in the default ILD detector.  The thicknesses of the layers in the HCal models are identical, so the extra layers correspond to an increase in the total thickness of the HCal.  This is needed to compensate for the effects of leakage at the higher energies seen by the CLIC experiment in comparison to the ILC. 

\begin{table}[h!]
\centering
\begin{tabular}{ l r }
\hline
Final State & Cross Section 1.4 TeV [fb] \\ 
\hline
$\text{e}^{+}\text{e}^{-} \rightarrow \nu{\nu}\text{qqqq}$ & 24.7 \\
$\text{e}^{+}\text{e}^{-} \rightarrow \text{l}\nu\text{qqqq}$ & 110.4\\
$\text{e}^{+}\text{e}^{-} \rightarrow \text{llqqqq}$ & 62.1\\
$\text{e}^{+}\text{e}^{-} \rightarrow \text{qqqq}$ & 1245.1\\
$\text{e}^{+}\text{e}^{-} \rightarrow \nu{\nu}\text{qq}$ & 787.7\\
$\text{e}^{+}\text{e}^{-} \rightarrow \text{l}\nu\text{qq}$ & 4309.7\\
$\text{e}^{+}\text{e}^{-} \rightarrow \text{llqq}$ & 2725.8\\
$\text{e}^{+}\text{e}^{-} \rightarrow \text{qq}$ & 4009.5\\
$\gamma_{\text{EPA}}\text{e}^{-} \rightarrow \text{qqqq}\text{e}^{-}$ & 287.1\\
$\gamma_{\text{BS}}\text{e}^{-} \rightarrow \text{qqqq}\text{e}^{-}$ & 1160.7\\
$\text{e}^{+}\gamma_{\text{EPA}} \rightarrow \text{qqqq}\text{e}^{+}$ & 286.9\\
$\text{e}^{+}\gamma_{\text{BS}} \rightarrow \text{qqqq}\text{e}^{+}$ & 1156.3\\
$\gamma_{\text{EPA}}\text{e}^{-} \rightarrow \text{qqqq}\nu$ & 32.6\\
$\gamma_{\text{BS}}\text{e}^{-} \rightarrow \text{qqqq}\nu$ & 136.9\\
$\text{e}^{+}\gamma_{\text{EPA}} \rightarrow \text{qqqq}\nu$ & 32.6\\
$\text{e}^{+}\gamma_{\text{BS}} \rightarrow \text{qqqq}\nu$ & 136.4\\
$\gamma_{\text{EPA}}\gamma_{\text{EPA}} \rightarrow \text{qqqq}$ & 753.0\\
$\gamma_{\text{EPA}}\gamma_{\text{BS}} \rightarrow \text{qqqq}$ & 4034.8\\
$\gamma_{\text{BS}}\gamma_{\text{EPA}} \rightarrow \text{qqqq}$ & 4018.7\\
$\gamma_{\text{BS}}\gamma_{\text{BS}} \rightarrow \text{qqqq}$ & 21406.2\\
\hline
\end{tabular}
\caption[]{Cross sections of signal and background processes at 1.4 TeV. In the above table q $\in$ u, $\bar{\text{u}}$, d, $\bar{\text{d}}$, s, $\bar{\text{s}}$, c, $\bar{\text{c}}$, b or $\bar{\text{b}}$ while l $\in$ $\text{e}^{\pm}$, $\mu^{\pm}$ or $\tau^{\pm}$ and $\nu$ $\in$ $\nu_{e}$, $\nu_{\mu}$ and $\nu_{\tau}$.  The subscript EPA or BS for the incoming photons indicate whether the photon is generated from the equivalent photon approximation or beamstrahlung.}
\label{table:crosssectionfull}
\end{table}




\section{Modelling of Anomalous Gauge Couplings}
\label{sec:eventweights}

\subsection{Cross Section Sensitivity}
\label{sec:crosssectioncheck}
To determine which final states are sensitive to $\alpha_{4}$ and $\alpha_{5}$ a comparison was made between the cross section using the standard model values of $\alpha_{4}$ and $\alpha_{5}$, i.e. 0, and the same calculation using non-zero values of these couplings.  This comparison was performed on all final states that would be relevant either as signal or background processes, for an analysis involving the purely hadronic decay channels of a vector boson scattering process.  In full the states that were tested are:

\begin{itemize}
\item Vector boson scattering signal final states that are expected to show sensitivity to the anomalous couplings: $\text{e}^{+}\text{e}^{-} \rightarrow \nu\nu\text{qqqq}$, $\text{e}^{+}\text{e}^{-} \rightarrow \text{l}\nu\text{qqqq}$ and $\text{e}^{+}\text{e}^{-} \rightarrow \text{llqqqq}$
\item Four jet final states arising from $\text{e}^{+}\text{e}^{-}$ interactions: $\text{e}^{+}\text{e}^{-} \rightarrow \text{qqqq}$.
\item Two jet final states arising from $\text{e}^{+}\text{e}^{-}$ interactions: $\text{e}^{+}\text{e}^{-} \rightarrow \nu{\nu}\text{qq}$, $\text{e}^{+}\text{e}^{-} \rightarrow \text{l}\nu\text{qq}$, $\text{e}^{+}\text{e}^{-} \rightarrow \text{llqq}$ and $\text{e}^{+}\text{e}^{-} \rightarrow \text{qq}$.
\item Four jet final states arising from the interactions of either $\text{e}^{+}$ or $\text{e}^{-}$ with a beamstrahlung photon: $\gamma_{\text{BS}}\text{e}^{-} \rightarrow \text{qqqq}\text{e}^{-}$, $\text{e}^{+}\gamma_{\text{BS}} \rightarrow \text{qqqq}\text{e}^{+}$, $\gamma_{\text{BS}}\text{e}^{-} \rightarrow \text{qqqq}\nu$ and $\text{e}^{+}\gamma_{\text{BS}} \rightarrow \text{qqqq}\nu$.
\item Four jet final states arising from the interactions of either $\text{e}^{+}$ or $\text{e}^{-}$ with the electromagnetic field of the opposing beam particle.  These cross sections are calculated using equivalent photon approximation (EPA), which represents the electromagnetic field of the opposing beam particle as a series of photon and so the final states appear as interactions of $\text{e}^{+}$ or $\text{e}^{-}$ with photons: $\gamma_{\text{EPA}}\text{e}^{-} \rightarrow \text{qqqq}\text{e}^{-}$, $\text{e}^{+}\gamma_{\text{EPA}} \rightarrow \text{qqqq}\text{e}^{+}$, $\gamma_{\text{EPA}}\text{e}^{-} \rightarrow \text{qqqq}\nu$ and $\text{e}^{+}\gamma_{\text{EPA}} \rightarrow \text{qqqq}\nu$.
\item Four jet final states arising from the interaction of the electromagnetic fields of opposing beam particles using the EPA approximation: $\gamma_{\text{EPA}}\gamma_{\text{EPA}} \rightarrow \text{qqqq}$.
\item Four jet final states arising from the interaction of the electromagnetic field of either $\text{e}^{+}$ or $\text{e}^{-}$ using the EPA approximation with a beamstrahlung photon: $\gamma_{\text{EPA}}\gamma_{\text{BS}} \rightarrow \text{qqqq}$ or $\gamma_{\text{BS}}\gamma_{\text{EPA}} \rightarrow \text{qqqq}$.
\item Four jet final states arising from the interaction of two beamstrahlung photons: $\gamma_{\text{BS}}\gamma_{\text{BS}} \rightarrow \text{qqqq}$.
\end{itemize}

Note: In the above list q = u, d, s, c and b, l = e, $\mu$, $\tau$ and $\nu$ = $\nu_{e}$, $\nu_{\mu}$ and $\nu_{\tau}$.

The cross section was found to differ when using non-zero values for the anomalous couplings in comparison to the standard model prediction for the vector boson scattering signal final states $\nu\nu\text{qqqq}$, $\text{l}\nu\text{qqqq}$ and llqqqq.  The cross section comparisons for these final states can be found in table \ref{table:crosssectionsensitivity1400} for 1.4 TeV samples.  In reality, non-zero anomalous couplings would change the cross sections of all processes considered, however, the sensitivity would only arise from high order terms in the Lagrangian.  Such terms would not be dominant in determining the cross section and so are omitted from the generator making certain final states appear invariant to changes in the anomalous couplings.

\begin{table}[h!]
\centering
\begin{tabular}{ l r r r r }
\hline
Final State & Cross Section [fb] & Cross Section [fb] & Percentage & CLIC Cross Section \\ 
& ($\alpha_{4} = \alpha_{5} = 0.00$) & ($\alpha_{4} = \alpha_{5} = 0.05$) & Change[\%] & [fb] \\ 
\hline
$\text{e}^{+}\text{e}^{-} \rightarrow \nu{\nu}\text{qqqq}$ & 20.8 & 34.6 & +66.3 & 24.7 \\
$\text{e}^{+}\text{e}^{-} \rightarrow \text{l}{\nu}\text{qqqq}$ & 112 & 113 & +0.9 & 115.3 \\
$\text{e}^{+}\text{e}^{-} \rightarrow \text{llqqqq}$ & 59.7 & 68.6 & +14.9 & 62.1 \\
\hline
\end{tabular}
\caption{Cross section for selected processes for given value of $\alpha_{4}$ and $\alpha_{5}$ at 1.4 TeV.}
\label{table:crosssectionsensitivity1400}
\end{table}

The cross section calculations show that the most sensitive final state to the anomalous gauge couplings is \nu{\nu}qqqq, therefore, this analysis will focus solely upon this final state.  Furthermore, as the l{\nu}qqqq final state has a much reduced sensitivity in comparison to the \nu{\nu}qqqq state and as the llqqqq can be easily vetoed from the analysis, as will be shown in subsequent chapters, it is only necessary to consider the sensitivity of the \nu{\nu}qqqq final state.  For the aforementioned reasons the l{\nu}qqqq and llqqqq final states will be treated as backgrounds that are invariant to changes in the anomalous couplings $\alpha_{4}$ and $\alpha_{5}$.  

The sensitivity of an individual event to the anomalous gauge couplings is determined through an event weight. This weight is given by the ratio of the squares of the matrix element used in the cross section calculation, one matrix element using non-zero values of $\alpha_{4}$ and $\alpha_{5}$ and the other matrix element using the standard model values of $\alpha_{4}$ and $\alpha_{5}$, i.e. 0.  The weight varies as a function of $\alpha_{4}$ and $\alpha_{5}$ as well as varying on an event by event basis as the kinematics of the final state changes.  Examples of the event weights as a function of $\alpha_{4}$ and $\alpha_{5}$ for selected events is shown in figure \ref{fig:eventweights1400raw} for 1.4 TeV events.

\begin{figure}
\centering
\subfloat[]{\label{fig:weight1}\includegraphics[width=0.5\textwidth]{PhysicsAnalysis/Plots/EventWeights/1400GeV/EventWeightsForEvent100001009_1400GeV_SPFOs_kt_0p70_10Bins_Start_0_End_10_1400GeV_Raw.pdf}}
\subfloat[]{\label{fig:weight2}\includegraphics[width=0.5\textwidth]{PhysicsAnalysis/Plots/EventWeights/1400GeV/EventWeightsForEvent100001014_1400GeV_SPFOs_kt_0p70_10Bins_Start_0_End_10_1400GeV_Raw.pdf}} \hfill
\subfloat[]{\label{fig:weight3}\includegraphics[width=0.5\textwidth]{PhysicsAnalysis/Plots/EventWeights/1400GeV/EventWeightsForEvent100001044_1400GeV_SPFOs_kt_0p70_10Bins_Start_0_End_10_1400GeV_Raw.pdf}}
\subfloat[]{\label{fig:weight4}\includegraphics[width=0.5\textwidth]{PhysicsAnalysis/Plots/EventWeights/1400GeV/EventWeightsForEvent100001051_1400GeV_SPFOs_kt_0p70_10Bins_Start_0_End_10_1400GeV_Raw.pdf}}
\caption[Event weights from Whizard for 1.4TeV \nu{\nu}qqqq final state events.]{A selection of plots showing how the event weight changes when varying the anomalous couplings $\alpha_{4}$ and $\alpha_{5}$ for 1.4TeV \nu{\nu}qqqq final state events.}
\label{fig:eventweights1400raw}
\end{figure}

This reweighting procedure has many advantages over the alternative of generating new samples with fixed $\alpha_{4}$ and $\alpha_{5}$, notably the absence of systematic errors arising from new event generation, simulation and reconstruction.  

\subsection{Validation of Samples}
\label{sec:validationofsamples}
The CLIC experiment has a repository of simulated and reconstructed samples that can be used for physics analyses, however, it is not possible to calculate the event weights for these samples as the raw Whizard format event files are missing.  Therefore, a new $\text{e}^{+}\text{e}^{-} \rightarrow \nu\nu\text{qqqq}$ sample was generated with the relevant files to make reweighting as a function of $\alpha_{4}$ and $\alpha_{5}$ possible.  An identical setup to that used for the production of official CLIC samples was used for the event generation, detector simulation and reconstruction.  Mimicking this production chain made it possible to use the official CLIC samples for the background final states used in this analysis.  

Several reconstructed level distributions were compared to the official CLIC samples and were found to be comparable to each other. A selection of these distributions is shown in figure \ref{fig:cliccomp}.

In order to determine the event weights it was necessary to use the anomalous gauge coupling model in Whizard, which in turn enforces a unit CKM matrix.  In the context of vector boson scattering this will restrict the hadronic decays of the $\text{W}^{-}$ boson to d$\bar{\text{u}}$ and s$\bar{\text{c}}$, the $\text{W}^{+}$ boson to u$\bar{\text{d}}$ and c$\bar{\text{s}}$ and the Z boson to u$\bar{\text{u}}$ , d$\bar{\text{d}}$, s$\bar{\text{s}}$, c$\bar{\text{c}}$ and b$\bar{\text{b}}$.  In contrast the official CLIC samples use a non-unit CKM matrix, which gives rise to alternative hadronic decay modes for the W and Z bosons.  When comparing the unit CKM matrix and the non-unit CKM $\nu\nu\text{qqqq}$ final state samples it was found that there were negligible differences in a variety of reconstructed level distributions, such as those found in figure \ref{fig:cliccomp}.  Furthermore, as flavour tagging information is not used in this analysis this difference was deemed insignificant.  

\begin{figure}
\centering
\subfloat[Visible Mass]{\label{fig:cliccomp1400_1}\includegraphics[width=0.5\textwidth]{PhysicsAnalysis/Plots/CLICSampleComparison/1400GeV/InvariantMassSystem.pdf}}
\subfloat[$\text{cos}\theta_{Missing}$ - The cosine of the polar angle of the missing momentum for an event.]{\label{fig:cliccomp1400_2}\includegraphics[width=0.5\textwidth]{PhysicsAnalysis/Plots/CLICSampleComparison/1400GeV/CosThetaMissing.pdf}} 
\caption[Comparison of various distributions between samples used in this analysis and the official CLIC samples for the \nu{\nu}qqqq final state at 1.4 TeV.]{Comparison of various distributions between samples used in this analysis and the official CLIC samples for the \nu{\nu}qqqq final state at 1.4 TeV.  Both samples have been normalised to the correct luminosity for CLIC running at 1.4 TeV.}
\label{fig:cliccomp}
\end{figure}

\section{Data Analysis}
\label{sec:dataanalysis}
The focus of this section is to describe the post reconstruction procedure applied to the signal and background events, described in \ref{sec:crosssectioncheck}, to extract the relevant information needed for this sensitivity study. 

\subsection{Jet Finding} 
\label{sec:jetpairing}
The MarlinFastJet processor, a wrapper for the FastJet \cite{Cacciari:2011ma} processor, is used to cluster the signal and background events into four jets.  In the case of the signal final state, $\nu\nu$qqqq, it is assumed these four jets have arisen from the hadronic decays of the bosons involved in the vector scattering process.  These jets are paired up on the assumption that the correct pairing is achieved when the invariant masses of the two pairs are closest together. 

The longitudinally invariant $k_{t}$ jet algorithm in exclusive mode was used for the jet clustering.  In contrast to the inclusive mode, the exclusive mode allows the user to request a fixed number of jets in the output from MarlinFastJet.  The longitudinally invariant $k_{t}$ algorithm proceeds as follows:
\begin{itemize}
\item For each pair of particles, i and j, the $k_{t}$ distance, $d_{ij}$, and beam distance, $d_{iB} = p_{t}^{2}$, are calculated.
\begin{equation}
d_{ij} = \text{min}(p_{ti}^{2}, p_{tj}^{2}){\Delta}R^{2}_{ij}/R^{2}
\end{equation}
where ${\Delta}R^{2}_{ij} = (y_{i} - y_{j})^2 + (\phi_{i} - \phi_{j})^2$, $p_{t}$ is the transverse momentum of the particle with respect to the beam axis, $y_{i}$ is the rapidity of particle i and $\phi_{i}$ is the azimuthal angle of particle i. $R$ is a configurable parameter that typically is of the order of 1.
\item The minimum distance, $d_\text{min}$, of all the $k_{t}$ and beam distances is found.  If the minimum occurs for a $k_{t}$ distance, merge particles i and j, summing their 4-momenta in the energy combination scheme.  If the beam distance is the minimum, declare particle i to be apart of the "beam" jet.  Remove this particle from the list of particles and do not included in the final jet output.
\item Repeat until the requested number of jets have been created, or inclusive mode no particles are left in the event.
\end{itemize}

Two other clustering algorithms were considered, but, as figure \ref{fig:invariantmassalgoveto} shows, were found to be inappropriate for the experimental conditions at CLIC.  These alternative algorithm choices are applied in the same manor as the longitudinally invariant $k_{t}$ algorithm, however, they differ in their for the $k_{t}$ distance, $d_{ij}$, and beam distance, $d_{iB}$.

The first alternative jet algorithm considered was the $k_{t}$ algorithm for $\text{e}^{+}\text{e}^{-}$ colliders (or Durham algorithm) where $d_{ij} = 2\text{min}(E_{i}^{2}, E_{j}^{2})(1-cos\theta_{ij})$ and $d_{iB}$ is not used.  $\theta_{ij}$ is the opening angle of particles i and j meaning that in the collinear limit $d_{ij}$ corresponds to the relative transverse momenta of the particles.  The major failure of this algorithm when applied to CLIC is the absence of $d_{iB}$, which leads to many beam related background particles being associated to jets.  As figure \ref{fig:invariantmassalgoveto} shows, the invariant mass of the paired jets, which is expected to be centred around the W and Z boson masses, is much larger than expected, due to the presence of the beam related backgrounds in the jets.  Also this algorithm is not invariant to boosts along the beam direction making it inappropriate to use at CLIC as the beam induced backgrounds modify the nominal collision kinematics.  

The second alternative jet algortihm considered was the Cambridge/Aachen jet algorithm where $d_{ij} = {\Delta}R_{ij}^{2}/R^2$ and $d_{iB} = 1$.  This algorithm gave poor performance as does not account for the transverse momentum or energy of the particles being clustered. In essence, this is a cone clustering algorithm with a cone radius defined through ${\Delta}R_{ij} = R$, which even for large R was found to discard too much energy in the event to be useful for this analysis.  This can be seen in figure \ref{fig:invariantmassalgoveto} as the invariant mass of the paired jets is much lower than expected.  This algorithm is useful for events with highly boosted jets, but at CLIC the jets are too disperse for this algorithm to be successfully applied.

\begin{figure}
\centering
\subfloat[]{\label{fig:invariantmassalgoveto1400GeV}\includegraphics[width=0.5\textwidth]{PhysicsAnalysis/Plots/SimpleInvMassPlot/InvariantMassesAlgorithmVeto.pdf}}
\caption[Reconstructed invariant masses for different choices of jet algorithm for 1.4 TeV \nu{\nu}qqqq events.]{Reconstructed masses for different choices of jet algorithm for 1.4 TeV \nu{\nu}qqqq events. These masses arise by forcing the reconstructed events into 4 jets and then pairing up the jets into pairs such that the reconstructed invariant masses of the pairs are closest to each other. These samples should be dominated by vector boson scattering involving pairs of W bosons and so it is expected that a peak at the W boson true mass should be observed. As this does not occur for the Cambridge-Aachen algorithm or the ee\_kt algorithm they were deemed unsuitable for this analysis at both 1.4. In the case of the kt algorithm and the ee\_kt algorithm an R parameter of 0.7 was used.}
\label{fig:invariantmassalgoveto}
\end{figure}

\subsection{Lepton Finding} 
\label{sec:isolatedleptonfinding}

Brief 
\textcolor{red}{Unsure of details needed in this section.}

An isolated lepton finder is included in the analysis chain in an attempt to reject background events containing leptons. 

\subsection{Discriminant Variables} 
\label{sec:analysisprocessor}
\textcolor{red}{Best way to show this?}

Finally, an analysis processor is run, which calculates a number of variables used downstream in the analysis. Included in these are:
\begin{itemize}
\item Number of PFOs in the jets and the paired up bosons.
\item Number of charged PFOs in the jets and paired up bosons.
\item Highest energy PFO: energy, momentum, transverse momentum, $cos\theta$.
\item Highest energy electron PFO: energy, momentum, transverse momentum, $cos\theta$.
\item Highest energy muon PFO: energy, momentum, transverse momentum, $cos\theta$.
\item Highest energy photon PFO: energy, momentum, transverse momentum, $cos\theta$.
\item (If in existence) Highest and second highest energy isolated lepton: energy, momentum, transverse momentum, $cos\theta$.
\item Bosons: energy, momentum, transverse momentum, $cos\theta$.
\item Invariant mass of the boson pair.
\item Jets: energy, momentum, transverse momentum, $cos\theta$.
\item $Cos\theta$ Of the missing 3-momentum vector.
\item Recoil mass.
\item Invariant mass of the visible system.
\item $y_{i}$, $y_{i+1}$. Jet clustering parameters ranging from i = 0 to 6.
\item $Cos\theta^{*}_{Jet}$.  This is the opening angle of a pair of jets, assumed to be from a signle boson, in the rest frame of the boson.
\item $Cos\theta^{*}_{Boson}$.  This is the opening angle of a pair of bosons, assumed to be from vector boson scattering, in the rest frame of the di-boson pair.
\item Transverse momentum and energy of the event.
\item Acolinearity of the jet pairs forming the bosons and the acoilinearity of the boson pair.
\item Principle thrust $T$ and the thrust axes $\bar{\textbf{n}}$. Note $\bar{\textbf{n}}$ is a unit vector. These are defined by the following equation
\begin{equation}
T = \text{max}_{\bar{\textbf{n}}} (\frac{\Sigma_{i} \textbf{p}_{i}.\bar{\textbf{n}}}{\Sigma_{i} |\textbf{p}_{i}|^{2}})
\end{equation}
\item The major and minor thrust values. These are defined with respect to the thrust axes $\bar{\textbf{n}}$ in the following way:
\begin{equation}
T = \text{max}_{\bar{\textbf{n}}_{major}} (\frac{\Sigma_{i} \textbf{p}_{i}.\bar{\textbf{n}}_{major}}{\Sigma_{i} |\textbf{p}_{i}|^{2}})
\end{equation}
where $\bar{\textbf{n}}_{major}.\bar{\textbf{n}} = \textbf{0}$. Similarly the minor thrust value is defined as 
\begin{equation}
T = \frac{\Sigma_{i} \textbf{p}_{i}.\bar{\textbf{n}}_{minor}}{\Sigma_{i} |\textbf{p}_{i}|^{2}}
\end{equation}
where $\bar{\textbf{n}}_{minor}.\bar{\textbf{n}} = \bar{\textbf{n}}_{minor}.\bar{\textbf{n}}_{major} =\textbf{0}$
\item Sphericity. This is defined using the sphericity tensor $S^{ab}$ defined as:
\begin{equation}
S^{ab} = \frac{\Sigma_{i}p^{\alpha}_{i}p^{\alpha}_{j}}{\Sigma_{i,\alpha=1,2,3}|p^{\alpha}_{i|^{2}}}
\end{equation}
Where $p_{i}$ are the components of the momenta of particle i in the frame of the detector and the sum runs over all particles in the event. Sphericity is defined as $\text{S} = \frac{3}{2}(\lambda_{2} + \lambda_{3})$, where $\lambda_{i}$ are the eigenvalues of the sphericity tensor defined such $\lambda_{1} \geq \lambda_{2} \geq \lambda_{3}$.  This provides a measure of how spherical the reconstructed event topology is with isotropic events having $S \approx 1$, while two jet events have $S \approx 0$.  (Also $\lambda_{1} + \lambda_{2} + \lambda_{3} = 1$.)
\item Aplanarity. Aplanarity is defined as $\frac{3}{2} \lambda_{3}$ where $\lambda_{3}$ is an eigenvalue of the sphericity tensor.  This provides a measure of whether an event is linear or planar.
\item B and C tag values for the jets, the min and max B and C tag values for the bosons.
\end{itemize}

Alongside these variables, for the \nu{\nu}qqqq final state a number of Monte-Carlo variables are calculated for informative purposes and are not used in the analysis. These include:
\begin{itemize}
\item The quark and neutrino 4 momenta.
\item Invariant mass of boson pair using MC pairing and MC energy.
\item Invariant mass of boson pair using MC pairing and reconstructed jet energy.
\end{itemize}

\section{Event Selection}

As discussed earlier the signal events for this analysis contain the \nu{\nu}qqqq final state. The processes to be considered in this analysis alongside the signal are events that would topologically look similar to signal in the detector. This includes events that could be confused with 4 jet events with missing energy, while excluding those events with large numbers of high energy leptons that could be vetoed easily during the analysis stage. In full the list includes:

\begin{table}[h!]
\centering
\begin{tabular}{ l r }
\hline
Final State & Cross Section 1.4 TeV [fb] \\ 
\hline
$\text{e}^{+}\text{e}^{-} \rightarrow \nu{\nu}\text{qqqq}$ & 24.7 \\
$\text{e}^{+}\text{e}^{-} \rightarrow \text{l}\nu\text{qqqq}$ & 110.4\\
$\text{e}^{+}\text{e}^{-} \rightarrow \text{llqqqq}$ & 62.1\\
$\text{e}^{+}\text{e}^{-} \rightarrow \text{qqqq}$ & 1245.1\\
$\text{e}^{+}\text{e}^{-} \rightarrow \nu{\nu}\text{qq}$ & 787.7\\
$\text{e}^{+}\text{e}^{-} \rightarrow \text{l}\nu\text{qq}$ & 4309.7\\
$\text{e}^{+}\text{e}^{-} \rightarrow \text{llqq}$ & 2725.8\\
$\text{e}^{+}\text{e}^{-} \rightarrow \text{qq}$ & 4009.5\\
$\gamma_{\text{EPA}}\text{e}^{-} \rightarrow \text{qqqq}\text{e}^{-}$ & 287.1\\
$\gamma_{\text{BS}}\text{e}^{-} \rightarrow \text{qqqq}\text{e}^{-}$ & 1160.7\\
$\text{e}^{+}\gamma_{\text{EPA}} \rightarrow \text{qqqq}\text{e}^{+}$ & 286.9\\
$\text{e}^{+}\gamma_{\text{BS}} \rightarrow \text{qqqq}\text{e}^{+}$ & 1156.3\\
$\gamma_{\text{EPA}}\text{e}^{-} \rightarrow \text{qqqq}\nu$ & 32.6\\
$\gamma_{\text{BS}}\text{e}^{-} \rightarrow \text{qqqq}\nu$ & 136.9\\
$\text{e}^{+}\gamma_{\text{EPA}} \rightarrow \text{qqqq}\nu$ & 32.6\\
$\text{e}^{+}\gamma_{\text{BS}} \rightarrow \text{qqqq}\nu$ & 136.4\\
$\gamma_{\text{EPA}}\gamma_{\text{EPA}} \rightarrow \text{qqqq}$ & 753.0\\
$\gamma_{\text{EPA}}\gamma_{\text{BS}} \rightarrow \text{qqqq}$ & 4034.8\\
$\gamma_{\text{BS}}\gamma_{\text{EPA}} \rightarrow \text{qqqq}$ & 4018.7\\
$\gamma_{\text{BS}}\gamma_{\text{BS}} \rightarrow \text{qqqq}$ & 21406.2\\
\hline
\end{tabular}
\caption[]{Cross sections of signal and background processes at 1.4 TeV. In the above table q $\in$ u, $\bar{\text{u}}$, d, $\bar{\text{d}}$, s, $\bar{\text{s}}$, c, $\bar{\text{c}}$, b or $\bar{\text{b}}$ while l $\in$ $\text{e}^{\pm}$, $\mu^{\pm}$ or $\tau^{\pm}$ and $\nu$ $\in$ $\nu_{e}$, $\nu_{\mu}$ and $\nu_{\tau}$.  The subscript EPA or BS for the incoming photons indicate whether the photon is generated from the equivalent photon approximation or beamstrahlung.}
\label{table:crosssectionfull}
\end{table}

Equivalent Photon Approximation (EPA) processes model the electromagnetic field of a charged particle as virtual photons.  BS (beamstrahlung) processes involve photons that have been radiated from incoming charged particles due to interactions with the electromagnetic field of the opposite beam.   The energy spectrum of the incoming particles for CLIC at the relevant operating energy is used to model the energy of these incoming photons.  Included in this study are photon-photon interactions from photons appearing from the EPA and beamstrahlung processes.

\subsection{Pre Selection}
\label{sec:preselection1400GeV}
The primary selection of the \nu{\nu}qqqq signal will be done using a multivariate analysis, however, in an attempt to veto trivial backgrounds a simple cut based preselection is applied. Cuts are applied to the transverse momentum, invariant mass of the visible system and the number of isolated leptons. The raw distributions of these variables is
shown in figure \ref{fig:preselection1400}. Based on these distributions the following cuts were applied:

\begin{itemize}
\item Transverse momentum > 100 GeV. This cut is effective due to the presence of missing energy in the form of neutrinos in the signal final state.
\item Visible mass of the system > 200 GeV. This cut is effective for accounting for the missing energy of the neutrinos in the final state along the longitudinal direction of the detector instead.
\item Number of isolated leptons = 0. This cut vetoes a large number of events with leptons in the final state.
The effect of these preselection cuts can be found in table 1.3. While a large fraction of the signal events are lost through these cuts, particularly the transverse momentum cuts, a much large fraction of background events are removed justifying the cut.
\end{itemize}

The event numbers for the signal and background are shown in table \ref{table:preselectionnumbers1400GeV} as these cuts are cumulatively applied.  These numbers are normalised to the correct luminosity for CLIC running at 1.4 TeV.  As is expected the large transverse momentum cut removes practically all backgrounds containing no missing energy.  The invariant mass cut removes significant fractions of two quark and missing energy events.  Finally, the isolated lepton finder cut removes backgrounds containing visible leptonic final states.  

\begin{table}[h!]
\centering
\begin{tabular}{ l r r r r }
\hline
Final State & Raw Event  & $p_{T}$ > 100 GeV & $p_{T}$ > 100 GeV \& & $p_{T}$ > 100 GeV \& \\ 
& Numbers & & $M_{\text{Vis}}$ > 200 GeV & $M_{\text{Vis}}$ > 200 GeV \&\\ 
& & & & $N_{\text{Isolated Leptons}}$ = 0\\ 
\hline
$\text{e}^{+}\text{e}^{-} \rightarrow \nu{\nu}\text{qqqq}$ & 37,050 & 23,800 & 21,080 & 21,020\\
$\text{e}^{+}\text{e}^{-} \rightarrow \text{l}\nu\text{qqqq}$ & 165,600 & 81,620 & 80,840 & 42,410\\
$\text{e}^{+}\text{e}^{-} \rightarrow \text{llqqqq}$ & 93,150 & 1,151 & 1,140 & 700\\
$\text{e}^{+}\text{e}^{-} \rightarrow \text{qqqq}$ & 1,868,000 & 6,487 & 6,467 & 6,445\\
$\text{e}^{+}\text{e}^{-} \rightarrow \nu{\nu}\text{qq}$ & 1,181,000 & 514,100 & 50,260 & 50,150\\
$\text{e}^{+}\text{e}^{-} \rightarrow \text{l}\nu\text{qq}$ & 6,464,000 & 2,003,000 & 1,259,000 & 567,600\\
$\text{e}^{+}\text{e}^{-} \rightarrow \text{llqq}$ & 4,088,000 & 7,754 & 7,351 & 5,643\\
$\text{e}^{+}\text{e}^{-} \rightarrow \text{qq}$ & 6,011,000 & 34,610 & 34,130 & 34,070\\
$\gamma_{\text{EPA}}\text{e}^{-} \rightarrow \text{qqqq}\text{e}^{-}$ & 430,600 & 2,463 & 2,446 & 865\\
$\gamma_{\text{BS}}\text{e}^{-} \rightarrow \text{qqqq}\text{e}^{-}$ & 1,306,000 & 1,382 & 1,340 & 1,002\\
$\text{e}^{+}\gamma_{\text{EPA}} \rightarrow \text{qqqq}\text{e}^{+}$ & 430,300 & 2,846 & 2,823 & 1,121\\
$\text{e}^{+}\gamma_{\text{BS}} \rightarrow \text{qqqq}\text{e}^{+}$ & 1,301,000 & 654 & 643 & 469\\
$\gamma_{\text{EPA}}\text{e}^{-} \rightarrow \text{qqqq}\nu$ & 48,890 & 17,450 & 13,490 & 8,852\\
$\gamma_{\text{BS}}\text{e}^{-} \rightarrow \text{qqqq}\nu$ & 154,000 & 56,380 & 36,350 & 35,900\\
$\text{e}^{+}\gamma_{\text{EPA}} \rightarrow \text{qqqq}\nu$ & 48,890 & 17,520 & 13,550 & 8,928\\
$\text{e}^{+}\gamma_{\text{BS}} \rightarrow \text{qqqq}\nu$ & 153,400 & 56,280 & 36,340 & 35,900\\
$\gamma_{\text{EPA}}\gamma_{\text{EPA}} \rightarrow \text{qqqq}$ & 1,129,000 & 3,160 & 3,079 & 1,563\\
$\gamma_{\text{EPA}}\gamma_{\text{BS}} \rightarrow \text{qqqq}$ & 4,539,000 & 5,325 & 5,270 & 3,987\\
$\gamma_{\text{BS}}\gamma_{\text{EPA}} \rightarrow \text{qqqq}$ & 4,521,000 & 3,810 & 3,730 & 2,318\\
$\gamma_{\text{BS}}\gamma_{\text{BS}} \rightarrow \text{qqqq}$ & 20,550,000 & 2,445 & 2,445 & 1,673\\
\hline
\end{tabular}
\caption[Number of events passing the various cuts applied in the preselection at 1.4TeV.]{Number of events passing the various cuts applied in the preselection at 1.4TeV.  Event numbers are normalised to the correct luminosity for CLIC at 1.4 TeV.  $p_{T}$ is the transverse momentum of the event,  $M_{\text{Vis}}$ is the visible mass and $N_{\text{Isolated Leptons}}$ is the number of isolated leptons in the event.  In the above table q $\in$ u, $\bar{\text{u}}$, d, $\bar{\text{d}}$, s, $\bar{\text{s}}$, c, $\bar{\text{c}}$, b or $\bar{\text{b}}$ while l $\in$ $\text{e}^{\pm}$, $\mu^{\pm}$ or $\tau^{\pm}$ and $\nu$ $\in$ $\nu_{e}$, $\nu_{\mu}$ and $\nu_{\tau}$.  The subscript EPA or BS for the incoming photons indicate whether the photon is generated from the equivalent photon approximation or beamstrahlung.}
\label{table:preselectionnumbers1400GeV}
\end{table}

\begin{figure}
\centering
\subfloat[Transverse momentum of system.]{\label{fig:preselection1400_1}\includegraphics[width=0.5\textwidth]{PhysicsAnalysis/Plots/PreSelection/1400GeV/TransverseMomentum.pdf}}\hfill
\subfloat[Invariant mass of the visible system.]{\label{fig:preselection1400_2}\includegraphics[width=0.5\textwidth]{PhysicsAnalysis/Plots/PreSelection/1400GeV/InvariantMassSystem.pdf}}
\subfloat[Number of isolated leptons.]{\label{fig:preselection1400_3}\includegraphics[width=0.5\textwidth]{PhysicsAnalysis/Plots/PreSelection/1400GeV/NumberOfIsolatedLeptons.pdf}}
\caption[Distribution of variables cut on in the preselection at 1.4 TeV.]{Distribution of variables cut on in the preselection at 1.4 TeV.}
\label{fig:preselection1400}
\end{figure}

\subsection{MVA}
\label{sec:mva1400GeV}
A multivariate analysis was applied to the data set to refine the selection using the TMVA toolkit \cite{Hocker:2007ht}. The following variables were used for training the TMVA selection.  

\begin{itemize}
\item Number of PFOs in the event.
\item Highest energy PFO type.
\item Transverse momentum of the event.
\item $\text{cos}\theta_{Missing}$.  The cosine of the polar angle of the missing momentum.
\item $\text{cos}\theta_{\text{Highest Energy Track}}$.  The cosine of the polar angle of the track with the largest momentum.
\item $y_{i}$, $y_{i+1}$. Jet clustering parameters ranging from i = 0 to 6.
\item Principle thrust, sphericity and aplanarity as defined in section BLAH.
\item Energy of the highest energy electron in the event.
\item Energy of the highest energy PFO in the event.
\item Energy of the reconstructed bosons.
\item Acolinearity of the reconstructed boson pair.
\item Invariant mass of the reconstructed bosons.
\item Acolinearity of the jets forming the reconstructed bosons. 
\end{itemize}

It was found that the best MVA algorithm for both performance and speed was the booted decision tree (BDT) when comparing different methods using the default settings.  Add plot here.

The BDT was further optimised by varying the number of trees used, the depth of the trees and the number of cuts applied.  The results shown in the rest of this section use the optimal configuration.  For the optimal BDT configuration a significance of S/$\sqrt(\text{S + B}) = 53.6$ was obtained.  

The event numbers passing the BDT cut can be found in table \ref{table:postmvanumbers1400GeV}.  The performance of the BDT is shown in figure \ref{fig:synbosonmass1400GeVMVAimpact}, which shows the change in the distribution of the the invariant mass of the reconstructed bosons as the MVA is applied. As expected the dominant background processes after the MVA is applied are those that will look identical to the visible signal process i.e. qqqq and missing energy.  Two smaller sources of background that pass the MVA exists, those where two jets and missing energy are confused as four jets and missing energy and those where a lepton is not properly reconstructed and the events look like four jets and missing energy.  

\begin{figure}
\centering
\subfloat[No cuts applied.]{\label{fig:nocutssynbosonmass1400GeVMVAimpact}\includegraphics[width=0.5\textwidth]{PhysicsAnalysis/Plots/PostMVASelection/1400GeV/InvariantMassSynBosons_1400GeV_No_Cuts_StackPlot.pdf}}\hfill
\subfloat[Preselection cuts applied.]{\label{fig:nocutssynbosonmass1400GeVMVAimpact}\includegraphics[width=0.5\textwidth]{PhysicsAnalysis/Plots/PostMVASelection/1400GeV/InvariantMassSynBosons_1400GeV_Pt_gt100GeV_MVis_gt200GeV_NIsoLep_eq0_Cuts_StackPlot.pdf}}
\subfloat[Preselection cuts and MVA applied.]{\label{fig:postmvasynbosonmass1400GeVMVAimpact}\includegraphics[width=0.5\textwidth]{PhysicsAnalysis/Plots/PostMVASelection/1400GeV/InvariantMassSynBosons_1400GeV_PostPreSelection_PostMVA_Cuts_StackPlot.pdf}} 
\caption[Impact of preselection and MVA on the reconstructed invariant mass of the bosons arising from jet pairing at 1.4 TeV.]{Impact of preselection and MVA on the reconstructed invariant mass of the bosons arising from jet pairing at 1.4 TeV.}
\label{fig:synbosonmass1400GeVMVAimpact}
\end{figure}

\begin{table}[h!]
\centering
\begin{tabular}{ l r r }
\hline
Final State & Raw Event Numbers & Post MVA Selection Numbers \\ 
\hline
$\text{e}^{+}\text{e}^{-} \rightarrow \nu{\nu}\text{qqqq}$ & 37,050 & 14,770 \\
$\text{e}^{+}\text{e}^{-} \rightarrow \text{l}\nu\text{qqqq}$ & 165,600 & 6,159 \\
$\text{e}^{+}\text{e}^{-} \rightarrow \text{llqqqq}$ & 93,150 & 80 \\
$\text{e}^{+}\text{e}^{-} \rightarrow \text{qqqq}$ & 1,868,000 & 1,264 \\
$\text{e}^{+}\text{e}^{-} \rightarrow \nu{\nu}\text{qq}$ & 1,181,000 & 3,286 \\
$\text{e}^{+}\text{e}^{-} \rightarrow \text{l}\nu\text{qq}$ & 6,464,000 & 6,262 \\
$\text{e}^{+}\text{e}^{-} \rightarrow \text{llqq}$ & 4,088,000 & 234 \\
$\text{e}^{+}\text{e}^{-} \rightarrow \text{qq}$ & 6,011,000 & 1,016 \\
$\gamma_{\text{EPA}}\text{e}^{-} \rightarrow \text{qqqq}\text{e}^{-}$ & 430,300 & 20 \\
$\gamma_{\text{BS}}\text{e}^{-} \rightarrow \text{qqqq}\text{e}^{-}$ & 1,306,000 & 42 \\
$\text{e}^{+}\gamma_{\text{EPA}} \rightarrow \text{qqqq}\text{e}^{+}$ & 430,300 & 19 \\
$\text{e}^{+}\gamma_{\text{BS}} \rightarrow \text{qqqq}\text{e}^{+}$ & 1,301,000 & 44 \\
$\gamma_{\text{EPA}}\text{e}^{-} \rightarrow \text{qqqq}\nu$ & 48,890 & 3,552 \\
$\gamma_{\text{BS}}\text{e}^{-} \rightarrow \text{qqqq}\nu$ & 154,000 & 18,540 \\
$\text{e}^{+}\gamma_{\text{EPA}} \rightarrow \text{qqqq}\nu$ & 48,890 & 3,652 \\
$\text{e}^{+}\gamma_{\text{BS}} \rightarrow \text{qqqq}\nu$ & 153,400 & 18,770 \\
$\gamma_{\text{EPA}}\gamma_{\text{EPA}} \rightarrow \text{qqqq}$ & 1,129,000 & 68 \\
$\gamma_{\text{EPA}}\gamma_{\text{BS}} \rightarrow \text{qqqq}$ & 4,539,000 & 55 \\
$\gamma_{\text{BS}}\gamma_{\text{EPA}} \rightarrow \text{qqqq}$ & 4,521,000 & 0 \\
$\gamma_{\text{BS}}\gamma_{\text{BS}} \rightarrow \text{qqqq}$ & 20,550,000 & 0 \\
\hline
\end{tabular}
\caption[Number of events passing the MVA selection at 1.4TeV.]{Number of events passing the MVA selection at 1.4TeV.  Event numbers are normalised to the correct luminosity for CLIC at 1.4 TeV.   The subscript EPA or BS for the incoming photons indicate whether the photon is generated from the equivalent photon approximation or beamstrahlung.}
\label{table:postmvanumbers1400GeV}
\end{table}

The summary of the selection procedure is given in table \ref{table:selectionsummary1400GeV}.

\begin{table}[h!]
\centering
\begin{tabular}{ l r r r }
\hline
Final State & $\epsilon_{\text{presel}}$ & $\epsilon_{\text{BDT}}$ & $N_{\text{BDT}}$ \\ 
\hline
$\text{e}^{+}\text{e}^{-} \rightarrow \nu{\nu}\text{qqqq}$ & 56.7\% & 39.9\% & 14,770 \\
$\text{e}^{+}\text{e}^{-} \rightarrow \text{l}\nu\text{qqqq}$ & 25.7\% & 3.7\% & 6,159 \\
$\text{e}^{+}\text{e}^{-} \rightarrow \nu{\nu}\text{qq}$ & 4.3\% & 0.3\% & 3,286 \\
$\text{e}^{+}\text{e}^{-} \rightarrow \text{l}\nu\text{qq}$ & 8.8\% & 0.1\% & 6,262 \\
$\gamma_{\text{EPA}}\text{e}^{-} \rightarrow \text{qqqq}\nu$ & 18.0\% & 7.3\% & 3,552 \\
$\gamma_{\text{BS}}\text{e}^{-} \rightarrow \text{qqqq}\nu$ & 23.2\% & 12.0\% & 18,540 \\
$\text{e}^{+}\gamma_{\text{EPA}} \rightarrow \text{qqqq}\nu$ & 18.2\% & 7.5\% & 3,652 \\
$\text{e}^{+}\gamma_{\text{BS}} \rightarrow \text{qqqq}\nu$ & 23.4\% & 12.2\% & 18,770 \\
\hline
\end{tabular}
\caption[Selection summary at 1.4TeV.]{Selection summary at 1.4TeV.   The subscript EPA or BS for the incoming photons indicate whether the photon is generated from the equivalent photon approximation or beamstrahlung.  Channels omitted from this table have less than 1,500 events in the post MVA selection.}
\label{table:selectionsummary1400GeV}
\end{table}








\section{Effect of Anomalous Coupling/Fitting Methodology}
The metric used to determine the optimal jet algorithm for this analysis is the sensitivity of the pure signal final state, $\nu\nu\text{qqqq}$, to the anomalous gauge couplings.  Pure signal was used in the optimisation to avoid the need to process the large number of background files for each iteration of the jet algorithm considered, while still basing the optimisation on the physics of interest.  The focus of this section in the description of the fitting technique that will be used for both jet algorithm optimisation and in the final sensitivity study.

\subsection{Choice of Fitting Distribution}
The sensitivity of CLIC to the anomalous gauge couplings is determined through the use of a $\chi^{2}$ fit to the distribution of $\text{cos}\theta^{*}_{Jets}$.  For a given event, the jet clustering and pairing proceeds as described in section \ref{sec:analysis} and leads to the event being clustered into four jets, which are then paired up to give two candidate bosons.  $\theta^{*}_{Jets}$ is defined as the opening angle of the jets in the rest frame of these candidate bosons.  The distribution of $\text{cos}\theta^{*}_{Jets}$ proved to be highly sensitive to the anomalous gauge couplings as shown in figure \ref{fig:costhetastarjets}.

The distribution of $\text{cos}\theta^{*}_{Bosons}$ was also considered for this sensitivity study, however, it proved to be less sensitive than $\text{cos}\theta^{*}_{Jets}$.  This can be seen when comparing figures \ref{fig:costhetastarjets} and \ref{fig:costhetastarbosons}.  $\theta^{*}_{Bosons}$ is defined as the opening angle between the two candidate bosons two bosons in the rest frame of the candidate boson pair.  Furthermore, it was found that the $\chi^{2}$ distribution formed from the two dimensional distribution of $\text{cos}\theta^{*}_{Jets}$ against $\text{cos}\theta^{*}_{Bosons}$ did not significantly benefit the sensitivity in comparison using the one dimensional distribution of $\text{cos}\theta^{*}_{Jets}$ and therefore was not considered for this analysis.

\begin{figure}
\subfloat[1.4 TeV Events]{\label{fig:costhetastarjets1400GeV} \includegraphics[width=0.5\textwidth]{PhysicsAnalysis/Plots/SensitiveDistributions/CosThetaStarSynJets_SPFOs_kt_0p70_1400GeV.pdf}}
\subfloat[3 TeV Events]{\label{fig:costhetastarjets3000GeV} \includegraphics[width=0.5\textwidth]{PhysicsAnalysis/Plots/SensitiveDistributions/CosThetaStarSynJets_SPFOs_kt_0p70_3000GeV.pdf}}
\caption[Sensitivity of $\text{cos}\theta^{8}_{Jets}$ to the anomalous gauge couplings $\alpha_{4}$ and $\alpha_{5}$ at 1.4 and 3 TeV.]{Sensitivity of $\text{cos}\theta^{*}_{Jets}$ to anomalous couplings at 1.4 and 3 TeV. The jet algorithm used for this example was the longitudinally invariant kt algorithm with an R parameter of 0.7. This sample corresponds to pure signal of hadronic decays in vector boson scattering i.e. \nu{\nu}qqqq.}
\label{fig:costhetastarjets}
\end{figure}

\begin{figure}
\subfloat[1.4 TeV Events]{\label{fig:costhetastarbosons1400GeV} \includegraphics[width=0.5\textwidth]{PhysicsAnalysis/Plots/SensitiveDistributions/CosThetaStarSynBosons_SPFOs_kt_0p70_1400GeV.pdf}}
\subfloat[3 TeV Events]{\label{fig:costhetastarbosons3000GeV} \includegraphics[width=0.5\textwidth]{PhysicsAnalysis/Plots/SensitiveDistributions/CosThetaStarSynBosons_SPFOs_kt_0p70_3000GeV.pdf}}
\caption[Sensitivity of $\text{cos}\theta^{8}_{Bosons}$ to the anomalous gauge couplings $\alpha_{4}$ and $\alpha_{5}$ at 1.4 and 3 TeV.]{Sensitivity of $\text{cos}\theta^{*}_{Bosons}$ to anomalous couplings at 1.4 and 3 TeV. The jet algorithm used for this example was the longitudinally invariant kt algorithm with an R parameter of 0.7. This sample corresponds to pure signal of hadronic decays in vector boson scattering i.e. \nu{\nu}qqqq.}
\label{fig:costhetastarbosons}
\end{figure}


\subsection{Application Of Anomalous Gauge Coupling Event Weights}
\label{sec:eventweightsinterpolation}
As described in section \ref{sec:eventweights}, event weights are used to determine the sensitivity of CLIC to the anomalous gauge couplings.  These event weights are extracted on an event by event basis for the signal final state $\nu\nu\text{qqqq}$ from the generator software Whizard.  To achieve a smooth $\chi^{2}$ distribution a fine sampling of the $\text{cos}\theta^{*}_{Jets}$ distribution in the $\alpha_{4}$ and $\alpha_{5}$ space is needed.  However, as extracting the event weights is highly CPU intensive, it is unfeasible to produce a finely sampled grid of event weights on an event by event basis by calling the generator.  To resolve this issue, an interpolation scheme was applied to determine the event weights within a sampled region of the $\alpha_{4}$ and $\alpha_{5}$ space.  This allows for an infinite sampling of the $\text{cos}\theta^{*}_{Jets}$ distribution in the space of $\alpha_{4}$ and $\alpha_{5}$ within the sampled region, without having to call the generator an infinite number of times.

A bicubic interpolation scheme, cubic interpolation along two dimensions, was applied to the event weights that were extracted from the generator.  This procedure is best illustrated by showing the interpolated surface superimposed with the raw event weights from the generator, which is shown for several $\nu\nu\text{qqqq}$ events at 1.4 TeV in figure \ref{fig:eventweights1400interpolated} (ADD 3 TEV).  This interpolation scheme produces a smooth and continuous surface that is sufficiently accurate for the fitting procedure applied in this analysis.  

For reference at 1.4 TeV event weights were produced from the generator, Whizard, by stepping along $\alpha_{4}$ and $\alpha_{5}$ in steps of 0.01 ranging from -0.07 to 0.07, as shown in figure \ref{fig:eventweights1400raw}, while at 3 TeV event weights were samples in steps of 0.00025 from -0.0045 to 0.0045.  These ranges proved to be sufficient for the contours of interest for the CLIC sensitivity analysis at these energies.

\begin{figure}
\centering
\subfloat[]{\label{fig:weight1}\includegraphics[width=0.5\textwidth]{PhysicsAnalysis/Plots/EventWeights/1400GeV/EventWeightsForEvent100001009_1400GeV_SPFOs_kt_0p70_10Bins_Start_0_End_10_1400GeV_Interpolated.pdf}}
\subfloat[]{\label{fig:weight2}\includegraphics[width=0.5\textwidth]{PhysicsAnalysis/Plots/EventWeights/1400GeV/EventWeightsForEvent100001014_1400GeV_SPFOs_kt_0p70_10Bins_Start_0_End_10_1400GeV_Interpolated.pdf}} \hfill
\subfloat[]{\label{fig:weight3}\includegraphics[width=0.5\textwidth]{PhysicsAnalysis/Plots/EventWeights/1400GeV/EventWeightsForEvent100001044_1400GeV_SPFOs_kt_0p70_10Bins_Start_0_End_10_1400GeV_Interpolated.pdf}}
\subfloat[]{\label{fig:weight4}\includegraphics[width=0.5\textwidth]{PhysicsAnalysis/Plots/EventWeights/1400GeV/EventWeightsForEvent100001051_1400GeV_SPFOs_kt_0p70_10Bins_Start_0_End_10_1400GeV_Interpolated.pdf}}
\caption[Event weights from Whizard for 1.4TeV \nu{\nu}qqqq final state events with interpolated surface.]{A selection of plots showing how the event weight changes when varying the anomalous couplings $\alpha_{4}$ and $\alpha_{5}$ for 1.4TeV \nu{\nu}qqqq final state events.  The hollow circles show the event weight produced from the generator while the surface shown is found using bicubic interpolation between those points.}
\label{fig:eventweights1400interpolated}
\end{figure}

\subsection{Determination of Sensitivity}
The sensitivity of CLIC to the anomalous gauge couplings $\alpha_{4}$ and $\alpha_{5}$ was determined using a $\chi^{2}$ of the following form:

\begin{equation}
\chi^{2} = \Sigma_{i} \frac{(O_{i} - E_{i})^{2}}{E_{i}}
\end{equation}

Here, where $O_{i}$ is the observed, or data, bin content for bin i in the distribution of $\text{cos}\theta^{*}_{Jets}$ produced with event weights corresponding to zero $\alpha_{4}$ and $\alpha_{5}$ and $E_{i}$ is the expected, or Monte-Carlo, bin content for bin i produced with event weights corresponding to non-zero $\alpha_{4}$ and $\alpha_{5}$.  The distribution of $\text{cos}\theta^{*}_{Jets}$ was binned in a histograms containing 10 bins ranging from 0 to 1, as shown in figure \ref{fig:costhetastarjets}.  This binning was selected to maximise the sensitivity of the distribution, while minimising the effect of large bin by bin fluctuations arising from individual events with large event weights.

Confidence limits that describe the sensitivity of CLIC to the anomalous gauge couplings, were found by examining the $\chi^{2}$ surface in $\alpha_{4}$ and $\alpha_{5}$ space.  Deviations about the minima of this surface, which by construction occurs at $\alpha_{4} = \alpha_{5} = 0$, yield confidence limits that indicate the probability of observing a particular value of $\alpha_{4}$ and $\alpha_{5}$ based on the $\text{cos}\theta^{*}_{Jets}$ distribution.  The confidence limits used in subsequent sections, 68\%, 90\% and 99\%, are defined using fixed deviations from the minima of $\chi^{2}$ contours of 2.28, 4.61 and 9.21 respectively.  These numbers arise from the integral of the two dimensional $\chi^{2}$ function.

It proved useful to consider the sensitivities to the individual parameters $\alpha_{4}$ and $\alpha_{5}$ independently.  This was done by projecting out the $\alpha_{4} = 0$ or $\alpha_{5} = 0$ one dimensional $\chi^{2}$ distribution from the two dimensional $\chi^{2}$ previously discussed.  It was then possible to extract the sensitivity to an individual parameters using confidence limits arising from the integral of the one dimensional $\chi^{2}$ function i.e. 68\% confidence limit occurs for $\chi^{2} = 0.989$.  In subsequent chapters these are the sensitivities quoted for individual anomalous gauge coupling parameters. 

HERE

\section{Results}
The sensitivity of the CLIC experiment to the anomalous gauge couplings $\alpha_{4}$ and $\alpha_{5}$ at 1.4 TeV is shown in figure \ref{fig:finalresult1400GeV}.  This result shows the sensitivity after the application of preselection and MVA described in sections \ref{sec:preselection1400GeV} and \ref{sec:mva1400GeV} purposed to remove the included background channels, described in section \ref{sec:eventgenerationandbackgrounds}.  These contours yield the one $\sigma$ confidence limit on the measurement of $\alpha_{4}$ to the range -0.00831, 0.0130 and similarly for the measurement of $\alpha_{5}$ the range is -0.00606, 0.00904.

\begin{figure}
\centering
\subfloat[$\chi^{2}$ sensitivity contours in $\alpha_{4}$ and $\alpha_{5}$ space.]{\label{fig:finalresult1400GeV}\includegraphics[width=0.5\textwidth]{PhysicsAnalysis/Plots/FinalResult/1400GeV/Final.pdf}}\hfill
\subfloat[$\chi^{2}$ as a function of $\alpha_{4}$ assuming $\alpha_{5} = 0$.]{\label{fig:a4finalresult1400GeV}\includegraphics[width=0.5\textwidth]{PhysicsAnalysis/Plots/FinalResult/1400GeV/Final_alpha4.pdf}}
\subfloat[$\chi^{2}$ as a function of $\alpha_{5}$ assuming $\alpha_{4} = 0$.]{\label{fig:a5finalresult1400GeV}\includegraphics[width=0.5\textwidth]{PhysicsAnalysis/Plots/FinalResult/1400GeV/Final_alpha5.pdf}}
\caption[$\chi^{2}$ sensitivity distributions at 1.4 TeV arising from a fit to $\text{cos}\theta^{*}_{\text{Jets}}$.  Results include the effect of backgrounds after the application of preselection and MVA.]{$\chi^{2}$ sensitivity distributions at 1.4 TeV arising from a fit to $\text{cos}\theta^{*}_{\text{Jets}}$.  Results include the effect of backgrounds after the application of preselection and MVA.}
\label{fig:allfinalresult1400GeV}
\end{figure}



\section{Sensitivity at 3 TeV}

\begin{table}[h!]
\centering
\begin{tabular}{ l r r }
\hline
Final State & Cross Section 1.4 TeV [fb] & Cross Section 3 TeV [fb]  \\ 
\hline
$\text{e}^{+}\text{e}^{-} \rightarrow \nu{\nu}\text{qqqq}$ & 24.7 & 71.5 \\
$\text{e}^{+}\text{e}^{-} \rightarrow \text{l}\nu\text{qqqq}$ & 110.4 & 106.6 \\
$\text{e}^{+}\text{e}^{-} \rightarrow \text{llqqqq}$ & 62.1 & 169.3 \\
$\text{e}^{+}\text{e}^{-} \rightarrow \text{qqqq}$ & 1245.1 & 546.5 \\
$\text{e}^{+}\text{e}^{-} \rightarrow \nu{\nu}\text{qq}$ & 787.7 & 1317.5 \\
$\text{e}^{+}\text{e}^{-} \rightarrow \text{l}\nu\text{qq}$ & 4309.7 & 5560.9 \\
$\text{e}^{+}\text{e}^{-} \rightarrow \text{llqq}$ & 2725.8 & 3319.6 \\
$\text{e}^{+}\text{e}^{-} \rightarrow \text{qq}$ & 4009.5 & 2948.9 \\
$\gamma_{\text{EPA}}\text{e}^{-} \rightarrow \text{qqqq}\text{e}^{-}$ & 287.1 & 287.8 \\
$\gamma_{\text{BS}}\text{e}^{-} \rightarrow \text{qqqq}\text{e}^{-}$ & 1160.7 & 1268.6 \\
$\text{e}^{+}\gamma_{\text{EPA}} \rightarrow \text{qqqq}\text{e}^{+}$ & 286.9 & 287.8 \\
$\text{e}^{+}\gamma_{\text{BS}} \rightarrow \text{qqqq}\text{e}^{+}$ & 1156.3 & 1267.3 \\
$\gamma_{\text{EPA}}\text{e}^{-} \rightarrow \text{qqqq}\nu$ & 32.6 & 54.2 \\
$\gamma_{\text{BS}}\text{e}^{-} \rightarrow \text{qqqq}\nu$ & 136.9 & 262.5 \\
$\text{e}^{+}\gamma_{\text{EPA}} \rightarrow \text{qqqq}\nu$ & 32.6 & 54.2 \\
$\text{e}^{+}\gamma_{\text{BS}} \rightarrow \text{qqqq}\nu$ & 136.4 & 262.3 \\
$\gamma_{\text{EPA}}\gamma_{\text{EPA}} \rightarrow \text{qqqq}$ & 753.0 & 402.7 \\
$\gamma_{\text{EPA}}\gamma_{\text{BS}} \rightarrow \text{qqqq}$ & 4034.8 & 2423.1 \\
$\gamma_{\text{BS}}\gamma_{\text{EPA}} \rightarrow \text{qqqq}$ & 4018.7 & 2420.6 \\
$\gamma_{\text{BS}}\gamma_{\text{BS}} \rightarrow \text{qqqq}$ & 21406.2 & 13050.3 \\
\hline
\end{tabular}
\caption[]{Cross sections of signal and background processes at 1.4 and 3 TeV. In the above table q $\in$ u, $\bar{\text{u}}$, d, $\bar{\text{d}}$, s, $\bar{\text{s}}$, c, $\bar{\text{c}}$, b or $\bar{\text{b}}$ while l $\in$ $\text{e}^{\pm}$, $\mu^{\pm}$ or $\tau^{\pm}$ and $\nu$ $\in$ $\nu_{e}$, $\nu_{\mu}$ and $\nu_{\tau}$.  The subscript EPA or BS for the incoming photons indicate whether the photon is generated from the equivalent photon approximation or beamstrahlung.}
\label{table:crosssectionfull}
\end{table}

\begin{table}
\centering
\begin{tabular}{ l r r r r }
\hline
Final State & Cross Section [fb] & Cross Section [fb] & Percentage & CLIC Cross Section \\ 
& ($\alpha_{4} = \alpha_{5} = 0.000$) & ($\alpha_{4} = \alpha_{5} = 0.005$) & Change[\%] & [fb] \\ 
\hline
$\text{e}^{+}\text{e}^{-} \rightarrow \nu{\nu}\text{qqqq}$ & 51.2 & 77.7 & +51.8 & 71.5 \\
$\text{e}^{+}\text{e}^{-} \rightarrow \text{l}{\nu}\text{qqqq}$ & 111.9 & 115.9 & +3.6 & 106.6 \\
$\text{e}^{+}\text{e}^{-} \rightarrow \text{llqqqq}$ & 169.7 & 161.7 & -4.9 & 169.3 \\
\hline
\end{tabular}
\caption{Cross section for selected processes for given value of $\alpha_{4}$ and $\alpha_{5}$ at 3 TeV.}
\label{table:crosssectionsensitivity3000}
\end{table}

\begin{figure}
\centering
\subfloat[Visible Mass, 3 TeV]{\label{fig:cliccomp3000_1}\includegraphics[width=0.5\textwidth]{PhysicsAnalysis/Plots/CLICSampleComparison/3000GeV/InvariantMassSystem.pdf}} \hfill
\subfloat[$\text{cos}\theta_{Missing}$ - The cosine of the polar angle of the missing momentum for an event, 3 TeV.]{\label{fig:cliccomp3000_2}\includegraphics[width=0.5\textwidth]{PhysicsAnalysis/Plots/CLICSampleComparison/3000GeV/CosThetaMissing.pdf}} 
\caption[Comparison of various distributions between samples used in this analysis and the official CLIC samples for the \nu{\nu}qqqq final state at 3 TeV.]{Comparison of various distributions between samples used in this analysis and the official CLIC samples for the \nu{\nu}qqqq final state at 3 TeV.  Both samples have been normalised to the correct luminosity for CLIC running at 3 TeV.}
\label{fig:cliccomp}
\end{figure}

\begin{figure}
\centering
\subfloat[]{\label{fig:invariantmassalgoveto3000GeV}\includegraphics[width=0.5\textwidth]{PhysicsAnalysis/Plots/SimpleInvMassPlot/InvariantMassesAlgorithmVeto3000GeV.pdf}}
\caption[Reconstructed invariant masses for different choices of jet algorithm for 3 TeV \nu{\nu}qqqq events.]{Reconstructed masses for different choices of jet algorithm for 3 TeV \nu{\nu}qqqq events. These masses arise by forcing the reconstructed events into 4 jets and then pairing up the jets into pairs such that the reconstructed invariant masses of the pairs are closest to each other. These samples should be dominated by vector boson scattering involving pairs of W bosons and so it is expected that a peak at the W boson true mass should be observed. As this does not occur for the Cambridge-Aachen algorithm or the ee\_kt algorithm they were deemed unsuitable for this analysis at both 3 TeV. In the case of the kt algorithm and the ee\_kt algorithm an R parameter of 0.7 was used.}
\label{fig:invariantmassalgoveto}
\end{figure}

\subsection{Pre Selection - 3 TeV}

\begin{table}[h!]
\centering
\begin{tabular}{ l r r r r }
\hline
Final State & Raw Event  & $p_{T}$ > 100 GeV & $p_{T}$ > 100 GeV \& & $p_{T}$ > 100 GeV \& \\ 
& Numbers & & $M_{\text{Vis}}$ > 200 GeV & $M_{\text{Vis}}$ > 200 GeV \&\\ 
& & & & $N_{\text{Isolated Leptons}}$ = 0\\ 
\hline
$\text{e}^{+}\text{e}^{-} \rightarrow \nu{\nu}\text{qqqq}$ & 143,000 & 106,600 & 99,390 & 99,130 \\ 
$\text{e}^{+}\text{e}^{-} \rightarrow \text{l}\nu\text{qqqq}$ & 213,200 & 129,800 & 127,300 & 82,880 \\
$\text{e}^{+}\text{e}^{-} \rightarrow \text{llqqqq}$ &338,600 & 32,750 & 31,010 & 23,550 \\
$\text{e}^{+}\text{e}^{-} \rightarrow \text{qqqq}$ & 1,093,000 & 40,180 & 37,360 & 37,300 \\ 
$\text{e}^{+}\text{e}^{-} \rightarrow \nu{\nu}\text{qq}$ & 2,634,000 & 1,333,000 & 380,100 & 379,500 \\
$\text{e}^{+}\text{e}^{-} \rightarrow \text{l}\nu\text{qq}$ & 11,120,000 & 4,240,000 & 2,479,000 & 1,836,000 \\
$\text{e}^{+}\text{e}^{-} \rightarrow \text{llqq}$ & 6,639,000 & 131,400 & 84,980 & 54,780 \\ 
$\text{e}^{+}\text{e}^{-} \rightarrow \text{qq}$ &5,897,000 & 79,440 & 66,790 & 66,730 \\
$\gamma_{\text{EPA}}\text{e}^{-} \rightarrow \text{qqqq}\text{e}^{-}$ & 575,600 & 57,920 & 54,640 & 34,480 \\ 
$\gamma_{\text{BS}}\text{e}^{-} \rightarrow \text{qqqq}\text{e}^{-}$ & 2,004,000 & 99,930 & 90,750 & 83,440 \\
$\text{e}^{+}\gamma_{\text{EPA}} \rightarrow \text{qqqq}\text{e}^{+}$ & 575,600 & 57,990 & 54,290 & 34,190 \\
$\text{e}^{+}\gamma_{\text{BS}} \rightarrow \text{qqqq}\text{e}^{+}$ & 2,002,000 & 100,300 & 90,830 & 83,960 \\   
$\gamma_{\text{EPA}}\text{e}^{-} \rightarrow \text{qqqq}\nu$ & 108,400 & 63,780 & 60,660 & 46,380 \\
$\gamma_{\text{BS}}\text{e}^{-} \rightarrow \text{qqqq}\nu$ & 414,700 & 233,800 & 215,600 & 213,600 \\
$\text{e}^{+}\gamma_{\text{EPA}} \rightarrow \text{qqqq}\nu$ & 108,400 & 64,230 & 61,130 & 46,720 \\ 
$\text{e}^{+}\gamma_{\text{BS}} \rightarrow \text{qqqq}\nu$ & 414,400 & 236,400 & 219,000 & 217,000 \\
$\gamma_{\text{EPA}}\gamma_{\text{EPA}} \rightarrow \text{qqqq}$ & 805,400 & 54,010 & 48,720 & 37,730 \\ 
$\gamma_{\text{EPA}}\gamma_{\text{BS}} \rightarrow \text{qqqq}$ & 3,828,000 & 150,800 & 131,600 & 114,500 \\ 
$\gamma_{\text{BS}}\gamma_{\text{EPA}} \rightarrow \text{qqqq}$ & 3,825,000 & 150,600 & 133,600 & 116,900 \\ 
$\gamma_{\text{BS}}\gamma_{\text{BS}} \rightarrow \text{qqqq}$ & 18,010,000 & 123,500 & 115,400 & 105,000 \\
\hline
\end{tabular}
\caption[Number of events passing the various cuts applied in the preselection at 3 TeV.]{Number of events passing the various cuts applied in the preselection at 3 TeV.  Event numbers are normalised to the correct luminosity for CLIC at 3 TeV.  $p_{T}$ is the transverse momentum of the event,  $M_{\text{Vis}}$ is the visible mass and $N_{\text{Isolated Leptons}}$ is the number of isolated leptons in the event.  In the above table q $\in$ u, $\bar{\text{u}}$, d, $\bar{\text{d}}$, s, $\bar{\text{s}}$, c, $\bar{\text{c}}$, b or $\bar{\text{b}}$ while l $\in$ $\text{e}^{\pm}$, $\mu^{\pm}$ or $\tau^{\pm}$ and $\nu$ $\in$ $\nu_{e}$, $\nu_{\mu}$ and $\nu_{\tau}$.  The subscript EPA or BS for the incoming photons indicate whether the photon is generated from the equivalent photon approximation or beamstrahlung.}
\label{table:preselectionnumbers3000GeV}
\end{table}

\begin{figure}
\centering
\subfloat[Transverse momentum of system.]{\label{fig:preselection3000_1}\includegraphics[width=0.5\textwidth]{PhysicsAnalysis/Plots/PreSelection/3000GeV/TransverseMomentum.pdf}}\hfill
\subfloat[Invariant mass of the visible system.]{\label{fig:preselection3000_2}\includegraphics[width=0.5\textwidth]{PhysicsAnalysis/Plots/PreSelection/3000GeV/InvariantMassSystem.pdf}}
\subfloat[Number of isolated leptons.]{\label{fig:preselection3000_3}\includegraphics[width=0.5\textwidth]{PhysicsAnalysis/Plots/PreSelection/3000GeV/NumberOfIsolatedLeptons.pdf}}
\caption[Distribution of variables cut on in the preselection at 3 TeV.]{Distribution of variables cut on in the preselection at 3 TeV.}
\label{fig:preselection3000}
\end{figure}

\subsection{MVA - 3 TeV}

\begin{table}[h!]
\centering
\begin{tabular}{ l r r }
\hline
Final State & Raw Event Numbers & Post MVA Selection Numbers \\ 
\hline
$\text{e}^{+}\text{e}^{-} \rightarrow \nu{\nu}\text{qqqq}$ & 143,000 & 64,750 \\
$\text{e}^{+}\text{e}^{-} \rightarrow \text{l}\nu\text{qqqq}$ & 213,200 & 23,310 \\
$\text{e}^{+}\text{e}^{-} \rightarrow \text{llqqqq}$ & 338,600 & 2,409 \\
$\text{e}^{+}\text{e}^{-} \rightarrow \text{qqqq}$ & 1,093,000 & 3,069 \\
$\text{e}^{+}\text{e}^{-} \rightarrow \nu{\nu}\text{qq}$ & 2,634,000 & 19,040 \\
$\text{e}^{+}\text{e}^{-} \rightarrow \text{l}\nu\text{qq}$ & 11,120,000 & 27,910 \\
$\text{e}^{+}\text{e}^{-} \rightarrow \text{llqq}$ & 6,639,000 & 786 \\
$\text{e}^{+}\text{e}^{-} \rightarrow \text{qq}$ & 5,897,000 & 1,335 \\
$\gamma_{\text{EPA}}\text{e}^{-} \rightarrow \text{qqqq}\text{e}^{-}$ & 575,600 & 2,860 \\
$\gamma_{\text{BS}}\text{e}^{-} \rightarrow \text{qqqq}\text{e}^{-}$ & 2,004,000 & 8,352 \\
$\text{e}^{+}\gamma_{\text{EPA}} \rightarrow \text{qqqq}\text{e}^{+}$ & 575,600 & 3,063 \\
$\text{e}^{+}\gamma_{\text{BS}} \rightarrow \text{qqqq}\text{e}^{+}$ & 2,002,000 & 8,090 \\
$\gamma_{\text{EPA}}\text{e}^{-} \rightarrow \text{qqqq}\nu$ & 108,400 & 17,950 \\
$\gamma_{\text{BS}}\text{e}^{-} \rightarrow \text{qqqq}\nu$ & 414,700 & 108,000 \\
$\text{e}^{+}\gamma_{\text{EPA}} \rightarrow \text{qqqq}\nu$ & 108,400 & 17,980 \\
$\text{e}^{+}\gamma_{\text{BS}} \rightarrow \text{qqqq}\nu$ & 414,400 & 109,700 \\
$\gamma_{\text{EPA}}\gamma_{\text{EPA}} \rightarrow \text{qqqq}$ & 805,400 & 3,058 \\
$\gamma_{\text{EPA}}\gamma_{\text{BS}} \rightarrow \text{qqqq}$ & 3,828,000 & 9,812 \\
$\gamma_{\text{BS}}\gamma_{\text{EPA}} \rightarrow \text{qqqq}$ & 3,825,000 & 8,880 \\
$\gamma_{\text{BS}}\gamma_{\text{BS}} \rightarrow \text{qqqq}$ & 18,010,000 & 2,213 \\
\hline
\end{tabular}
\caption[Number of events passing the MVA selection at 3 TeV.]{Number of events passing the MVA selection at 3 TeV.  Event numbers are normalised to the correct luminosity for CLIC at 3 TeV.   The subscript EPA or BS for the incoming photons indicate whether the photon is generated from the equivalent photon approximation or beamstrahlung.}
\label{table:postmvanumbers3000GeV}
\end{table}

\begin{table}[h!]
\centering
\begin{tabular}{ l r r r }
\hline
Final State & $\epsilon_{\text{presel}}$ & $\epsilon_{\text{BDT}}$ & $N_{\text{BDT}}$ \\ 
\hline
$\text{e}^{+}\text{e}^{-} \rightarrow \nu{\nu}\text{qqqq}$ & 69.4\% & 45.3\% & 64.750 \\
$\text{e}^{+}\text{e}^{-} \rightarrow \text{l}\nu\text{qqqq}$ & 38.9\% & 10.9\% & 23,310 \\
$\text{e}^{+}\text{e}^{-} \rightarrow \nu{\nu}\text{qq}$ & 14.4\% & 0.7\% & 19,040 \\
$\text{e}^{+}\text{e}^{-} \rightarrow \text{l}\nu\text{qq}$ & 16.5\% & 0.3\% & 27,910 \\
$\gamma_{\text{BS}}\text{e}^{-} \rightarrow \text{qqqq}\text{e}^{-}$ & 4.1\% & 0.4\% & 8,352 \\
$\text{e}^{+}\gamma_{\text{BS}} \rightarrow \text{qqqq}\text{e}^{+}$ & 4.2\% & 0.4\% & 8,090 \\
$\gamma_{\text{EPA}}\text{e}^{-} \rightarrow \text{qqqq}\nu$ & 42.8\% & 16.6\% & 17,950 \\
$\gamma_{\text{BS}}\text{e}^{-} \rightarrow \text{qqqq}\nu$ & 51.6\% & 26.0\% & 108,000 \\
$\text{e}^{+}\gamma_{\text{EPA}} \rightarrow \text{qqqq}\nu$ & 43.1\% & 16.6\% & 17,980 \\
$\text{e}^{+}\gamma_{\text{BS}} \rightarrow \text{qqqq}\nu$ & 52.3\% & 26.5\% & 109,700 \\
$\gamma_{\text{EPA}}\gamma_{\text{BS}} \rightarrow \text{qqqq}$ & 3.0\% & 0.3\% & 9,812 \\
$\gamma_{\text{BS}}\gamma_{\text{EPA}} \rightarrow \text{qqqq}$ & 3.1\% & 0.2\% & 8,880 \\
\hline
\end{tabular}
\caption[Selection summary at 3 TeV.]{Selection summary at 3 TeV.   The subscript EPA or BS for the incoming photons indicate whether the photon is generated from the equivalent photon approximation or beamstrahlung.  Channels omitted from this table have less than 6,000 events in the post MVA selection.}
\label{table:selectionsummary3000GeV}
\end{table}

















\section{Optimisation of Jet Reconstruction} \label{sec:optimisationjetalgo}
The jet algorithm used in this analysis is the longitudinally invariant kt algorithm as described in section \ref{sec:jetpairing}.  The parameters considered in this optimisation were the R parameter, used in the kt algorithm definition, and the PFO selection.  

In attempt to remove the effects of the overlaid $\gamma\gamma \rightarrow \text{Hadron}$ background event a number of cuts \cite{arXiv:1209.4039} are applied to the transverse momenta and the timing information of the PFOs produced by PandoraPFA to reduce the PFOs into a subset that are believed to originate from the desired interaction.  Different values of these cuts give rise to the tight, default and loose selected PFOs that were considered in this optimisation.  



\subsection{1.4 TeV Optimal Jet Reconstruction}
At 1.4 TeV the optimal sensitivity is achieved for either loose selected PFOs with an R parameter of 0.7 or default selected PFOs with an R parameter of 0.9 as can be seen from tables \ref{table:precisiona4signaljetalgo1400GeV} and \ref{table:precisiona5signaljetalgo1400GeV}.  As a tie breaker between these options the separation power, the fraction of events misidentified as either arising from a WW pair or a ZZ pair, was considered.  Again performance was similar, but there was a slight preference towards the use of selected PFOs and an R parameter of 0.9.  While not used in the primary analysis the separation of samples into WW and ZZ events is important for an extension analysis found in section BLAH.  

The optimal contours can be found in figure \ref{fig:chi2jetalgoideal1400GeV} and the optimal 1D plot used to produce the errors references in the tables \ref{table:precisiona4signaljetalgo1400GeV} and \ref{table:precisiona5signaljetalgo1400GeV} can be found in figures \ref{fig:a4chi2jetalgoideal1400GeV} and \ref{fig:a5chi2jetalgoideal1400GeV} respectively.  All other contours and plots for this optimisation can be found in the appendices.  There are minimal performance differences between the various jet algorithm configurations at 1.4 TeV.

\begin{table}[h!]
\centering
\begin{tabular}{ l l l l }
\hline
PFO Selection & Tight Selected PFOs & Selected PFOs & Loose Selected PFOs \\ 
R Parameter & & & \\ 
\hline
0.7 & $-0.00391$ $+0.00497$ & $-0.00385$ $+0.00500$ & $-0.00368$ $+0.00465$ \\
0.9 & $-0.00405$ $+0.00508$ & $-0.00375$ $+0.00464$ & $-0.00379$ $+0.00475$ \\
1.1 & $-0.00406$ $+0.00509$ & $-0.00392$ $+0.00502$ & $-0.00403$ $+0.00496$ \\
\hline
\end{tabular}
\caption[$1\sigma$ precision on measurement of $\alpha_{4}$ for different jet reconstruction parameters considering pure signal at 1.4 TeV.]{Precision on measurement of $\alpha_{4}$ at 1.4 TeV for different jet reconstruction parameters considering pure signal and applying a $\chi^{2}$ fit to $\text{cos}\theta^{*}_{Jets}$.}
\label{table:precisiona4signaljetalgo1400GeV}
\end{table}

\begin{table}[h!]
\centering
\begin{tabular}{ l l l l }
\hline
PFO Selection & Tight Selected PFOs & Selected PFOs & Loose Selected PFOs \\ 
R Parameter & & & \\ 
\hline
0.7 & $-0.00267$ $+0.00313$ & $-0.00266$ $+0.00318$ & $-0.00255$ $+0.00302$ \\
0.9 & $-0.00280$ $+0.00320$ & $-0.00259$ $+0.00302$ & $-0.00258$ $+0.00303$ \\
1.1 & $-0.00281$ $+0.00321$ & $-0.00272$ $+0.00319$ & $-0.00282$ $+0.00315$ \\
\hline
\end{tabular}
\caption[$1\sigma$ precision on measurement of $\alpha_{5}$ for different jet reconstruction parameters considering pure signal at 1.4 TeV.]{Precision on measurement of $\alpha_{5}$ at 1.4 TeV for different jet reconstruction parameters considering pure signal and applying a $\chi^{2}$ fit to $\text{cos}\theta^{*}_{Jets}$.}
\label{table:precisiona5signaljetalgo1400GeV}
\end{table}

\begin{figure}
\centering
\subfloat[$\chi^{2}$ sensitivity contours in $\alpha_{4}$ and $\alpha_{5}$ space.]{\label{fig:chi2jetalgoideal1400GeV}\includegraphics[width=0.5\textwidth]{PhysicsAnalysis/Plots/Chi2ContoursOptimisation/1400GeV/KtSPFOsR0p90.pdf}}\hfill
\subfloat[$\chi^{2}$ as a function of $\alpha_{4}$ assuming $\alpha_{5} = 0$.]{\label{fig:a4chi2jetalgoideal1400GeV}\includegraphics[width=0.5\textwidth]{PhysicsAnalysis/Plots/Chi2ContoursOptimisation/1400GeV/KtSPFOsR0p90_alpha4Optimal.pdf}}
\subfloat[$\chi^{2}$ as a function of $\alpha_{5}$ assuming $\alpha_{4} = 0$.]{\label{fig:a5chi2jetalgoideal1400GeV}\includegraphics[width=0.5\textwidth]{PhysicsAnalysis/Plots/Chi2ContoursOptimisation/1400GeV/KtSPFOsR0p90_alpha5Optimal.pdf}}
\caption[$\chi^{2}$ sensitivity distributions for the $\text{qqqq}\nu\nu$ final state arising from a fit to $\text{cos}\theta^{*}_{\text{Jets}}$ at 1.4 TeV for the optimal jet reconstruction parameters.]{$\chi^{2}$ sensitivity distributions for the $\text{qqqq}\nu\nu$ final state arising from a fit to $\text{cos}\theta^{*}_{\text{Jets}}$ at 1.4 TeV for the optimal jet reconstruction parameters.} 
\label{fig:allchi2jetalgoideal1400GeV}
\end{figure}

\subsection{3 TeV Optimal Jet Reconstruction}
At 3 TeV the optimal sensitivity for the reconstructions considered is achieved for tight selected PFOs with an R parameter of 1.1 as can be seen from tables \ref{table:precisiona4signaljetalgo3000GeV} and \ref{table:precisiona5signaljetalgo3000GeV}.  The optimal contours can be found in figure \ref{fig:chi2jetalgoideal3000GeV} and the optimal 1D plot used to produce the errors references in the tables \ref{table:precisiona4signaljetalgo3000GeV} and \ref{table:precisiona5signaljetalgo3000GeV} can be found in figures \ref{fig:a4chi2jetalgoideal3000GeV} and \ref{fig:a5chi2jetalgoideal3000GeV} respectively.  All other contours and plots for this optimisation can be found in the appendices.  

The gains in optimising the jet algorithm at 3 TeV are larger than those found at 1.4 TeV.  The preference for the tight selected PFOs is to be expected as this configuration minimises the effect of beam induced backgrounds, which are more prominent at higher energies.  

\begin{table}[h!]
\centering
\begin{tabular}{ l l l l }
\hline
PFO Selection & Tight Selected PFOs & Selected PFOs & Loose Selected PFOs \\ 
R Parameter & & & \\ 
\hline
0.7 & $-0.000530$ $+0.000525$ & $-0.000502$ $+0.000507$ & $-0.000547$ $+0.000555$ \\
0.9 & $-0.000566$ $+0.000555$ & $-0.000539$ $+0.000520$ & $-0.000568$ $+0.000553$ \\
1.1 & $-0.000472$ $+0.000472$ & $-0.000508$ $+0.000492$ & $-0.000504$ $+0.000490$ \\
\hline
\end{tabular}
\caption[$1\sigma$ precision on measurement of $\alpha_{4}$ for different jet reconstruction parameters considering pure signal at 3 TeV.]{Precision on measurement of $\alpha_{4}$ at 3 TeV for different jet reconstruction parameters considering pure signal and applying a $\chi^{2}$ fit to $\text{cos}\theta^{*}_{Jets}$.}
\label{table:precisiona4signaljetalgo3000GeV}
\end{table}

\begin{table}[h!]
\centering
\begin{tabular}{ l l l l }
\hline
PFO Selection & Tight Selected PFOs & Selected PFOs & Loose Selected PFOs \\ 
R Parameter & & & \\ 
\hline
0.7 & $-0.000393$ $+0.000370$ & $-0.000356$ $+0.000348$ & $-0.000357$ $+0.000348$ \\
0.9 & $-0.000394$ $+0.000365$ & $-0.000392$ $+0.000361$ & $-0.000396$ $+0.000368$ \\
1.1 & $-0.000351$ $+0.000337$ & $-0.000374$ $+0.000354$ & $-0.000353$ $+0.000336$ \\
\hline
\end{tabular}
\caption[$1\sigma$ precision on measurement of $\alpha_{5}$ for different jet reconstruction parameters considering pure signal at 3 TeV.]{Precision on measurement of $\alpha_{5}$ at 3 TeV for different jet reconstruction parameters considering pure signal and applying a $\chi^{2}$ fit to $\text{cos}\theta^{*}_{Jets}$.}
\label{table:precisiona5signaljetalgo3000GeV}
\end{table}

\begin{figure}
\centering
\subfloat[$\chi^{2}$ sensitivity contours in $\alpha_{4}$ and $\alpha_{5}$ space.]{\label{fig:chi2jetalgoideal3000GeV}\includegraphics[width=0.5\textwidth]{PhysicsAnalysis/Plots/Chi2ContoursOptimisation/3000GeV/KtTPFOsR1p10.pdf}}\hfill
\subfloat[$\chi^{2}$ as a function of $\alpha_{4}$ assuming $\alpha_{5} = 0$.]{\label{fig:a4chi2jetalgoideal3000GeV}\includegraphics[width=0.5\textwidth]{PhysicsAnalysis/Plots/Chi2ContoursOptimisation/3000GeV/KtTPFOsR1p10_alpha4Optimal.pdf}}
\subfloat[$\chi^{2}$ as a function of $\alpha_{5}$ assuming $\alpha_{4} = 0$.]{\label{fig:a5chi2jetalgoideal3000GeV}\includegraphics[width=0.5\textwidth]{PhysicsAnalysis/Plots/Chi2ContoursOptimisation/3000GeV/KtTPFOsR1p10_alpha5Optimal.pdf}}
\caption[$\chi^{2}$ sensitivity distributions for the $\text{qqqq}\nu\nu$ final state arising from a fit to $\text{cos}\theta^{*}_{\text{Jets}}$ at 3 TeV for the optimal jet reconstruction parameters.]{$\chi^{2}$ sensitivity distributions for the $\text{qqqq}\nu\nu$ final state arising from a fit to $\text{cos}\theta^{*}_{\text{Jets}}$ at 3 TeV for the optimal jet reconstruction parameters.} 
\label{fig:allchi2jetalgoideal3000GeV}
\end{figure}

\iffalse
$\text{e}^{+}\text{e}^{-} \rightarrow \nu{\nu}\text{qqqq}$
$\text{e}^{+}\text{e}^{-} \rightarrow \text{l}\nu\text{qqqq}$
$\text{e}^{+}\text{e}^{-} \rightarrow \text{llqqqq}$
$\text{e}^{+}\text{e}^{-} \rightarrow \text{qqqq}$
$\text{e}^{+}\text{e}^{-} \rightarrow \nu{\nu}\text{qq}$
$\text{e}^{+}\text{e}^{-} \rightarrow \text{l}\nu\text{qq}$
$\text{e}^{+}\text{e}^{-} \rightarrow \text{llqq}$
$\text{e}^{+}\text{e}^{-} \rightarrow \text{qq}$
$\gamma_{\text{EPA}}\text{e}^{-} \rightarrow \text{qqqq}\text{e}^{-}$
$\gamma_{\text{BS}}\text{e}^{-} \rightarrow \text{qqqq}\text{e}^{-}$
$\text{e}^{+}\gamma_{\text{EPA}} \rightarrow \text{qqqq}\text{e}^{+}$
$\text{e}^{+}\gamma_{\text{BS}} \rightarrow \text{qqqq}\text{e}^{+}$
$\gamma_{\text{EPA}}\text{e}^{-} \rightarrow \text{qqqq}\nu$
$\gamma_{\text{BS}}\text{e}^{-} \rightarrow \text{qqqq}\nu$
$\text{e}^{+}\gamma_{\text{EPA}} \rightarrow \text{qqqq}\nu$
$\text{e}^{+}\gamma_{\text{BS}} \rightarrow \text{qqqq}\nu$
$\gamma_{\text{EPA}}\gamma_{\text{EPA}} \rightarrow \text{qqqq}$
$\gamma_{\text{EPA}}\gamma_{\text{BS}} \rightarrow \text{qqqq}$
$\gamma_{\text{BS}}\gamma_{\text{EPA}} \rightarrow \text{qqqq}$
$\gamma_{\text{BS}}\gamma_{\text{BS}} \rightarrow \text{qqqq}$
\fi




\chapter{Calorimeter Optimisation Studies}
\label{chap:MoreStuff}

\chapterquote{There, sir! that is the perfection of vessels!}
{Jules Verne, 1828--1905}

\section{Calorimeter Optimisation Studies}

If the future linear collider is to reach it's maximum potantial in terms of energy resolution then, optimisation of the detector will be essential.  The energy resolution in the paricle flow paradigm is dependant upon several detector components.  The momentum of charged particles arises from the shape of the tracks depostied within the detector while the energy of uncharged particles arise from calorimetric measurements.  Application of sophisticated pattern recognition algorithms allows the particle type to be inferred for the charged particles.  In tern this allows for the conversion of the track momentum into an energy measure for the charged particles.  The particle identification algorithms use the toplogical information acquired from the calorimetric energy deposited to infer particle type for a subset of charged particles.  

The calorimetric energy deposits are therefore used in a twofold manner: (i) as energy measurements for neutral particles and (ii) as input for particle idenfitication algorithms.  There is potential for significant gains to be made in physics performance by optimising the calorimters due to their dominant role in energy measurements.  In this chapter the optimisation of the calorimeters is considered.  Parameters such as the longitudinal granularity, transverse granularity and material choices for the calorimeters are considered.  

This chapter concludes with an optimisation of several global parameters for the detector.  These parameters are not calorimeter specific, but the optimisation procedure developed for the calorimeters is appropriate to use.  These parameters relate to the global detector size and the magnetic field applied throughout solenoid in the detector.

\section{Electromagnetic Calorimeter Optimisation}
\label{optstud:sec:ecal}

The first section

\subsection{ECal Transverse Granularity}
\label{optstud:sec:ecal:cellsize}

\begin{figure}
  \includegraphics[width=\largefigwidth]{OptimisationStudies/Plots/JER_vs_SiliconECalCellSize.pdf}
  \caption[Jet energy resolution as a function of ECal cell size for the silicon tungsten ECal option.]{Jet energy resolution is shown for several fixed energy jets as a function of ECal cell size for the silicon tungsten ECal option.}
  \label{optstud:fig:siecalcells}
\end{figure}

\begin{figure}
  \includegraphics[width=\largefigwidth]{OptimisationStudies/Plots/JER_vs_ScintillatorECalCellSize.pdf}
  \caption[Jet energy resolution as a function of ECal cell size for the scintillator tungsten ECal option.]{Jet energy resolution is shown for several fixed energy jets as a function of ECal cell size for the scintillator tungsten ECal option.}
  \label{optstud:fig:scecalcells}
\end{figure}

\subsection{ECal Longitudinal Granularity}
\label{optstud:sec:ecal:nlayers}

\begin{figure}
  \includegraphics[width=\largefigwidth]{OptimisationStudies/Plots/JER_vs_SiliconECalNumberofLayers.pdf}
  \caption[Jet energy resolution as a function of the number of layers in the ECal for the silicon tungsten ECal option.]{Jet energy resolution is shown for several fixed energy jets as a function of the number of layers in the ECal for the silicon tungsten ECal option.}
  \label{optstud:fig:siecallayers}
\end{figure}

\begin{figure}
  \includegraphics[width=\largefigwidth]{OptimisationStudies/Plots/JER_vs_ScintillatorECalNumberofLayers.pdf}
  \caption[Jet energy resolution as a function of the number of layers in the ECal for the scintillator tungsten ECal option.]{Jet energy resolution is shown for several fixed energy jets as a function of the number of layers in the ECal for the scintillator tungsten ECal option.}
  \label{optstud:fig:scecallayers}
\end{figure}

\section{Hadronic Calorimeter Optimsation}
\label{optstud:sec:hcal}

\subsection{HCal Transverse Granularity}
\label{optstud:sec:hcal:cellsize}

\begin{figure}
  \includegraphics[width=\largefigwidth]{OptimisationStudies/Plots/JER_vs_HCalCellSize.pdf}
  \caption[Jet energy resolution as a function of HCal cell size for the scintillator steel HCal option.]{Jet energy resolution is shown for several fixed energy jets as a function of HCal cell size for the scintillator steel HCal option.}
  \label{optstud:fig:hcalcells}
\end{figure}

\subsection{HCal Longitudinal Granularity}
\label{optstud:sec:hcal:nlayers}

\begin{figure}
  \includegraphics[width=\largefigwidth]{OptimisationStudies/Plots/JER_vs_NumberOfLayersInTheHCal.pdf}
  \caption[Jet energy resolution as a function of the number of layers in the HCal for the scintillator steel HCal option.]{Jet energy resolution is shown for several fixed energy jets as a function of the number of layers in the HCal for the scintillator steel HCal option.}
  \label{optstud:fig:hcalnlayers}
\end{figure}

\begin{figure}
  \includegraphics[width=\largefigwidth]{OptimisationStudies/Plots/JER_vs_NumberOfNuclearInterationLengthsInTheHCal.pdf}
  \caption[Jet energy resolution as a function of the number of nuclear interaction lengths in the HCal for the scintillator steel HCal option.]{Jet energy resolution is shown for several fixed energy jets as a function of the number of nuclear interation lenghts in the HCal for the scintillator steel HCal option.}
  \label{optstud:fig:hcaldepth}
\end{figure}

\begin{figure}
  \includegraphics[width=\largefigwidth]{OptimisationStudies/Plots/JER_vs_SamplingFractionInTheHCal.pdf}
  \caption[Jet energy resolution as a function of the sampling fraction in the HCal for the scintillator steel HCal option.]{Jet energy resolution is shown for several fixed energy jets as a function of the sampling fraction in the HCal for the scintillator steel HCal option.}
  \label{optstud:fig:hcalsampfrac}
\end{figure}

\subsection{HCal Absorber Material}
\label{optstud:sec:hcal:absmat}

\begin{figure}
  \includegraphics[width=\largefigwidth]{OptimisationStudies/Plots/JER_vs_HCalAbsorberMaterial.pdf}
  \caption[Jet energy resolution as a function of the absorber matieral in the HCal.]{Jet energy resolution is shown for several fixed energy jets as a function of the absorber material in the HCal.}
  \label{optstud:fig:hcalabsmat}
\end{figure}

\section{Global Detector Parameter Optimisation}
\label{optstud:sec:global}

\subsection{Magnetic Field Strength}
\label{optstud:sec:glob:bfield}

\begin{figure}
  \includegraphics[width=\largefigwidth]{OptimisationStudies/Plots/JER_vs_MagneticField.pdf}
  \caption[Jet energy resolution as a function of magnetic field strength.]{Jet energy resolution is shown for several fixed energy jets as a function of magetic field strength.}
  \label{optstud:fig:bfield}
\end{figure}

\subsection{Inner ECal Radius}
\label{optstud:sec:glob:ecalinnerrad}

\begin{figure}
  \includegraphics[width=\largefigwidth]{OptimisationStudies/Plots/JER_vs_ECalInnerRadius.pdf}
  \caption[Jet energy resolution as a function of the ECal inner radius.]{Jet energy resolution is shown for several fixed energy jets as a function of the inner ECal radius.}
  \label{optstud:fig:ecalinnerrad}
\end{figure}

\iffalse
ECalCellSizeBoth.pdf                                       JER_vs_ScintillatorECalCellSize.pdf
ECalInnerRadius.pdf                                        JER_vs_ScintillatorECalCellSize_500GeV_DiJet_Breakdown.C
HCalCellSize.pdf                                           JER_vs_ScintillatorECalCellSize_500GeV_DiJet_Breakdown.pdf
HCalNumberIntLenght.pdf                                    JER_vs_ScintillatorECalCellSize_91GeV_DiJet_Breakdown.C
HCalNumberOfLayers.pdf                                     JER_vs_ScintillatorECalCellSize_91GeV_DiJet_Breakdown.pdf
HCalSampFraction.pdf                                       JER_vs_ScintillatorECalNumberofLayers.C
JER_vs_BothECalCellSize.C                                  JER_vs_ScintillatorECalNumberofLayers.pdf
JER_vs_BothECalNumberofLayers.C                            JER_vs_SiliconECalCellSize.C
JER_vs_ECalInnerRadius.C                                   JER_vs_SiliconECalCellSize.pdf
JER_vs_ECalInnerRadius.pdf                                 JER_vs_SiliconECalCellSize_500GeV_DiJet_Breakdown.C
JER_vs_HCalAbsorberMaterial.C                              JER_vs_SiliconECalCellSize_500GeV_DiJet_Breakdown.pdf
JER_vs_HCalAbsorberMaterial.pdf                            JER_vs_SiliconECalCellSize_91GeV_DiJet_Breakdown.C
JER_vs_HCalCellSize.C                                      JER_vs_SiliconECalCellSize_91GeV_DiJet_Breakdown.pdf
JER_vs_HCalCellSize.pdf                                    JER_vs_SiliconECalNumberofLayers.C
JER_vs_HCalCellSize1000000GeVMHHHE.C                       JER_vs_SiliconECalNumberofLayers.pdf
JER_vs_HCalCellSize1000000GeVMHHHE.pdf                     MakePlots.C
JER_vs_HCalCellSize1GeVMHHHE.C                             NumberLayersECalBoth.pdf
JER_vs_HCalCellSize1GeVMHHHE.pdf                           ScECalCells.pdf
JER_vs_MagneticField.pdf                                   ScECalConfusion500GeV.pdf
JER_vs_MagneticFieldStrength.C                             ScECalConfusion91GeV.pdf
JER_vs_MagneticFieldStrength.pdf                           ScECalLayers.pdf
JER_vs_NumberOfLayersInTheHCal.C                           SiECalCells.pdf
JER_vs_NumberOfLayersInTheHCal.pdf                         SiECalConfusion500GeV.pdf
JER_vs_NumberOfNuclearInterationLengthsInTheHCal.C         SiECalConfusion91GeV.pdf
JER_vs_NumberOfNuclearInterationLengthsInTheHCal.pdf       SiECalLayers.pdf
JER_vs_SamplingFractionInTheHCal.C                         SiECalNumberOfLayers.pdf
JER_vs_SamplingFractionInTheHCal.pdf                       SiliconECalCells.pdf
JER_vs_ScintillatorECalCellSize.C
\fi



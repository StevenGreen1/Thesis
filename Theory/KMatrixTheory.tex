\chapter{Theory}
\label{chap:theory}

\chapterquote{There, sir! that is the perfection of vessels!}
{Jules Verne, 1828--1905}

\section{Unitarisation}
\label{sec:unitarisation}

\section{The K - Matrix}
\label{sec:kmatrix}

The probability relating an initial state $\Phi_{i}$ to a final state $\Phi_{f}$ is given by

\begin{equation}
  S_{fi} = \bra{\Phi_{f}} S \ket{\Phi_{i}}
  \label{eq:sdefinition}
\end{equation}

where the \textit{S} operator is defined as... (fill in)

The conservation of probability currents demand that the \textit{S} operator be unitary.  

\begin{equation}
  SS^{\dagger} = S^{\dagger}S = \textbf{1}
  \label{eq:sunitarity}
\end{equation}

The transition operator, whose purpose is to encompass changes to states in the given process, is defined as follows 

\begin{equation}
   S = \textbf{1} + iT
\label{eq:tdefinition}
\end{equation}

The uniarity property of the S operator leads to the following identity, also known as the optical theorem (reference), for the transition operator.  

\begin{equation}
  iT^{\dagger}T = T - T^{\dagger} 
  \label{eq:opticaltheorem}
\end{equation}

By applying a Cayley transform to the unitary scattering operator \textit{S} it is possible to map this operator onto a Hermitian operator \textit{K}.  

\begin{equation}
  S = \frac{\textbf{1} + iK/2}{\textbf{1} - iK/2} 
  \label{eq:kdef}
\end{equation}

This is a beneficial construct for comparisons to the interaction Hamiltonian due to the property of Hermiticiy, which is not shared with the \textit{S} operator.

Inverting the above relationships yeilds

\begin{equation}
  K = 2i \frac{\textbf{1} - S}{\textbf{1} + S} = \frac{T}{\textbf{1} + iT/2}
  \label{eq:kdef2}
\end{equation}

This construction is used as the basis for the unitarisation scheme applied in this analysis.  

\section{Unitarisation}
\label{sec:unitarisation}

In a basis where the \textit{S} operator has been diagonalised both \textit{T} and \textit{K} operators will be diagonal from equations \ref{eq:tdefinition} and \ref{eq:kdef2} respectively.  Consdier the complex eigenvalues \textit{t} of the transition operator \textit{T}.  If \textit{t} \= 2\textit{a} then the optical theorem yeilds

\begin{equation}
  |a - i/2| = 1/2
  \label{eq:toperatoreval}
\end{equation}

Therefore, the eigenvalues, \textit{a}, of the \textit{T} operator are found on a circle in the Argand plance centred at i/2 with radius 1/2.  Defining the real eigenvalue of the \textit{K} matrix to be $\textit{k} = 2\textit{a}_{\textit{K}}$ they can be realated to \textit{a} as follows

\begin{equation}
  a_{K} = \frac{a}{1+ia}
  \label{eq:akdef}
\end{equation}

In this basis the Cayley transform is identical to an inverse sterographic projection of the transition operator \textit{T} onto the space of Hermitian operators.  DIAGRAM

It is possible to apply a unitarisation scheme using a \textit{K} matrix calculated up to finite order.  This is done by calculating the eigenvalues of \textit{S} or \textit{T} operators using the inverse of equation \ref{eq:akdef}

\begin{equation}
  a = \frac{a_{K}}{1-ia_{K}}
  \label{eq:adef}
\end{equation}

If we take the \textit{n}-th order approximation to the \textit{T} operator, which is represented by the eigenamplitudes $a_0^{n}$ we can calculate the \textit{n}-th order approximation to the \textit{K} operator using equation \ref{eq:akdef}

\begin{equation}
  a_{K}^{(n)} &= \frac{a_{0}^{(n)}}{1+ia_{0}^{(n)}} = a_{0}^{(1)} + \text{Re} a_{0}^{(2)} + i( \text{Im}a_{0}^{(2)} - ( a_{0}^{(1)} )^{2} ) + ... \\
              &= a_{0}^{(1)} + \text{Re} a_{0}^{(2)} 
  \label{eq:akdef}
\end{equation}

Where in the last step the first order expansion of the optical theorem was applied and terms were kept to second order only.

The next step involves substituting the truncated pertubation series for $a_{K}^{(n)}$ into equation \ref{eq:adef}.  No truncation of the series is applied in this step.  If the theory allows for a perturbative method then this method yeilds an \textit{S} operator that is unitary and reproducibile on an order by order basis. 

Expansion of optical theorem to first order.  

Represent the \textit{n}-th order approximation, $T_0^{n}$, of the \textit{T} operator by the eigenamplitudes $a_0^{n}$.  Substitute this into \ref{eq:opticaltheorem}.

\begin{equation}
  iT^{\dagger}T = T - T^{\dagger} \\
  iT_0^{(n)*}T_0^{(n)} = T_0^{(n)} - T_0^{(n)*}
  \label{eq:opticaltheoremexp}
\end{equation}

Applying this approximation to second order yeilds

\begin{equation}
  4i(a_0^{(1)} + a_0^{(2)})^{*}(a_0^{(1)} + a_0^{(2)}) = 2(a_0^{(1)} + a_0^{(2)}) - 2(a_0^{(1)} + a_0^{(2)})^{*} 
  \label{eq:opticaltheoremexp2}
\end{equation}

Given the \textit{K} operator is Hermitian the values of $a_{K}^{(n)}$ are real for a complete series.  Therefore, we assume the first eigenamplitude $a_{K}^{(1)}$ is real.  Using this assumption equation \ref{eq:opticaltheoremexp2} and keeping only the lowest order terms gives

\begin{equation}  
  (a_0^{(1)})^2 = \text{Im}a_0^{(2)} 
  \label{eq:opticaltheoremexp2}
\end{equation}

%\begin{subequations}
%  \label{eq:cosThetaCk}
%  \begin{equation}
%    \textit{S} = \textbf{1} + i\textit{T}
%    \label{eq:cosThetaCkApprox}
%  \end{equation}
%\end{subequations}


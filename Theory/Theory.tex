\chapter{Theory}
\label{chap:theory}

\chapterquote{There, sir! that is the perfection of vessels!}
{Jules Verne, 1828--1905}

\section{Particle Flow Calorimetry}
\label{sec:unitarisation}

\subsection{Overview}

Particle flow calorimetry is a method of calorimetry where the goal is to reconstruct every visible particle in any given event.  This prcoesses has numerous advantages over traditional calorimetry in that it yeilds superior energy resolutions as well as more topological information that can be used further downstream in physics analyses.  Careful use of this approach to calorimetry allows us to make signficant strides forward in physics understanding in comparison to traditional calorimetric methods.

The application of particle flow calorimetry creates new challenges for detector response on both the hardware and software side.  Fine granularity calorimeters are crucial to be able to track the energy deposits from individual particles and sophisticated pattern recongition software is essential for piecing these energy deposits back together into reconstructed particles.  For particle flow calorimetry to be successful there has to be a synergy between the hardware and software so that the two work together well with success being dictated by physics performance.  

The immense benefits offered by particle flow calorimetry have made it the front runner in terms of the calorimetric approach and as such has been adopted by the future linear lepton-lepton collider.  

\subsection{Paradigm}

The principle of particle flow calorimetry is a simple one, that is to record the energy deposited by a particle in a dectector in the subsystem that offers the best energy resolution.  While a relatively simple aim the application of such a paradigm is challenging as different particles deposit energy throughout the detector in different regions.  

The stable particles that it is possible to measure in a particle collider detector are relatively few in number and can be broken into three categories depending upon their energy depositions.  There are:

\begin{itemize}
\item Charged hardons.  These produce energy depoits in the tracker and both the electromangetic and hadronics calorimeters.
\item Neutral hadrons.  These produce energy depoits in the electromangetic and hadronics calorimeters.
\item Photons.  These produce energy deposits primarily in the electromangetic calorimeters. 
\end{itemize}

As the energy resolution offered by the tracker is signifcantly better that that offered by the calorimeters it is desirable to measure the energy of charged particles in the tracker.  As photons and neutral hadrons do not produce tracks these energy deposits must come from the calorimeter.  This approach is to be contrasted with traditional calorimetry where all energy deposits arise from the calorimeters.  


\section{Physics Theory }

\chapter{Theory}
\label{chap:theory}

\chapterquote{There, sir! that is the perfection of vessels!}
{Jules Verne, 1828--1905}

\section{Calorimetry}
% Probably start taking about EM showers and then move onto differences later.
% Particle Showers

\subsection{Electromagnetic Showers}
% Explain what happens when an electron passes through matter
When an electron passes through a material there are several ways different eenrgy loss mechanisms such as ionisation, nuclear excitation, nucelar inrerations, bremsstrahlung and \delta ray production (high energy electron knoack on).  These all have energy dependancies and vary from material to material, however, at the highest energies, such as those found in modern day particle collider experiments, the dominant energy loss meachnism is bremsstrahlung.  Bremsstrahlung is the process by which a photon is radiated off a charged particle as it interacts with the electromagnetic field of a nucleus.  The energy spectrum of the photons is continuous with values ranging up to the energy of the incoming electron (passage of particles through matter).  The peak of the photon energy spectrum from bremsstrahlung occurs at low values of the fractional energy lossed from the incoming particle, as shown in figure BLAH (p20 pptm, 32.12) indicating that in general several photons will be produced from a decelerating charged particle.  

% Explain what happens when a photon passes through matter
The incoming electron energy is thus distributed into serveal bremsstrahlung photons, which in turn deposit energy within the absorbing material through different mechanisms such as the photoelectric effect, Compton scattering and pair-production.  Again, at high energies a single process is dominant, which in this case is pair production whereby an $\text{e}^{+}\text{e}^{-}$ pair are produced from the interaction of a photon in the field of a charged particle, which in this case would be from the nucleus or the electrons of the absorber material atoms (p22 pptm, 32.15).  The $\text{e}^{+}$ will then go onto annihilate with an the $\text{e}^{-}$ in the absorber material producing two photons (spin conservation) and, if the energies are sufficiecntly high, the $\text{e}^{-}$ from the pair production will produce further bremsstrahlung photons.  

% Explain how both combine to make EM shower
The combination of these two mechanims mean that when an electron enters a material a shower of particles are produced from a chain reaction bremsstrahlung and pair production mechanisms.  This is known as an electromagentic shower.  The same shower mechanics also apply if the incoming particle were a positron, annihilation occuring first, or a photon, pair-production occuring first.  

\subsection{Detection of Electromagnetic Showers}
The goal of effective calorimetry is to get an accurate estimate of the energy of the incoming particle based on the measurements that can be made from the showering particles.  A commen technique for this would be to count the number of ionisation tracks produced in the electromagnetic shower.  The number of tracks will be directly proportional to the energy of the incoming particle as the higher the energy of the incoming particle the larger the number of showering particles and so the larger the number of tracks appearing in the shower.  




Calorimetry relies upon the assumption that the energy deposited within a detector is directly proprtional to the energy of the incoming particle.  The incoming particle if charged will deposit energy throughout the calorimeter via ionisation.  However, the incoming particle may also begin a particle shower.  This occurs when the incoming particle interacts via some process with the material inside the calorimeter and produces a cascase of secondard particles.  These secondary particles may then go on to shower later in the calorimeter.  Each particle in this heirarchy will deposit energy within the detector in various ways such as ionisation, exciting of atomic nuclei, collision with nuclei etc.  These energy deposits if recorded can then be summed to give a measurement of the initial particles energy.

% Why Gaussian
Effective calorimetry then becomes a counting exercise to measure as many of these energy deposits as well as possible.  These energy deposits can be recorded in multiple ways depending upon the choice of detector concept, however, these methods exclusively involve counting either the number of charged tracks, e.g. silicon detectors, or the number of photons, e.g. PMT detector, within the particle shower.

For example consider a silicon detector.  If we denote the sum of all the ionisation track lengths of a particle shower within this detector as $T_{0}$ then this is propartional to the number of ionising particles in the shower $N$.  In this case $N = E_{0}/\epsilon$ where $E_{0}$ is the energy of the particle initiating the shower and $\epsilon$ is the average particle energy within the shower.  This means the energy measurement $E_{0}$ is directly prortional to the number of ionised tracks in the shower.  So if $E \propto N$ then $\sigma_{E} \propto \sigma_{N} = \sqrt{N}$.  The last statement holds true as Poissonian statistics hold for measuring the signal $N$.  In the limit of large $N$, which is the typical case for a calorimeter the Poissonian statistics behave as Gaussian statistics and calorimetry yeilds Gaussian distributions for reconstructed energy measurements.  Therefore, the resolution in an ideal calorimeter goes as $\frac{\sigma_{E}}{E} \propto \frac{1}{\sqrt{E}}$.  The same arguement can be applied when we consider counting the number of photons instead of ionisation tracks in the shower.

% Resolution Terms
There are, however, other sources contributing to the resolution of a calorimeter that must be considered.  In general calorimeter energy resolutions are quotes using the following form:

\begin{equation}
\frac{\sigma_{E}}{E} = \frac{a}{\sqrt{E}} \oplus \frac{b}{E} \oplus c
\end{equation}

There are many different contributing sources to the resolution, these include:
\begin{enumerate}
\item Stoichastic term:  This term contributes to the resolution due to the statistical nature of counting quanta to get a measurement.
\item Noise term:  A noise term can also be added to the resolution that scales as $\frac{1}{E}$.  This is the case as the effect of noise on $\sigma_{E}$ is independant of energy, therefore$\sigma_{E}{E} \propto \frac{1}{E}$.
\item Constant term:  Accounts for effects such as leakage that grow with energy.  If $\sigma_{E} \propto E$ then a constant term is added to the resolution.
\end{enumerate}

% Sampling
In a sampling calorimetry the energy is deposited in the active layers of the calorimeter and this information is used to infer the energy deposited in the adjacent absorber layers.







\section{Particle Flow Calorimetry}
\label{sec:unitarisation}

\subsection{Overview}

Particle flow calorimetry is a method of calorimetry where the goal is to reconstruct every visible particle in any given event.  This prcoesses has numerous advantages over traditional calorimetry in that it yeilds superior energy resolutions as well as more topological information that can be used further downstream in physics analyses.  Careful use of this approach to calorimetry allows us to make signficant strides forward in physics understanding in comparison to traditional calorimetric methods.

The application of particle flow calorimetry creates new challenges for detector response on both the hardware and software side.  Fine granularity calorimeters are crucial to be able to track the energy deposits from individual particles and sophisticated pattern recongition software is essential for piecing these energy deposits back together into reconstructed particles.  For particle flow calorimetry to be successful there has to be a synergy between the hardware and software so that the two work together well with success being dictated by physics performance.  

The immense benefits offered by particle flow calorimetry have made it the front runner in terms of the calorimetric approach and as such has been adopted by the future linear lepton-lepton collider.  

\subsection{Paradigm}

The principle of particle flow calorimetry is a simple one, that is to record the energy deposited by a particle in a dectector in the subsystem that offers the best energy resolution.  While a relatively simple aim the application of such a paradigm is challenging as different particles deposit energy throughout the detector in different regions.  

The stable particles that it is possible to measure in a particle collider detector are relatively few in number and can be broken into three categories depending upon their energy depositions.  There are:

\begin{itemize}
\item Charged hardons.  These produce energy depoits in the tracker and both the electromangetic and hadronics calorimeters.
\item Neutral hadrons.  These produce energy depoits in the electromangetic and hadronics calorimeters.
\item Photons.  These produce energy deposits primarily in the electromangetic calorimeters. 
\end{itemize}

As the energy resolution offered by the tracker is signifcantly better that that offered by the calorimeters it is desirable to measure the energy of charged particles in the tracker.  As photons and neutral hadrons do not produce tracks these energy deposits must come from the calorimeter.  This approach is to be contrasted with traditional calorimetry where all energy deposits arise from the calorimeters.  


\section{Physics Theory }

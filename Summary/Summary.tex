\chapter{Summary}
\label{chap:summary}

\chapterquote{These are the things God has revealed to us by his Spirit.}
{1 Corinthians 2:10}

%========================================================================================

% Validation new technology for CLIC vertex detector
% Pushed boundaries of energy resolution
% Validated calibration of detector simulation
% Optimisation studies of calorimeters
% VBS + AGC Analysis shows good results compared to LHC

The work presented in this thesis has contributed to the future linear collider experiments in terms of detector design, event reconstruction and demonstration of physics potential.  

A number of capacitively coupled pixel detectors were prototyped and tested, using both lab and test beam measurements, to determine whether they were viable for use in the CLIC vertex detector.  The performance of these prototyped devices was extremely good.  As an offset between the sensor and readout ASICs could be accidentally introduced to the devices during the manufacturing procedure, a number of devices were examined that contained a known offset.  Even devices contains an offset of up to $\frac{1}{4}$ of a pixel were found to have comparable performance to the ideally aligned devices.  Although modifications would be required for the final design of the sensor and readout ASICs to fully optimise performance, the technique of capacitive coupling of sensor to readout ASICs was found to be viable for use at the future linear collider experiments.  

An optimisation study of the calorimeter design for use at the future linear collider was performed.  This study clarified those detector parameters that are crucial for achieving outstanding performance in the particle flow paradigm.  Furthermore, this work has made it possible to make informed decisions about the detector design that minimise the cost, while retaining outstanding jet energy resolutions.  Reliability in the conclusions drawn from this study could only be achieved by employing the calibration procedure that was developed for the linear collider simulation.  

Development of novel software techniques, which make full use of the segmentation of the linear collider calorimeters, led to a significant improvement in the energy resolution of the linear collider detector.  This improvement would be extremely expensive to obtain if it were achieved by modifying the design of the calorimeters, therefore, as well as extending the physics reach of the detector, a significant cost reduction was made.   

The final study presented in this thesis determined the sensitivity of the CLIC experiment to the anomalous gauge couplings $\alpha_{4}$ and $\alpha_{5}$ through the vector boson scattering process.  The signal final state ${\nu}{\nu}$qqqq was selected for this analysis based on the relative sensitivities of final states showing sensitivity to these couplings.  Background processes were then selected based on whether they could be confused with the signal.  An event selection procedure was applied to separate the signal and backgrounds.  The significance obtained from this event selection was 52.7 (90.6) for CLIC running at 1.4 (3) TeV.  Finally, a $\chi^{2}$ fit was applied to the distribution of the invariant mass of the system to determine the sensitivity of the CLIC experiment to the anomalous gauge couplings.  The sensitivity manifested itself in the form of event weights for the signal final state.  Using this procedure the one $\sigma$ confidence limits on the couplings, assuming the corresponding coupling is zero, were found to be
%
\begin{equation}
-0.0082 < \alpha_{4} < 0.0116 \text{,} \\
-0.0055 < \alpha_{5} < 0.0078 \text{,}
\end{equation}
%
\noindent at 1.4 TeV and
%
\begin{equation}
-0.0010 < \alpha_{4} < 0.0011 \text{,} \\
-0.0007 < \alpha_{5} < 0.0007 \text{,}
\end{equation}
%
\noindent at 3 TeV.  These limits significantly improve on the measurements made at the LHC, Run 1, by a factor of approximately 10 (100) at 1.4 (3) TeV \cite{Green:2016trm}.  This is a large gain in sensitivity indicating just one aspect of the physics capabilities of the linear collider.  This study adds further weight to the argument for the construction of a linear collider.   

%========================================================================================

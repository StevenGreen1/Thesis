\chapter{Summary}
\label{chap:summary}

\chapterquote{There, sir! that is the perfection of vessels!}
{Jules Verne, 1828--1905}

%========================================================================================

% Validation new technology for CLIC vertex detector
% Pushed boundaries of energy resolution
% Validated calibration of detector simulation
% Optimisation studies of calorimeters
% VBS + AGC Analysis shows good results compared to LHC

The work presented in this thesis has made significant contributions to the future linear collider in terms of both detector design, event reconstruction and demonstration of physics potential.  

A capacitively coupled pixel was prototyped and tested using both lab and test beam measurements to determine whether the devices were viable for use at the CLIC vertex detector.  The performance of these prototyped devices was extremely good, even with significant offsets between the sensor and readout ASICs that could appear in the manufacturing process.  Although modifications would be required for the final design of the sensor and readout ASICs, the technique of capacitively coupling is viable for use at the future linear collider.  

Studies into the calorimeter design have helped to clarify the detector parameters that are crucial for achieving outstanding performance when using particle flow calorimetry.  This allows for informed decisions to be made that minimise the cost of the detector, while retaining exceptional jet energy resolutions.  Reliability in the conclusions of this study could only be achieved by employing the calibration procedure that was developed for the linear collider simulation.  

Development of novel software techniques, which make full use of the segmentation of the linear collider calorimeters, led to a significant improvement in the energy resolution of the linear collider detector.  This improvement in energy resolution would be extremely expensive if it were achieved by modifying the design of the calorimeters, therefore, as well as extending the physics reach of the detector a significant cost saving has been made.   

This final study presented determined the sensitivity of the CLIC experiment to the anomalous gauge couplings $\alpha_{4}$ and $\alpha_{5}$ using the vector boson scattering process.  The signal final state ${\nu}{\nu}$qqqq was selected for this analysis based on the relative sensitivities of final states showing sensitivity to these couplings.  Background processes were then selected based on whether they could be confused with the signal.  An event selection procedure was applied to separate the signal and backgrounds.  The significance obtained from this event selection was 52.7 (90.6) for CLIC running at 1.4 (3) TeV.  Finally, a $\chi^{2}$ fit was applied to the distribution of the invariant mass of the system to determine the sensitivity of the CLIC experiment to the anomalous gauge couplings.  The sensitivity manifested itself in the form of event weights for the signal final state.  Using this procedure the one $\sigma$ confidence limits on the couplings, assuming the corresponding coupling is zero, were found to be:
%
\begin{equation}
-0.0082 < \alpha_{4} < 0.0116 \text{,} \\
-0.0055 < \alpha_{5} < 0.0078 \text{,} \\
\end{equation}
%
\noindent at 1.4 TeV and:
%
\begin{equation}
-0.0010 < \alpha_{4} < 0.0011 \text{,} \\
-0.0007 < \alpha_{5} < 0.0007 \text{,} \\
\end{equation}
%
\noindent at 3 TeV.  These limits significantly improve on the measurements made at the LHC, Run 1, by a factor of approximately 10 (100) at 1.4 (3) TeV \cite{Green:2016trm}.  This is a significant improvement indicating just one aspect of the physics capabilities of the linear collider.  This study adds further weight to the argument for the construction of a linear collider.   

%========================================================================================
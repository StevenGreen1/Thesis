\chapter{Energy Estimators}
\label{chap:energyestimators}

\chapterquote{There, sir! that is the perfection of vessels!}
{Jules Verne, 1828--1905}

%========================================================================================
%========================================================================================

\section{Motivation}
\label{sec:motivation}
The separation of energy deposits from charged and neutral particles in the calorimeters is crucial for achieving good energy resolutions in the particle flow paradigm.  This is only possible if the energy estimators for those energy deposits are accurate.  The ultimate goal of the calibration procedure outlined in this chapter is to obtain the best energy estimator for particles showering in the calorimeters.  

Calorimeters are designed to measure the energy of particles.  

Calorimeters initiate particle showers 

Particle showers are a cascade of secondary particles that are produced from an interaction between a high energy particle and a dense material.  



When a particle showers within a calorimeter it will create several energy deposits, or hits.  Depending on the segmentation of the calorimeter, many hits can be created when a particle showers.  The energy of the showering particle, $E_{Cluster}$, is determined by grouping these energy deposits together into clusters and summing their energy:
%
\begin{equation}
E_{Cluster} & = \sum_{ECal \text{ } hits \text{, }i} E^{i}_{ECal} +\sum_{HCal \text{ } hits \text{, }i} E^{i}_{HCal} \text{ ,}
\end{equation}
%
\noindent where $E^{i}_{ECal}$ is the energy of ECal hit $i$, $E^{i}_{HCal}$ is the energy HCal hit $i$ and $\sum$ is the summation over all hits in a given calorimeter.  In this example, the energy deposits made by the showering particle are assumed to be split across an ECal and a HCal, therefore, the sum runs over the hits in both calorimeters.  

At the linear collider experiments, the calorimeters are sampling calorimeters.  Sampling calorimeters are comprised of alternating layers of active and absorber materials \cite{Fabjan:2003aq};  The absorber layers are designed to initiate particle showers and propagate their development, while the active layers are designed to give a response that is proportional to the energy deposited within them.  It is only the response in the active layers of a sampling calorimeter that are measured, which means that the energy deposited in the absorber medium has to be estimated based on the active layer energies.  



This is a naive energy estimator that will act as a starting point for the development of more sophisticated procedures aimed at improving detector performance.  


However, before going further the calorimeter hit energies must be determined, which for a sampling calorimeter is non-trivial.  

As the response of a sampling calorimeter gives a measure of the energy deposited in the active layers only, the energy deposited in the absorber medium has to be estimated based on the active layer energies.  This estimation is made by assuming the energy deposited across a calorimeter hit, that is one active and one absorber layer, is uniform.  Working under this assumption, the total calorimeter hit energy is proportional to the active layer hit energy.  This estimation procedure is loosely referred to as digitisation and, in this way, the cluster energy estimator introduced above can be written as:
%
\begin{equation}
E_{Cluster} = \sum_{ECal \text{ } hits \text{, }i} \epsilon^{i}_{ECal} \alpha_{ECal} + \sum_{HCal \text{ } hits \text{, }i} \epsilon^{i}_{HCal} \alpha_{HCal} \text{ ,}
\end{equation}
%
\noindent where $\alpha_{ECal}$ and $\alpha_{HCal}$ are digitisation constants for the ECal and HCal respectively, $\epsilon^{i}_{ECal}$ is the energy response in the active medium for ECal hit $i$, $\epsilon^{i}_{HCal}$ is the energy response in the active medium for a HCal hit $i$ and $\sum$ is the summation over all hits in a given calorimeter.  The first stage of the calibration procedure presented in this chapter covers the determination of these digitisation constants.  

Once the basic energy estimator has been calibrated, it is possible to apply more advanced procedures designed to give a compensating calorimeter response \cite{arXiv:0907.3577}.  A compensating calorimeter produces an identical response to a particle shower irrespective of whether the particle shower is electromagnetic or hadronic in nature.  It is the atomic properties of the materials used in a calorimeter that determine whether a calorimeter is compensating.  The primary cause of a difference in the response of a calorimeter to electromagnetic and hadronic showers is the invisible energy component that is found in hadronic showers.  As this component cannot be measured by the calorimeter, the response is typically lower for hadronically showering particles than for electromagnetically showering particles.  The invisible component exists due to a combination of effects such as neutrons stopping within the calorimeter and nuclear binding energy losses.  If left unchecked, this difference would lead to a systematic loss of energy for hadronic showers that would harm detector performance.  

There are two distinct routes available for negating this unwanted effect and achieving a compensating response from a calorimeter:  The first is hardware compensation \cite{Derrick:1991tq}, whereby calorimeters are constructed using materials that yield extra energy in response to hadronic showers, and the second is software compensation \cite{Tran:2017tgr}, whereby the uncompensated calorimetric energies for hadronic showers are modified at the software level.  The linear collider lends itself to software compensation as the fine segmentation of the calorimeters and precise reconstruction of individual particles makes identification of hadronic showers, and modifying their energies, feasible.  A basic form of software compensation included in the linear collider reconstruction is the modification of the electromagnetic cluster energy estimator to:
%
\begin{equation}
E_{EM \text{ } Cluster} & = \sum_{ECal \text{ } hits \text{, }i} E^{i}_{ECal} \beta^{EM}_{ECal} + \sum_{HCal \text{ } hits \text{, }i} E^{i}_{HCal} \beta^{EM}_{HCal} \text{ ,}
\end{equation}
%
\noindent and the hadronic cluster energy to:
%
\begin{equation}
E_{Had \text{ } Cluster} & = \sum_{ECal \text{ } hits \text{, }i} E^{i}_{ECal} \beta^{Had}_{ECal} + \sum_{HCal \text{ } hits \text{, }i} E^{i}_{HCal} \beta^{Had}_{HCal} \text{ ,}
\end{equation}
%
\noindent where the $\beta$s are scaling factors that are applied to the energy of clusters of calorimeter hits associated with electromagnetic and hadronic clusters in the ECal and HCal.  This simple scaling of energies compensates the response of the calorimeters, which leads to better detector performance.  Determination of these energy scale setting constants is the second stage of the calibration procedure that is presented in this chapter.  

While this scaling of energies improves detector performance, it does not account for any changes to the $\beta$ scaling factors as a function of the total energy deposited.  An energy dependence in the scaling factors is expected as the mechanisms governing the propagation of hadronic showers are sensitive to the shower energy \cite{Wigmans:2000vf}.  To account for this, more sophisticated software techniques have been developed that vary the calorimeter cluster energy estimator as a function of energy to achieve a compensating response across a wider range of energies.  These techniques make use of the fine segmentation of the linear collider calorimeters to identify hadronic showers.  These techniques also address the problem of spuriously high energy calorimeter hits that if left unchecked would damage the reconstruction.  These high energy hits are caused by Landau fluctuations \cite{Landau:1944if}, which originate from high energy knock-on electrons appearing within particle showers \cite{Bichsel:2004ej}.  In this chapter, following a description of the calibration procedure outlined above, is an explanation of these novel energy estimators and the impact they have on detector performance.  

This chapter concludes with a study determining the impact on detector performance of timing cuts that are applied to the calorimeter hits.  These cuts form part of the software trigger that will be used at the linear collider experiment.  Details regarding how all the detector performance metrics used in this chapter are calculated can be found in section \ref{sec:optstudiesmetric}.

%========================================================================================

\subsubsection{Hardware Compensation}
A novel example of hardware compensation is the ZEUS calorimeter \cite{Derrick:1991tq}.  The ZEUS calorimeter was constructed using uranium as the absorber material.  In response to neutral hadrons the uranium undergoes fission producing extra energy that increases the hadronic response of the calorimeter.  The amount of uranium was carefully chosen to achieve a fully compensating calorimeter response i.e. identical calorimeter response to electromagnetic and hadronic showers.  While hardware compensation is possible for the linear collider calorimeters, restrictions on calorimeter construction and the use of a large amount of radioactive material are highly undesirable.  

%========================================================================================

\subsubsection{Calibration and detector optimisation}
Optimising the detector at a future linear collider will be crucial to exploit the full physics potential available to it.  An extensive optimisation of the calorimeters was performed and the results can be found in chapter \ref{sec:optimisationstudies}.  For each detector model considered, the calibration procedure outlined in section \ref{sec:overviewcalibration} was applied to ensure optimal energy estimators were used to quantify the detector performance.  This made it possible to perform unbiased comparisons between detector models, which ensured reliability in the conclusions drawn from those studies.

%========================================================================================

\section{Calibration in the particle flow paradigm}
\label{sec:overviewcalibration}
The calibration procedure described in the following section discusses how the digitisation constants, $\alpha$, and the scaling factors $\beta$ that were introduced in section \ref{sec:motivation} are determined.  In addition to this, minimum ionising particle (MIP) scale setting is included in the calibration procedure to ensure optimal detector performance is achieved.  

The MIP scale is the average energy response, on a per hit level, of a calorimeter when a normally incident MIP passes through it.  This energy scale is used by the digitisation processor and PandoraPFA to apply noise vetoing cuts to the calorimeter hit energies.  The digitisation processor will only create a calorimeter hit if the corresponding active layer energy exceeds a threshold energy, which is defined in units of MIPs.  Similarly, PandoraPFA will only use a calorimeter hit in the reconstruction if the calorimeter hit energy exceeds a threshold energy, which is also defined in units of MIPs.  Both of these cuts are designed to veto noise that would be present in a real detector.  While noise is not explicitly added to the detector simulations, these thresholds are still applied to ensure the simulation better reflects the performance of a real detector.  In addition to this, the MIP scale is used by the digitiser for simulating a number of realistic detector effects.  This includes a truncation of the active layer calorimeter hit energies, which is set in units of MIPs.  This truncation mimics saturation effects in the calorimeter readout technology.  Setting the MIP scale is crucial for ensuring the noise vetoing cuts and realistic effects are correctly applied.
 
The MIP scale in the digitiser and PandoraPFA, while intrinsically linked, have to be calculated separately.  The digitiser requires the MIP scale to be defined using the active layer hit energies while, PandoraPFA requires the definition from the full hit, the active and absorber layer, energies.  It may be expected that the digitisation MIP scale could be related to the PandoraPFA MIP scale by the digitisation constants, however, this is not necessarily the case; additional cuts are applied to the full calorimeter hit energies that are not applied to the active layer hit energies e.g. timing cuts.  Therefore, the conservative approach of independently recalculating the MIP scale for PandoraPFA was taken.   

The $\alpha$ and $\beta$ constants are determined by tuning the mean of reconstructed energy distributions.  A number of cuts are applied when populating these reconstructed energy distributions that ensure the relevant reconstructed energy is being tuned.  The application of these cuts means that linear scaling of the $\alpha$ and $\beta$ constants does not lead to a linear shift in the mean of the reconstructed energy distributions.  Therefore, when calibrating the $\alpha$ and $\beta$ constants an iterative approach is taken; the next iteration of the calibration constant is determined by repeating the reconstruction using the current iteration of the constant and adjusting the constant based on the mean of the reconstructed energy distribution.  

%========================================================================================

\subsubsection{Ordering of Calibration Procedure}
\label{sec:ordercalibration}
The calibration procedure is broadly split into four separate operations: determination of digitisation constants, determination of scaling factor constants and  MIP scale setting in the digitiser and PandoraPFA.  The ordering of each of these calibration steps is crucial as it is possible to get interference between the different stages if applied in an arbitrary order.  With that in mind, the procedure order to minimise interference is as follows:

\begin{enumerate} 
\item Setting the MIP scale setting in the digitiser.  No dependancies on other calibration constants.  
\item Setting the digitisation constants, $\alpha$s.  This relies upon the MIP scale in the digitiser being set.
\item Setting the MIP scale setting in PandoraPFA.  This relies upon correct calibration of calorimeter hit energies, therefore, the $\alpha$s must be calibrated first.
\item Setting the scaling factors, $\beta$s.  This relies upon correct calibration of calorimeter hit energies and correct MIP scale setting in PandoraPFA.
\end{enumerate} 

%========================================================================================

\subsubsection{Photon Likelihood Data}
Some of the algorithms in PandoraPFA make use of likelihood data to identify electromagnetic showers.  This likelihood data consists of a series of probability density functions (PDFs) that describe a number of topological properties of PFOs.  The data contained in these PDFs is related solely to the ECal, which means that every time the ECal is changed from that of the nominal ILD detector this likelihood data must be retrained.  Should retraining be necessary it should be performed after the application of the calibration procedure as described in section \ref{sec:ordercalibration}.  To ensure PandoraPFA gives optimal performance for jet reconstruction, the likelihood data is trained using off-shell mass Z boson events, at 500~GeV, decaying into light quarks (u, d, s).  

%========================================================================================

\subsection{Digitisation Implementation}
\label{sec:digi}
This section discusses how the digitisation constants, $\alpha$, introduced in section \ref{sec:motivation} are determined.  The digitisation constant for a given calorimeter depends upon several factors such as the material properties of the active and absorber layers, the magnetic field strength and energy losses occurring within the gaps in the detector.  Therefore, each calorimeter in the ILD detector model has a distinct constant that must be independently determined. 

%========================================================================================

\subsubsection{ECal Digitisation Implementation}
\label{sec:ecaldigi}
The procedure for determining the digitisation constants in the ECal involves simulation of single $\gamma$ events at energy $E_{MC} = 10$~GeV.  $\gamma$ events are ideal for calibration of the ECal as $\gamma$s are, at this energy, largely contained within the ECal, as shown in figure \ref{fig:ecaldigiphotonsplit}.  This makes them ideal for isolating the ECal digitisation calibration from that of the HCal digitisation calibration.  Events are only used for calibrating the ECal digitisation if they are confined to the ECal.  To that extent, cuts are applied ensuring that the sum of the reconstructed energy found outside the ECal is less than 1\% of $E_{MC}$ and that the $\text{cos}(\theta) < 0.95$, where $\theta$ is the polar angle of the $\gamma$.  The polar angle cut veto events where the $\gamma$ goes down the beam pipe.  $\gamma$ conversions are also vetoed in this event sample at MC level.  The impact of these cuts on the sum of ECal hit energies for the $E_{MC} = 10$~GeV $\gamma$ events is shown in figure \ref{fig:ecaldigiselection}.

\begin{figure}[h!]
\subfloat[]{\label{fig:ecaldigiphotonsplit}\includegraphics[width=0.5\textwidth]{Calibration/Plots/Calibration/Digitsation/ECal/ECalHCalPhotonSplit.pdf}}
\subfloat[]{\label{fig:ecaldigiselection}\includegraphics[width=0.5\textwidth]{Calibration/Plots/Calibration/Digitsation/ECal/DigitisationECalSelection.pdf}}
\caption[\protect\subref{fig:ecaldigiphotonsplit} The sum of calorimeter hit energies in ECal and HCal for 10~GeV $\gamma$ events.  \protect\subref{fig:ecaldigiselection} The sum of the ECal calorimeter hit energies for 10~GeV $\gamma$ events with and without the selection cuts.]{\protect\subref{fig:ecaldigiphotonsplit} The sum of calorimeter hit energies in ECal and HCal for 10~GeV $\gamma$ events.  \protect\subref{fig:ecaldigiselection} The sum of the ECal calorimeter hit energies for 10~GeV $\gamma$ events with and without the selection cuts.}
\label{fig:ecaldigi}
\end{figure}

The calibration of the digitisation in the ECal is an iterative procedure, which begins with the simulation of single $\gamma$ events using a trial calibration, with digitisation constant in the ECal $\alpha^{0}_{\text{ECal}}$.  Next the distribution of the sum of calorimeter hit energies within the ECal is produced for events passing the selection cuts, as shown in figure \ref{fig:ecaldigiselection}.  For an ideal calorimeter this distribution should be Gaussian, as described in chapter \ref{chap:detopt}, therefore, a Gaussian fit is applied to this distribution and the mean, $E_{\text{Fit}}$, extracted.  To remove the effect of any outliers in this distribution, the fit is applied to the range of data with the smallest root mean square that contains at least 90 \% of the data.  An example of such a fit is shown in figure \ref{fig:ecaldigifit}.  In the case of ideal calibration, the mean of this fit, $E_{\text{Fit}}$, would be equal $E_{MC}$.  It is assumed that any difference between the two is due to the calibration, therefore, to correct this the digitisation constant from the trial calibration, $\alpha^{0}_{\text{ECal}}$, is rescaled by the ratio of the $E_{MC}$ to $E_{\text{Fit}}$:
%
\begin{equation}
\alpha^{0}_{\text{ECal}} \rightarrow \alpha_{\text{ECal}} = \alpha^{0}_{\text{ECal}} \times \frac{E_{MC}}{E_{Fit}}\text{ ,}
\end{equation}
%
This procedure is then repeated until the $E_{\text{Fit}}$ falls within a specified tolerance.  The tolerance applied here was $|E_{\text{Fit}} - E_{\text{MC}}| < E_{\text{MC}} \times 5 \%$.  The binning used for the fitted histogram is chosen such that the bin width is equal to the desired tolerance on $E_{\text{Fit}}$ e.g. $E_{\text{MC}} \times 5 \% = 0.5$~GeV.  This tolerance is somewhat loose, however, it is tight enough to ensure successful application of PFA.  It should also be emphasised that the PFO energies used for downstream analyses have the electromagnetic and hadronic energy scale corrections applied, which are calibrated to a much tighter accuracy.

\begin{figure}[h!]
\includegraphics[width=0.5\textwidth]{Calibration/Plots/Calibration/Digitsation/ECal/DigitisationECalFit.pdf}
\caption[Gaussian fit to sum of the ECal calorimeter hit energies for 10~GeV $\gamma$ events with selection cuts.  The coarse binning reflects the tolerance on the digitisation constant calibration.]{Gaussian fit to sum of the ECal calorimeter hit energies for 10~GeV $\gamma$ events with selection cuts.  The coarse binning reflects the tolerance on the digitisation constant calibration.}
\label{fig:ecaldigifit}
\end{figure}

%========================================================================================

\subsubsection{HCal Digitisation Implementation}
\label{sec:hcaldigi}
The calibration for the digitisation in the HCal proceeds in a similar manor to that described for the ECal with a few key differences.  This calibration uses $K^{0}_{L}$ events at $E_{MC} = 20$~GeV as these neutral hadrons will deposit the bulk of their energy in the HCal.  The higher energy, with respect to the ECal digitisation, is used to create larger particle showers that sample deeper into the HCal.  As the $K^{0}_{L}$s must pass through the ECal before arriving at the HCal and, as the ECal contains $\approx 1 \lambda_{I}$, some of the $K^{0}_{L}$s begin showering in the ECal, as shown by figure \ref{fig:hcaldigikaonsplit}.  Such events are unsuitable for calibration of the HCal digitisation constants as rescaling $\alpha^{0}_{\text{HCal}}$ would not lead to a linear change in $E_{\text{Fit}}$.  These events are vetoed in the even selection by requiring events deposit less than less than 5\% of $E_{MC}$ outside of the HCal.  In addition to this, the last layer of the HCal where energy is deposited is required to be in the innermost 90\% of the HCal.  This cut vetoes events that shower late in the HCal and deposit a significant amount of energy in the uninstrumented coil region of the detector.  The impact of these cuts on the sum of HCal calorimeter hit energies for the $E_{MC} = 20$~GeV $K^{0}_{L}$ events is shown in figure \ref{fig:hcaldigiselection}.  There are two HCal digitisation constants used in the detector simulation, one applied for the barrel and another for the endcap.  This is to account for differences in hadronic shower dynamics between the two, such as differing magnetic field configurations in the barrel and endcap.  Both parameters are calibrated in the same manor, but have different cuts on $\theta$, the polar angle of the $K^{0}_{L}$.  For the barrel region of the HCal events are selected if $0.2 < \text{cos}(\theta) < 0.6$, while for the endcap events are selected if $0.8 < \text{cos}(\theta) < 0.9$.  These angular cuts are conservative to account for the transverse profile of the hadronic showers and ensure that they are confined to the relevant sub-detector.  One further difference to the ECal digitisation procedure is that the target reconstructed energy for the $K^{0}_{L}$ samples is the kinetic energy as opposed to the total energy.  As the majority of the neutral hadrons appearing in jets are neutrons and their accessible energy is their kinetic energy, calibrating to the kinetic energy should give the best performance for jet reconstruction.  

\begin{figure}[h!]
\subfloat[]{\label{fig:hcaldigikaonsplit}\includegraphics[width=0.5\textwidth]{Calibration/Plots/Calibration/Digitsation/HCal/ECalHCalKaon0LSplit.pdf}}
\subfloat[]{\label{fig:hcaldigiselection}\includegraphics[width=0.5\textwidth]{Calibration/Plots/Calibration/Digitsation/HCal/DigitisationHCalSelection.pdf}}
\caption[\protect\subref{fig:hcaldigikaonsplit} Sum of calorimeter hit energies in ECal and HCal for 20~GeV $K^{0}_{L}$ events.  \protect\subref{fig:hcaldigiselection} Sum of the HCal calorimeter hit energies for a 20~GeV $K^{0}_{L}$ events with and without the selection cuts.]{\protect\subref{fig:hcaldigikaonsplit} Sum of calorimeter hit energies in ECal and HCal for 20~GeV $K^{0}_{L}$ events.  \protect\subref{fig:hcaldigiselection} Sum of the HCal calorimeter hit energies for a 20~GeV $K^{0}_{L}$ events with and without the selection cuts.}
\label{fig:hcaldigi}
\end{figure}

Using these cuts the calibration procedure for the digitisation of the HCal barrel and endcap proceeds in the same manor as was described for the ECal, the details of which can be found in section \ref{sec:ecaldigi}.  Examples of the Gaussian fits applied to the sum of the calorimeter hit energies in the HCal barrel and endcap can be found in figure \ref{fig:hcaldigifit}.  

\begin{figure}[h!]
\subfloat[HCal barrel.]{\label{fig:hcaldigibarrel}\includegraphics[width=0.5\textwidth]{Calibration/Plots/Calibration/Digitsation/HCal/DigitisationHCalbarrelFit.pdf}}
\subfloat[HCal endcap.]{\label{fig:hcaldigiendcap}\includegraphics[width=0.5\textwidth]{Calibration/Plots/Calibration/Digitsation/HCal/DigitisationHCalendcapFit.pdf}}
\caption[Gaussian fit to sum of the HCal calorimeter hit energies for 20~GeV $K^{0}_{L}$ events with selection cuts.]{Gaussian fit to sum of the HCal calorimeter hit energies for 20~GeV $K^{0}_{L}$ events with selection cuts.}
\label{fig:hcaldigifit}
\end{figure}

%========================================================================================

\subsubsection{HCal Ring Digitisation Implementation}
\label{sec:hcalringdigi}
The HCal ring, illustrated in figure \ref{fig:calorimeters}, also has an independent digitisation constant to account for any difference in the hadronic shower development between the ring, barrel and endcap.  The procedure used to calibrate this constant has to differs from that presented in section \ref{sec:hcaldigi} as it is unfeasible, due to the depth of the ring, to produce events that are wholly contained within it.  Fortunately, the size of the HCal ring means it plays a minimal role in the reconstruction, so precise calibration is not crucial.  To ensure that the calibration is approximately correct for the HCal ring, $\alpha_{\text{HCal ring}}$ is assumed to equal $\alpha_{\text{HCal endcap}}$ multiplied by several factors designed to accounts for changes in the active layer thickness, absorber layer thickness and the MIP response between the HCal endcap and ring.  In detail:
%
\begin{equation}
\alpha_{\text{HCal ring}} = \alpha_{\text{HCal endcap}} \times \frac{\langle \text{cos}(\theta_\text{endcap}) \rangle}{\langle \text{cos}(\theta_\text{ring}) \rangle} \times \frac{P_\text{endcap} }{P_\text{ring} } \times \frac{L^{Absorber}_\text{endcap}}{L^{Absorber}_\text{ring} } \times \frac{L^{Active}_\text{ring}}{L^{Active}_\text{endcap}} \text{ ,}
\end{equation}
%
\noindent where $\theta$ is the incident angle of the incoming particle to the calorimeter determined using the 20~GeV $K^{0}_{L}$ events, $L^{Active}$ is the active layer thickness and $L^{Absorber}$ is the absorber layer thickness. $P$ is the position of the MIP peak in the distribution of active layer hit energies, which has been corrected so that the MIP appears to enter the calorimeter at normal incidence, and is determined using 10~GeV $\mu^{-}$ events.  Details on how $P$ is determined can be found in section \ref{sec:mipresponse}.

\begin{figure}[h!]
\subfloat[]{\label{fig:ecal}\includegraphics[width=0.33\textwidth]{Calibration/Plots/Calibration/VisualDisplay/ECalWithScale.pdf}}
\subfloat[]{\label{fig:hcal}\includegraphics[width=0.33\textwidth]{Calibration/Plots/Calibration/VisualDisplay/HCal.png}}
\subfloat[]{\label{fig:hcalring}\includegraphics[width=0.33\textwidth]{Calibration/Plots/Calibration/VisualDisplay/HCalring.png}}
\caption[A PandoraPFA event display showing the nominal ILD calorimeters.  \protect\subref{fig:ecal} the ECal, \protect\subref{fig:hcal} the full HCal and \protect\subref{fig:hcalring} the HCal ring.]{A PandoraPFA event display showing the nominal ILD calorimeters.  \protect\subref{fig:ecal} the ECal, \protect\subref{fig:hcal} the full HCal and \protect\subref{fig:hcalring} the HCal ring.}
\label{fig:calorimeters}
\end{figure}

%========================================================================================

\subsection{MIP Scale Determination}
\label{sec:mipresponse}
The digitiser MIP scale was defined as the, non-zero, peak in the distribution of the active layer calorimeter hit energies for normally incident 10~GeV $\mu^{-}$ \cite{Bichsel:2004ej}, as shown in figure \ref{fig:digitisermip}.  This distribution is made for each calorimeter where the MIP scale needs to be determined.  As the average energy deposited per hit in a given sub-detector is relevant for setting the MIP scale, as opposed to the total energy deposited in a sub-detector, no selection cuts are required.  When populating the active layer energy distribution, a direction correction factor of $\text{cos}(\theta)$, where $\theta$ is the incident angle of the $\mu^{-}$ to the calorimeter hit, was applied to the hit energy to generate the effect of having the $\mu^{-}$ enter the calorimeter at normal incidence.  The MIP scale was determined separately for the ECal, HCal barrel, HCal endcap and HCal ring, however, only a single HCal MIP scale, taken as the HCal barrel, was required by the digitisation processor.  The HCal endcap and ring MIP scales were calculated for the purposes of the HCal ring digitisation described in section \ref{sec:hcalringdigi}.  No MIP scale setting was required in the digitisation processor for the muon chamber.  

\begin{figure}[h!]
\subfloat[]{\label{fig:digitisermipecal}\includegraphics[width=0.5\textwidth]{Calibration/Plots/Calibration/MIPScale/Digitiser/MIPScaleDigitiserECal.pdf}}
\subfloat[]{\label{fig:digitisermiphcalbarrel}\includegraphics[width=0.5\textwidth]{Calibration/Plots/Calibration/MIPScale/Digitiser/MIPScaleDigitiserHCalbarrel.pdf}} \\
\subfloat[]{\label{fig:digitisermiphcalendcap}\includegraphics[width=0.5\textwidth]{Calibration/Plots/Calibration/MIPScale/Digitiser/MIPScaleDigitiserHCalendcap.pdf}}
\subfloat[]{\label{fig:digitisermiphcalring}\includegraphics[width=0.5\textwidth]{Calibration/Plots/Calibration/MIPScale/Digitiser/MIPScaleDigitiserHCalOther.pdf}}
\caption[The active layer calorimeter hit energy distributions for \protect\subref{fig:digitisermipecal} the ECal, \protect\subref{fig:digitisermiphcalbarrel} the HCal barrel, \protect\subref{fig:digitisermiphcalendcap} the HCal endcap and \protect\subref{fig:digitisermiphcalring} the HCal ring for 10~GeV $\mu^{-}$ events.]{The active layer calorimeter hit energy distributions for \protect\subref{fig:digitisermipecal} the ECal, \protect\subref{fig:digitisermiphcalbarrel} the HCal barrel, \protect\subref{fig:digitisermiphcalendcap} the HCal endcap and \protect\subref{fig:digitisermiphcalring} the HCal ring for 10~GeV $\mu^{-}$ events.}
\label{fig:digitisermip}
\end{figure}

A similar procedure was employed for calculation of the MIP peak in PandoraPFA.  The distribution used to set the MIP scale in PandoraPFA is the distribution of the calorimeter hit energies, i.e. the active and absorber layer energies, as opposed to the active layer energies used for setting the MIP scale in the digitisation processor.   Examples of the distributions used to set the MIP scale in PandoraPFA can be found in figure \ref{fig:pandoramip}.  There are few populated low calorimeter hit energy bins in this these distributions as cuts are applied in the digitiser on the active layer energy.  The double peak structure observed in the ECal calorimeter hit energy distribution is expected given the ECal absorber material thickness doubling in the back 10 layers of the ECal.  Further differences between the MIP scale setting in the digitiser and PandoraPFA are as follows: The MIP scale setting in PandoraPFA combines the HCal sub-detectors, the barrel, endcap and ring, together when creating the calorimeter hit energy distributions; PandoraPFA requires the MIP scale to be set in the muon chamber unlike the digitisation processor and, therefore, the muon chamber hit energy distribution must also be created.  

\begin{figure}[h!]
\subfloat[]{\label{fig:pandoramipecal}\includegraphics[width=0.5\textwidth]{Calibration/Plots/Calibration/MIPScale/PandoraPFA/MIPScalePandoraPFAECal.pdf}}
\subfloat[]{\label{fig:pandoramiphcal}\includegraphics[width=0.5\textwidth]{Calibration/Plots/Calibration/MIPScale/PandoraPFA/MIPScalePandoraPFAHCal.pdf}} \\
\subfloat[]{\label{fig:pandoramipmuon}\includegraphics[width=0.5\textwidth]{Calibration/Plots/Calibration/MIPScale/PandoraPFA/MIPScalePandoraPFAMuon.pdf}}
\caption[The calorimeter hit energy distributions for \protect\subref{fig:pandoramipecal} the ECal, \protect\subref{fig:pandoramiphcal} the HCal and \protect\subref{fig:pandoramipmuon} the muon chamber for 10~GeV $\mu^{-}$ events.]{The calorimeter hit energy distributions for \protect\subref{fig:pandoramipecal} the ECal, \protect\subref{fig:pandoramiphcal} the HCal and \protect\subref{fig:pandoramipmuon} the muon chamber for 10~GeV $\mu^{-}$ events.}
\label{fig:pandoramip}
\end{figure}

%========================================================================================

\subsection{Electromagnetic and Hadronic Scale Setting}
\label{sec:scalesetting}

%========================================================================================

\subsubsection{Electromagnetic scale setting}
\label{sec:emscalesetting}
The electromagnetic scale in the ECal, $\beta^{EM}_{ECal}$, is determined using $\gamma$ events at $E_{MC} = 10$~GeV.  $\gamma$ events are ideal for setting the electromagnetic scale as they procedure electromagnetic showers that are primarily confined to the ECal, which is shown by figure \ref{fig:ecaldigiphotonsplit}.  To ensure that the events used for this part of the calibration are largely confined to the ECal, a cut requiring less than 1\% of the reconstructed energy to be found outside the ECal is applied.  Furthermore, a cut requiring a single $\gamma$ be reconstructed are added to veto events with pattern recognition failures.  $\gamma$ conversions are excluded at MC level to ensure energy measurements used in the calibration arise from the calorimeters and not the charged particle tracks.  The impact of these cuts on the electromagnetic energy measured in the ECal for 10~GeV $\gamma$ events is shown in figure \ref{fig:ecalemscaleselection}.  In figure \ref{fig:ecalemscaleselection}, the peak with zero electromagnetic energy in the ECal is due to events traveling down the beam pipe and events with $\gamma$ conversions.  In $\gamma$ conversion events the reconstructed energy is taken from the $\text{e}^{\pm}$ tracks, which means the electromagnetic energy in the ECal will register as zero.  The tail of events with low electromagnetic energy in the ECal occurs primarily due pattern recognition failures in $\gamma$ conversion events.  In such cases, some of the calorimetric energy deposits are not associated to the $\text{e}^{\pm}$ tracks and instead form new $\gamma$s that contain only a small fraction of the true MC energy.  

The fitting procedure follows that used for the ECal digitisation, described in section \ref{sec:ecaldigi}, whereby a trial calibration for the electromagnetic energy scale in the ECal, $\beta^{EM0}_{ECal}$, is assumed and the single $\gamma$ events simulated.  The distribution of the electromagnetic energy in the ECal is created and a Gaussian fit applied to the range of data with the smallest root mean square containing at least 90 \% of the data.  The mean of this fit, $E_{\text{Fit}}$, is then used to scale $\beta^{EM0}_{ECal}$ in the following way:
%
\begin{equation}
\beta^{EM0}_{ECal} \rightarrow \beta^{EM}_{ECal} = \beta^{EM0}_{ECal} \times \frac{E_{MC}}{E_{Fit}}\text{ .}
\end{equation}
%
An example distribution and fit used in the calibration of the nominal ILD detector model can be found in figure \ref{fig:ecalemscalefit}.  This procedure is repeated using the updated $\beta^{EM}_{ECal}$ until $E_{\text{Fit}}$ falls within a specified tolerance.  The tolerance applied here was $|E_{\text{Fit}} - E_{\text{MC}}| < E_{\text{MC}} \times 0.5 \%$.  The binning for the fitted histogram is chosen such that the bin width is equal to the desired target tolerance on $E_{\text{Fit}}$ e.g. $E_{\text{MC}} \times 0.5 \% = 0.05$~GeV.  This tolerance is tighter than was applied for the digitisation as it is these energies that are used in downstream analyses.   
 
\begin{figure}[h!]
\subfloat[]{\label{fig:ecalemscaleselection}\includegraphics[width=0.5\textwidth]{Calibration/Plots/Calibration/EMScaleSetting/EMScaleECalSelection.pdf}}
\subfloat[]{\label{fig:ecalemscalefit}\includegraphics[width=0.5\textwidth]{Calibration/Plots/Calibration/EMScaleSetting/EMScaleSettingECalFit.pdf}}
\caption[\protect\subref{fig:ecalemscaleselection} The sum of the electromagnetic energy measured in the ECal for 10~GeV $\gamma$ events with and without the selection cuts.  \protect\subref{fig:ecalemscalefit} Gaussian fit to sum of the electromagnetic energy deposited in the ECal for 10~GeV $\gamma$ events with selection cuts.  The fine binning reflects the tolerance on the electromagnetic scale calibration constant in the ECal.]{\protect\subref{fig:ecalemscaleselection} The sum of the electromagnetic energy measured in the ECal for 10~GeV $\gamma$ events with and without the selection cuts.  \protect\subref{fig:ecalemscalefit} Gaussian fit to sum of the electromagnetic energy deposited in the ECal for 10~GeV $\gamma$ events with selection cuts.  The fine binning reflects the tolerance on the electromagnetic scale calibration constant in the ECal.}
\label{fig:ecalemscale}
\end{figure}
 
The electromagnetic scale in the HCal, $\beta^{EM}_{HCal}$, is chosen to be equal to the hadronic scale in the HCal, $\beta^{Had}_{HCal}$.  The details of the determination of $\beta^{Had}_{HCal}$ can be found in section \ref{sec:hadscalesetting}.  For the ILC and CLIC, $\beta^{EM}_{HCal}$ is not a critical parameter in the reconstruction as $\gamma$s are largely contained within the ECal meaning little to no electromagnetic energy is measured in the HCal.  

%========================================================================================

\subsubsection{Hadronic scale setting}
\label{sec:hadscalesetting}
The hadronic energy scale factors for the ECal and HCal, $\beta^{Had}_{ECal}$ and $\beta^{Had}_{HCal}$, are determined using $K^{0}_{L}$ events at $E_{MC} = 20$~GeV.  The hadronic scale in the ECal, $\beta^{Had}_{ECal}$, is important to detector performance as a non-negligible amount of hadronic energy will be measured in the ECal.  As the ECal contains $\approx 1 \lambda_{I}$, the hadronic scale in the ECal cannot be independently set as it is unfeasible to create a large sample of 20~GeV $K^{0}_{L}$ events that are fully contained within it.  Therefore, the hadronic scale in the ECal and HCal have to be set simultaneously.  

For the reasons outlined in section \ref{sec:hcaldigi}, the target reconstructed energy for this sample is the kinetic energy, $E_{K}$, of the $K^{0}_{L}$ as opposed to the total energy.  To ensure the events used are not affected by leakage of energy out of the back of the HCal, a cut is applied that vetoes events where energy is deposited in the outermost 10\% of the HCal.  In addition to this, a cut requiring a single neutral hadron to be reconstructed is applied to veto events with reconstruction failures.  Finally, it is required that the total hadronic energy measured within the calorimeters fall within three $\sigma$ of the kinetic energy of the $K^{0}_{L}$, where $\sigma$ is defined to be $55\% \times \sqrt{E_{K}}$~GeV.  This definition for $\sigma$ is approximately the energy resolution for neutral hadrons using the nominal ILD HCal \cite{Behnke:2013lya}.  This cut ensures that when fitting the two dimensional distribution of hadronic energy measured in the ECal and HCal outliers do not skew the fit.   The impact of cuts is illustrated in figure \ref{fig:hadscaleselection}.

\begin{figure}[h!]
\subfloat[]{\label{fig:hadscaleselectionnocuts}\includegraphics[width=0.5\textwidth]{Calibration/Plots/Calibration/HadScaleSetting/HadScaleECalHCalSelectionNoCuts.pdf}}
\subfloat[]{\label{fig:hadscaleselectioncuts}\includegraphics[width=0.5\textwidth]{Calibration/Plots/Calibration/HadScaleSetting/HadScaleECalHCalSelectionCuts.pdf}}
\caption[The distribution of hadronic energy measured in the ECal and HCal for 20~GeV $K^{0}_{L}$ events with and without selection cuts.]{The distribution of hadronic energy measured in the ECal and HCal for 20~GeV $K^{0}_{L}$ events \protect\subref{fig:hadscaleselectionnocuts} without selection cuts and \protect\subref{fig:hadscaleselectioncuts} with selection cuts.}
\label{fig:hadscaleselection}
\end{figure}

This part of the calibration procedure is again iterative and begins by assuming trial values, $\beta^{Had0}_{ECal}$ and $\beta^{Had0}_{HCal}$, for the hadronic scale calibration factors $\beta^{Had}_{ECal}$ and $\beta^{Had}_{HCal}$.  Following this the 20~GeV $K^{0}_{L}$ events are simulated and reconstructed using these scale factors.  Then a linear fit is applied to the two dimensional distribution of the reconstructed hadronic energies measured in the ECal and HCal for events passing the selection cuts.  The best fit is obtained by minimising $\chi^{2}$ with respect to variables describing a linear fit to the distribution.  In this case, $\chi^{2}$ is defined as:
%
\begin{equation}
\chi^{2}(\delta^{Had}_{ECal}, \delta^{Had}_{HCal}) = \sum_{i} \frac{r_{i}}{\sigma_{r_{i}}}\text{ ,}
\end{equation}
%
\noindent where $r_{i}$ is the perpendicular distance in the two dimensional plane of hadronic energies measured in the ECal and HCal from the point $(x_{i}, y_{i})$ to a straight line passing through the points $(\delta^{Had}_{ECal}, 0)$ and $(0, \delta^{Had}_{HCal})$.  In this definition, $x_{i}$ and $y_{i}$ are the hadronic energies measured in the ECal and HCal respectively for event $i$.  The variables $\delta^{Had}_{ECal}$ and $\delta^{Had}_{HCal}$ describe a linear fit to the hadronic energy distribution, which are to be varied when minimising $\chi^{2}$.  The explicit definition of $r_{i}$ is given in equation \ref{equ:xicalc}, however it is best illustrated by considering figure \ref{fig:hadscalechi2calc}.  The uncertainty on $r_{i}$ is given by $\sigma_{r_{i}}$, which is explicitly defined in equation \ref{equ:sigmaxicalc}.  This uncertainty is calculated by propagating the uncertainties on $x_{i}$ and $y_{i}$, which are assumed to be $\sigma_{x_{i}/y_{i}} = 55\% \times \sqrt{x_{i}/y_{i}}$, into the expression for $r_{i}$.  The sum runs over all events, $i$, passing the selection cuts.  
%
\begin{equation}
r_{i} = \frac{y_{i} \delta^{Had}_{ECal} + x_{i} \delta^{Had}_{HCal} - \delta^{Had}_{ECal} \delta^{Had}_{HCal}}{\sqrt{(\delta^{Had}_{ECal})^{2} + (\delta^{Had}_{HCal})^{2}}}\text{ ,}
\label{equ:xicalc}
\end{equation}
\begin{equation}
\sigma_{i} = \frac{(\sigma_{y_{i}}  \delta^{Had}_{ECal})^{2} + (\sigma_{x_{i}} \delta^{Had}_{HCal})^{2}}{\sqrt{(\delta^{Had}_{ECal})^{2} + (\delta^{Had}_{HCal})^{2}}}\text{ ,}
\label{equ:sigmaxicalc}
\end{equation}
%
\begin{figure}[h!]
\includegraphics[width=0.5\textwidth]{Calibration/Plots/Calibration/HadScaleSetting/HadScaleECalHCalSelectionExample.pdf}
\caption[An example showing the definition of $r_{i}$, the variable used for the calculation of $\chi^{2}(\delta^{Had}_{ECal}, \delta^{Had}_{HCal})$ in determining the hadronic energy scale factors.]{An example showing the definition of $r_{i}$, the variable used for the calculation of $\chi^{2}(\delta^{Had}_{ECal}, \delta^{Had}_{HCal})$ in determining the hadronic energy scale factors.  For an event that has been measured with hadronic energy $x_{i}$ in the ECal and $y_{i}$ in the HCal, the geometric interpretation of $r_{i}$ is shown.  The blue dotted line is defined as $y_{i} = \delta^{Had}_{HCal} - x_{i} \frac{\delta^{Had}_{HCal}}{\delta^{Had}_{ECal}}$.}
\label{fig:hadscalechi2calc}
\end{figure}
%
\noindent The minimisation of $\chi^{2}$ is done by stepping over a range of $\delta^{Had}_{ECal}$ and $\delta^{Had}_{HCal}$ centred about the ideal value of $E_{K}$ in search for the minimum $\chi^{2}$.  Once the minima in $\chi^{2}$ is found the trial calibration factors $\beta^{Had0}_{ECal}$ and $\beta^{Had0}_{ECal}$ are rescaled to correct for any deviation from the desired fit as follows:
%
\begin{equation}
\beta^{Had0}_{ECal} \rightarrow \beta^{Had}_{ECal} = \beta^{Had0}_{ECal} \times \frac{E_{K}}{\Delta^{Had}_{ECal}} \text{ ,}\\
\beta^{Had0}_{HCal} \rightarrow \beta^{Had}_{HCal} = \beta^{Had0}_{HCal} \times \frac{E_{K}}{\Delta^{Had}_{HCal}}\text{ ,}
\end{equation}
%
\noindent where $\Delta^{Had}_{ECal}$ and $\Delta^{Had}_{ECal}$ are the values of $\delta^{Had}_{ECal}$ and $\delta^{Had}_{ECal}$ giving the minimum $\chi^{2}$.  The step size used for minimising $\chi^{2}$ with respect to $\delta^{Had}_{ECal}$ and $\delta^{Had}_{ECal}$ was chosen such that a single step would correspond to the final tolerance on $\delta^{Had}$, which in this case is $\approx$ 0.1~GeV.  This procedure is then repeated using the updated hadronic scaling factors until $\Delta^{Had}_{ECal}$ and $\Delta^{Had}_{ECal}$ both fall within a specified final tolerance, which in this case it taken to be $|\Delta^{Had}_{E/HCal} - E_{\text{{K}}}| < E_{\text{{K}}} \times 0.5 \% \approx 0.1 \text{GeV}$.

%========================================================================================
%========================================================================================

%% For normal draft builds (figs undisplayed hence fast compile)
%\documentclass[hyperpdf,nobind,draft,oneside]{hepthesis}
%\documentclass[hyperpdf,nobind,draft,twoside]{hepthesis}

%% For short draft builds (breaks citations by necessity)
%\documentclass[hyperpdf,nobind,draft,hidefrontback]{hepthesis}

%%For Cambridge soft-bound version
\documentclass[hyperpdf,bindnopdf]{hepthesis}
%% For Cambridge hard-bound version (must be one-sided)
%\documentclass[hyperpdf,oneside]{hepthesis}

%% Load special font packages here if you wish
%\usepackage{lmodern}
\usepackage{mathpazo}
\usepackage{amssymb}
%\usepackage{euler}

%% Put package includes etc. into preamble.tex for convenience
\input{preamble}

%% You can set the line spacing this way
%\setallspacing{double}
%% or a section at a time like this
%\setfrontmatterspacing{double}


%% Define the thesis title and author
\title{Calorimetry at a Future Linear Collider}
\author{Steven Green}

%% Doc-specific PDF metadata
\makeatletter
\@ifpackageloaded{hyperref}{%
\hypersetup{%
  pdftitle = {Calorimetry at a Future Linear Collider},
  pdfsubject = {Steven Green's PhD thesis},
  pdfkeywords = {Linear Collider, CLIC, ILC, ILD},
  pdfauthor = {\textcopyright\ Steven Green}
}}{}
\makeatother

%% Start the document
\begin{document}

%% Define the un-numbered front matter (cover pages, rubrik and table of contents)
\begin{frontmatter}
  %% Title
\titlepage[of Emmanuel College]{%
  A dissertation submitted to the University of Cambridge\\ for the degree of Doctor of Philosophy}

%% Abstract
\begin{abstract}%[\smaller \thetitle\\ \vspace*{1cm} \smaller {\theauthor}]
  %\thispagestyle{empty}
This thesis describes the optimisation of the calorimeter design for collider experiments
at the future Compact LInear Collider (CLIC) and the International Linear Collider (ILC). 
The detector design of these experiments is built around high-granularity Particle Flow Calorimetry
that, in contrast to traditional calorimetry, uses the energy measurements for charged particles from 
the tracking detectors. This can only be realised if calorimetric energy deposits from
charged particles can be separated from those of neutral particles. This is made possible with fine
granularity calorimeters and sophisticated pattern recognition software, which is provided by the
PandoraPFA algorithm. This thesis presents results on Particle Flow calorimetry performance for a number
of detector configurations. To obtain these results a new calibration procedure was developed 
and applied to the detector simulation and reconstruction to ensure optimal performance 
was achieved for each detector configuration considered.

This thesis also describes the development of a software compensation technique that vastly improves the
intrinsic energy resolution of a Particle Flow Calorimetry detector. This technique is implemented
within the PandoraPFA framework and demonstrates the gains that can be made by fully exploiting
the information provided by the fine granularity calorimeters envisaged at a future linear collider.

A study of the sensitivity of the CLIC experiment to anomalous gauge couplings that
effect vector boson scattering processes is presented. These anomalous couplings provide insight
into possible beyond standard model physics. This study, which utilises the excellent jet energy resolution from
Particle Flow Calorimetry, was performed at centre-of-mass energies of 1.4 TeV and 3 TeV with integrated 
luminosities of 1.5$\text{ab}^{-1}$ and 2$\text{ab}^{-1}$ respectively. The precision achievable at CLIC is shown to be approximately 
one to two orders of magnitude better than that currently offered by the LHC.

In addition, a study into various technology options for the CLIC vertex detector is described.
\end{abstract}


%% Declaration
\begin{declaration}
  This dissertation is the result of my own work, except where explicit
  reference is made to the work of others, and has not been submitted
  for another qualification to this or any other university. This
  dissertation does not exceed the word limit for the respective Degree
  Committee.
  \vspace*{1cm}
  \begin{flushright}
    Steven Green
  \end{flushright}
\end{declaration}


%% Acknowledgements
\begin{acknowledgements}
  Of the many people who deserve thanks, some are particularly prominent,
  such as my supervisor\dots
\end{acknowledgements}


%% Preface
%\begin{preface}
%  This thesis describes my research on various aspects of the \LHCb
%  particle physics program, centred around the \LHCb detector and \LHC
%  accelerator at \CERN in Geneva.

%  \noindent
%  For this example, I'll just mention \ChapterRef{chap:SomeStuff}
%  and \ChapterRef{chap:MoreStuff}.
%\end{preface}

%% ToC
\tableofcontents


%% Strictly optional!
\frontquote{%
  Writing in English is the most ingenious torture\\
  ever devised for sins committed in previous lives.}%
  {James Joyce}
%% I don't want a page number on the following blank page either.
\thispagestyle{empty}

\end{frontmatter}

%% Start the content body of the thesis
\begin{mainmatter}
  %% Introudction 
% \chapter{Introduction}
\label{chap:introduction}

\chapterquote{There, sir! that is the perfection of vessels!}
{Jules Verne, 1828--1905}

%========================================================================================

The Standard Model has proven to be one of the greatest accomplishments of modern day particle physics.  It has have been used to make countless predictions of various physics processes across a wide range of energies that have proven to be consistent with experimental measurements.  The final piece of the Standard Model to be discovered was the Higgs boson, which was found by the ATLAS \cite{Aad:2012tfa} and CMS \cite{Chatrchyan:2012xdj} experiments at the Large Hadron Collider (LHC) in 2012.  

Despite the remarkable descriptive power of the Standard Model, there are a number of features in the universe that it does not provide a description for.  How does gravity fit into the Standard Model?  Why is there an excess of matter over antimatter in the observable universe?  How does the "dark matter" predicted by astronomers couple with the particles in the Standard Model?  What are the properties of the Higgs field in the Standard Model?  While the LHC and previous generations of particle collider experiment have had enormous success in validating the Standard Model and searching for new physics, it is clear that there is more work to be done. 

The linear collider experiments are proposals for the next generation of particle collider experiment.  These experiments are $\text{e}^{+}\text{e}^{-}$ colliders with a focus on precision measurements.  The physics program for the linear collider is designed to complement and extend the work done at the LHC and to develop our understanding of particle physics.  One of the primary goals of the linear collider experiments is to study the Higgs field of the Standard Model.  A detailed description of the Higgs field is likely to help in the description of "dark matter" as many extensions of the Standard Model Higgs field contain particles that fit the properties of "dark matter".  The linear collider experiments will also provide a detailed description of the properties of the top quark.  This will complement the Higgs study as the strongest couplings for the Higgs in the Standard Model occurs with the top quark.  Another goal of the linear collider experiments is to provide high precision measurements of the electroweak sector in the Standard Model.  As the electroweak sector is the only place in the Standard Model where CP violation can occur, a detailed description will help to determine why there is an excess of matter over antimatter in the universe.  Furthermore, the linear collider will expand the descriptive reach for many Standard Model extensions such as supersymmetry (SUSY).

The linear collider experiments place emphasis on precision measurements.  As well as searching for beyond Standard Model physics, precision measurements will guide the future direction of experimental particle physics.  Precision measurements have helped to guide the course of particle physics experiments in the past; LEP electroweak data, which gave indirect information about the lightness of the Higgs boson, was used to build the physics case for the LHC.  By colliding $\text{e}^{+}\text{e}^{-}$, the experimental conditions found at the linear collider will be far cleaner than those at the LHC, which makes it easier to perform precision measurements.  High precision measurements are made possible at the linear collider due to the use of particle flow calorimetry, which is a revolutionary technique in detector design that offers exceptional energy resolution for jets.  This paradigm shift means the linear collider detectors are significantly different from those found in previous generations of particle colliders.  As the detector design is continually evolving, the ongoing research in this area is vital for determining the overall success of these experiments.  

This thesis is organised as follows.  Chapter \ref{chap:anomalousgaugecouplingtheory} contains a summary of the Standard Model as well as an outline of the physics of interest related to the analysis presented in chapter \ref{chap:PhysicsAnalysis}.  Chapter \ref{chap:clicvertex} presents a study into a novel technology option for the Compact LInear Collider (CLIC) vertex detector.  Chapter \ref{chap:energyestimators} contains numerous studies related to the treatment of energy deposits in the linear collider simulation.  This begins with an outline of the calibration procedure for the linear collider detector simulation.  This is followed by a number of novel software techniques aimed at improving the energy resolution of a calorimeter designed for particle flow calorimetry.  Finally, the chapter concludes with a study of the timing requirements applied in the software trigger that will be used at the linear collider experiment.  Chapter \ref{chap:detopt} presents an optimisation study of the linear collider calorimeters.  The starkest contract in detector design when comparing particle flow calorimeter and tradition calorimeter is the design of the calorimeters.  As the linear collider experiments will be the first experiments purposefully built with particle flow calorimetry in mind, there is no precedence for the detector design.  Therefore, these studies are vital for guiding detector design.  Chapter \ref{chap:PhysicsAnalysis} contains a study into anomalous gauge couplings that are sensitive to massive gauge boson quartic vertices at the CLIC experiment.  This study is of particular interest as it provides a detailed probe of the electroweak symmetry breaking sector of the Standard Model as well as showing CLICs sensitivity to a possible extension to the Standard Model.  The thesis concludes with a summary in chapter \ref{chap:summary}.

%========================================================================================

\section{Future Linear Colliders}
There are two proposed future linear collider experiments; the International Linear Collider (ILC) and the Compact LInear Collider (CLIC).  These colliders are both $\text{e}^{+}\text{e}^{-}$ colliders with focus upon precision measurements, however, they operate at different collision energies, which presents each experiment with its own unique challenges.  One benefit of a linear collider is that it is possible to stage the experiment at several different energies throughout the experiments lifetime

%========================================================================================

\subsection{The International Linear Collider}
The ILC, shown in figure \ref{fig:ilc}, initially plans to operate at a centre-of-mass energy of 250~GeV to study the Higgs boson in detail through the Higgstrahlung process ($\text{e}^{+}\text{e}^{-} \rightarrow ZH$).  This analysis of this process makes it possible to examine all the decays of the Higgs boson with high precision.  The next phase of operation will increase the collision energy to 500~GeV.  This will extend the study of the Higgs, making it possible to observe the Higgs coupling with the top quark and to determine self interactions of the Higgs.  Furthermore, at this energy, it will be possible to search for evidence for SUSY and extended Higgs states.  Finally, there is an option to increase the centre-of-mass up to 1~TeV, which would extend the search for SUSY and composite Higgs models.

\begin{figure}[h!]
\includegraphics[width=1.0\textwidth]{Introduction/Plots/ILC.jpg}
\caption[Schematic layout of the ILC, indicating all the major subsystems (not to scale).  Figure taken from \cite{Behnke:2013xla}.]{Schematic layout of the ILC, indicating all the major subsystems (not to scale).  Figure taken from \cite{Behnke:2013xla}.}
\label{fig:ilc}
\end{figure}

%========================================================================================

\subsection{The Compact Linear Collider}
The CLIC experiment, shown in figure \ref{fig:clic}, plans to operate with maximum collision energy of 3~TeV.  CLIC will also operate at intermediate energy stages, however, these energies are to be determined by the ongoing work at the LHC.  The large collision energy of CLIC gives it a greater physics reach for searching for extensions to the Standard Model, e.g. SUSY, that would be inaccessible at ILC-like energies \cite{Linssen:2012hp}.  Although the exact energies for the staging of the CLIC experiment are not certain, CLIC will operate at a low collision energy during staging, $\sim 500$~GeV, to study the Higgs.  The higher energy stages of the CLIC experiment will provide access to different channels for studying Higgs couplings, as shown by figure \ref{fig:higssprodclic}.

\begin{figure}[h!]
\includegraphics[width=1.0\textwidth]{Introduction/Plots/CLIC.jpg}
\caption[CLIC layout at 3~TeV.  Figure taken from \cite{Aicheler:2012bya}.]{CLIC layout at 3~TeV.  Figure taken from \cite{Aicheler:2012bya}.}
\label{fig:clic}
\end{figure}

\begin{figure}[h!]
\includegraphics[width=0.5\textwidth]{Introduction/Plots/CDRPlots/HiggsCrossSectionCLIC.pdf}
\caption[Cross section for production mechanisms of the Standard Model Higgs boson as a function of the collision energy.  The cross sections were calculated assuming a Higgs mass of 120~GeV.  Figure taken from \cite{Linssen:2012hp}.]{Cross section for production mechanisms of the Standard Model Higgs boson as a function of the collision energy.  The cross sections were calculated assuming a Higgs mass of 120~GeV.  Figure taken from \cite{Linssen:2012hp}.}
\label{fig:higssprodclic}
\end{figure}

%========================================================================================

\subsubsection{Experimental Conditions at CLIC}
The CLIC experiment will operate in a unique environment in comparison to either the ILC or previous generations of lepton colliders.  It is vital that this is properly accounted for when determining the physics potential that CLIC has to offer.  The following aspects of the CLIC experiment present the largest challenges to the physics potential:

\begin{itemize}
\item The high bunch charge density.  The small beam size at the impact point produces very large electromagnetic fields.  These fields can interact with the opposite beam particles causing them to radiate photons in an effect known as beamstrahlung.  Beamstrahlung acts to reduce the collision energy of the $\text{e}^{+}\text{e}^{-}$ pairs.   
\item Beam related backgrounds.  Beamstrahlung photons can subsequently interact to produce background events that must be accounted for.  Dominant backgrounds of this form that cannot be easily vetoed in the reconstruction include incoherent pair production of $\text{e}^{+}\text{e}^{-}$ and $\gamma\gamma \rightarrow \text{Hadron}$.  While these backgrounds are also problematic for the ILC experiment, the lower collision energy means it is has a much smaller impact on performance.
\item Fast readout technology.  The CLIC bunch train consists of 312 bunches with a repetition rate of 50~Hz.  Each bunch is separated by 0.5~ns, therefore, it will be necessary to integrate over multiple bunch crossing when reading out the detectors.  This places tight constraints on all detector electrical readout speeds and time resolutions.   
\end{itemize}

%========================================================================================

\subsubsection{Beam-Related Backgrounds at CLIC}
\label{sec:beamrelatedbackgrounds}
The primary sources of background for the CLIC experiment are as follows:
\begin{itemize}
\item $\text{e}^{+}\text{e}^{-}$ pair creation from the interaction of a beamstrahlung photons with the opposing beam.  The different mechanisms for pair creation are as follows:
\begin{itemize}
\item \textbf{Coherent pair production}.  This mechanism involves the interaction of a real beamstrahlung photon with the electromagnetic field from the opposing beam.
\item \textbf{Trident pair production}.  This mechanism involves the interaction of a virtual beamstrahlung photon with the electromagnetic field from the opposing beam.
\item \textbf{Incoherent pair production}.  This mechanism involves the interaction of a real or virtual beamstrahlung photon with the individual particles in the opposing beam.
\end{itemize}
\item $\gamma\gamma \rightarrow \text{Hadron}$ events from the interaction of real or virtual beamstrahlung photons with each other.  
\item Beam halo muons that arise from interactions of the beam particles during collimation.  The dominant mechanisms producing beam halo muons are photon conversions into muon pairs ($\gamma \text{e}^{-} \rightarrow \mu^{+}\mu^{-}\text{e}^{-}$) and annihilation of positrons with atomic $\text{e}^{-}$ into muon pairs ($\text{e}^{+}\text{e}^{-} \rightarrow \mu^{+}\mu^{-}$) \cite{Pilicer:2015ijy}.
\end{itemize}
Each of these has to be properly addressed to get a true measure of the physics potential at CLIC.  Coherent and trident pair production is not a dominant source of background as they are produced at low transverse momenta, as figure \ref{fig:backgroundangle} shows, and a simple cut would veto these backgrounds.  This is not the case for incoherent pair production of $\text{e}^{+}\text{e}^{-}$, which are dominant in the forward regions of the detector, and $\gamma\gamma \rightarrow \text{Hadron}$, which are dominant in the tracker and the calorimeters (with the exception of low radii in the calorimeter endcaps) \cite{Linssen:2012hp, Sailer:2012mfa}.  Beam halo muons are not a major source of background either as they can be easily removed during the reconstruction as they produce a clear signal in the detector.  An algorithm was developed within the PandoraPFA framework for this purpose and it was found to be highly effective at removing the beam halo muons background \cite{Linssen:2012hp}.  

\begin{figure}[h!]
\includegraphics[width=0.5\textwidth]{Introduction/Plots/CDRPlots/BackgroundAngleCut.pdf}
\caption[Angular distribution of number of particles for beam induced backgrounds for CLIC at a centre of mass energy of 3~TeV.  Figure taken from \cite{Linssen:2012hp}.]{Angular distribution of number of particles for beam induced backgrounds for CLIC at a centre of mass energy of 3~TeV.  Figure taken from \cite{Linssen:2012hp}.}
\label{fig:backgroundangle}
\end{figure}

$\gamma\gamma \rightarrow \text{Hadron}$ events are the most dominant source of background to consider at CLIC as these events deposit more energy throughout the detector than incoherent pair production of $\text{e}^{+}\text{e}^{-}$ events \cite{Linssen:2012hp}.  Although incoherent pairs are still a source of background, they will produce a second order effect in comparison to the $\gamma\gamma \rightarrow \text{Hadron}$ events.

%========================================================================================
%========================================================================================

%Physics introduction leading to chapter description.
%Max 2-3 pages
%Standard model extremely successful, missing gravity though
%Higgs discovery added crucial piece for mass generation
%Properties missing:
%Dark matter coupling
%CP violation -> Matter > antimatter
%ILC designed to study Higgs at 125 GeV, e+e-->Zh peak cross section at 250 GeV allows decay of Higgs to be measured by recoil of Z
%Top quark mass also studies.  Heaviest particle so coupling to H will be strong.
%Ultra-Precision for EW sector, which is only known CP violation 
%CLIC EW symmetry breaking at TeV scale
%and SUSY searches
%Improvement to LHC
%Precision Quantitative improvement of what is know , Jump in physics e.g. GUT in SUSY only proposed from precision EW  Lightness of Higgs from LEP electroweak data
%Discovery reach from processes with low production cross section at LHC
%Precision from PFlow
%Chapter Vertex
%Chapter Energy Est
%Chapter Calo Opt
%Chapter Physics Analysis
%========================================================================================
  %% Theory 
% \chapter{Anomalous Gauge Coupling Theory}
\label{chap:anomalousgaugecouplingtheory}

\chapterquote{"Meaningless! Meaningless!" says the Teacher.  "Utterly meaningless! Everything is meaningless."}
{Ecclesiastes 1:2}

Presented in chapter \ref{chap:PhysicsAnalysis} is an analysis of the sensitivity of the CLIC experiment to the anomalous gauge couplings $\alpha_{4}$ and $\alpha_{5}$ through the vector boson scattering process.  Here, a brief description of the Standard Model of particle physics and a deeper discussion of the anomalous coupling theory studied in chapter \ref{chap:PhysicsAnalysis} is given.  

%========================================================================================
%========================================================================================
%
\section{The Standard Model}
The Standard Model is a non-abelian gauge theory of the $\text{SU}(3) \times \text{SU}(2)_{\text{L}} \times \text{U}(1)$ symmetry group.  It provides a description of three of the four fundamental forces of nature: the electromagnetic, weak and strong nuclear forces \cite{Griffiths:1987tj,Peskin:1995ev}.  The Standard Model contains a total of 24 fermion fields: six flavours of quark each with three colours and six leptons.  A summary of the properties of these particles is given in table \ref{table:smleptons} and \ref{table:smquarks}.  As these fields, $\psi$, are spin-$\frac{1}{2}$, they obey the Dirac equation
%
\begin{equation}
\mathcal{L} = \overline{\psi}(i \slashed{\partial} - m)\psi \text{ ,}
\end{equation}
%
\noindent where $\mathcal{L}$ is the Lagrangian density and $m$ is a mass term.  The derivative term, $\slashed{\partial} = \gamma^{\mu}\partial_{\mu}$, represents a summation over the partial \textcolor{blue}{derivative}, $\partial^{\mu} = (\frac{\partial}{\partial{t}},\frac{\partial}{\partial{x}},\frac{\partial}{\partial{y}},\frac{\partial}{\partial{z}})$, of the field $\psi$ and the gamma matrices, $\gamma^{\mu}$.  Each of the gauge transformations of the Standard Model are defined by a unitary operator \textrm{U}, which acts to transform the vector space, $\Psi$, formed from a combination of fermion fields, $\psi$, in the following way
%
\begin{equation}
\Psi \rightarrow \Psi' = \textrm{U}\Psi \text{ .}
\end{equation}
%
\begin{table}[h!]
\centering
\begin{tabular}{l l r r r}
\hline
Generation & Particle & Mass [MeV] & Spin & \textcolor{blue}{\textit{Q}}/\textit{e} \\
\hline
1 & $e^{-}$ & $0.548579909070\pm0.000000000016$ & 1/2 & $-1$ \\
& $\nu_{e}$ & - & 1/2 & 0 \\
\hline
2 & $\mu^{-}$ & $105.6583745\pm0.0000024$ & 1/2 & $-1$ \\
& $\nu_{\mu}$ & - & 1/2 & 0 \\
\hline
3 & $\tau^{-}$ & $1776.86\pm0.12$ & 1/2 & $-1$ \\
& $\nu_{\tau}$ & - & 1/2 & 0 \\
\end{tabular}
\caption[The mass, spin and electric charge (\textcolor{blue}{\textit{Q}}) of the leptons found in the Standard Model \cite{Beringer:1900zz}.  Neutrino masses have not been included in the above table as precise measurements are yet to be made.  However, oscillations between different neutrino flavour states have been observed, which indicates that the flavour and mass eigenstates differ and that the neutrinos have a non-zero mass.  The current upper bound on neutrino mass measurements is 2 eV.]{The mass, spin and electric charge (\textcolor{blue}{\textit{Q}}) of the leptons found in the Standard Model \cite{Beringer:1900zz}.  Neutrino masses have not been included in the above table as precise measurements are yet to be made.  However, oscillations between different neutrino flavour states have been observed, which indicates that the flavour and mass eigenstates differ and that the neutrinos have a non-zero mass.  The current upper bound on neutrino mass measurements is 2 eV.}
\label{table:smleptons}
\end{table}
%
\begin{table}[h!]
\centering
\begin{tabular}{l l r r r}
\hline
Generation & Particle & Mass [MeV] & Spin & \textcolor{blue}{\textit{Q}}/\textit{e} \\
\hline
1 & $u$ & $2.2^{+0.6}_{-0.4}$ & 1/2 & $+2$/3 \\
 & $d$ & $4.7^{+0.5}_{-0.4}$ & 1/2 & $-1$/3 \\
\hline
2 & $c$ & $1270\pm30$ & 1/2 & $+2$/3 \\
 & $s$ & $98^{+8}_{-4}$ & 1/2 & $+2$/3 \\
\hline
3 & $t$ & $173210 \pm 510 \pm 710$ & 1/2 & $+2$/3 \\
 & $b$ & $4180^{+40}_{-30}$ & 1/2 & $-1$/3 \\
\end{tabular}
\caption[The mass, spin and electric charge (\textcolor{blue}{\textit{Q}}) of the quarks found in the Standard Model \cite{Beringer:1900zz}.  Each of the particles in the above table corresponds to three fermion fields, one for each of the three colours of the SU(3) symmetry.]{The mass, spin and electric charge (\textcolor{blue}{\textit{Q}}) of the quarks found in the Standard Model \cite{Beringer:1900zz}.  Each of the particles in the above table corresponds to three fermion fields, one for each of the three colours of the SU(3) symmetry.} 
\label{table:smquarks}
\end{table}

In the Standard Model, the Lagrangian density describing the fermion fields is invariant under a $\text{SU}(3)$, $\text{SU}(2)_{\text{L}}$ and U(1) gauge transformations.  The $\text{SU}(2)_{L}$ gauge symmetry acts on doublets formed of pairs of left handed chiral components of the fermion fields, $\psi_{L} = \frac{1}{2}(1-\gamma_{5})\psi$, while the right handed components, $\psi_{R} = \frac{1}{2}(1+\gamma_{5})\psi$, transform trivially as singlets \cite{Weinberg:1967tq}.  Similarly, the SU(3) symmetry acts on triplets formed of the fermion fields for each flavour of quark.  All fields transform under the fundamental representation of U(1).  The invariance of the Standard Model Lagrangian to these gauge transformations is established by introducing 12 gauge fields, summarised in table \ref{table:smbosons}, through the covariant \textcolor{blue}{derivative} of the fermion fields
%
\begin{equation}
\partial^{\mu} \rightarrow D^{\mu} = \partial^{\mu} + ig_{1}YB^{\mu} + ig_{2} \textbf{T} \cdot \textbf{W}^{\mu} + ig_{3}\textbf{X} \cdot \textbf{G}^{\mu} \text{ ,}
\end{equation}
%
\noindent where $B^{\mu}$ is the gauge field for the U(1) symmetry, $\textbf{W}^{\mu}$ ($\text{W}^{\mu}_{j}, j =1,2,3$) are the fields of the $\text{SU}(2)_{\text{L}}$ symmetry and $\textbf{G}^{\mu}$ ($\text{G}^{\mu}_{j}, j =1,..,8$) are the fields of the SU(3).  $Y$ is the weak hypercharge, which relates to the chirality and flavour of the fermion field that it is associated to.  The three coefficients $g_{1}$, $g_{2}$ and $g_{3}$ are coupling constants related to the three gauged symmetry groups in the Standard Model.  Mixing of the gauge fields for the U(1) and SU(2) symmetry of the form
%
\begin{equation}
\text{Z}_{\mu} = \text{cos}{\theta_{W}} \text{W}^{3}_{\mu} - \text{sin}{\theta_{W}} \text{B}_{\mu} \text{ ,}\\
\text{A}_{\mu} = \text{sin}{\theta_{W}} \text{W}^{3}_{\mu} + \text{cos}{\theta_{W}} \text{B}_{\mu} \text{ ,}\\
\text{W}^{\pm}_{\mu} = \frac{1}{\sqrt{2}}(\text{W}^{1}_{\mu} \mp i \text{W}^{2}_{\mu}) \text{ ,}
\end{equation}
%
\noindent where
%
\begin{equation}
\text{cos}{\theta_{W}} = \frac{g_{2}}{g_{1}+g_{2}} \text{ and } \text{sin}{\theta_{W}} = \frac{g_{1}}{g_{1}+g_{2}} \text{ ,}
\end{equation}
%
\noindent gives the electroweak gauge bosons; $\text{W}^{\pm}$, Z and $\gamma$.  This mixing ensures that the $\text{W}^{\pm}$ and Z bosons become massive, while the $\gamma$ remains massless.  The $\text{G}^{\mu}_{j}$ fields are the eight massless gluons of the strong force.   $\textbf{T}$ and $\textbf{X}$ are the generators for the SU(2) and SU(3) symmetries, which are typically chosen as
%
\begin{equation}
T_{i} = \frac{1}{2}\tau_{i} \text{ ,} \\
X_{i} = \frac{1}{2}\lambda_{i} \text{ ,} \\
\end{equation}
%
\noindent where $\tau$ and $\lambda$ are the Pauli and the Gell-Mann matrices\textcolor{blue}{,} respectively.  
%
\begin{table}[h!]
\centering
\begin{tabular}{l l r r r}
\hline
Force & Particle & Mass [GeV] & Spin & \textcolor{blue}{\textit{Q}}/\textit{e} \\
\hline
Electromagnetic & $\gamma$ & 0 & 1 & 0 \\
\hline
Weak Nuclear & $\text{W}^{\pm}$ & $80.385 \pm 0.015$ & 1 & $\pm1$ \\
& Z & $91.1876 \pm 0.0021$ & 1 & 0 \\
\hline
Strong Nuclear & $g$ ($\times 8$ colours) & 0 & 1 & 0 \\
\hline
Higgs & H & $125.1 \pm 0.3$ & 0 & 0 \\
\end{tabular}
\caption[The mass, spin and electric charge (\textcolor{blue}{\textit{Q}}) of the gauge bosons found in the Standard Model \cite{Beringer:1900zz}.  The $\gamma$ and $g$s theoretically have zero mass, which is consistent with measurements.  The upper bound on the $\gamma$ mass has been measured at $10^{-18}$ eV, while gluon masses of up to a few MeV have not been precluded.  The upper bound on the magnitude of the charge of the $\gamma$ is measured at $10^{-35}$.]{The mass, spin and electric charge (\textcolor{blue}{\textit{Q}}) of the gauge bosons found in the Standard Model \cite{Beringer:1900zz}.  The $\gamma$ and $g$s theoretically have zero mass, which is consistent with measurements.  The upper bound on the $\gamma$ mass has been measured at $10^{-18}$ eV, while gluon masses of up to a few MeV have not been precluded.  The upper bound on the magnitude of the charge of the $\gamma$ is measured at $10^{-35}$.}
\label{table:smbosons}
\end{table}
%
The gauge fields of the Standard Model, $B_{\mu}$, $\textbf{W}_{\mu}$ and $\textbf{G}_{\mu}$, transform under the gauge transformations as
%
\begin{equation}
K_{\mu} \rightarrow K'_{\mu} = UK_{\mu}U^{\dagger} + \frac{i}{g}(\partial^{\mu}U)U^{\dagger} \text{ ,} 
\end{equation}
%
\noindent where $K_{\mu}$ is any of $B_{\mu}$, $\textbf{W}_{\mu}$ and $\textbf{G}_{\mu}$ and $g$ is the coupling constant associated to the relevant gauged symmetry group.  As the $B_{\mu}$, $\textbf{W}_{\mu}$ and $\textbf{G}_{\mu}$ gauge fields are spin-1, they are described by the Proca Lagrangian density
%
\begin{equation}
\mathcal{L} = -\frac{1}{4}F^{{\mu}{\nu}}_{i}F_{{\mu}{\nu}{i}} + \frac{1}{2}m_{K}^{2}K_{i\mu}K^{\mu}_{i} \text{ ,} 
\end{equation}
%
\noindent where
%
\begin{equation}
F^{{\mu}{\nu}}_{i} = \partial^{\mu}K^{\nu}_{i} - \partial^{\nu}K^{\mu}_{i} - gf_{ijk}K^{\mu}_{j}K^{\nu}_{k} \text{ ,} 
\end{equation}
%
\noindent \textcolor{blue}{and} $f_{ijk}$ are the fully anti-symmetric structure constants of the group, $K^{\mu}_{i}$ is the $i^{th}$ gauge field of the group and $m_{K}$ is a mass term for the gauge boson.  The structure constants are defined from the commutation relations between generators of the symmetry group
%
\begin{equation}
[T_{i},T_{j}] = if_{ijk}T_{k} \text{.} 
\end{equation}
%
\noindent These structure constants govern the self-interactions for the gauge bosons.  There is only one structure constant for the U(1) symmetry, which is zero, as the U(1) symmetry is abelian.  The SU(2) symmetry structure constants are $f_{ijk} = \epsilon_{ijk}$, where $\epsilon_{ijk}$ is the Levi-Civita tensor.  Due to the symmetries that are present in the Standard Model, $m_{K} = 0$ for all the gauge fields, however, it is clear that this is not the case.  

%========================================================================================
%========================================================================================

\section{Higgs Physics}
\label{sec:higgsphysics}
Mass terms are generated in the Standard Model by introducing a Higgs field that undergoes spontaneous symmetry breaking.  This allows the gauge bosons, as well as the quarks and leptons, to obtain a mass, while still respecting the gauge symmetries found in the Standard Model.  

%========================================================================================

\subsection{Spontaneous Symmetry Breaking}
\label{sec:ssb}
To illustrate spontaneous symmetry breaking, consider a complex scalar field \psi with the Klein-Gordon Lagrangian
%
\begin{equation}
\mathcal{L} = \partial^{\mu} \psi^{*} \partial_{\mu} \psi -m^{2} |\psi|^{2} = \partial^{\mu} \psi^{*} \partial_{\mu} \psi - V(\psi) \text{ ,}
\label{equ:kleingordon}
\end{equation}
%
\noindent where $m$ is a mass term and $V(\psi)$ is the potential \textcolor{blue}{of} the field $\psi$.  This Lagrangian density is invariant under the global symmetry $\psi \rightarrow e^{i\alpha} \psi$.  By adding extra terms to the Lagrangian, which retain the invariance to this global symmetry, it is possible to modify the interactions of this scalar field.  Consider modifying the potential of the scalar field to the following
%
\begin{equation}
\text{V}(\psi) = m^{2}|\psi|^{2} + \lambda |\psi|^{4} \text{ .}
\end{equation}
%
\noindent If $m^{2} > 0$, the potential has a \textcolor{blue}{minimum} at zero, however, if $m^{2} < 0$ then the minima exists on a circle in the complex $\psi$ plane, which is centred at $(0,0)$ and has radius $v = \sqrt{-m^{2}/\lambda}$.  To quantise this theory it is necessary to expand about the \textcolor{blue}{minimum} of the potential.  However, in the case of $m^{2} < 0$ there are an infinite number of choices of minima to expand about.  Irrespective of the choice of \textcolor{blue}{minimum} used to expand the field about, the symmetry $\psi \rightarrow e^{i\alpha} \psi$ is broken.  Fluctuations about the \textcolor{blue}{minimum} along the degenerate direction leave the potential unchanged, which is a consequence of the breaking of the $\psi \rightarrow e^{i\alpha} \psi$ symmetry; this is known as spontaneous symmetry breaking.  Goldstone's theorem \cite{Goldstone:1962es} implies that, for Lorentz-invariant theories, spontaneous symmetry breaking always leads to the existence of massless particles known as Goldstone bosons.  If the complex scalar field $\psi$ is expanded about the non-zero minima, $\psi$ takes the form
%
\begin{equation}
\psi = \frac{1}{\sqrt{2}}(v + \psi_{1} + i \psi_{2}) \text{ ,}
\label{equ:minima}
\end{equation}
%
\noindent where $\psi_{1}$ and $\psi_{2}$ are real fields and $v = \sqrt{-m^{2}/\lambda}$.  Applying this parameterisation to the Lagrangian yields a mass term of $\sqrt{-m^{2}}$ for the $\psi_{1}$ field.  However, there is no corresponding mass term for the $\psi_{2}$ field, which indicates that it is massless as predicated by Goldstone's theorem
%
\begin{equation}
\mathcal{L} = \frac{1}{2}\partial^{\mu} \psi_{1} \partial_{\mu} \psi_{1} + \frac{1}{2}\partial^{\mu} \psi_{2} \partial_{\mu} \psi_{2} - m^{2}|\psi_{1}|^{2} + ... \text{ \textcolor{blue}{.}}
\end{equation}

Spontaneous symmetry breaking is the origin of gauge boson mass terms when applied to local symmetries instead of global ones.  For example, consider the global symmetry, $\psi \rightarrow e^{i\alpha} \psi$ that exists in equation \ref{equ:kleingordon}.  If this global symmetry is promoted to a local symmetry by letting $\alpha \rightarrow \alpha(x)$ and $\partial^{\mu} \rightarrow D^{\mu} = \partial^{\mu} + iA^{\mu}$, where $A^{\mu}$ \textcolor{blue}{is} the gauge field that transforms as $A^{\mu} \rightarrow A^{\mu} - \partial^{\mu}\alpha(x)$, the Lagrangian becomes
%
\begin{equation}
\mathcal{L} = (D^{\mu} \psi)^{*} (D_{\mu} \psi) - m^{2} |\psi|^{2} - \lambda |\psi|^{4} \text{ .}
\end{equation}
%
\noindent If the $\psi$ field is expanded about a non-zero \textcolor{blue}{minimum} in the potential, i.e. $m^{2} < 0$ and $v = \sqrt{-m^{2}/\lambda}$, as was done in equation \ref{equ:minima}, then a gauge boson mass term, $+\frac{v^{2}}{2} A^{\mu} A_{\mu}$, is generated from the $(D^{\mu} \psi)^{*} (D_{\mu} \psi)$ term.

%========================================================================================

\subsection{Electroweak Interactions}
\label{sec:ewint}
The electroweak sector of the Standard Model is that related to the $\text{SU}(2)_{\text{L}} \times \text{U}(1)$ symmetry \cite{Ellis:2013jnq}.  In this sector, spontaneous symmetry breaking must occur in such a way as to give three massive gauge bosons,  $\text{W}^{\pm}$ and Z, and one massless gauge boson, the $\gamma$.  This can be achieved through a Higgs field, H, that transforms as a doublet under the $\text{SU}(2)_{\text{L}}$ symmetry.  The Lagrangian for this field is
%
\begin{equation}
\mathcal{L}_{Higgs} = (D_{\mu}\text{H})^{\dagger}D^{\mu}\text{H} - \text{V}(\text{H}) \text{ .}
\end{equation}
%
\noindent The Higgs potential, V(H), is
%
\begin{equation}
\text{V}(\text{H}) = -\mu^{2}\text{H}^{\dagger}\text{H} + \lambda (\text{H}^{\dagger}\text{H})^{2} \text{ ,}
\end{equation}
%
\noindent where $\mu$ and $\lambda$ are constants.  The covariant derivative of this Higgs field must satisfy the $\text{SU}(2)_{\text{L}} \times \text{U}(1)$ gauge symmetry meaning it takes the form
%
\begin{equation}
D_{\mu} \text{H} = (\partial_{\mu} + ig_{1}YB_{\mu} + ig_{2}\frac{\tau^{i}}{2}W^{i}_{\mu})\text{H} \text{ ,}
\end{equation}
%
\noindent where $g_{1}$ and $g_{2}$ are coupling constants for the U(1) and $\text{SU}(2)_{\text{L}}$ gauged symmetries\textcolor{blue}{,} respectively, $Y = \frac{1}{2}$ is the weak hypercharge of the Higgs and $\tau^{i}$ are the Pauli matrices.  $B_{\mu}$ and $W^{i}_{\mu}$ are the gauge fields for the U(1) and $\text{SU}(2)_{\text{L}}$ gauged symmetries respectively.  

Consider spontaneously breaking the symmetry in the Higgs sector by expanding the Higgs field about a non-zero vacuum expectation value (vev)
%
\begin{equation}
\langle \text{H} \rangle = \binom{0}{\frac{v}{\sqrt{2}}} \text{ ,}
\label{equ:higgsmin}
\end{equation}
%
\noindent where the minima of the field is defined as
%
\begin{equation}
\frac{v}{\sqrt{2}} = \sqrt{\frac{\mu^{2}}{2\lambda}} \text{ ,}
\end{equation}
%
\noindent where $v$ real.  In that case, the kinematic term in the Higgs Lagrangian, $D^{\mu} \text{H}^{\dagger} D_{\mu} \text{H}$, contains mass terms for the gauge bosons
%
\begin{equation}
D^{\mu} \text{H}^{\dagger} D_{\mu} \text{H} \subset \frac{v^{2}}{2}(ig_{1}YB^{\mu} + ig_{2}\frac{\tau^{i}}{2}W^{i\mu})(ig_{1}YB_{\mu} + ig_{2}\frac{\tau^{i}}{2}W^{i}_{\mu}) \text{ .}
\end{equation}
%
\noindent If there is mixing of the $\text{SU}(2)_{\text{L}}$ and U(1) fields of the form
%
\begin{equation}
\text{Z}_{\mu} = \text{cos}{\theta_{W}} \text{W}^{3}_{\mu} - \text{sin}{\theta_{W}} \text{B}_{\mu} \text{ ,}\\
\text{A}_{\mu} = \text{sin}{\theta_{W}} \text{W}^{3}_{\mu} + \text{cos}{\theta_{W}} \text{B}_{\mu} \text{ ,}\\
\text{W}^{\pm}_{\mu} = \frac{1}{\sqrt{2}}(\text{W}^{1}_{\mu} \mp i \text{W}^{2}_{\mu}) \text{ ,}
\end{equation}
%
\noindent then the following gauge boson mass terms are generated
%
\begin{equation}
\frac{(gv)^{2}}{4} \text{W}^{+}_{\mu} \text{W}^{-\mu} + \frac{(g^{2} + g'^{2})v^{2}}{8} \text{Z}_{\mu} \text{Z}^{\mu} \text{ .}
\end{equation}
%
\noindent The gauge boson masses generated by spontaneous symmetry breaking of the Higgs field are
%
\begin{equation}
\begin{aligned}
m_{\text{W}} = & \frac{gv}{2} \text{ ,} \\
m_{\text{Z}} = & \frac{v\sqrt{g^{2} + g'^{2}}}{2} = \frac{m_{W}}{\text{cos}{\theta_{W}}} \text{ ,} \\
m_{\text{A}} = & 0 \text{ ,}
\end{aligned}
\end{equation}
%
\noindent where $\theta_{W}$ is the Weinberg angle.  This mixing produces a massless gauge boson, the $\gamma$, and three massive gauge bosons, the $W^{\pm}$ and $Z$.  By acquiring a non-zero vev, the Higgs field breaks the $\text{SU}(2)_{\text{L}} \times \text{U}(1)$ symmetry that was present in the Lagrangian to the $\text{U}(1)_{em}$ symmetry of electromagnetism.  

The ratio of the masses of the $W^{\pm}$ and $Z$ bosons is predicted when spontaneous symmetry breaking occurs in the Higgs sector.  This prediction sets the $\rho$ parameter to unity, where the $\rho$ parameter is defined as
%
\begin{equation}
\rho = \frac{m_{\text{W}}^{2}}{m_{\text{Z}}^{2}\text{cos}{\theta_{W}}^{2}} = 1\text{ .}
\label{equ:custodialsymmetry}
\end{equation}
%
\noindent This is a consequence of the Higgs potential containing custodial symmetry \cite{Beringer:1900zz}.  As the $\rho$ parameter has been experimentally measured to be $1.00040 \pm 0.00024$ \cite{Agashe:2014kda}, it is clear that any extension to the Standard Model should retain this result.

%========================================================================================

\subsubsection{Custodial Symmetry}
The Standard Model Higgs field is defined by the Lagrangian
%
\begin{equation}
\mathcal{L}_{Higgs} = (D_{\mu}\text{H})^{\dagger}D^{\mu}\text{H} - \text{V}(\text{H}) \text{,}
\end{equation}
%
\noindent where
%
\begin{equation}
\text{V}(\text{H}) = -\mu^{2}\text{H}^{\dagger}\text{H} + \lambda (\text{H}^{\dagger}\text{H})^{2} \text{ ,}
\end{equation}
%
\noindent and $\mu$ and $\lambda$ are constants.  By construction the Higgs sector of the Standard Model is invariant under local $\text{SU}(2)_{\text{L}} \times \text{U}(1)$ gauge transformations.  However, a larger global symmetry also exists in this sector, which can be seen by examining the Higgs doublet \cite{Ellis:1991qj}
%
\begin{equation}
\text{H} = \binom{\psi^{+}}{\psi^{0}} = \binom{\psi_{1} + i\psi_{2}}{\psi_{3} + i\psi_{4}} \text{ .}
\end{equation}
%
\noindent All the terms in the Higgs potential involve $\text{H}^{\dagger}\text{H} = \psi_{1}^{2} + \psi_{2}^{2} + \psi_{3}^{2} + \psi_{4}^{2}$, which is invariant under any rotation of these four components and hence under \textcolor{blue}{an} SO(4) global symmetry.  In general, $\text{SO}(4) \cong \text{SU}(2) \times \text{SU}(2)$, where $\cong$ denotes an isomorphism.  In the case of the Higgs sector $\text{SO}(4) \cong \text{SU}(2)_{L} \times \text{SU}(2)_{R}$ where the $\text{SU}(2)_{L}$ symmetry is the gauged symmetry of the Standard Model.  This symmetry can be manifested using an alternative parameterisation \cite{Andersen:2011yj} of the Higgs field
%
\begin{equation}
\Phi = (i\tau_{2}\text{H}, \text{H}) = 
\begin{pmatrix}
\psi^{0*} & \psi^{+} \\
-\psi^{+*} & \psi^{0}
\end{pmatrix} 
\text{ .}
\end{equation}
%
\noindent In this parametrisation the Higgs Lagrangian, $\mathcal{L}_{Higgs}$, becomes
%
\begin{equation}
\mathcal{L}_{Higgs} =  \frac{1}{2}\text{Tr}[(D_{\mu}\Phi)^{\dagger}D^{\mu}\Phi] + \mu^{2}\text{Tr}[\Phi^{\dagger}\Phi] - \lambda \text{Tr}[\Phi^{\dagger}\Phi\Phi^{\dagger}\Phi] \text{ ,}
\end{equation}
%
\noindent which is invariant under transformations of the form
%
\begin{equation}
\Phi \rightarrow U_{L} \Phi U_{R}^{\dagger} \text{ ,}
\end{equation}
%
\noindent where $U_{L}$ and $U_{R}$ are transformations of the $\text{SU}(2)_{L}$ and $\text{SU}(2)_{R}$ symmetry groups respectively.

When the Higgs field acquires a non-zero vev the $\text{SU}(2)_{L} \times \text{SU}(2)_{R}$ symmetry of the Higgs potential is broken to \textcolor{blue}{an} $\text{SU}(2)_{C}$ symmetry, which is known as custodial symmetry \cite{Grojean:2007zz}.  As $\text{SO}(3) \cong \text{SU}(2)$, symmetry breaking in the Higgs sector is equivalent to \textcolor{blue}{an} SO(4) symmetry being broken to \textcolor{blue}{an} SO(3) symmetry.  This becomes clear when inspecting the form of the Higgs potential after symmetry breaking.  After expanding the Higgs field about the non-zero vev that is defined in equation \ref{equ:higgsmin}, the terms in the Higgs potential involve $\text{H}^{\dagger}\text{H} = (\psi_{3}-v)^{2} + \psi_{1}^{2} + \psi_{2}^{2} + \psi_{4}^{2}$.  Since $\text{H}^{\dagger}\text{H} $ is only invariant to rotations between the $\psi_{1}$, $\psi_{2}$ and $\psi_{4}$ fields, the SO(4) global symmetry of the Higgs potential has been broken by spontaneous symmetry breaking to \textcolor{blue}{an} SO(3) symmetry.

The Higgs field, H, transforms a singlet under this $\text{SU}(2)_{C}$ custodial symmetry, while the $\text{SU}(2)_{L}$ gauge boson fields, $W^{i}_{\mu}$, transform as a triplet.  It is the transformation of the $W^{i}_{\mu}$ fields under the $\text{SU}(2)_{C}$ symmetry that enforces the relationship between the masses of the $W^{\pm}$ and $Z$ gauge bosons and that $\rho$ should equal unity.  It should be noted that the $\text{SU}(2)_{L} \times \text{SU}(2)_{R}$ symmetry only exists in the Higgs sector of the Standard Model.  The $\text{SU}(2)_{R}$ symmetry in the Standard Model is broken by Yukawa couplings of the Higgs to quarks and leptons and by a non-zero coupling to the U(1) gauge symmetry of the Standard Model, $g_{1}$.  However, this breaking of the $\text{SU}(2)_{R}$ symmetry is weak, which means the deviations of $\rho$ from unity are minimal \cite{Grojean:2007zz}.  

%========================================================================================
\section{\textcolor{blue}{Anomalous Gauge Couplings}}

\textcolor{blue}{The existence of a light Higgs boson, as discovered by the LHC \cite{Chatrchyan:2012xdj, Aad:2012tfa}, suggests the presence of new physics beyond the Standard Model.  The naturalness principle suggests that the mass of the Higgs boson, $\sim 125$~GeV, should be similar to the Plank energy, $\sim \mathcal{O}(10^{19})$~GeV, however, is not the case.  Such a disparity can only be resolved with a large amount of fine-tuning, which is undesirable, or beyond Standard Model physics involving new particles that protect the Higgs mass from large radiative corrections.  Searching the electroweak sector of the Standard Model for deviations from Standard Model predictions is a natural starting point when searching for new physics that could resolve the naturalness problem of the Higgs.  One such search is presented in chapter \ref{chap:PhysicsAnalysis} and uses an effective field theory (EFT) framework to parameterise deviations from the Standard Model through anomalous gauge couplings.  The EFT framework is discussed in section \ref{sec:eft} and the electroweak chiral Lagrangian from which the anomalous gauge couplings arise is discussed in section \ref{sec:ewchiral}.}

%========================================================================================

\subsection{Effective Field Theory}
\label{sec:eft}
There are a number of features in the observable universe that cannot be accounted for using the Standard Model of particle physics.  However, the Standard Model provides a very good description of the interactions between particles at the energies being probed at modern particle collider experiments.  Any underlying theory governing the interactions of particles must, therefore, behave like the Standard Model over these energies, or distance scales.  Above such energies the theory will deviate from the Standard Model in order to account for the full underlying theory.  Effective field theories (EFTs) work from this premise by assuming that the complete theory has a momentum scale, $\Lambda$, below which Standard Model behaviour is replicated \cite{Degrande:2013rea, Arzt:1993gz}.  

Quantum field theories must be renormalizable to ensure that non-infinite predictions of the coefficients in the Lagrangian can be made and tested \cite{Gripaios:2015qya}.  Infinities arise from non-renormalizable theories due to divergent integrals from loop diagrams that assume the theory being applied is valid at all energy and length scales.  Effective field theories act to avoid such problems by only integrating up to the momentum scale $\Lambda$ and not above it.  At the energy scale being considered, any infinities arising from the loop calculations in the EFT can be absorbed into a finite number of parameters.  This methodology avoids the assumption that the theory in question is applicable to all energy scales and allows measurable predictions to be made.  

As the Standard Model should be replicated at the low energy scale, it is appropriate when creating an EFT Lagrangian to append new operators to the Standard Model Lagrangian to account for areas of new physics.  This gives the general form for an EFT Lagrangian as \cite{Degrande:2013rea}
%
\begin{equation}
\mathcal{L}_{EFT} = \mathcal{L}_{SM} + \sum_{\text{dimension d>4}} \sum_{i} \frac{c_{i}^{(d)}}{\Lambda^{d-4}} \mathcal{O}_{i}^{(d)} \text{ ,}
\label{equ:eft}
\end{equation}
%
\noindent where $\mathcal{L}_{SM}$ is the Standard Model Lagrangian, $c_{i}^{(d)}$ are free parameters, $\mathcal{O}_{i}^{(d)}$ is the $i^{th}$ unique operator with dimension $d$ in the EFT and $\Lambda$ is the EFT momentum scale.  The sum runs over all unique operators with dimension greater than four.  The presence of the $\Lambda^{d-4}$ in the denominator is required to ensure correct dimensionality of the new terms being added to the Lagrangian.  

New physics is introduced by the operators $\mathcal{O}_{i}^{(d)}$, but suppressed by the momentum scale $\Lambda$.  It is assumed that $\Lambda$ is large with respect to the momentum scales that have been examined at pre\textcolor{blue}{-}existing particle collider experiments, therefore, any new physics is suppressed.  Under this assumption, new operators with dimension less than, or equal to, four can be vetoed from the EFT as their effects would be readily observed at preexisting particle collider experiments, due to the $\Lambda^{4-d}$ coefficient.  At energies below the momentum scale, $\Lambda$, it is possible to find the dominant new physics terms in the EFT and consider these as corrections to the Standard Model.  Above this scale the EFT breaks down as operator $\mathcal{O}_{i}^{(d)}$ in $\mathcal{L}_{EFT}$ has a non-negligible coefficient.  In the extremal limit, $\Lambda \rightarrow \infty$, the Standard Model is recovered as new physics is too far out of reach to have any impact on observables.

%========================================================================================

\subsection{Electroweak Chiral Lagrangian}
\label{sec:ewchiral}
The introduction of a Higgs field undergoing spontaneous symmetry breaking is able to produce mass terms in the Lagrangian for the $\text{W}^{\pm}$ and Z bosons.  However, it is possible to introduce these terms by parameterising the Higgs field using the gauge boson fields of the $\text{SU}(2)_{\text{L}}$ Standard Model symmetry \cite{Herrero:1994tj}.  In this approach, the pattern of spontaneous symmetry breaking mirrors that found in the Higgs sector of the Standard Model i.e. a global $\text{SU}(2)_{L} \times \text{SU}(2)_{R}$ symmetry is broken to \textcolor{blue}{an} $\text{SU}(2)_{C}$ symmetry.  This will ensure that the $\rho$ parameter, introduced in section \ref{sec:ewint}, retains a value of unity which is consistent with experimental measurements.  The Standard Model spontaneous symmetry breaking pattern can be replicated using a field, $\Sigma(x)$, which transforms under the $\text{SU}(2)_{L} \times \text{SU}(2)_{R}$ global symmetries as
%
\begin{equation}
\Sigma \rightarrow U_{L} \Sigma U_{R}^{\dagger} \text{ ,}
\end{equation}
%
\noindent where $U_{L}$ and $U_{R}$ are transformations of the $\text{SU}(2)_{L}$ and $\text{SU}(2)_{R}$ symmetry groups\textcolor{blue}{,} respectively\textcolor{blue}{,} and $\Sigma(x)$ is
%
\begin{equation}
\Sigma(x) = \text{exp} \bigg(\frac{-i}{v} \Sigma^{3}_{a=1} \pi^{a}\tau^{a}\bigg)\text{ ,}
\end{equation}
%
\noindent where $\pi^{a}$ are the three would-be Goldstone bosons that exist when the $\text{SU}(2)_{\text{L}} \times \text{U}(1)$ symmetry is broken to $\text{U}(1)_{em}$ \cite{Longhitano:1980tm}.  The $\text{SU}(2)_{\text{L}}$ and $\text{U}(1)$ symmetries of the Standard Model are gauged in the usual way by defining the covariant derivate of the $\Sigma$ field
%
\begin{equation}
\mathcal{D}_{\mu} \Sigma(x) = \partial_{\mu} \Sigma(x) + \frac{ig_{2}}{2}\text{W}_{\mu}^{a}\tau^{a}\Sigma(x) - \frac{ig_{1}}{2}\text{B}_{\mu}\tau^{3}\Sigma(x) \text{ ,}
\end{equation}
%
\noindent where $g_{1}$ and $g_{2}$ are coupling constants for the U(1) and SU$(2)_\text{L}$ symmetries respectively and $\tau^{a}$ are the Pauli spin matrices.  The lowest order derivative term for this $\Sigma$ field that could appear in the Lagrangian is
%
\begin{equation}
\mathcal{L}_{\Sigma} = \frac{v^{2}}{4} \text{Tr} (\mathcal{D}^{\mu} \Sigma^{\dagger} \mathcal{D}_{\mu} \Sigma) = -\frac{v^{2}}{4}\text{Tr}(V_{\mu} V^{\mu}) \text{ ,} 
\end{equation}
%
\noindent where $V_{\mu} = (\mathcal{D}_{\mu}\Sigma) \Sigma^{\dagger}$.  This terms respects all the symmetries present in the Higgs sector of that Standard Model, including the custodial symmetry in the limit $g_{1} \rightarrow 0$.  Furthermore, by expanding this field about a non-zero vev, the $\text{SU}(2)_{L} \times \text{SU}(2)_{R}$ global symmetry is broken to \textcolor{blue}{an} $\text{SU}(2)_{C}$ symmetry exactly as it is in the Standard Model.  For example, if this field is expanded about the point $\Sigma = \textbf{1}$, i.e. the unitary gauge, mass terms for the electroweak gauge bosons are generated that match those produced from spontaneous symmetry breaking of the Higgs field as described in section \ref{sec:ssb}
%
\begin{equation}
\frac{v^{2}}{4}\text{Tr}[V^{\mu}V_{\mu}] = - \frac{(gv)^{2}}{4} \text{W}^{+}_{\mu} \text{W}^{-\mu} - \frac{(g^{2} + g'^{2})v^{2}}{8} \text{Z}_{\mu} \text{Z}^{\mu} \\
\begin{aligned}
m_{\text{A}} = & 0 \text{ ,} \\
m_{\text{W}} = & \frac{gv}{2} \text{ ,} \\
m_{\text{Z}} = & \frac{v\sqrt{g^{2} + g'^{2}}}{2} = \frac{m_{\text{W}}}{\text{cos}{\theta_{W}}} \text{ \textcolor{blue}{.}}
\end{aligned}
\end{equation}

So far, all that has been done is a parameterisation of the Higgs field, however, it was shown by Longhitano \cite{Longhitano:1980tm} that there are several relevant operators involving the $\Sigma$ field that are $\text{SU}(2)_{\text{L}} \times \text{U}(1)$ invariant.  As these operators obey the same symmetries as those found in the Standard Model they should be considered.  This can be done using EFT approach, as discussed in section \ref{sec:eft}.  Of the operators introduced by Longhitano, only two involve quartic massive gauge boson vertices and preserve the custodial symmetry \cite{Belyaev:1998ih}.  These are
%
\begin{equation}
\alpha_{4}\text{Tr}[V^{\mu}V_{\nu}]\text{Tr}[V^{\nu}V_{\mu}] \quad \text{ and } \quad \alpha_{5}\text{Tr}[V^{\mu}V_{\mu}]^{2} \text{ .}
\label{equ:newterms}
\end{equation}
%
\noindent These terms contribute to the massive gauge boson quartic vertices shown in figure \ref{fig:agcvertices}.  The Standard Model already contains triple and quartic vertices involving the electroweak gauge bosons, shown in figure \ref{fig:smtripleandquarticverticestheory}, and these are also present in this EFT approach.  These vertices originate from the kinematic terms in the Proca Lagrangian density $\mathcal{L}_{kin} = -\frac{1}{4}B_{\mu\nu}B^{\mu\nu} - \frac{1}{4}W_{\mu\nu}W^{\mu\nu}$.  Of the vertices showing sensitivity to $\alpha_{4}$ and $\alpha_{5}$, only the vertex shown in figure \ref{fig:agcvertex3} is not present in the Standard Model.  

Both terms shown in equation \ref{equ:newterms} contain dimension 8 operators \cite{Degrande:2013rea} and, with respect to the EFT approach (i.e. equation \ref{equ:eft}) their coefficients are proportional to $\Lambda^{-4}$, where $\Lambda$ is the momentum scale of the new physics being modelled.  In the limit that the momentum scale of new physics is beyond experimental reach, i.e. $\Lambda \rightarrow \infty$, these terms do not contribute to measurable observables and the Standard Model is recovered.  It should be noted that in this case, the Standard Model has been parameterised using the $\Sigma$ field; so, in the limit $\Lambda \rightarrow \infty$, the gauge boson mass terms generated from $\mathcal{L}_{\Sigma}$ do not vanish.  

\begin{figure}[h!]
\subfloat[]{\label{fig:agcvertex1}\includegraphics[width=0.3\textwidth]{AnomalousCouplingsTheory/Plots/AGCVertex1.pdf}} 
\subfloat[]{\label{fig:agcvertex2}\includegraphics[width=0.3\textwidth]{AnomalousCouplingsTheory/Plots/AGCVertex3.pdf}} 
\subfloat[]{\label{fig:agcvertex3}\includegraphics[width=0.2275\textwidth]{AnomalousCouplingsTheory/Plots/AGCVertex2.pdf}} 
\caption[Gauge boson self-coupling vertices that are sensitive to the anomalous gauge couplings $\alpha_{4}$ and $\alpha_{5}$.]{Gauge boson self-coupling vertices that are sensitive to the anomalous gauge couplings $\alpha_{4}$ and $\alpha_{5}$.}
\label{fig:agcvertices}
\end{figure}
% Feynam diagrams of triple and quartic vertices in the standard model.
\begin{figure}[h!]
\subfloat[]{\label{fig:smvertex1}\includegraphics[width=0.3\textwidth]{PhysicsAnalysis/Plots/FeynmanDiagrams/SMVertex1.pdf}} 
\subfloat[]{\label{fig:smvertex2}\includegraphics[width=0.3\textwidth]{PhysicsAnalysis/Plots/FeynmanDiagrams/SMVertex2.pdf}} 
\subfloat[]{\label{fig:smvertex3}\includegraphics[width=0.3\textwidth]{PhysicsAnalysis/Plots/FeynmanDiagrams/SMVertex3.pdf}} \hfill
\subfloat[]{\label{fig:smvertex4}\includegraphics[width=0.3\textwidth]{PhysicsAnalysis/Plots/FeynmanDiagrams/SMVertex4.pdf}} 
\subfloat[]{\label{fig:smvertex5}\includegraphics[width=0.3\textwidth]{PhysicsAnalysis/Plots/FeynmanDiagrams/SMVertex5.pdf}} 
\caption[Gauge boson self-coupling vertices in the Standard Model.]{Gauge boson self-coupling vertices in the Standard Model.}
\label{fig:smtripleandquarticverticestheory}
\end{figure}

%========================================================================================

  %% Linear Collider Detectors and Particle Flow
 \chapter{Particle Flow Calorimetry and Linear Collider Detectors}
\label{chap:reconstructionchain}

\chapterquote{I am fond of pigs.  Dogs look up to us.  Cats look down on us.  Pigs treat us as equals.}
{Winston Churchill}

\section{Particle Flow Calorimetry}

The premise of particle flow calorimetry is to measure the energy of individual particles in detector using the sub-detector that offers the best energy resolution.  For particle collider experiments, the biggest contrast to tradition calorimetry is that the energy of charged particles is measured using the curvature of the tracks they produce in the tracker instead of measuring their energy in calorimeters.  The energy resolution for these charged particles is significantly better when using the particle flow approach to calorimetry, which leads to exceptionally good jet energy resolutions that can be used for characterising multi-jet final states in physics processes of interest at the linear collider experiment.  Furthermore, these energy resolutions are highly beneficial for quantifying those final states involving charged leptons and missing momentum, due to the presence of neutrinos.

\begin{figure}
\centering
\includegraphics[width=0.5\textwidth]{LCDetectorsAndPFlow/Plots/Pictures/PFlow.png}
\caption[A typical simulated 250 GeV jet in the CLIC\_ILD detector, with labels identifying constituent particles.  Image taken from  \cite{arXiv:1209.4039}.]{A typical simulated 250 GeV jet in the CLIC\_ILD detector, with labels identifying constituent particles.  Image taken from  \cite{arXiv:1209.4039}.}
\label{fig:particleflowpic}
\end{figure} 

Particle flow calorimetry is challenging to put into practice as it requires a precise reconstruction for all long-lived particles within a detector.  Charged particles have their energy measurements taken from the curvature of the track they transverse, but they also produce calorimetric energy deposits, as shown in figure \ref{fig:particleflowpic}, and if both energy measurements are included the energy of the charged particle will be double counted.  Therefore, the precise reconstruction has to associate charged particle tracks to their corresponding calorimetric energy deposits.  This can only be realised by using calorimeters with fine segmentation so that it is possible to resolve individual particle showers within them.  This is the basis for the design of the linear collider calorimeters.  While double counting of energy of charged particles is possible, it is also possible to omit the energy measurements of neutral particles.  This occurs if a neutral particle calorimetric energy deposit, which is the only energy measurement produced for a neutral particle, is incorrectly associated to a track.  In such a case the calorimetric energy deposit will not be used as the reconstruction believes the energy is from a charged particle and so will come from the track and the neutral hadron energy will be lost.  These two effects form the confusion contribution to the jet energy resolution, which acts to degrade the energy resolution of a particle flow calorimetry based detector.  

\begin{table}[h!]
\centering
\begin{tabular}{ l l l l l}
\hline
Jet  & Detector & Energy & Energy & Jet Energy \\
Component &  & Fraction & Resolution & Resolution Contribution \\
\hline
Charged & Tracker & $\sim 0.6 E_{j}$ & $10^{-4} \times E_{X^{\pm}}^{2}$ & $< 3.6 \times 10^{-5} \times E_{j}^{2}$ \\
Particles ($X^{\pm}$) & & & & \\
Photons & ECal & $\sim 0.3 E_{j}$ & $0.15 \times \sqrt{E_{\gamma}}$ & $0.08 \times \sqrt{E_{j}}$ \\
($\gamma$) & & & & \\
Neutral & HCal &$\sim 0.1 E_{j}$ & $0.55 \times \sqrt{E_{X^{0}}}$ & $0.17 \times \sqrt{E_{j}}$ \\
Hadrons ($X^{0}$) & & & & \\
\hline
\end{tabular}
\caption[The approximate fractions, energy resolutions and jet energy resolution contributions made by charged particles ($X^{\pm}$), photons ($\gamma$) and neutral hadrons ($X^{0}$).  Table taken from \cite{arXiv:0907.3577}.]{The approximate fractions, energy resolutions and jet energy resolution contributions made by charged particles ($X^{\pm}$), photons ($\gamma$) and neutral hadrons ($X^{0}$).  Table taken from \cite{arXiv:0907.3577}.}
\label{table:pflowjet}
\end{table}

The magnitude of the improvements offered by particle flow calorimetry can be explicitly seen when considering the different contributions to the measurement of jet energies, which is summarised in table \ref{table:pflowjet}.  After the decay of short lived particles approximately 60\% of the energy of a jet is carried in the form of charged particles, 30\% in the form of $\gamma$s and 10\% in the form of neutral hadrons.  A negligible amount of energy is also carried in the form of invisible energy i.e. neutrinos.  In the traditional calorimetric approach $\gamma$s are measured within the ECal, with an energy resolution of $\sim 0.15 \times \sqrt{E_{\gamma}}$, and the remaining particles are measured in the HCal, with an energy resolution of $\sim 0.55 \times \sqrt{E_{X}}$.  This gives contributions to the jet energy, $E_{j}$, resolution of $\frac{0.08}{\sqrt{E_{j}}}$ and $\frac{0.46}{\sqrt{E_{j}}}$ from $\gamma$s and other particles respectively.  These add in quadrature to give a total jet energy resolution of $\frac{0.47}{\sqrt{E_{j}}}$.  In the particle flow paradigm the energy of charged particles is measured in the tracker, which has such a good energy resolution that its contribution to the jet energy resolution is negligible.  Therefore, contributions to the jet energy resolutions only come from $\gamma$s and from neutral hadrons, which when added in quadrature give a total jet energy resolution of $\frac{0.19}{\sqrt{E_{j}}}$.  This jet energy resolution is significantly better than that offered by the traditional calorimetric approach, however, it must be emphasised that this is an upper limit on the performance as the effect of confusion will degrade the jet energy resolution.  By applying sophisticated pattern recognition algorithms this confusion can be minimised and exceptional performance achieved.  While numerous approximations have been made in the above calculation it is clear that particle flow calorimetry has the potential to revolutionise detector deign for high energy physics experiments.

\section{Linear Collider Detectors}
\label{sec:ild}
All detector concepts for the linear collider have been purposely built to make particle flow calorimetry possible.  While there are a number of different concepts that are under consideration for both the ILC and CLIC one of the most prominent, and the focus of this work, is the International Large Detector (ILD).  The ILD detector, shown in figure \ref{fig:ild} realises very high spatial resolution for all sub-detector systems thanks to its highly granular calorimeters and central tracking system, all of which is encompassed within a 3.5 T magnetic field.  When combined with sophisticated pattern recognition software provided by PandoraPFA, particle flow calorimetry can be realised and the jet energy resolution can reach the goal of $3.8 \%$ which is required to allow separation of hadronic decays from W and Z bosons.  Details on each of the various sub-detector systems for ILD will now be discussed.

\begin{figure}
\centering
\subfloat[]{\label{fig:ild1}\includegraphics[width=0.5\textwidth]{LCDetectorsAndPFlow/Plots/Pictures/ILD.jpg}}
\subfloat[]{\label{fig:ild2}\includegraphics[width=0.5\textwidth]{LCDetectorsAndPFlow/Plots/Pictures/ILD_2.jpg}}
\caption[\protect\subref{fig:ild1} Quadrant view of the ILD detector concept.  The interaction point is in the lower right corner of the picture.  Dimensions are in mm.  \protect\subref{fig:ild2} View of the ILD detector concept.  Figures taken from  \cite{Behnke:2013lya}.]{\protect\subref{fig:ild1} Quadrant view of the ILD detector concept.  The interaction point is in the lower right corner of the picture.  Dimensions are in mm.  \protect\subref{fig:ild2} View of the ILD detector concept.  Figures taken from  \cite{Behnke:2013lya}.}
\label{fig:ild}
\end{figure} 

\subsection{Tracking System}
The tracking system for the ILD detector consist of a multi-layer pixel-vertex detector, which is surrounded by a system of silicon strip and pixel detectors.  These are purposed to give precise information about displaced vertices with respect to the impact point, which are crucial for the study of short lived particles such as the $D$ or $B$ mesons.  Outside of the vertex detector the central tracker of ILD, which is a Time Projection Chamber (TPC).  The TPC allows each charged particle track to be sampled at many space points giving precise information that can be used to extract the curvature of the track and the momentum of the charged particle transversing the track.  Finally, a further silicon strip detector surrounds the TPC to give an additional, high precision, space point to aid in the tracking performance. 

\subsubsection{Vertex System}
The main goal of the ILD vertex detector is to achieve a resolution on the impact parameter of charged particle tracks of $\sigma_{b} < 5 \oplus \frac{10}{p\text{sin}(\theta)^{3/2}} \mu$m, where the first term is the transverse impact parameters resolution and the second is a multiple-scattering term.  This makes it possible to precisely tag secondary vertices from charm and bottom mesons, which typically have relatively short proper lifetimes, $\tau$, such that $c\tau \approx \mathcal{O}(100-500) \mu$m.  To achieve this impact parameter resolution a spatial resolution of better than 3 {\mu}m is required near the IP will be required.  Furthermore, a low material budget of less than 0.15 \% $X_{0}$ per layer is needed to ensure that little energy is lost and that few electromagnetic showers are initiated within the tracker.  A low pixel occupancy will be essential for determining the trajectory of individual tracks in the detector.  The detector will have to be radiation hard to cope with the intense conditions found close to the IP due to the beam induced background, predominantly beamstrahlung.  Furthermore, consideration will have to be given as to the mechanical structure of the detector, power consumption and cooling.  

There are a number of different pixel technology options under consideration for the vertex detector for the ILD detector and this is an active area of ongoing research and development for the linear collider collaboration.  The current design of the vertex detector consists of three concentric layers of double-sided ladders that are close to being cylindrical.  Each ladder has two pixel sensors on each side and the ladder thickness is approximately 2 mm.  The inner most radii of the ladders ranges from 16 mm to 60mm from the IP.  In the simulation of the vertex detector silicon is used as the sensitive material and both support material and a cryostat is included for realism.  

\subsubsection{Silicon Tracking System}
There are four components that make up the silicon tracking system for ILD, shown in figure \ref{fig:vertex}.  These are the:

\begin{figure}
\centering
\subfloat[]{\label{fig:vertex1}\includegraphics[width=0.5\textwidth]{LCDetectorsAndPFlow/Plots/Pictures/Vertex1.jpg}}
\subfloat[]{\label{fig:vertex2}\includegraphics[width=0.5\textwidth]{LCDetectorsAndPFlow/Plots/Pictures/Vertex2.pdf}}
\caption[\protect\subref{fig:vertex1} A quadrant view of the ILD silicon envelope made of the four components SIT, SET, ETD and FTD as included in the full MOKKA simulation.  \protect\subref{fig:vertex2} a 3D detailed GEANT4 simulation description of the silicon system.  Figures taken from  \cite{Behnke:2013lya}.]{\protect\subref{fig:vertex1} A quadrant view of the ILD silicon envelope made of the four components SIT, SET, ETD and FTD as included in the full MOKKA simulation.  \protect\subref{fig:vertex2} a 3D detailed GEANT4 simulation description of the silicon system.  Figures taken from  \cite{Behnke:2013lya}.}
\label{fig:vertex}
\end{figure} 

\begin{itemize}
\item Silicon Inner Tracker (SIT) and Silicon External Tracker (SET).  These are both barrel components, which are positioned immediately inside and outside the TPC.  They act to provide additional space points that can be used in track fitting.  In particular these help to link the vertex detector with the TPC and help with extrapolation of TPC tracks into the calorimeter.  These silicon sensors will have a 50 {\mu}m pitch and will contain 200 m{\mu} thick silicon.  
\item Endplate of the TPC (ETD).  This sensor is identical to the SET, but is positioned in front of the ECal endcap calorimeter to extend the coverage of this silicon envelope. 
\item Forward tracker (FTD).  This detector consists of seven silicon disks, which extends the coverage of the tracking down to small angles which the TPC does not cover.  
\end{itemize}

The requirements for these sensors is similar to those of the vertex detector in terms of requiring low material budget and low occupancy, however, as the sensors are further away from the IP radiation hardness is less crucial.  The technology options for these sensors is also under development as was the case for the vertex detector.  In the detector model simulations all of these elements are included with additional material added to represent the support structure.  

\subsubsection{TPC}
The central tracking system for the ILD detector is a TPC, which is shown in figure \ref{fig:tracker}.  The TPC consists of two chambers of gas that has a high voltage applies across it.  Charged particles passing through the TPC ionise the gas and the ionised molecules drift in the high voltage to the end plates where they are collected and measured.  The drift time is then used to calculate the position of the ionisation point.  TPCs have the advantage over silicon tracking as they continuously track any charged particle passing through them unlike silicon detectors, which are only sensitive within the silicon layer.  This compensates for the worse single point resolution that TPCs have in comparison to a silicon detectors and makes TPCs a viable option for the ILD detector.  Furthermore, the TPC has a very low material budget, which benefits calorimetry in ILD.  The TPC will operate within a 3.5 T magnetic field and under these conditions a point resolution of better than 100 {\mu}m and a double hit resolution in $\phi$ of less than 2 mm can be achieved.  Several readout technology options that are dependent upon the gas mixture used for the TPC are currently under development.  For all potential options it is envisaged that the readout pads would be $\approx 1 \times 6 \text{mm}^{2}$ giving a total of approximately $10^{6}$ on the TPC endplates.

In the detector simulation the TPC is simulated as a cylindrical volume of the gas mixture, Ar:$\text{CH}_4$:$CO_{2}$ (95:3:2) \cite{Abe:2010aa}, which is surrounded by a realistic field cage.  Furthermore, a conservative estimate of the endplate is included in the simulation, which accounts for the support structure, electronics and cooling pipes.  Estimations have also been made for the material budget for power and readout cables that will serve the inner tracking detector.  These are included in the simulation as an aluminium cylinder between the beam pipe and the field cage of the TPC.

\begin{figure}
\centering
\subfloat[]{\label{fig:tracker1}\includegraphics[width=0.5\textwidth]{LCDetectorsAndPFlow/Plots/Pictures/Tracker1.png}}
\subfloat[]{\label{fig:tracker2}\includegraphics[width=0.5\textwidth]{LCDetectorsAndPFlow/Plots/Pictures/Tracker2.png}}
\caption[\protect\subref{fig:tracker1} Drawing of the propose end-plate for the TPC.  \protect\subref{fig:tracker2} Conceptual sketch of the TPC system showing the main parts of the TPC (not to scale).  Figures taken from  \cite{Behnke:2013lya}.]{\protect\subref{fig:tracker1} Drawing of the propose end-plate for the TPC.  \protect\subref{fig:tracker2} Conceptual sketch of the TPC system showing the main parts of the TPC (not to scale).  Figures taken from  \cite{Behnke:2013lya}.}
\label{fig:tracker}
\end{figure} 

\subsection{Electromagnetic Calorimeter}
A highly segmented electromagnetic sampling calorimeter (ECal) surrounds the ILD tracking system, which has been designed with particle flow calorimetry in mind.  To that extend the spatial resolution of particle showers within the ECal takes as much, if not more, precedence than the energy resolution.  The nominal ILD ECal is a silicon tungsten sampling calorimeter, which contains 30 layers and uses square cells with side length 5 mm, however, a scintillator strip option is also being considered.  

The primary goal of the ECal is to induce electromagnetic particles to shower within it and to record the energy deposited by those showers.  To that extent the ECal is constructed using tungsten as the absorber material.  As well as containing a large number of radiation lengths ($X_{0}$) per unit length, see table \ref{table:absorberoptions}, tungsten also has a small Moli�re radius and a large ratio of the radiation length to the nuclear interaction length.  The small Moli�re radius will lead to compact electromagnetic showers and make the separation of nearby showers easier, while the large ratio of the radiation length to nuclear interaction length will lead to greater longitudinal separation between electromagnetic and hadronic showers.  The use of tungsten in the nominal ILD ECal allows for a large number of radiation lengths, $\approx 24 X_{0}$, to be compacted within a relatively short distance, $\approx 20$ cm, which is sufficient for containing all but the highest energy electromagnetic showers.  The compact nature of the ECal also helps to reduce the overall size and cost of the detector.   A good energy resolution can be achieved with this configuration if 30 sampling layers are used.  The the tungsten thickness is 2.1mm for the inner 20 layers and 4.2mm for the last 10 layers to reduce the number of readout channels and cost, while maintaining a high sampling rate at the start of the calorimeter.  It should be noted that this offers no major gains in terms of energy resolutions in comparison to preexisting particle collider experiments \ref{Chatrchyan:2013dga, Perret:2014owa} because the focus of this calorimeter is split between imaging the particle showers and recording their energy as opposed to purely focusing on the energy measurement.  The 5 mm cell size for the ECal was chosen as a balance between being able to resolve nearby particle showers as well as reducing the overall cost of the calorimeter, which scales with the number of readout channels needed.  An optimisation study of the various ECal parameters for the ILD detector can be found in section \ref{sec:ecal}.

\begin{table}[h!]
\centering
\begin{tabular}{ l l l l l}
\hline
Material & $\lambda_{I}$ (cm) & $X_{0}$ (cm) & $\rho_{M}$ (cm) & $ \frac{\lambda_{I}}{X_{0}}$ \\
\hline
Fe & 16.8 & 1.76 & 1.69 & 9.5 \\
Cu & 15.1 & 1.43 & 1.52 & 10.6 \\
W & 9.6 & 0.35 & 0.93 & 27.4 \\
Pb & 17.1 & 0.56 & 1.00 & 30.5 \\
\hline
\end{tabular}
\caption[Comparison of the nuclear interaction length $\lambda_{I}$, radiation length $X_{0}$ and Moli�re radius for iron, copper, tungsten and lead.  Table taken from \cite{arXiv:0907.3577}.]{Comparison of the nuclear interaction length $\lambda_{I}$, radiation length $X_{0}$ and Moli�re radius for iron, copper, tungsten and lead.  Table taken from \cite{arXiv:0907.3577}.}
\label{table:absorberoptions}
\end{table}

As well as including the silicon tungsten sampling calorimeter, the simulation of the ILD ECal contains additional material to represent the instrumented region of the sensor and a heat shield as shown in figure \ref{fig:ecal}.

\begin{figure}
\centering
\subfloat[]{\label{fig:ecal1}\includegraphics[width=0.4\textwidth]{LCDetectorsAndPFlow/Plots/Pictures/SiECal.png}} 
\hspace{1cm}
\subfloat[]{\label{fig:ecal2}\includegraphics[width=0.4\textwidth]{LCDetectorsAndPFlow/Plots/Pictures/ScECal.png}}
\caption[Cross section through ECal layer for \protect\subref{fig:ecal1} silicon and \protect\subref{fig:ecal2} scintillator option.  Figures taken from  \cite{Behnke:2013lya}.]{Cross section through ECal layer for \protect\subref{fig:ecal1} silicon and \protect\subref{fig:ecal2} scintillator option.  Figures taken from  \cite{Behnke:2013lya}.}
\label{fig:ecal}
\end{figure} 

\subsection{Hadronic Calorimeter}
Surrounding the ECal is a finely segmented hadronic calorimeter (HCal), which has the primary goal of measuring the energy deposits from charged and neutral hadrons.  Similarly to the ECal the focus of this HCal is split between being able to resolve nearby particle showers and measuring their energy with a good resolution.  The nominal ILD HCal is a scintillator steel sampling calorimeter, which contains 48 layers and uses square cells with side length 30 mm.      
 
Iron is used as the absorber material for the HCal as it has excellent mechanical properties that allow the HCal to be constructed without auxiliary supports, which if required would act as dead regions in the detector.  Furthermore, iron is relatively inexpensive and given the nuclear interaction length is sufficiently small, it is possible to achieve a compact calorimeter design for low cost.  The relatively short radiation length found in iron is useful for fine sampling of the electromagnetic shower core that is found in hadronic showers, which exists due to the decays of $\pi^{0}$ and $\eta$ mesons.  This fine sampling leads to good energy resolution for the HCal for these shower components.  The nominal ILD HCal contains approximately $6 \lambda_{I}$, which when combined with the $1 \lambda_{I}$ in the ECal is sufficent for containing the majority of hadronic showers at ILC like energies.  The 48 layers are identical and are comprised of 20 mm of steel absorber with a 3 mm scintillator active medium.  The use of a square cell size of 30 mm is again a balance between reducing the cost of the detector, which is proportional to the number of readout channels and thus wants to lower the cell size, and achieving the required spatial resolution to make particle flow calorimetry possible.  Overall, the segmentation of the ILD HCal is able to give excellent spatial and energy resolution that can make particle flow a reality.  An optimisation study of the various HCal parameters for the ILD detector can be found in section \ref{sec:hcal}.

Simulation of the ILD HCal has a number of realistic features including detailed modelling of the electronics, detector gaps and the implementation of Birk's law for the scintillator sensitive detector elements.

\subsection{Forward Calorimetry}

Three additional sampling calorimeters are envisaged for the linear collider, LumiCal, LHCal and BeamCal, that will extend the coverage of the detector towards 4\pi and monitor the beam quality.  The LumiCal will aim to measure the luminosity with a precision of less than $10^{-3}$ at 500 GeV using Bhabha scattering, $\text{e}^{+}\text{e}^{-} \rightarrow \text{e}^{+}\text{e}^{-}(\gamma)$, as a gauge process \cite{Abe:2010aa}, while the BeamCal will make a bunch-bunch estimate of the luminosity and assist in the beam tuning.  Alongside the LumiCal is the LHCal, which extends the coverage of the HCal to low polar angle as shown in figure \ref{fig:fcal}.  The LumiCal covers polar angles between 31 and 77 mrad, while the BeamCal covers the range between 5 and 40 mrad.  The presence of beam-induced backgrounds along the beam line means these calorimeters will have to be radiation hard.  
 
\begin{figure}
\centering
\includegraphics[width=0.5\textwidth]{LCDetectorsAndPFlow/Plots/Pictures/FCal.png}
\caption[The very forward revion of the ILD detector.  LumiCal, BeamCal and LHCal are carried by the support tube for the final focusing quadruple, QD0, and the beam pipe.  Figure taken from  \cite{Behnke:2013lya}.]{The very forward revion of the ILD detector.  LumiCal, BeamCal and LHCal are carried by the support tube for the final focusing quadruple, QD0, and the beam pipe.  Figure taken from  \cite{Behnke:2013lya}.}
\label{fig:fcal}
\end{figure} 

As the primary focus of these calorimeters is measuring the $\text{e}^{+}\text{e}^{-}$  beam, they are all constructed using tungsten absorber material to ensure narrow electromagnetic showers form within them.  The LumiCal layer configuration mirrors that of the ECal giving it a total of $\approx 24 X_{0}$ across its 30 layers.  Silicon is used as sensitive detector element for the LumiCal.  The LHCal uses silicon readout sensors identical to those found in the LumiCal and in total the LHCal contains $4 \lambda_{I}$ across 40 layers.  The BeamCal sensitive detector material is currently being developed as, due to the high occupancy from the beam induced backgrounds, a fast readout is required.  The cell sizes for these calorimeters is yet to be confirmed.  

In the context of particle flow calorimetry these calorimeters play a minimal role as few particles from the hard physics interaction will have their energy measured in these calorimeters and so they are not used in the reconstruction.

\subsection{Muon Chamber}
The ILD outer detector surrounds the HCal and is comprised of a coil, which generates a 3.5 T magnetic field, followed by an iron yoke.  The coil is one of the major cost drivers for the ILD detector and to minimise the size it does not encompass the iron yoke.  The yoke supplements measurements in the calorimeters by acting as a tail catcher for energy leaking out of the calorimeters and, furthermore, is used in the identification of muons.  The yoke is an iron scintillator sampling calorimeter, which consists of 10 layers spaced 14 cm apart followed by 2 (3) layers spaced 60 cm apart for the barrel (endcap) region of the detector as shown in figure \ref{fig:muon}.  There is also an additional sensitive layer for the barrel region placed immediately outside the HCal to help with association energy deposits between the calorimeters and the yoke.   

\begin{figure}
\centering
\includegraphics[width=0.5\textwidth]{LCDetectorsAndPFlow/Plots/Pictures/Muon.png}
\caption[The sensitive layers of the ILD muon system.  Figure taken from  \cite{Behnke:2013lya}.]{The sensitive layers of the ILD muon system.  Figure taken from  \cite{Behnke:2013lya}.}
\label{fig:muon}
\end{figure}   

The yoke uses scintillator strip readout technology and in the simulations a square cell size of 30 mm is assumed.  This is in contrast to the ILD baseline, which plans to use 3 cm wide and 1 m long strips, however, as the tail-catcher plays a minimal role in particle flow at ILC like energies this difference should have negligible impact.  

\subsection{CLIC ILD}
It is inappropriate to use the nominal ILD detector for simulations of the CLIC experiment due to the increased collision energy.  Therefore, the CLIC experiment has modified the nominal detector model to create a new detector model, CLIC\_ILD \cite{Linssen:2012hp} shown in figure \ref{fig:clicild}, which is more suited to CLIC like conditions.  The key differences between the nominal ILD detector and CLIC\_ILD  are:

\begin{itemize}
\item The higher energies found at the CLIC experiment lead to more intense beam induced backgrounds, which is especially problematic for detectors close to the IP where the occupancies will be extremely high.  To attempt to compensate for these effects the inner vertex detector in CLIC\_ILD is moved 15 mm further out from the IP.    
\item The HCal thickness is increased from 6 $\lambda_{I}$ to 7.5 $\lambda_{I}$.  This helps to contain the high energy particle showers found in the CLIC experiment to the calorimeters.
\item The HCal absorber material in the barrel is tungsten as opposed to iron.  This reduces the overall thickness of the HCal and keeps the coil size, one of the driving cost factors for the detectors, similar for the nominal ILD and CLIC\_ILD detectors.  In the endcaps iron is again used as the absorber material as there are no spatial requirements and it will lower the detector cost.  Furthermore, the shower development time in steel is faster than in tungsten making effective time stamping of energy deposits easier, which is crucial for the CLIC experiment for vetoing beam induced backgrounds.  
\item The magnetic field strength in the CLIC\_ILD detector is increased to 4 T.  This was found to benefit the reconstruction, particularly at high energies, as it leads to greater separation of charged particle tracks.  Furthermore, it was possible to achieve this increase in field strength using the nominal ILD coil design.   
\end{itemize}   

\begin{figure}
\centering
\subfloat[]{\label{fig:clicild1}\includegraphics[width=0.4\textwidth]{LCDetectorsAndPFlow/Plots/Pictures/CLIC_ILD.png}}
\hspace{1cm}
\subfloat[]{\label{fig:clicild2}\includegraphics[width=0.4\textwidth]{LCDetectorsAndPFlow/Plots/Pictures/CLIC_ILD_2.png}}
\caption[\protect\subref{fig:ild1} Longitudinal (top quadrant) and \protect\subref{fig:ild2} transverse cross section of the CLIC\_ILD detector.  Figures taken from \cite{Linssen:2012hp}.]{\protect\subref{fig:ild1} Longitudinal (top quadrant) and \protect\subref{fig:ild2} transverse cross section of the CLIC\_ILD detector.  Figures taken from \cite{Linssen:2012hp}.}
\label{fig:clicild}
\end{figure} 

\section{PandoraPFA}
Particle flow calorimetry relies upon correct associations being made between calorimetric energy deposits and charged particle tracks.  Even with a finely segmented detector, such as the ILD detector described in section \ref{sec:ild}, correctly making these associations is highly a non-trivial task and must be done using advanced pattern recognition software.  This is provided by the PandoraPFA particle flow algorithm \cite{arXiv:0907.3577, arXiv:1209.4039}.

PandoraPFA applies the pattern recognition logic in eight main stages:
\being{enumerate}
\item Track selection.  The input track collections are examined to determine whether $V^{0}$ decays, two charged tracks originating from a point displaced from the IP, or kinks, where a charged particle has decayed into a single charged particle and a number of neutral ones, are present.  Such information will be propagated in the reconstruction to final PFO creation.  
\item Calorimeter hit selection.  The various collection of, post digitisation, calorimeter hits are passed into the Pandora framework and converted into Pandora calorimeter hits.  These objects are self describing so that the Pandora pattern recognition logic has no dependancy on the external software framework.  A minimum ionising particle equivalent energy cut is applied to the calorimeter hits at this stage.  If a calorimeter hit contains less than 0.5 (0.3) of the energy of a normally incident MIP passing through the ECal (HCal) calorimeter cell it is not used in the reconstruction.  
\item Clustering.
\item Topological cluster merging.
\item Statistical re-clustering.
\item Photon identification and recovery.
\item Fragment removal.
\item Formation of particle flow objects.
\end{enumerate}












 %~~ Add these to above
%~~ \chapter{Future Linear Collider Experiments}
\label{chap:futurelinearcolliderexperiments}

\section{The International Linear Collider}

\section{The Compact Linear Collider}

\subsection{Experimental Conditions at CLIC}
The CLIC experiment will operate in a unique environment in comparison to previous generations of lepton colliders and this must be properly accounted for to get an accourate measure of the physics potential that CLIC has to offer.  The following aspects of the CLIC experiment present the largest challenges to the physics potential for the CLIC experiment:

\begin{itemize}
\item The high bunch charge density.  The small beam size at the impact point produces very large electromagnetic fields.  These fields can interact with the opposite beam particles causing them to radiate photons in an effect known as beamstrahlung.  Beamstrahlung acts to reduce the collision energy of the $\text{e}^{+}\text{e}^{-}$ pairs.   
\item Beam related backgrounds.  Beamstrahlung photons can subsequently interact to produce background events that must be accounted for.  Dominant backgrounds of this form that cannot be easily vetoed in the reconstruction include incoherent pair production of $\text{e}^{+}\text{e}^{-}$ and $\gamma\gamma \rightarrow \text{Hadron}$.  
\item Fast readout technology is crucial.  The CLIC bunch train consists of 312 bunches with a repetition rate of 50 Hz.  Each bunch is separated by 0.5ns, therefore, it will be necessary to integrate over multiple bunch crossing when reading out the detectors.  This places tight constraints on all detector electrical readout speeds and time resolutions.   
\end{itemize}

\subsubsection{Beam-Related Backgrounds at CLIC}
The primary sources of background for the CLIC experiment are as follows:
\begin{itemize}
\item $\text{e}^{+}\text{e}^{-}$ pair creation from the interaction of a beamstrahlung photons with the opposing beam.  The different mechanisms for pair creation are as follows:
\begin{itemize}
\item \textbf{Coherent pair production}.  This mechanism involves the interaction of a real beamstrahlung photon with the electromagnetic field from the opposing beam.
\item \textbf{Trident pair production}.  This mechanism involves the interaction of a virtual beamstrahlung photon with the electromagnetic field from the opposing beam.
\item \textbf{Incoherent pair production}.  This mechanism involves the interaction of a real or virtual beamstrahlung photon with the individual particles in the opposing beam.
\end{itemize}
\item $\gamma\gamma \rightarrow \text{Hadron}$ from the interaction of real or virtual beamstrahlung photons with each other.  Example Feynman diagrams for such processes is shown in figure ??. 
\item Beam halo muons that arise from interactions of the beam particles during collimation.  The dominant mechanisms producing beam halo muons are photon conversions into muon pairs ($\gamma \text{e}^{-} \rightarrow \mu^{+}\mu^{-}\text{e}^{-}$) and annihilation of positrons with atomic $\text{e}^{-}$ into muon pairs ($\text{e}^{+}\text{e}^{-} \rightarrow \mu^{+}\mu^{-}$) \cite{Pilicer:2015ijy}.
\end{itemize}

Each of these has to be properly addressed to get a true measure of the physics potential at CLIC.  Coherent and trident pair production is not a dominant source of background as they are produced at low transverse momenta, as figure \ref{fig:backgroundangle} shows, and a simple cut would veto these backgrounds.  This is not the case for incoherent pair production of $\text{e}^{+}\text{e}^{-}$, which are dominant in the forward regions of the detector, and $\gamma\gamma \rightarrow \text{Hadron}$, which are dominant in the tracker and the calorimeters (with the exception of low radii in the calorimeter endcaps) \cite{Linssen:2012hp, Sailer:2012mfa}.  Beam halo muons are not a major source of background either as they can be easily removed during the reconstruction due to the clear signal they create in the detector.  An algorithm was developed within the PandoraPFA framework for this purpose and it was found to be highly effective at removing the beam halo muons background \cite{Linssen:2012hp}.  

$\gamma\gamma \rightarrow \text{Hadron}$ events are the most dominant source of background to consider at CLIC as these events deposit more energy throughout the detector than incoherent pair production of $\text{e}^{+}\text{e}^{-}$ events \cite{Linssen:2012hp}.  The effect of the $\gamma\gamma \rightarrow \text{Hadron}$ background is incorporated into this analysis by overlaying $\gamma\gamma \rightarrow \text{Hadron}$ events onto the event samples used in this analysis.  The overlaid backgrounds are added prior to reconstruction so that their effect on the reconstruction is fully accounted for.  For a given event the exact number of background events overlaid is drawn from a Poisson distribution with a mean of 3.2 (1.3) events per bunch crossing at 3 (1.4) TeV.  While incoherent pairs are still a source of background they will produce a second order effect in comparison to the $\gamma\gamma \rightarrow \text{Hadron}$ events.

The PFO choices described in section \ref{sec:optimisationjetalgo} are applied to veto the effect of PFOs that arise from the overlaid $\gamma\gamma \rightarrow \text{Hadron}$ events.

\begin{figure}
\includegraphics[width=0.5\textwidth]{FutureLinearColliders/Plots/CDRPlots/BackgroundAngleCut.pdf}
\caption[]{Angular distribution of number of particles for beam induced backgrounds for CLIC at $\sqrt{s} = 3$ TeV.  Taken from CLIC CDR.}
\label{fig:backgroundangle}
\end{figure}
  %% CLIC Vertex Work
% \chapter{Capacitively Coupled Pixel Detectors for the CLIC Vertex Detector}
\label{chap:theory}

\chapterquote{There, sir! that is the perfection of vessels!}
{Jules Verne, 1828--1905}

\section{Introduction}

Successful identification of heavy-flavour quarks and tau-leptons relies upon precise reconstruction of the secondary displaced vertices produced in the decay of these particles as well as accurate association of the daughter tracks to those vertices.  To achieve this for the CLIC experiment very high spatial resolution, of approximately 3 {\mu}m and good geometric coverage extending to low $\theta$ values are essential.  The vertex detector must also have a low material budget, less than 0.2 $\text{X}_{0}$ per layer, as to not impact the performance of the other sub detectors and a low occupancy, aided by time-tagging to an accuracy of 10 ns, to counteract the high beam-induced backgrounds found near the impact point.  

There are no commercially available technology options that fulfil all the criteria for the vertex detector, which had led the CLIC experiment to consider a variety of new technology options.  The focus of this chapter is the use of high voltage complementary metal-oxide-semiconductor (HV-CMOS) active sensors coupled to a separate readout ASIC for the CLIC vertex detector.  

% Active sensor does something to signal when recorded, passive just records signel.  HV-CMOS is active as there's an amplification of signal step.

\subsection{HC-CMOS}
There are two classifications for pixel detectors; hybrid detectors where a passive sensor is bump-bonded to a separate readout chip and fully integrated where the collection diode is built upon the same wafer as the readout circuitry.  Both of these technology options find the CLIC experimental conditions extremely challenging.  Hybrid technologies struggle to achieved both the radiation tolerance and the functionality in the readout circuitry, while fully integrated circuits have too slow readout times due to limitations on the applied bias voltage.  

HV-CMOS is adapted to the CLIC experimental conditions as the n-MOS and p-MOS transistors forming the integrated amplifier (or generic in-pixel logic operations) for collecting the signal are embedded within a deep n-well, as shown in figure \ref{fig:hvcmos}.  This acts as both the collection diode as well as providing shielding to the circuitry from the beam induced radiation.  With the integrated circuitry shielded from the p-substrate it becomes possible to apply a large bias voltage to the substrate to widen the depletion region meaning that the main part of any signal deposited in the detector will be transferred via drift as opposed to diffusion, which provides the fast readout times required by the CLIC experiment.  

\begin{figure}
\centering
\includegraphics[width=0.5\textwidth]{CLICdpVertex/Plots/HV-CMOSDiagram.pdf}
\caption[HV-CMOS diagram.]{HV-CMOS diagram.}
\label{fig:hvcmos}
\end{figure}

HV-CMOS devices are strong candidates for the CLIC vertex detector, however, they do have limitations such as noise from interference between the n and p doped wells of the n-MOS and p-MOS transistors that sit within the deep n well.  This noise will grow with the number of n-MOS and p-MOS devices on the wafer and so ultimately restricts the complexity of the in-pixel operations that can be performed.  There are also topological difficulties such as the difficulty of applying the CMOS process to all sizes and the fact that the deep n well does not occupy the full space of the pixel.  

To minimise the material budget for the vertex detector, the pixels used are designed to be as thin as possible.  This means the signal from the HV-CMOS will be small as the depletion region will be thin.  To counter this, in-pixel signal amplification was applied to the HV-CMOS devices, as shown in figure \ref{fig:ccpdandclicpix}.  This increases the signal going to the readout ASIC, which also counteracts the intrinsically small capacitance between the HV-CMOS and readout ASIC.

\begin{figure}
\centering
\includegraphics[width=0.5\textwidth]{CLICdpVertex/Plots/schematic.pdf}
\caption[Schematic of CCPDv3 and CLICpix pixels.]{Schematic of CCPDv3 and CLICpix pixels.}
\label{fig:ccpdandclicpix}
\end{figure}

\subsection{CLICpix}

The readout ASIC in this study is the CLICpix, which is a charge integrating amplifier connected to a discriminator as shown in figure \ref{fig:ccpdandclicpix}.  The output to this discriminator is then used as the input for further logic operations that record the magnitude, using a Time over Threshold (ToT) measurement, and time of arrival of the collected charge.

\section{Construction}
Description of construction information based on fabrication note.

\section{Device Characterisation}

\subsection{CLICPix Calibration}
Compare the HV-CMOS pulse height to the ToT recorded on the CLICPix using strontium 90 sources.  Gives indication of gluing layer and CLICPix capacitances.  

\begin{itemize}
\item ToT vs pulse heights.  
\item ToT vs rise times.  
\end{itemize}

\begin{figure}
\centering
\includegraphics[width=0.5\textwidth]{CLICdpVertex/Plots/TargetToT_vs_PulseHeight.pdf}
\caption[Average ToT vs pulse height.]{Average ToT vs pulse height.}
\label{fig:avgtotvspulseheight}
\end{figure}

\begin{figure}
\centering
\includegraphics[width=0.5\textwidth]{CLICdpVertex/Plots/RiseTime_vs_PulseHeight.pdf}
\caption[Rise time vs pulse height.]{Rise time vs pulse height.}
\label{fig:risetimevspulseheight}
\end{figure}

\subsection{Cross Couplings}
ToT on adjacent cells vs pulse heights.  No charge sharing apparent except for SET16.  Possible issues with manufacturing other offset samples as some charge sharing is expcted.

\begin{figure}
\centering
\includegraphics[width=0.5\textwidth]{CLICdpVertex/Plots/ToT_X_vs_PulseHeight.pdf}
\caption[Average ToT on adjacent pixel vs pulse height.]{Average ToT on adjacent pixel vs pulse height.}
\label{fig:avgtotadjvspulseheight}
\end{figure}

\subsection{Test Pulse Calibration}
Inject pulse height of fixed size directly into CLICPix and recored ToT.  This cannot be done for the HV-CMOS due to the device construction preventing getting to the relevant input to the HV-CMOS.  Plots of average ToT vs pulse height, describe surrogate function fit and column structure.  

\section{Test Beam Analysis}
\subsection{Test Beam Area}
Description of test beam, site and telescope.

\subsection{Efficiency}

\begin{itemize}
\item Description of masks and why they need to be applied.
\item Alignment description.
\item Efficiency calculations and conclusions. 
\end{itemize}

\begin{figure}
\centering
\includegraphics[width=0.5\textwidth]{CLICdpVertex/Plots/ZoomedEfficiency.pdf}
\caption[Efficiency vs threshold.]{Efficiency vs threshold.}
\label{fig:efficiency}
\end{figure}




  

  %% Energy Estimators
% \chapter{Energy Estimators}
\label{chap:energyestimators}

\chapterquote{There, sir! that is the perfection of vessels!}
{Jules Verne, 1828--1905}

%========================================================================================
%========================================================================================

\section{Calibration}
\label{sec:calibration}
%========================================================================================

\subsection{Calibration in the particle flow paradigm}
% Results use detector model 85 and calibration variant 71

In any experiment, calibration is essential for ensuring reliability in measured quantities and the linear collider will be no exception to this.  In the particle flow paradigm measured energy deposits fall into two distinct classes (i) calorimetric energy deposits and (ii) charged particle tracks.  Calorimetric energy deposits are the essential building blocks for the application of the particle flow.  The separation of energy deposits from charged and neutral particles in the calorimeters is crucial for achieving good energy resolutions and this is only possible if the energy estimators for those energy deposits are accurate.  

%The other crucial energy deposit used in particle flow calorimetry are track energy deposits.  These are also crucial to physics performance, however, in the particle flow paradigm these energy deposits are topologically related to the energy of the reconstructed particle.  A spatial helix fit is applied to the track energy deposits which when combined with knowledge of the magnetic filed yields the momentum of the particle producing the track.  Therefore, there is no direct relationship between the energy deposited by the monte-carlo particle in the active medium and the energy of the reconstructed energy.  Therefore, precise calibration of the energy deposited by a charged particle track is less crucial than for calorimeter energy deposits.  For this reason the focus of this chapter is on the calibration of calorimeter energy deposits. 

Calibration of the linear collider detector simulation extends beyond the calorimeter hits and into the particle flow algorithm itself.  The fine calorimeter granularity required for particle flow calorimetry yields excellent separation of hadronic and electromagnetic showers.  The pattern recognition software used in the reconstruction, PandoraPFA, provides detailed particle identification that allows for distinct treatments of electromagnetic and hadronic particle shower energy estimators.  This distinction can be used to produce a response from the calorimeter that is compensating, i.e. a calorimeter that has an identical response to electromagnetic and hadronic showers initiated by particles of the same incident energy, despite the intrinsic response being non-compensating.  This would significantly improve the energy resolution of the detector and extend the physics potential of the linear collider experiment.  Therefore, incorporated within the calibration procedure is the setting of several, energy independent, rescaling factors.  These factors are applied to the energy measurements from electromagnetic and hadronic particle showers with the aim of achieving a compensating calorimeter response.  

It is possible to make further improvements to the energy resolution by applying energy corrections during the reconstruction.  With that aim in mind a study into two energy correction options will be presented.  These studies show the effect of the corrections on the energy resolution of the detector as well as the pattern recognition aspect of the reconstruction.  The first energy correction to be studies is a naive energy truncation for calorimeter cells in the HCal, while the second, software compensation, is a more sophisticated approach involving the use of the energy density of the calorimeter hits to determine an energy correction.  

The chapter concludes with a discussion of the impact of timing cuts applied to calorimeter hits in the software trigger for the linear collider.   

Details regarding how all the detector performance metrics used in this chapter are calculated can be found in section \ref{sec:optstudiesmetric}.

%========================================================================================

\subsection{Calibration and detector optimisation}
Optimising the detector at a future linear collider will be crucial to exploit the full physics potential available to it.  An extensive optimisation of the calorimeters was performed and the results can be found in chapter \ref{sec:optimisationstudies}.  For each detector model considered in this study the calibration procedure outlined in this section was applied to ensure optimal performance was achieved.  This made unbiased comparison between detector models performance possible and ensured reliability in the conclusions drawn from this study.

%========================================================================================

\subsection{Calibration Goals}
The calibration procedure aims to determine variables related to four aspects of the reconstruction, which are:

\begin{enumerate}
\item \textbf{Digitisation of calorimeter hits}.  Digitisation in this sense is the estimation of the energy deposited within a calorimeter cell, both the active and absorber layers, based on the signal measured in the sensitive region of the cell, the active layer.  
\item \textbf{Minimum ionising particle (MIP) scale setting in the digitisation processor and PandoraPFA}.  The MIP scale has to be set in the digitiser as it simulates the response of the readout technology, which includes a maximum readout value set in units of MIPs.  The digitiser also applies a minimum threshold on the active layer cell energy, in units of MIPs, for calorimeter hit to be created.  PandoraPFA uses the MIP scale to place further threshold cuts on the cell energy that must be exceeded for a calorimeter hit to be used in the reconstruction.  Both of these thresholds are designed to veto noise that would be present in a real detector.  While noise is not applied in these simulations, the cuts are applied to better reflect the performance of a real detector. 
\item \textbf{Electromagnetic and hadronic scale setting in PandoraPFA}.  As discussed in chapter CALORIMETER CHAPTER, the response of a calorimeter to electromagnetic and hadronic showers is different due to the fundamentally different mechanisms governing their propagation.  A key difference between the two is the presence of an invisible energy component within hadronic showers.  This leads to calorimeters to produce a lower response to hadronic showers than to electromagnetic showers initiated by particles of the same incident energy.  To account for this and any energy losses incurred due to the application of noise vetoing cuts in PandoraPFA the PFO energies are rescaled depending on whether the PFO has showered electromagnetically or hadronically.  Determination of these scaling factors is the setting of the electromagnetic and hadronic energy scales.  
\item \textbf{Retraining photon likelihood data}.  The PandoraPFA algorithm uses likelihood data to determine whether a reconstructed object is a photon.  This likelihood data has to be retrained every time the ECal is altered.
\end{enumerate}

The majority of these aspects must be addressed every time the detector model changes.  The ordering of each of these calibration steps is also crucial as it is possible to get interference between the different stages if applied in an arbitrary order.

%========================================================================================

\subsection{Digitisation}
\label{sec:digi}
Calibration of the digitisation of the calorimeter hits involves accurately estimating the energy deposited in a calorimeter cell, in both the active and absorber layers, based on the energy deposited in the sensitive element of the calorimeter, the active layer.  The relationship between the energy deposited in the active layers and the absorber layers of a calorimeter is linear as the energy deposited in both layers is proportional to the number of charged particle tracks passing through them.  This works under the assumption that the density of charged particle tracks across a calorimeter cell within a particle shower is uniform.  The ratio of the calorimeter cell energy to the energy deposited in the active layers is hereby called the digitisation constant.

The digitisation constant for a calorimeter depends upon factors including the material properties of the calorimeters, the magnetic field strength and energy losses occurring within the gaps between the active and absorber layers.  To account for the extra material in the detector due to the effect of instrumented read out technology additional material surrounding the active and absorber layers is added in the simulation of the calorimeters.  In comparison to the absorber layer, this extra material adds little to the detector, however the small energy losses incurred here will be accounted for by the digitisation calibration.  

%========================================================================================

\subsubsection{ECal Digitisation}
\label{sec:ecaldigi}
The procedure for determining the digitisation constants in the ECal involves simulation of single $\gamma$ events at energy $E_{MC} = 10$ GeV.  $\gamma$ events are ideal for the calibration of the ECal as $\gamma$ energy measurements are made primarily within the ECal.  Furthermore, at this energy, $\gamma$s are largely fully contained within the ECal, as shown in figure \ref{fig:ecaldigiphotonsplit}.  This makes them ideal for isolating the ECal digitisation calibration from that of the HCal digitisation calibration.   

\begin{figure}
\includegraphics[width=0.5\textwidth]{EnergyEstimators/Plots/Calibration/Digitsation/ECal/ECalHCalPhotonSplit.pdf}
\caption[The sum of calorimeter hit energies in ECal and HCal for 10 GeV $\gamma$ events.]{The sum of calorimeter hit energies in ECal and HCal for 10 GeV $\gamma$ events.}
\label{fig:ecaldigiphotonsplit}
\end{figure}

Events are only used for calibrating the ECal digitisation if they are confined to the ECal.  To that extent cuts are applied ensuring that the sum of any reconstructed energy found outside the ECal is less than 1\% of $E_{MC}$ and that the $\text{cos}(\theta) < 0.95$ where $\theta$ is the polar angle of the $\gamma$.  $\gamma$ conversions are also vetoed in this event sample at MC level.  The impact of these cuts on the sum of ECal hit energies for the $E_{MC} = 10$ GeV $\gamma$ events is shown in figure \ref{fig:ecaldigiselection}.

\begin{figure}
\includegraphics[width=0.5\textwidth]{EnergyEstimators/Plots/Calibration/Digitsation/ECal/DigitisationECalSelection.pdf}
\caption[The sum of the ECal calorimeter hit energies for 10 GeV $\gamma$ events with and without the selection cuts.]{The sum of the ECal calorimeter hit energies for 10 GeV $\gamma$ events with and without the selection cuts.}
\label{fig:ecaldigiselection}
\end{figure}

\begin{figure}
\includegraphics[width=0.5\textwidth]{EnergyEstimators/Plots/Calibration/Digitsation/ECal/DigitisationECalFit.pdf}
\caption[Gaussian fit to sum of the ECal calorimeter hit energies for 10 GeV $\gamma$ events with selection cuts.]{Gaussian fit to sum of the ECal calorimeter hit energies for 10 GeV $\gamma$ events with selection cuts.}
\label{fig:ecaldigifit}
\end{figure}

The calibration of the digitisation in the ECal is an iterative procedure and begins with the simulation of single $\gamma$ events using a trial calibration, with digitisation constant in the ECal $\alpha^{0}_{\text{ECal}}$ that may not be ideal.  Next the distribution of the sum of calorimeter hit energies within the ECal is produced for events passing the selection cuts, as shown in figure \ref{fig:ecaldigiselection}.  For an ideal calorimeter this distribution should be Gaussian, as was described in section CALORIMETER CHAPTER, therefore, a Gaussian fit is applied to this distribution and the mean, $E_{\text{Fit}}$, extracted.  To remove the effect of any outliers in this distribution, the fit is applied to the range of data with the smallest root mean square that contains at least 90 \% of the data.  An example of such a fit is shown in figure \ref{fig:ecaldigifit}.  In the case of ideal calibration the mean of this fit, $E_{\text{Fit}}$, would be equal $E_{MC}$.  It is assumed that any difference between the two is due to the calibration and to correct for this the digitisation constant from the trial calibration, $\alpha^{0}_{\text{ECal}}$, is rescaled by the ratio of the $E_{MC}$ to $E_{\text{Fit}}$.

\begin{equation}
\alpha^{0}_{\text{ECal}} \rightarrow \alpha_{\text{ECal}} = \alpha^{0}_{\text{ECal}} \times \frac{E_{MC}}{E_{Fit}}
\end{equation}

This procedure is then repeated until the $E_{\text{Fit}}$ falls within a specified tolerance.  The tolerance applied here was $|E_{\text{Fit}} - E_{\text{MC}}| < E_{\text{MC}} \times 5 \%$.  The binning used for the fitted histogram is chosen such that the bin width is equal to the desired tolerance on $E_{\text{Fit}}$ e.g. $E_{\text{MC}} \times 5 \% = 0.5$ GeV.  This tolerance is somewhat loose, however, it is tight enough to ensure successful application of PFA.  It should also be emphasised that the PFO energies used in downstream analyses have the electromagnetic and hadronic energy scale corrections applied, which are calibrated to a much tighter accuracy.

%========================================================================================

\subsubsection{HCal Digitisation}
\label{sec:hcaldigi}
The calibration for the digitisation in the HCal proceeds in a similar manor to that described for the ECal with a few key differences.  This calibration uses $K^{0}_{L}$ events at $E_{MC} = 20$ GeV as these neutral hadrons will deposit the bulk of their energy in the HCal.  The higher energy is used to create larger particle showers and sample deeper into the calorimeters.  

As in these events the $K^{0}_{L}$s have to pass through the ECal before arriving at the HCal and as the ECal contains $\approx 1 \lambda_{I}$, some of the $K^{0}_{L}$ begin showering in the ECal, as shown by figure \ref{fig:hcaldigikaonsplit}.  Such events are unsuitable for calibration of the HCal digitisation constants as rescaling $\alpha^{0}_{\text{HCal}}$ would not lead to a linear rescaling in $E_{\text{Fit}}$.  These events are vetoed in the even selections by applying selection cuts to ensure events used for the calibration deposit the bulk of their energy in the HCal.  

\begin{figure}
\includegraphics[width=0.5\textwidth]{EnergyEstimators/Plots/Calibration/Digitsation/HCal/ECalHCalKaon0LSplit.pdf}
\caption[Sum of calorimeter hit energies in ECal and HCal for 20 GeV $K^{0}_{L}$ events.]{Sum of calorimeter hit energies in ECal and HCal for 20 GeV $K^{0}_{L}$ events.}
\label{fig:hcaldigikaonsplit}
\end{figure}

Events are only considered in this analysis if a single neutral hadron PFO is reconstruction, the sum of any reconstructed energy found outside the HCal is less than 5\% of $E_{MC}$ and the last layer of the HCal where energy is deposited is in the first 90\% of the HCal.  The cut on the last HCal layer where energy is deposited is applied to veto events that shower late in the HCal and deposit a significant amount of energy in the uninstrumented coil region of the detector.  The impact of these cuts on the sum of HCal calorimeter hit energies for the $E_{MC} = 20$ GeV $K^{0}_{L}$ events is shown in figure \ref{fig:hcaldigiselection}.

\begin{figure}
\includegraphics[width=0.5\textwidth]{EnergyEstimators/Plots/Calibration/Digitsation/HCal/DigitisationHCalSelection.pdf}
\caption[Sum of the HCal calorimeter hit energies for a 20 GeV $K^{0}_{L}$ events with and without the selection cuts.]{Sum of the HCal calorimeter hit energies for a 20 GeV $K^{0}_{L}$ events with and without the selection cuts.}
\label{fig:hcaldigiselection}
\end{figure}

There are two HCal digitisation constants used in the detector simulation, one applied for the Barrel and another for the EndCap.  This is to account for differences in hadronic shower dynamics between the two, such as differing magnetic field configurations in the Barrel and EndCap.  Both parameters are calibrated in the same manor, but have different cuts on $\theta$, the polar angle of the $K^{0}_{L}$.  For the Barrel region of the HCal events are selected if $0.2 < \text{cos}(\theta) < 0.6$, while for the EndCap events are selected if $0.8 < \text{cos}(\theta) < 0.9$.  These angular cuts are conservative to account for the transverse profile of the hadronic showers and ensure that they are confined to the relevant sub-detector.

Using these cuts the calibration procedure for the digitisation of the HCal Barrel and EndCap proceeds in the same manor as was described for the ECal, the details of which can be found in section \ref{sec:ecaldigi}.  Examples of the Gaussian fits applied to the sum of the calorimeter hit energies in the HCal Barrel and EndCap can be found in figure \ref{fig:hcaldigifit}.  

A noteworthy difference to the ECal digitisation procedure is that the target reconstructed energy for the $K^{0}_{L}$ samples is the kinetic energy as opposed to the total energy.  This decision was made as the majority of the neutral hadrons appearing in jets are neutrons and their accessible energy, what they can deposit in the detector, is their kinetic energy and not their rest mass energy.  This is the case as neutrons will come to a rest rather than decaying in the detector.  Therefore, calibrating to the kinetic energy should give the best performance for jet reconstruction.  

\begin{figure}
\subfloat[HCal Barrel.]{\label{fig:hcaldigibarrel}\includegraphics[width=0.5\textwidth]{EnergyEstimators/Plots/Calibration/Digitsation/HCal/DigitisationHCalBarrelFit.pdf}}
\subfloat[HCal EndCap.]{\label{fig:hcaldigiendcap}\includegraphics[width=0.5\textwidth]{EnergyEstimators/Plots/Calibration/Digitsation/HCal/DigitisationHCalEndCapFit.pdf}}
\caption[Gaussian fit to sum of the HCal calorimeter hit energies for 20 GeV $K^{0}_{L}$ events with selection cuts.]{Gaussian fit to sum of the HCal calorimeter hit energies for 20 GeV $K^{0}_{L}$ events with selection cuts.}
\label{fig:hcaldigifit}
\end{figure}

%========================================================================================

\subsubsection{HCal Ring Digitisation}
\label{sec:hcalringdigi}

\begin{figure}
\subfloat[]{\label{fig:ecal}\includegraphics[width=0.33\textwidth]{EnergyEstimators/Plots/Calibration/VisualDisplay/ECal.png}}
\subfloat[]{\label{fig:hcal}\includegraphics[width=0.33\textwidth]{EnergyEstimators/Plots/Calibration/VisualDisplay/HCal.png}}
\subfloat[]{\label{fig:hcalring}\includegraphics[width=0.33\textwidth]{EnergyEstimators/Plots/Calibration/VisualDisplay/HCalRing.png}}
\caption[A PandoraPFA event display showing the nominal ILD calorimeters.  \protect\subref{fig:ecal} show the ECal, \protect\subref{fig:hcal} shows the full HCal and \protect\subref{fig:hcalring} shows the HCal Ring.]{A PandoraPFA event display showing the nominal ILD calorimeters.  \protect\subref{fig:ecal} show the ECal, \protect\subref{fig:hcal} shows the full HCal and \protect\subref{fig:hcalring} shows the HCal Ring.}
\label{fig:calorimeters}
\end{figure}

The HCal Ring, illustrated in figure \ref{fig:calorimeters}, also has an independent digitisation constant to account for any difference in the hadronic shower development between the Ring, Barrel and EndCap.  The procedure used to calibrate this constant has to differs from that presented in section \ref{sec:hcaldigi} as it is unfeasible, due to the depth of the ring, to produce events that are wholly contained within it.  Fortunately, the size of the HCal ring means it plays a minimal role in the reconstruction, so precise calibration is not crucial.  To ensure that the calibration is approximately correct for the HCal Ring, $\alpha_{\text{HCal Ring}}$ is assumed to equal $\alpha_{\text{HCal EndCap}}$ multiplied by several factors designed to accounts for changes in the active layer thickness, absorber layer thickness and the MIP response between the HCal EndCap and Ring.  In detail:

\begin{equation}
\alpha_{\text{HCal Ring}} = \alpha_{\text{HCal EndCap}} \times \frac{\langle \text{cos}(\theta_\text{EndCap}) \rangle}{\langle \text{cos}(\theta_\text{Ring}) \rangle} \times \frac{P_\text{EndCap} }{P_\text{Ring} } \times \frac{L^{Absorber}_\text{EndCap}}{L^{Absorber}_\text{Ring} } \times \frac{L^{Active}_\text{Ring}}{L^{Active}_\text{EndCap}}
\end{equation}

where $\theta$ is the incident angle of the incoming particle to the calorimeter cells determined using the 20 GeV $K^{0}_{L}$ events, $L^{Active}$ is the active layer thickness and $L^{Absorber}$ is the absorber layer thickness. $P$ is the position of the MIP peak in the distribution of active layer cell energies, which has been corrected so that the MIP appears to enter the cell at normal incidence, and is determined using 10 GeV $\mu^{-}$ events.  Details on how $P$ is determined can be found in section \ref{sec:mipresponse}.

%========================================================================================

\subsection{MIP Scale Setting}
\label{sec:mipresponse}
The response of the various sub-detectors to a MIP has to be determined for both the digitisation processor and for PandoraPFA as both apply cuts in units of MIP response.  The digitiser applies cuts related to the electronic readout range of the various active layer technology options and applies a threshold on the minimum active layer energy for the creation of calorimeter hits.  PandoraPFA applies cuts designed to veto noise that would be present in a real detector.  Both these MIP responses, while intrinsically linked, have to be calculated separately as the digitiser requires the MIP peak definition from the active layer cell energies while, PandoraPFA requires the definition from the full cell, active and absorber layer, energies.  In these studies a MIP was defined as a 10 GeV $\mu^{-}$ \cite{Bichsel:2004ej} and no selection cuts applied to the sample.  

For the digitiser the MIP scale was defined as the, non-zero, peak in the distribution of the active layer calorimeter cell energies for normally incident $\mu^{-}$ as shown in figure \ref{fig:digitisermip}.  This distribution was produced using a sample of $\mu^{-}$ events that are spatially isotropic about the impact point.  A direction correction factor, $\text{cos}(\theta)$ where $\theta$ is the incident angle of the incoming $\mu^{-}$ to the calorimeter cell, was applied to the active layer cell energies to generate the effect of having normally incident $\mu^{-}$.  

\begin{figure}
\subfloat[ECal.]{\label{fig:digitisermipecal}\includegraphics[width=0.5\textwidth]{EnergyEstimators/Plots/Calibration/MIPScale/Digitiser/MIPScaleDigitiserECal.pdf}}
\subfloat[HCal Barrel.]{\label{fig:digitisermiphcalbarrel}\includegraphics[width=0.5\textwidth]{EnergyEstimators/Plots/Calibration/MIPScale/Digitiser/MIPScaleDigitiserHCalBarrel.pdf}} \\
\subfloat[HCal EndCap.]{\label{fig:digitisermiphcalendcap}\includegraphics[width=0.5\textwidth]{EnergyEstimators/Plots/Calibration/MIPScale/Digitiser/MIPScaleDigitiserHCalEndCap.pdf}}
\subfloat[HCal Ring.]{\label{fig:digitisermiphcalring}\includegraphics[width=0.5\textwidth]{EnergyEstimators/Plots/Calibration/MIPScale/Digitiser/MIPScaleDigitiserHCalOther.pdf}}
\caption[The active layer calorimeter cell energy distributions for \protect\subref{fig:digitisermipecal} the ECal, \protect\subref{fig:digitisermiphcalbarrel} the HCal Barrel, \protect\subref{fig:digitisermiphcalendcap} the HCal EndCap and \protect\subref{fig:digitisermiphcalring} the HCal Ring for 10 GeV $\mu^{-}$ events.]{The active layer calorimeter cell energy distributions for \protect\subref{fig:digitisermipecal} the ECal, \protect\subref{fig:digitisermiphcalbarrel} the HCal Barrel, \protect\subref{fig:digitisermiphcalendcap} the HCal EndCap and \protect\subref{fig:digitisermiphcalring} the HCal Ring for 10 GeV $\mu^{-}$ events.}
\label{fig:digitisermip}
\end{figure}

In the digitiser processor a single value for the MIP peak was required for the HCal and that was taken as the MIP peak position for the HCal Barrel.  The MIP peaks were separately calculated for the HCal EndCap and Ring for the purposes of the HCal Ring digitisation described in section \ref{sec:hcalringdigi}.  The realistic digitisation features present in the simulation of the ECal and HCal are not available for the muon chamber simulation, therefore, no MIP peak setting for that digitisation step is required.

A similar procedure was employed for calculation of the MIP peak in PandoraPFA.  One important difference was the distribution used for setting the MIP scale in PandoraPFA is the distribution of calorimeter cell energies, i.e. the energy in the active and absorber layers of a cell, and not just the active layer energies.  Examples of the distributions used to set the MIP scale in PandoraPFA can be found in figure \ref{fig:pandoramip}.  There are few populated low calorimeter cell energy bins due to cuts applied in the digitiser on the minimum active layer energy required to make a calorimeter hit.  The double peak structure in the ECal calorimeter hit energy distribution is present due to the doubling of the thickness of the ECal absorber material, from 2.1 mm to 4.2 mm tungsten, in the ILD detector model that occurs for the back 10 layers of the 30 layer ECal.  Further differences between the MIP scale setting in the digitiser and PandoraPFA worthy of note are that in PandoraPFA the HCal MIP scale setting combines the HCal sub-detectors, the Barrel, EndCal and Ring, together and PandoraPFA requires the MIP scale to be set in the muon chamber, which requires the muon chamber cell energy distribution to be created.  

\begin{figure}
\subfloat[]{\label{fig:pandoramipecal}\includegraphics[width=0.5\textwidth]{EnergyEstimators/Plots/Calibration/MIPScale/PandoraPFA/MIPScalePandoraPFAECal.pdf}}
\subfloat[]{\label{fig:pandoramiphcal}\includegraphics[width=0.5\textwidth]{EnergyEstimators/Plots/Calibration/MIPScale/PandoraPFA/MIPScalePandoraPFAHCal.pdf}} \\
\subfloat[]{\label{fig:pandoramipmuon}\includegraphics[width=0.5\textwidth]{EnergyEstimators/Plots/Calibration/MIPScale/PandoraPFA/MIPScalePandoraPFAMuon.pdf}}
\caption[The calorimeter cell energy distributions for \protect\subref{fig:pandoramipecal} the ECal, \protect\subref{fig:pandoramiphcal} the HCal and \protect\subref{fig:pandoramipmuon} the muon chamber for 10 GeV $\mu^{-}$ events.]{The calorimeter cell energy distributions for \protect\subref{fig:pandoramipecal} the ECal, \protect\subref{fig:pandoramiphcal} the HCal and \protect\subref{fig:pandoramipmuon} the muon chamber for 10 GeV $\mu^{-}$ events.}
\label{fig:pandoramip}
\end{figure}

%========================================================================================

\subsection{Electromagnetic and hadronic scale setting}
\label{sec:scalesetting}
The electromagnetic and hadronic scales have to be independently set in the simulation to account for the different mechanisms governing the propagation of electromagnetic and hadronic showers.  The setting of the scales involves tuning four parameters in PandoraPFA that correspond to energy rescaling factors, which are applied to energy measurements from electromagnetic and hadronic showering particles in the ECal and HCal.  

%========================================================================================

\subsubsection{Electromagnetic scale setting}
\label{sec:emscalesetting}
The electromagnetic scale in the ECal, $\beta^{EM}_{ECal}$, is determined using $\gamma$ events at $E_{MC} = 10$ GeV.  $\gamma$ events are ideal for the setting of the electromagnetic scale as they procedure electromagnetic showers and are primarily confined to the ECal at the energy considered, which was shown in figure \ref{fig:ecaldigiphotonsplit}.  

Cuts are applied to ensure that only events where the bulk of the energy is deposited within the ECal are used for this part of the calibration.  Less than 1\% of the reconstructed energy is outside the ECal to ensure the event is contained.  Furthermore, cuts requiring a single $\gamma$ be reconstructed are added to veto pattern recognition failures.  $\gamma$ conversions are excluded at MC level to ensure energy measurements used in the calibration arise from the calorimeters and not the charged particle tracks.  The impact of these cuts on the electromagnetic energy measured in the ECal for 10 GeV $\gamma$ events is shown in figure \ref{fig:ecalemscaleselection}.

\begin{figure}
\includegraphics[width=0.5\textwidth]{EnergyEstimators/Plots/Calibration/EMScaleSetting/EMScaleECalSelection.pdf}
\caption[Sum of the electromagnetic energy measured in the ECal for 10 GeV $\gamma$ events with and without the selection cuts.]{Sum of the electromagnetic energy measured in the ECal for 10 GeV $\gamma$ events with and without the selection cuts.}
\label{fig:ecalemscaleselection}
\end{figure}

\begin{figure}
\includegraphics[width=0.5\textwidth]{EnergyEstimators/Plots/Calibration/EMScaleSetting/EMScaleSettingECalFit.pdf}
\caption[Gaussian fit to sum of the electromagnetic energy deposited in the ECal for 10 GeV $\gamma$ events with selection cuts.]{Gaussian fit to sum of the electromagnetic energy deposited in the ECal for 10 GeV $\gamma$ events with selection cuts.}
\label{fig:ecalemscalefit}
\end{figure}

The fitting procedure follows that used for the ECal digitisation, described in section \ref{sec:ecaldigi}, whereby a trial calibration for the electromagnetic energy scale in the ECal, $\beta^{EM0}_{ECal}$, is assumed and the single $\gamma$ events simulated.  The distribution of the electromagnetic energy in the ECal is created and a Gaussian fit applied to the range of data with the smallest root mean square containing at least 90 \% of the data.  The mean of this fit, $E_{\text{Fit}}$, is then used to scale $\beta^{EM0}_{ECal}$ in the following way

\begin{equation}
\beta^{EM0}_{ECal} \rightarrow \beta^{EM}_{ECal} = \beta^{EM0}_{ECal} \times \frac{E_{MC}}{E_{Fit}}
\end{equation}

An example distribution and fit used in the calibration of the nominal ILD detector model can be found in figure \ref{fig:ecalemscalefit}.  This procedure is repeated using the updated $\beta^{EM}_{ECal}$ until $E_{\text{Fit}}$ falls within a specified tolerance.  The tolerance applied here was $|E_{\text{Fit}} - E_{\text{MC}}| < E_{\text{MC}} \times 0.5 \%$.  The binning for the fitted histogram is chosen such that the bin width is equal to the desired target tolerance on $E_{\text{Fit}}$ e.g. $E_{\text{MC}} \times 0.5 \% = 0.05$ GeV.  This tolerance is tighter than was applied for the digitisation as it is these energies that are used in downstream analyses.   
 
The electromagnetic scale in the HCal, $\beta^{EM}_{HCal}$, is chosen to be equal to the hadronic scale in the HCal, $\beta^{Had}_{HCal}$.  The details of the determination of $\beta^{Had}_{HCal}$ can be found in section \ref{sec:hadscalesetting}.  For the ILC and CLIC, $\beta^{EM}_{HCal}$ is not a critical parameter in the reconstruction as photons are largely contained within the ECal meaning little to no electromagnetic energy is measured in the HCal.  

%========================================================================================

\subsubsection{Hadronic scale setting}
\label{sec:hadscalesetting}
The hadronic scale in the ECal, $\beta^{Had}_{ECal}$, is important to detector performance as a non-negligible amount of hadronic energy will be measured in the ECal.  As the ECal contains $\approx 1 \lambda_{I}$, the hadronic scale in the ECal cannot be independently set as it is unfeasible to create a large sample of 20 GeV $K^{0}_{L}$ events that are fully contained within it.  Therefore, the hadronic scale in the ECal and HCal have to be set simultaneously.  

Cuts are applied to select the $K^{0}_{L}$ events that are appropriate to use for the hadronic scale calibration.  The last layer in which energy is deposited in the HCal must not occur in the back 10 \% of the HCal.  This ensures that the event does not suffer from leakage of energy out of the back of the HCal.  A single neutral hadron must be reconstructed to veto events with reconstruction failures.  Finally, the total hadronic energy measured in ECal and HCal, $E^{Had}_{ECal} + E^{Had}_{HCal}$, must fall within 3 $\sigma$ of the desired hadronic energy distribution, $E^{Had}_{ECal} + E^{Had}_{HCal} = 20 \text {GeV} - m_{K^{0}_{L}} = E_{K}$.  $\sigma$ is defined to be $55\% \times \sqrt{E} = 2.46 \text{GeV}$ for 20 GeV $K^{0}_{L}$.  This definition for sigma is used as it matches the energy resolution as a function of energy for neutral hadrons using the nominal ILD HCal \cite{Behnke:2013lya}.  This cut ensures that when fitting the two dimensional distribution of hadronic energy measured in the ECal and HCal outliers do not skew the fit.  Once again the target reconstructed energy for this sample is the kinetic energy and not the total energy of the $K^{0}_{L}$ for the reasons outlined in section \ref{sec:hcaldigi}.  The impact of cuts are illustrated in figure \ref{fig:hadscaleselection}.

\begin{figure}
\subfloat[]{\label{fig:hadscaleselectionnocuts}\includegraphics[width=0.5\textwidth]{EnergyEstimators/Plots/Calibration/HadScaleSetting/HadScaleECalHCalSelectionNoCuts.pdf}}
\subfloat[]{\label{fig:hadscaleselectioncuts}\includegraphics[width=0.5\textwidth]{EnergyEstimators/Plots/Calibration/HadScaleSetting/HadScaleECalHCalSelectionCuts.pdf}}
\caption[The distribution of hadronic energy measured in the ECal and HCal for 20 GeV $K^{0}_{L}$ events with and without selection cuts.]{The distribution of hadronic energy measured in the ECal and HCal for 20 GeV $K^{0}_{L}$ events \protect\subref{fig:hadscaleselectionnocuts} without selection cuts and \protect\subref{fig:hadscaleselectioncuts} with selection cuts.}
\label{fig:hadscaleselection}
\end{figure}

This part of the calibration procedure is again iterative and begins by assuming trial values, $\beta^{Had0}_{ECal}$ and $\beta^{Had0}_{ECal}$, for the hadronic scale calibration factors $\beta^{Had}_{ECal}$ and $\beta^{Had}_{ECal}$.  The 20 GeV $K^{0}_{L}$ events are then simulated and reconstructed.  Following that a linear fit to the distribution of $E^{Had}_{ECal}$ against $E^{Had}_{HCal}$ for 20 GeV $K^{0}_{L}$ events passing the selection cuts is applied.  The fit is performed by minimising $\chi^{2}$, which is defined as

\begin{equation}
\chi^{2}(\delta^{Had}_{ECal}, \delta^{Had}_{HCal}) = \sum_{i} \frac{x_{i}}{\sigma_{x_{i}}}
\end{equation}

where $x_{i}$ is the perpendicular distance from $E^{Had}_{ECal}$ and $E^{Had}_{HCal}$ for event $i$ to the line $E^{Had}_{HCal} = \delta^{Had}_{HCal} - E^{Had}_{ECal} \frac{\delta^{Had}_{HCal}}{\delta^{Had}_{ECal}}$.   The definition of $x_{i}$ is given in equation \ref{equ:xicalc}, but best illustrated by considering figure \ref{fig:hadscalechi2calc}.  $\sigma_{x_{i}}$ is the uncertainty on $x_{i}$, which is calculated by propagating the uncertainties on $E^{Had}_{ECal}$ and $E^{Had}_{HCal}$, which are assumed to be $\sigma_{E^{Had}_{E/HCal}} = 55\% \times \sqrt{E^{Had}_{E/HCal}}$, into the expression for $x_{i}$.  The result of this propagation of errors is given in equation \ref{equ:sigmaxicalc}.  The sum runs over all events, $i$, passing the selection cuts.  

\begin{figure}
\includegraphics[width=0.5\textwidth]{EnergyEstimators/Plots/Calibration/HadScaleSetting/HadScaleECalHCalSelectionExample.pdf}
\caption[An example showing the definition of $x_{i}$, the variable used for the calculation of $\chi^{2}(\delta^{Had}_{ECal}, \delta^{Had}_{HCal})$ for the setting of the hadronic energy scale.]{An example showing the definition of $x_{i}$, the variable used for the calculation of $\chi^{2}$ for the setting of the hadronic energy scale.  For an event that has been measured with hadronic energy $E^{Had}_{ECal}$ in the ECal and $E^{Had}_{HCal}$ in the HCal, the geometric interpretation of $x_{i}$ is shown.  The blue dotted line is defined as $E^{Had}_{HCal} = \delta^{Had}_{HCal} - E^{Had}_{ECal} \frac{\delta^{Had}_{HCal}}{\delta^{Had}_{ECal}}$.}
\label{fig:hadscalechi2calc}
\end{figure}

\begin{equation}
x_{i} = \frac{E^{Had}_{HCal} \delta^{Had}_{ECal} + E^{Had}_{ECal} \delta^{Had}_{HCal} - \delta^{Had}_{ECal} \delta^{Had}_{HCal}}{\sqrt{(\delta^{Had}_{ECal})^{2} + (\delta^{Had}_{HCal})^{2}}}
\label{equ:xicalc}
\end{equation}

\begin{equation}
\sigma_{i} = \frac{(\sigma_{E^{Had}_{HCal}}  \delta^{Had}_{ECal})^{2} + (\sigma_{E^{Had}_{ECal}} \delta^{Had}_{HCal})^{2}}{\sqrt{(\delta^{Had}_{ECal})^{2} + (\delta^{Had}_{HCal})^{2}}}
\label{equ:sigmaxicalc}
\end{equation}

The minimisation is done by stepping a range of $\delta^{Had}_{ECal}$ and $\delta^{Had}_{HCal}$ centred about the ideal value of $20 \text { GeV} - m_{K^{0}_{L}}$ in search for the minimum $\chi^{2}$.  Once the minima in $\chi^{2}$ is found the trial calibration factors $\beta^{Had0}_{ECal}$ and $\beta^{Had0}_{ECal}$ are rescaled to correct for any deviation from the desired fit as follows

\begin{equation}
\beta^{Had0}_{ECal} \rightarrow \beta^{Had}_{ECal} = \beta^{Had0}_{ECal} \times \frac{E_{K}}{\Delta^{Had}_{ECal}} \\
\beta^{Had0}_{HCal} \rightarrow \beta^{Had}_{HCal} = \beta^{Had0}_{HCal} \times \frac{E_{K}}{\Delta^{Had}_{HCal}}
\end{equation}

where $\Delta^{Had}_{ECal}$ and $\Delta^{Had}_{ECal}$ are the values of $\delta^{Had}_{ECal}$ and $\delta^{Had}_{ECal}$ giving the minimum $\chi^{2}$.  The step sizes used for minimising $\chi^{2}$ with respect to $\delta^{Had}_{ECal}$ and $\delta^{Had}_{ECal}$ is chosen such that a single step corresponds to the target final tolerance on $\delta^{Had}$ i.e. $|\delta^{Had}_{E/HCal} - E_{\text{MC}}| < E_{\text{MC}} \times 0.5 \% \approx 0.1 \text{GeV}$.  

This procedure is then repeated until $\Delta^{Had}_{ECal}$ and $\Delta^{Had}_{ECal}$ both fall within a given tolerance, which in this case it taken to be $|\Delta^{Had}_{E/HCal} - E_{\text{MC}}| < E_{\text{MC}} \times 0.5 \% \approx 0.1 \text{GeV}$

%========================================================================================

\subsection{Calibration step ordering}
\label{sec:orderingcalib}

The calibration procedure has to be run in the following order so that the building blocks used for each stage of the calibration procedure can be assumed to be correctly calibrated:

\begin{enumerate}
\item MIP Scale setting in the digitiser as described in section \ref{sec:mipresponse}.
\item Calibration of digitisation of calorimeter hits in the ECal and HCal as described in section \ref{sec:digi}.
\item MIP Scale setting in PandoraPFA as described in section \ref{sec:mipresponse}.
\item Electromagnetic and hadronic scale settings in PandoraPFA as described in section \ref{sec:scalesetting}.
\end{enumerate}

%========================================================================================

\subsection{Retraining photon likelihood data}
PandoraPFA uses likelihood data in the identification of $\gamma$s.  Data related to the topology and energy of electromagnetic showers and the wider event environment is used to determine whether a given shower is likely to be a $\gamma$.  The likelihood data is trained using off-shell mass Z boson (Z') events at 500 GeV that decay into light quarks (u, d, s).  It is necessary to retrain this data only when varying the ECal as photons are contained largely within the ECal at the energies being considered and the likelihood data only uses measurements made in the ECal.

As this data uses post digitisation hits it is important to ensure that a fully calibrated detector is used when retraining the likelihood data.  Therefore, it is necessary to run the calibration procedure, as described in section \ref{sec:orderingcalib}, before retaining.  However, as the reconstruction uses likelihood data the calibration procedure must be performed twice.  Initially the calibration procedure is performed where PandoraPFA is run without the inclusion of this likelihood data, using PandoraSettingsMuon.xml.  Then the likelihood data is retrained using the results of the first calibration pass and then the retrained likelihood data is used in the second pass of the calibration procedure.  

In the optimisation studies presented in chapter OPTIMISATION STUDIES this procedure was followed whenever the ECal was modified so that optimal detector performance was achieved.  

%========================================================================================
%========================================================================================

\section{HCal Cell Truncation}
\label{sec:hcalcelltruncation}
A powerful tool used by PandoraPFA in event reconstruction is the application of a truncation to the maximum amount of energy that can be recorded in a calorimeter cell in the HCal.  The purpose of this truncation is to eliminate the effect of spuriously high energy calorimeter cells that would skew the reconstruction.  The origin of these high energy cells is twofold: showering particles may be moving within the plane of the active material, which can lead to an overestimation of the deposited energy if the shower is not sufficiently uniform across the full cell, and Landau fluctuations \cite{Landau:1944if}, which originate from high energy knock-on electrons appearing within particle showers \cite{Bichsel:2004ej}.  These effects are only relevant in hadronic showers as electromagnetic showers begin showering almost immediately within the ECal, meaning they are unlikely to be directed within the plane of the active material of the calorimeter, and the Landau distribution only describes the energy loss for charged particles.  This means showering $\gamma$ do not produce Landau fluctuations and neither do $e^{-}$ and $e^{+}$ as their energy, in the particle flow paradigm, arises from the curvature of the track they produce and not the calorimeter hits.  Therefor, as only hadronic showers produce spuriously high energy cells through these mechanisms, this truncation is only applied on HCal cells.  

A great deal of care has to be given to the truncation energy so that cells from typical hadronic shower development are not truncated, while the spuriously high energy cells are.  This can be illustrated by considering the single particle energy resolutions as a function of the HCal cell truncation that are shown in figure \ref{fig:ercelltrunc}.  The photon energy resolution is invariant to changes in the cell truncation as no photon energy is recorded in the HCal, while the energy resolution for the neutral hadrons is optimal for a truncation of 1 GeV.  This indicates that a  GeV truncation is sufficient for dealing with the Landau fluctuations and showering particles travelling within the active material of the calorimeter cell.  For cell truncations greater than 1 GeV the resolution degrades as the spuriously high energy cells are not accounted for, while truncations below 1 GeV truncate energy measurements from typical hadronic shower development.  If typical hadronic shower cells are truncated this leads to a poorer sampling of the shower and a degradation in the energy resolution.  

\begin{figure}
\subfloat[]{\label{fig:ercelltruncphotons}\includegraphics[width=0.5\textwidth]{EnergyEstimators/Plots/CellTruncation/ER_vs_PhotonCellTrunc_100GeVPhoton.pdf}}
\subfloat[]{\label{fig:ercelltrunckaons}\includegraphics[width=0.5\textwidth]{EnergyEstimators/Plots/CellTruncation/ER_vs_Kaon0LCellTrunc_50GeVKaon0L.pdf}}
\caption[The energy resolution as a function of HCal cell truncation for \protect\subref{fig:ercelltruncphotons} 100 GeV $\gamma$ events and \protect\subref{fig:ercelltrunckaons} 50 GeV $K^{0}_{L}$ events for the nominal ILD detector model.]{The energy resolution as a function of HCal cell truncation for \protect\subref{fig:ercelltruncphotons} 100 GeV $\gamma$ events and \protect\subref{fig:ercelltrunckaons} 50 GeV $K^{0}_{L}$ events for the nominal ILD detector model.}
\label{fig:ercelltrunc}
\end{figure}

Once again, this effect propagates into the jet energy resolutions as shown by figure \ref{fig:jercelltrunc}.  The trends in this plot are complex as the optimal cell truncation varies with the jet energy.  At low energies a 0.5 GeV truncation gives the best performance, however, when the jet energies reach $\approx$ 180 GeV a 1-2 GeV truncation giving the best performance.  This is to be expected based on the Landau fluctuations.  The Landau distribution is essentially a Gaussian with a high energy tail and as the jet energy increases the definition of cell energies falling in the high energy tail changes.  Therefore, the definition of the cells requiring truncating changes as a function of jet energy and a procedure with more degrees of freedom than a single truncation value is required to fully account for this.  As particle showers grow in size with increasing incident particle energy it is expected that the problem of particles travelling within the active material plane, and not uniformly across the rest of the cell, should not change significantly with changes to the jet energy.  

When examining the breakdown of the jet energy resolutions into the intrinsic energy resolution and the confusion it was noted that the cell truncation improves both the intrinsic energy resolution and the confusion contribution to the jet energy resolution.  This indicates that the pattern recognition performed by PandoraPFA benefits from the absence of these spuriously high energy cells, which if not omitted can skew energy comparisons made in the reconstruction.  Any skewed energy comparisons in the reconstruction leads to inaccurate association of calorimeter hits to charged particle tracks.  In turn this causes double counting and/or omission of energy deposits in the calorimeters leading to a degradation of the energy resolution.  

\begin{figure}
\includegraphics[width=0.5\textwidth]{EnergyEstimators/Plots/CellTruncation/JER_vs_JetEnergy_HCalCellTruncation.pdf}
\caption[The jet energy resolution as a function of jet energy for various hadronic cell truncations.  The results shown use the nominal ILD detector model.]{The jet energy resolution as a function of jet energy for various hadronic cell truncations.  The results shown use the nominal ILD detector model.}
\label{fig:jercelltrunc}
\end{figure}

While it is challenging to determine the optimal performance for a given detector model it is clear that applying an appropriate truncation produces significant improvement in detector performance.  Therefore, for the optimisation studies presented in section OPT STUDIES, the performance of each detector model is determined using a range of HCal cell truncations and the optimal resolutions quoted.  The HCal cell truncations considered in the optimisation were 0.5, 0.75, 1, 1.5, 2, 5, 10, and $10^{6}$ GeV (semi-infinite).  For the HCal cell size study the optimal performance for the 10, 20, 30, 40, 50 and 100 mm HCal cell size detector models was achieved using a 0.5, 0.75, 1, 1.5, 2 and 5 GeV truncation, for the tungsten HCal options the optimal truncation was 5 GeV and for all other detector models considered the optimal truncation was 1 GeV.  This optimisation has a significant impact on detector optimisation, which can be seen by comparing the jet energy resolutions using the optimised cell truncation and a uniform 1 GeV truncation, as found in \ref{fig:jerhcalcellopt}.  Without this optimisation of cell truncation the significance of the HCal cell size is overinflated and could have led to a misinformed detector design choice.  

\begin{figure}
\subfloat[]{\label{fig:jerhcalcelloptgoodtrunc}\includegraphics[width=0.5\textwidth]{OptimisationStudies/Plots/JetEnergyResolutions/JER_vs_HCalCellSize.pdf}}
\subfloat[]{\label{fig:jerhcalcelloptbadtrunc}\includegraphics[width=0.5\textwidth]{EnergyEstimators/Plots/CellTruncation/JER_vs_HCalCellSizeBadTruncation.pdf}}
\caption[The jet energy resolution as a function of HCal cell size using \protect\subref{fig:jerhcalcelloptgoodtrunc} an optimised HCal cell truncation and \protect\subref{fig:jerhcalcelloptbadtrunc} a fixed 1 GeV truncation.]{The jet energy resolution as a function of HCal cell size using \protect\subref{fig:jerhcalcelloptgoodtrunc} an optimised HCal cell truncation and \protect\subref{fig:jerhcalcelloptbadtrunc} a 1 GeV truncation.}
\label{fig:jerhcalcellopt}
\end{figure}

%========================================================================================
%========================================================================================

\section{Software Compensation}
\label{sec:softcomp}
As discussed in chapter CALORIMETERS CHAPTER, the response of a calorimeter to a hadronic shower is different to that of an electromagnetic showers.  The particle shower produced by a hadron when passing through a calorimeter has two components,\cite{Wigmans:2000vf}; an electromagnetic shower core, which originates from the production and decay of $\pi^{0}$, and a hadronic shower component originating from all other interacting and decaying hadrons in the shower.  Hadronic showers have an "invisible" energy component caused by various factors such as neutrons stopping within the calorimeter and nuclear binding energy losses.  This leads to a reduced response from a calorimeter to a hadronic showering particle in comparison to an electromagnetic showering particle of the same energy.  An event display showing the high energy density electromagnetic core of a hadronic cluster for a 500 GeV Z$\rightarrow$uds di-jet event can be found in figure \ref{fig:softcompeventdisplay}.  This different calorimetric response will lead to a degradation in the energy resolution if not properly compensated for.  

\begin{figure}
\subfloat[]{\label{fig:softcompfulleventdisplay}\includegraphics[width=0.5\textwidth]{EnergyEstimators/Plots/SoftComp/VisualDisplay/SoftComp1.png}}
\subfloat[]{\label{fig:softcompclustereventdisplay}\includegraphics[width=0.3\textwidth]{EnergyEstimators/Plots/SoftComp/VisualDisplay/SoftComp3.png}}
\caption[An event display for a 500 GeV Z$\rightarrow$uds di-jet event reconstructed using the nominal ILD detector.  \protect\subref{fig:softcompfulleventdisplay} shows the full event environment.  \protect\subref{fig:softcompclustereventdisplay} shows a single hadronic cluster from the same event where shading indicates the energy density in the HCal.  High energy density cells are coloured red, while lower energy density cells are coloured blue.  All ECal hits are shaded black.  The high energy density electromagnetic core of the selected hadronic cluster is clearly visible.]{An event display for a 500 GeV Z$\rightarrow$uds di-jet event reconstructed using the nominal ILD detector.  \protect\subref{fig:softcompfulleventdisplay} shows the full event environment.  \protect\subref{fig:softcompclustereventdisplay} shows a single hadronic cluster from the same event where shading indicates the energy density in the HCal.  High energy density cells are coloured red, while lower energy density cells are coloured blue.  All ECal hits are shaded black.  The high energy density electromagnetic core of the selected hadronic cluster is clearly visible.}
\label{fig:softcompeventdisplay}
\end{figure}

This compensation can be applied either at the hardware level, whereby a calorimeter is made to be intrinsically compensating, or at the software level, whereby hadronic showers are identified and their energy estimators modified.  

An example of hardware compensation would be the ZEUS calorimeter \cite{Derrick:1991tq} that was constructed using uranium as the absorber material.  In response to neutral hadrons the uranium underwent fission producing extra energy, which raises the hadronic response of the calorimeter.  The amount of uranium was carefully chosen to achieve a fully compensating calorimeter response i.e. identical calorimeter response to electromagnetic and hadronic showers.  While hardware compensation is possible for the linear collider calorimeters, restrictions on calorimeter construction and the use of a large amount of radioactive material are highly undesirable.  

The high granularity calorimeters and sophisticated pattern recognition software used at the linear collider give excellent resolution on individual particle showers.  This resolution means software compensation can be applied at the linear collider with greater effectiveness than has been possible for previous collider experiments.  

%========================================================================================

\subsection{Application}
The software compensation technique applied in this study involves reweighing HCal hits based on their energy density and the energy of the cluster those hits belong to.  This is applied in the PandoraPFA framework in the form of an energy correction function, which in effect means whenever the energy of a cluster of hits is considered by PandoraPFA the software compensated energy is used.  Applying software compensation in this way benefits the detector energy resolution in two ways; firstly the intrinsic energy resolution of the detector improves and secondly the confusion contribution to the energy resolution, from incorrect association of charged particle tracks to calorimeter hit clusters, is reduced.  Lowering the confusion benefits the energy resolution as it decreases the number calorimeter hits where the energy of the hit is effectively double counted, if charged particle hits are not associated to a track, or not counted at all, if neutral hadron hits are associated to a track.   

Software compensation is applied to clusters of calorimeter hits, as opposed to being applied directly to the final output PFOs, so that the more accurate energy estimators can be used during the reconstruction.  For a cluster of calorimeter hits with an initial energy of $E_{\text{Raw}}$, which is  calculated by summing the calorimeter hit energies, the software compensated cluster energy, $E_{SoftComp}$ \cite{Adloff:2012gv}, is given by 

\begin{equation}
E_{SoftComp} = E_{ECal} + \sum_{i} E_{i} \times \omega_{i}(E_{\text{Raw}}, \rho_{i}) + E_{\text{Muon Chamber}}
\label{equ:softcomp}
\end{equation}

where $E_{ECal}$ is the sum of the calorimeter hit energies measured in the ECal, $E_{i}$ and $\rho_{i}$ are the energy and energy density of HCal hit $i$ respectively, $\omega_{i}$ is the software compensation weight applied to hit $i$, $E_{\text{Muon Chamber}}$ is the cluster energy recorded in the muon chamber and the sum runs over all hits, $i$, in the HCal.  The weight function $\omega_{i}(E_{\text{Raw}}, \rho_{i})$ is defined as

\begin{equation}
\omega_{i}(E_{\text{Raw}}, \rho_{i}) = p_{1}(E_{\text{Raw}}) \times exp(p_{2}(E_{\text{Raw}}) \times \rho_{i}) + p_{3}(E_{\text{Raw}}) \\
p_{1} = p_{11} + p_{12} \times E_{\text{Raw}} + p_{13} \times E_{\text{Raw}}^{2} \\
p_{2} = p_{21} + p_{22} \times E_{\text{Raw}} + p_{23} \times E_{\text{Raw}}^{2} \\
p_{3} = \frac{p_{31}}{p_{32} + exp(p_{33} \times E_{\text{Raw}})}
\label{equ:softcompweight}
\end{equation}

where $p_{ij}$ are trained parameters.  The parameters $p_{ij}$ are determined by performing a $\chi^{2}$ fit of $E_{SoftComp}$ to the MC energy for samples of $K^{0}_{L}$ ranging from 10 to 100 GeV in steps of 10 GeV.  Using the fitted parameters, $p_{1}$,  $p_{2}$ and $p_{3}$ as a function of $E_{\text{Raw}}$ and $\omega(E_{\text{Raw}}, \rho)$ as a function of $\rho$ for various $E_{\text{Raw}}$ are shown in figure \ref{fig:softcompparams} and \ref{fig:softcompweights} respectively.  

\begin{figure}
\subfloat[]{\label{fig:softcompparam1}\includegraphics[width=0.33\textwidth]{EnergyEstimators/Plots/SoftComp/Weights/SoftwareCompensationParam1.pdf}}
\subfloat[]{\label{fig:softcompparam2}\includegraphics[width=0.33\textwidth]{EnergyEstimators/Plots/SoftComp/Weights/SoftwareCompensationParam2.pdf}}
\subfloat[]{\label{fig:softcompparam3}\includegraphics[width=0.33\textwidth]{EnergyEstimators/Plots/SoftComp/Weights/SoftwareCompensationParam3.pdf}}
\caption[]{}
\label{fig:softcompparams}
\end{figure}

\begin{figure}
\includegraphics[width=0.5\textwidth]{EnergyEstimators/Plots/SoftComp/Weights/SoftwareCompensationWeights.pdf}
\caption[The software compensation weight applied to a calorimeter hit as a function of calorimeter hit energy density for various cluster energies.]{The software compensation weight applied to a calorimeter hit as a function of calorimeter hit energy density for various cluster energies.}
\label{fig:softcompweights}
\end{figure}

As software compensation only modifies the energy of HCal hits there is freedom to apply further energy corrections to the ECal hits.  Application of the Clean Clusters energy correction logic, described in section \ref{sec:legacycorrections}, to the ECal hits alongside software compensation gave further improvements to the jet energy resolution.  Therefore, as standard the application of software compensation within PandoraPFA implicitly involves the application of the Clean Clusters logic to the ECal hits.  

Software compensation is trained using a maximum $K^{0}_{L}$ energy of 100 GeV, therefor, it is only applied to clusters where $E_{\text{Raw}} < 100$ GeV as the behaviour outside this range cannot be guaranteed.  While it would be possible to modify the energy range of the training sample to go to higher energies, hadronic clusters with energy greater than 100 GeV will be rare for the use case considered here.  This is the case as the ILD detector was designed for use at the ILC, which has a maximum running energy of 500 GeV.  

%========================================================================================

\subsection{Results}

%========================================================================================

\subsubsection{Legacy Energy Corrections}
\label{sec:legacycorrections}
Before examining the impact of software compensation on detector performance is it necessary to address the 'legacy' energy corrections that are used as default in PandoraPFA.  The three energy correction that were in use prior to the development of software compensation are:

\begin{itemize}
\item \textbf{HCal cell truncation}, the details of which can be found in section \ref{sec:hcalcelltruncation}.
\item \textbf{Clean Clusters}.  This algorithm checks to see whether the energy measured within a calorimeter hit is anomalously high.  Anomalously high energy cells are defined as cells where the energy contained within the cell is greater than 10\% of the energy of the cluster that the cell has been associated to.  If a cell is deemed to have an anomalously high energy and if this energy is above a threshold (0.5 GeV) the cell energy used by PandoraPFA is modified.  The updated cell energy is taken as the average cell energy in the calorimeter layers immediately before and after the layer containing the high energy cell.    
\item \textbf{Scale Hot Hadrons}.  This algorithm calculates the average number of MIP equivalent particles passing through each calorimeter cell in a cluster.  If this number is larger than a given value, default 15 MIPs per cell, the cluster energy is rescaled to give a lower average number of MIPs per hit, default is 5 MIPs per hit.  
\end{itemize}

Each of these energy corrections help to deal with the effects of spuriously high energy cells the origin of which is described in section \ref{sec:hcalcelltruncation}.  However, the algorithms are simplistic and software compensation is expected to give far better results than these 'legacy' options.  

The optimisation studies presented in section OPTIMISATION STUDIES use all three of these legacy options simultaneously, which was the default behaviour for PandoraPFA when the studies were undertaken.  The new default behaviour in PandoraPFA is to use software compensation.

%========================================================================================

\subsubsection{Energy Resolution}
\label{sec:softcomper}
The energy resolution as a function of the MC energy for single $K^{0}_{L}$ events is shown in figure \ref{fig:ersoftcomp} using various energy correction settings.  

When comparing the energy resolution given by software compensation to that obtained using no energy corrections, it can be seen that software compensation offers a gain of $\approx 2 \%$ in energy resolution across all energies considered.  The uniformity of this improvement is encouraging, indicating software compensation has been successfully trained across this energy range.   

Comparing the performance of software compensation to the legacy corrections it can be seen that software compensation gives a better energy resolution across almost the entire range of energies considered.  The only exception to this is around $E_{K^{0}_{L}} \approx 50$ where the performance of software compensation and the legacy corrections are comparable.  By removing the cell truncation from the legacy options it is clear that the changes in energy resolution when using the legacy options are being driven by the cell truncation.  This makes the trend in energy resolution observed using the legacy corrections clear as at low $K^{0}_{L}$ energies very few cells are affected by the truncation so the performance is comparable to not using any energy corrections.  At high $K^{0}_{L}$ energies the truncation is too aggressive and removes energy from cells that are not spuriously high leading to a worsening energy resolution.  Between these two extremes, $E_{K^{0}_{L}} \approx 50$, the truncation works ideally and improvement in energy resolution using the legacy corrections is the largest.  

\begin{figure}
\includegraphics[width=0.5\textwidth]{EnergyEstimators/Plots/SoftComp/EnergyResolution/ER_vs_Kaon0LSoftComp_Kaon0L.pdf}
\caption[The energy resolution as a function of the MC energy for single $K^{0}_{L}$ events using various energy correction settings.  The detector model used was the nominal ILD detector model.]{The energy resolution as a function of the MC energy for single $K^{0}_{L}$ events using various energy correction settings.  The detector model used was the nominal ILD detector model.}
\label{fig:ersoftcomp}
\end{figure}

%========================================================================================

\subsubsection{Jet Energy Resolution}

The improvements in the intrinsic energy resolution of the detector observed when using software compensation will propagate into the reconstruction of jets.  These effects are illustrated by examining the jet energy resolution as a function of jet energy, which is shown in figure \ref{fig:jersoftcomp}.  Again it is clear that software compensation is extremely beneficial to the detector performance.  

\begin{figure}
\includegraphics[width=0.5\textwidth]{EnergyEstimators/Plots/SoftComp/JetEnergyResolution/JER_vs_JetEnergy_Default.pdf}
\caption[The jet energy resolution as a function of the jet energy for a variety of different energy correction options.  These results were produced for the nominal ILD detector model.]{The jet energy resolution as a function of the jet energy for a variety of different energy correction options.  These results were produced for the nominal ILD detector model.}
\label{fig:jersoftcomp}
\end{figure}

Software compensation gives a significant reduction in the jet energy resolution in comparison to using no energy corrections.  It also reduces the jet energy resolution in comparison to using the legacy corrections.  

Further light can be shed on these trends by examining the contribution to the jet energy resolutions from the intrinsic energy resolution and the pattern recognition confusion, which are shown in figure \ref{fig:jerbreakdownsoftcomp}.  The intrinsic energy resolution contribution shows that software compensation is significantly better than all other energy corrections options, which is to be expected from the energy resolution studies presented in section \ref{sec:softcomper}.  Unlike the single particle study there is no jet energy for which the cell truncation matches the performance obtained using software compensation.  This is due to the fact that the energy resolution when using the cell truncation is only comparable to the energy resolution using software compensation for a narrow range of hadronic cluster energies.  As the jet contains a broad spectrum of hadronic cluster energies the performance obtained when using the cell truncation will always be worse than when using software compensation.  When comparing the jet energy resolution for the legacy corrections is again apparent that the term driving the jet energy resolution is the cell truncation.

The confusion contribution to the jet energy resolution when using software compensation and the legacy corrections is almost identical.  This indicates that the improvement seen in the jet energy resolution, shown in figure \ref{fig:jersoftcomp}, when using software compensation as opposed to the legacy corrections is being driven by improvements to the intrinsic energy resolution.  

At low jet energies the Clean Clusters and Scale Hot Hadrons energy corrections are beneficial at reducing the confusion contribution, while the cell truncation is largely redundant.  For high jet energies jets this trend is reversed.  As the use of the Clean Clusters and Scale Hot Hadrons energy corrections do not alter the intrinsic energy resolution of the detector it is apparently that these energy corrections are purposed to account for failures in the pattern recognition that occur largely at low jet energies.  On the other hand the cell truncation and software compensation techniques aim to improve the energy resolution of the hadronic clusters, which has a knock-on effect of improving the track cluster associations made in the pattern recognition.  These corrections work across all energy ranges, but have a greater impact at high energies.  

By extracting the Clean Clusters logic, which is the driving term reducing the confusion contribution to the jet energy resolution at low jet energies, and embedding it within the software compensation energy correction, it is possible to achieve exceptional jet energy resolutions that will extend the physics reach of the linear collider detector.  

\begin{figure}
\subfloat[]{\label{fig:jerbreakdownsoftcomp1}\includegraphics[width=0.5\textwidth]{EnergyEstimators/Plots/SoftComp/JetEnergyResolution/JER_vs_JetEnergy_PerfectPFA.pdf}}
\subfloat[]{\label{fig:jerbreakdownsoftcomp2}\includegraphics[width=0.5\textwidth]{EnergyEstimators/Plots/SoftComp/JetEnergyResolution/JER_vs_JetEnergy_TotalConfusion.pdf}}
\caption[The contributions to the jet energy resolution as a function of the jet energy for a variety of different energy correction options.  \protect\subref{fig:jerbreakdownsoftcomp1} is the intrinsic energy resolution of the detector and \protect\subref{fig:jerbreakdownsoftcomp2} is the total confusion term.  The quadrature sum of both yields the standard reconstruction performance.  These results were produced for the nominal ILD detector model.]{The contributions to the jet energy resolution as a function of the jet energy for a variety of different energy correction options.  \protect\subref{fig:jerbreakdownsoftcomp1} is the intrinsic energy resolution of the detector and \protect\subref{fig:jerbreakdownsoftcomp2} is the total confusion term.  The quadrature sum of both yields the standard reconstruction performance.  These results were produced for the nominal ILD detector model.}
\label{fig:jerbreakdownsoftcomp}
\end{figure}

%========================================================================================
%========================================================================================

\section{Timing Cuts}
The ILC and CLIC will operate using a trigger-less readout approach whereby the recorded data for each sub-detector is readout between collisions of $\text{e}^{+}$ and $\text{e}^{-}$ bunches.  The train structure for the ILC and CLIC at maximum operating energy is shown in table \ref{table:trainstructure}.  Event selection will proceed through the application of a software trigger.  This involves the identification of hard interactions, prior to full event reconstruction, and only putting data into the event reconstruction if it is measured within a chosen time window about this interaction.  Timing cuts placed on the calorimeter hits are corrected for straight time-of-flight to the IP.  This ensures that the amount of time particle showers have to develop in the calorimeters is independent of their position.  As the size of the time window around the hard interaction changes the amount of time particle showers have to develop varies and this will affect the performance of the detector. 

% THIS IS NEW!
% Maybe worth adding to timing chapter
%\begin{figure}[h!]
%\centering
%\includegraphics[width=0.5\textwidth]{OptimisationStudies/Plots/Description/CalorimeterHitTimes_91GeV_Z_uds_Steel.pdf}
%\caption[The distribution of the time of the calorimeter hits, corrected for time of flight to the impact point, for 91 GeV Z$\rightarrow$uds di-jet events.]{The distribution of the time of the calorimeter hits, corrected for time of flight to the impact point, for 91 GeV Z$\rightarrow$uds di-jet events.}
%\label{fig:calohittiming}
%\end{figure} 

\begin{table}[h!]
\centering
\begin{tabular}{l r r}
\hline
& ILC 500 GeV & CLIC 3 TeV \\
\hline
Electrons per bunch & 2.0 & 0.37 \\
Bunches per train & 2810 & 312 \\
Train repetition rate [Hz] & 5 & 50 \\
Bunch separation [ns] & 308 & 0.5 \\
\end{tabular}
\caption[The train structure for 500 GeV ILC and 3 TeV CLIC.]{The train structure for 500 GeV ILC and 3 TeV CLIC.  CITE}
\label{table:trainstructure}
\end{table}

For all choices of time window considered in this study the calibration procedure was reapplied.  This means that the mean of the reconstructed energy distributions will be invariant to changes in the calorimeter timing window as the calibration procedure compensates for any energy losses incurred by truncating the particle shower development time.  

The energy resolution for 100 GeV $\gamma$ and 50 GeV $K^{0}_{L}$ events as a function of the timing window applied to the calorimeter hits is shown in figure \ref{fig:ertimingcuts} for the nominal ILD detector model .  The timing cut makes little difference to the energy resolution of the $\gamma$ events as they produce electromagnetic showers that develop rapidly.  However, there is a significant decrease in the energy resolution for the neutral hadrons, which is expected as these showers develop much more slowly.  Truncating the measurement of the hadronic showers by having a small time window leads to a reduced sampling of the shower, as those hits passing the time window are no longer measured, and a broadening of the reconstructed energy distribution.  

\begin{figure}
\subfloat[]{\label{fig:ertimingcutsphotons}\includegraphics[width=0.5\textwidth]{EnergyEstimators/Plots/TimingCuts/ER_vs_PhotonTiming_100GeVPhoton.pdf}}
\subfloat[]{\label{fig:ertimingcutskaons}\includegraphics[width=0.5\textwidth]{EnergyEstimators/Plots/TimingCuts/ER_vs_Kaon0LTiming_50GeVKaon0L.pdf}}
\caption[The energy resolution as a function of calorimeter timing window for \protect\subref{fig:ertimingcutsphotons} 100 GeV $\gamma$ events and \protect\subref{fig:ertimingcutskaons} 50 GeV $K^{0}_{L}$ events for the nominal ILD detector model.]{The energy resolution as a function of calorimeter timing window for \protect\subref{fig:ertimingcutsphotons} 100 GeV $\gamma$ events and \protect\subref{fig:ertimingcutskaons} 50 GeV $K^{0}_{L}$ events for the nominal ILD detector model.}
\label{fig:ertimingcuts}
\end{figure}

The degradation in neutral hadron energy resolution with decreasing calorimeter time window also affects the jet energy resolution, which can be seen in figure \ref{fig:jertimingcuts}.  The sole exception to this is the 250 GeV jets for the 100 ns time window whereby the jet energy resolution is slightly better than when using the 300 ns and semi-infinite time windows.  As the magnitude of the changes to the jet energy resolution when varying the time window size are small in comparison to the absolute resolutions, this exception will most likely be due to a fluctuation in either the event sample used or in the reapplication of the calibration procedure.  

\begin{figure}
\includegraphics[width=0.5\textwidth]{EnergyEstimators/Plots/TimingCuts/JER_vs_JetEnergy_TimingCutStudies.pdf}
\caption[The jet energy resolution as a function of jet energy for various calorimeter timing cuts.  The results shown use the nominal ILD detector model.]{The jet energy resolution as a function of jet energy for various calorimeter timing cuts.  The nominal ILD detector model was used for this study.}
\label{fig:jertimingcuts}
\end{figure}

The time window applied to the calorimeter hits affects both the neutral hadron and jet energy resolutions with a larger timing window leading to better resolutions.  It can be seen that applying an aggressive choice of time window, such as 10 ns, damages the jet energy resolutions as many of the hadronic showers being sampled do not have time to fully develop.  However, even using a 10 ns timing cut the jet energy resolutions are still sufficiently low to give excellent detector performance.  Both the single particle and jet energy resolutions indicate that the majority of hadronic showers will have fully developed within 100 ns and that there are little gains to be made by extending the size of this window.  

For results presented in this chapter and the optimisation studies found in chapter OPT STUDIES a 100 ns timing window was applied across all models considered.  As the choice of timing window has yet to be finalised for the linear collider this value was chosen as it represents something that could be achieved using the readout technology options presently available CITE.  Furthermore, it adds additional realism to the detector simulation in comparison to omitting the effect of the calorimeter time window.  The categorisation of changes to the detector performance when varying the calorimeter timing window presented here can be used to discern the impact of changing the timing window used for the optimisation studies at a later date if so desired.  

%========================================================================================
%========================================================================================

  %% Optimisation Studies
% \chapter{Calorimeter Optimisation Studies}
\label{chap:detopt}

\chapterquote{The simple believes everything, but the prudent gives thought to his steps.}
{Proverbs 14:15}

%========================================================================================
%========================================================================================

\section{Introduction}
\label{sec:optimisationstudies}
This chapter describes the optimisation of the calorimeters used at the linear collider, with focus placed on obtaining the best energy resolution for jets.  Parameters such as the number of layers, cell size and material choices for the calorimeters are investigated.  Several global detector parameters such as the magnetic field strength and the inner radius of the ECal are also studied.  These parameters are not calorimeter specific, but affect the jet energy resolution obtained from particle flow. 

%========================================================================================
%========================================================================================

\section{Electromagnetic Calorimeter Optimisation}
\label{sec:ecal}
The purpose of an electromagnetic calorimeter (ECal) is to measure the energy deposits from electromagnetic showers.  The nominal ILD ECal, summarised in table \ref{table:defaultildecal}, is a silicon-tungsten sampling calorimeter.  It contains 29 readout layers and 24 radiation lengths ($\text{X}_{0}$), which is sufficient to contain all but the highest energy electromagnetic showers.  The absorber thickness of the last nine layers is twice that of the first 20 layers to reduce the number of readout channels and cost of the calorimeter.  The high longitudinal sampling frequency is crucial for the pattern recognition aspect of particle flow calorimetry, especially in the region where particle showers start developing.  

\begin{table}[h!]
\centering
\begin{tabular}{ l l}
\hline
Parameter & Default Value \\
\hline
Cell Size & $5 \times 5\text{ mm}^{2}$ square cells \\
Number of Layers & 29 readout layers \\
Active Material Choice & Silicon or Scintillator  \\
Active Material Thickness & 0.5 mm (Silicon) or 2 mm (Scintillator)  \\
Absorber Material Choice & Tungsten \\
Absorber Material Thickness & 20 layers of 2.1 mm followed by 9 layers of 4.2 mm \\
\hline
\end{tabular}
\caption[The configuration of the silicon and scintillator ECal options for the ILD detector model \cite{Behnke:2013lya}.]{The configuration of the silicon and scintillator ECal options for the ILD detector model \cite{Behnke:2013lya}.}
\label{table:defaultildecal}
\end{table}

The calorimeter performance was simulated for a number of detector models where the following detector parameters were varied:
\begin{itemize}
\item Cell size:  This is a vital aspect of the detector in the particle flow paradigm as smaller cell sizes leads to better separation between nearby showering particles, which helps to minimise the effect of confusion.  Modifying the cell size should have little effect on the intrinsic energy resolution of the detector.  
\item Longitudinal sampling frequency:  The longitudinal sampling frequency in the ECal was varied by changing the number of layers in the ECal while simultaneously changing the thicknesses of the layers such that the total depth, in radiation lengths, was held constant.  Increasing the number of layers in a sampling calorimeter means any particles showering within it are sampled more, which leads to a reduction in the stochastic contribution to the energy resolution.  Therefore, varying the number of layers is expected to change in intrinsic energy resolution of the calorimeter.  
\item Active material choice:  The options under consideration for the active sensor material are silicon or plastic scintillator.  As well as providing different intrinsic energy resolutions the readout mechanics of these two options are significantly different.  There is no clear prior knowledge as to which should provide better performance. 
\end{itemize}

%========================================================================================

\subsection{ECal Cell Size}
\label{sec:ecalcells}
Different detector models were considered where the cell size in the ECal was varied about the nominal value of $5 \times 5 \text{ mm}^{2}$ square cells.  The granularities considered were $3 \times 3 \text{ mm}^{2}$, $5 \times 5 \text{ mm}^{2}$, $7 \times 7 \text{ mm}^{2}$, $10 \times 10 \text{ mm}^{2}$, $15 \times 15 \text{ mm}^{2}$ and $20 \times 20 \text{ mm}^{2}$ square cells for both the silicon and scintillator active material options.  

The energy resolution, using 100~GeV photons, as a function of the ECal cell size is shown in figure \ref{fig:ecalsicellsize100gamma} for the silicon option and in figure \ref{fig:ecalsccellsize100gamma} for the scintillator option.  At this energy, the photons will be largely contained within the ECal and the reported energy resolution reflects solely the performance of the ECal.  For both the silicon and scintillator ECal options the energy resolution does not depend strongly on the ECal cell size.  This is to be expected as there is no change in the number of layers, which is the main factor in determining the energy resolution of a sampling calorimeter.  

The only statistically significant variation observed occurs for the scintillator ECal option.  A degradation in the energy resolution of $\sim10$\% is observed when reducing the ECal cell size from $5 \times 5 \text{ mm}^{2}$ to $3 \times 3 \text{ mm}^{2}$.  The most likely cause is the "dead" region in the active material, which represents the readout multi pixel photon counter (MPPC) \cite{arXiv:1006.3396}.  The MPPC occupies a fixed area of the cell, irrespective of cell size, and so the dead region of the cell fractionally increases as cell size is reduced.  The larger this dead region, the worse the sampling of the electromagnetic showers in the ECal and the worse the resolution.  While this effect will be present in all scintillator ECal options, it will only be significant for the small cell sizes when the dead region is fractionally the largest.    

\begin{figure}[h!]
\centering
\subfloat[]{\label{fig:ecalsicellsize100gamma}\includegraphics[width=0.5\textwidth]{OptimisationStudies/Plots/EnergyResolution/ER_vs_SiECalCellSize_100GeVPhoton.pdf}}
\subfloat[]{\label{fig:ecalsccellsize100gamma}\includegraphics[width=0.5\textwidth]{OptimisationStudies/Plots/EnergyResolution/ER_vs_ScECalCellSize_100GeVPhoton.pdf}}
\caption[The energy resolution as a function of ECal cell size for 100~GeV photons using the nominal ILD detector model with \protect\subref{fig:ecalsicellsize100gamma} the silicon and \protect\subref{fig:ecalsccellsize100gamma} the scintillator ECal option.]{The energy resolution as a function of ECal cell size for 100~GeV photons using the nominal ILD detector model with \protect\subref{fig:ecalsicellsize100gamma} the silicon and \protect\subref{fig:ecalsccellsize100gamma} the scintillator ECal option.}
\label{fig:ecalcellsizegamma}
\end{figure}

The ability to separate nearby electromagnetic particle showers within a calorimeter is limited by the Moli�re radius of the absorber material and the cell size.  The Moli�re radius controls the width of the electromagnetic shower, while the cell size controls how the transverse shower profile is sampled.  By reducing the cell size, it becomes easier to resolve nearby electromagnetic showers, which in turn reduces the effect of confusion.  Therefore, it is expected that the jet energy resolution will be sensitive to the ECal cell size, even though the intrinsic energy resolution is not.  The jet energy resolution as a function of ECal cell size is shown in figure \ref{fig:ecalsicellsize} for the silicon option and figure \ref{fig:ecalsccellsize} for the scintillator option.  There is a strong dependance on the ECal cell size, with smaller cell sizes leading to lower values of the jet energy resolution; the jet energy resolution for 250~GeV jets for both ECal options goes from $\sim 3.0\%$ to $\sim 4.3\%$ when the ECal cell size goes from $3 \times 3 \text{ mm}^{2}$ to $20 \times 20 \text{ mm}^{2}$.  The origin of this trend is best illustrated by considering the intrinsic energy resolution and confusion contributions to the jet energy resolution.  These contributions are shown as a function of ECal cell size for 45 and 250~GeV jets in figure \ref{fig:ecalcellsizebreak}.  It is clear from these contributions that the intrinsic energy resolution of the detector does not change when varying the cell size, which agrees with both prior expectations of calorimeter behaviour and the single particle energy resolution study.  As expected, it can be seen that the trend in jet energy resolution as a function of the ECal cell size is being driven purely by changes to the confusion contribution and, in particular, the confusion caused by the reconstruction of photons.  

\begin{figure}[h!]
\centering
\subfloat[]{\label{fig:ecalsicellsize}\includegraphics[width=0.5\textwidth]{OptimisationStudies/Plots/JetEnergyResolutions/JER_vs_SiliconECalCellSize.pdf}}
\subfloat[]{\label{fig:ecalsccellsize}\includegraphics[width=0.5\textwidth]{OptimisationStudies/Plots/JetEnergyResolutions/JER_vs_ScintillatorECalCellSize.pdf}} \hfill
\caption[The fractional jet energy resolution as a function of ECal cell size for various jet energies using the nominal ILD detector model with \protect\subref{fig:ecalsicellsize} the silicon and \protect\subref{fig:ecalsccellsize} the scintillator ECal option.]{The fractional jet energy resolution as a function of ECal cell size for various jet energies using the nominal ILD detector model with \protect\subref{fig:ecalsicellsize} the silicon and \protect\subref{fig:ecalsccellsize} the scintillator ECal option.}
\label{fig:ecalcellsize}
\end{figure}

\begin{figure}[h!]
\centering
\subfloat[]{\label{fig:ecalsicellsize45break}\includegraphics[width=0.5\textwidth]{OptimisationStudies/Plots/JetEnergyResolutions/JER_vs_SiliconECalCellSize_91GeV_DiJet_Breakdown.pdf}}
\subfloat[]{\label{fig:ecalsccellsize45break}\includegraphics[width=0.5\textwidth]{OptimisationStudies/Plots/JetEnergyResolutions/JER_vs_ScintillatorECalCellSize_91GeV_DiJet_Breakdown.pdf}} \hfill
\subfloat[]{\label{fig:ecalsicellsize250break}\includegraphics[width=0.5\textwidth]{OptimisationStudies/Plots/JetEnergyResolutions/JER_vs_SiliconECalCellSize_500GeV_DiJet_Breakdown.pdf}}
\subfloat[]{\label{fig:ecalsccellsize250break}\includegraphics[width=0.5\textwidth]{OptimisationStudies/Plots/JetEnergyResolutions/JER_vs_ScintillatorECalCellSize_500GeV_DiJet_Breakdown.pdf}}
\caption[Contributions to the jet energy resolution shown as function of ECal cell size using the nominal ILD detector model for \protect\subref{fig:ecalsicellsize45break} the silicon ECal option and 45~GeV jets, \protect\subref{fig:ecalsccellsize45break} the scintillator ECal option and 45~GeV jets, \protect\subref{fig:ecalsicellsize250break} the silicon ECal option and 250~GeV jets and \protect\subref{fig:ecalsccellsize250break} the scintillator ECal option and 250~GeV jets.  The black curves correspond to the standard reconstruction, the blue curves to the intrinsic energy resolution contribution to the jet energy resolution, the red curves to the confusion contribution to the jet energy resolution and the magenta curves to the confusion contribution to the jet energy resolution related solely to photon reconstruction.]{Contributions to the jet energy resolution shown as function of ECal cell size using the nominal ILD detector model for \protect\subref{fig:ecalsicellsize45break} the silicon ECal option and 45~GeV jets, \protect\subref{fig:ecalsccellsize45break} the scintillator ECal option and 45~GeV jets, \protect\subref{fig:ecalsicellsize250break} the silicon ECal option and 250~GeV jets and \protect\subref{fig:ecalsccellsize250break} the scintillator ECal option and 250~GeV jets.  The black curves correspond to the standard reconstruction, the blue curves to the intrinsic energy resolution contribution to the jet energy resolution, the red curves to the confusion contribution to the jet energy resolution and the magenta curves to the confusion contribution to the jet energy resolution related solely to photon reconstruction}
\label{fig:ecalcellsizebreak}
\end{figure}

It is clear that the ECal cell size is extremely important for jet energy measurements, although it has little bearing on the intrinsic energy resolution of the ECal.  Separation of the hadronic decays of the W and Z bosons, i.e. $\sigma_{E}/E \lesssim 3.8\%$ \cite{arXiv:0907.3577}, can be achieved across the jet energy range considered here using a maximum ECal cell size of $15 \times 15 \text{ mm}^{2}$.  However, as reducing the ECal cell size further continues to benefit the jet energy resolution, minimising the ECal cell size is desirable.

%========================================================================================

\subsection{ECal Longitudinal Sampling Frequency} 
\label{sec:ecalnlayers}
The detector performance was simulated where the number of layers in the ECal was varied, while keeping the total material budget ($\text{X}_{0}$) approximately constant.  This study was performed for both the silicon and scintillator active material options.  In all cases tungsten was used for the ECal absorber material and the active layer thicknesses were not changed from those used in the nominal ILD ECal summarised in table \ref{table:defaultildecal}.  The different ECal layouts considered are summarised in table \ref{table:nlayersecaloption}.  

\begin{table}[h!]
\centering
\begin{tabular}{ r r r r r r}
\hline
Total Number & $N_{Layers}$ & Absorber & $N_{Layers}$ & Absorber & Total  \\
of Layers & Region 1 & Thickness & Region 2 & Thickness & Thickness \\
$N_{\text{Layers ECal}}$ & & Region 1 [mm] & &  Region 2 [mm] &  [$\text{X}_{0}$] \\
\hline
30 & 20 & 2.10 & 9 & 4.20 & 22.77 \\
26 & 17 & 2.40 & 8 & 4.80 & 22.60 \\
20 & 13 & 3.15 & 6 & 6.30 & 22.47 \\
16 & 10 & 4.00 & 5 & 8.00 & 22.31\\
\hline
\end{tabular}
\caption[The longitudinal structure of the ECal models considered in the optimisation study.  The radiation length of tungsten absorber is 3.504~mm \cite{Olive:2016xmw}.  Note that a presampler layer contributes one extra layer to the cumulative number of layers.]{The longitudinal structure of the ECal models considered in the optimisation study.  The radiation length of tungsten absorber is 3.504~mm \cite{Olive:2016xmw}.  Note that a presampler layer contributes one extra layer to the cumulative number of layers.}
\label{table:nlayersecaloption}
\end{table}

The energy resolution, for 100~GeV photons, as a function of the number of layers in the ECal is shown in figure \ref{fig:ecalsinlayers100gamma} for the silicon option and in figure \ref{fig:ecalscnlayers100gamma} for the scintillator option.  When the number of layers is increased $\sigma_{E}$/E decreases, which is expected because the energy resolution for a sampling calorimeter is $\propto$ 1/$\sqrt{E \times N_{Layers}}$, where $E$ is the reconstructed energy and $N_{Layers}$ is the number of layers in the calorimeter.

\begin{figure}[h!]
\centering
\subfloat[]{\label{fig:ecalsinlayers100gamma}\includegraphics[width=0.5\textwidth]{OptimisationStudies/Plots/EnergyResolution/ER_vs_SiECalNLayers_100GeVPhoton.pdf}}
\subfloat[]{\label{fig:ecalscnlayers100gamma}\includegraphics[width=0.5\textwidth]{OptimisationStudies/Plots/EnergyResolution/ER_vs_ScECalNLayers_100GeVPhoton.pdf}}
\caption[The energy resolution as a function of number of layers in the ECal for 100~GeV photons using the nominal ILD detector model with \protect\subref{fig:ecalsinlayers100gamma} the silicon and \protect\subref{fig:ecalscnlayers100gamma} the scintillator ECal option.]{The energy resolution as a function of number of layers in the ECal for 100~GeV photons using the nominal ILD detector model with \protect\subref{fig:ecalsinlayers100gamma} the silicon and \protect\subref{fig:ecalscnlayers100gamma} the scintillator ECal option.}
\label{fig:ecalnlayersgamma}
\end{figure}

When the number of layers in the ECal is increased, the intrinsic energy resolution benefits; the intrinsic energy resolution of the ECal improves by $\sim 25\%$ in both ECal options when increasing the number of layers from 16 to 30.  This has the knock-on effect of reducing the confusion contribution to the jet energy resolution, which can be seen in figures \ref{fig:ecalsinlayers} and \ref{fig:ecalscnlayers} for the silicon and scintillator ECal options respectively.  In both cases, the jet energy resolution was found to improve when the number of layers in the ECal was increased; the jet energy resolution goes from $\sim 4.4$ to $\sim 3.6\%$ for the silicon option and from $\sim 4.1$ to $\sim 3.6\%$ for the scintillator option when increasing the number of layers from 16 to 30.  The magnitude of the change in jet energy resolution is dependent upon the jet energy, with a stronger dependancy being observed for low energy jets.  This is expected from the stochastic contribution to the energy resolution for a sampling calorimeter.  For high jet energies, changing the number of layers in the ECal does not significantly affect the jet energy resolution because the jet energy resolution is dominated by confusion.  For low jet energies, the stochastic contribution to the energy resolution is bigger making it possible to resolve the changes to it when varying the number of layers in the ECal.  

The decomposition of the jet energy resolution into the intrinsic energy resolution and confusion contributions for 45 and 250~GeV jets are shown, for both the silicon and scintillator ECal options, in figure \ref{fig:ecalnlayersbreak}.  As expected, the improvement to the intrinsic energy resolution seen when increasing the number of layers in the ECal leads to the knock-on effect of lowering the confusion.  However, significantly the magnitude of the change to the intrinsic energy resolution and confusion contributions to the jet energy resolution when varying the number of layers in the ECal are comparable in size.  This shows that pattern recognition is as important for detector performance in the particle flow paradigm than intrinsic energy resolution.  

\begin{figure}[h!]
\centering
\subfloat[]{\label{fig:ecalsinlayers}\includegraphics[width=0.5\textwidth]{OptimisationStudies/Plots/JetEnergyResolutions/JER_vs_SiliconECalNumberofLayers.pdf}}
\subfloat[]{\label{fig:ecalscnlayers}\includegraphics[width=0.5\textwidth]{OptimisationStudies/Plots/JetEnergyResolutions/JER_vs_ScintillatorECalNumberofLayers.pdf}} \hfill
\caption[The jet energy resolution as a function of number of layers in the ECal for various jet energies using the nominal ILD detector model with \protect\subref{fig:ecalsinlayers} the silicon and \protect\subref{fig:ecalscnlayers} the scintillator ECal option.]{The jet energy resolution as a function of number of layers in the ECal for various jet energies using the nominal ILD detector model with \protect\subref{fig:ecalsinlayers} the silicon and \protect\subref{fig:ecalscnlayers} the scintillator ECal option.}
\label{fig:ecalnlayers}
\end{figure}

\begin{figure}[h!]
\centering
\subfloat[]{\label{fig:ecalsinlayers45break}\includegraphics[width=0.5\textwidth]{OptimisationStudies/Plots/JetEnergyResolutions/JER_vs_SiliconECalNumberofLayers_91GeV_DiJet_Breakdown.pdf}}
\subfloat[]{\label{fig:ecalscnlayers45break}\includegraphics[width=0.5\textwidth]{OptimisationStudies/Plots/JetEnergyResolutions/JER_vs_ScintillatorECalNumberofLayers_91GeV_DiJet_Breakdown.pdf}} \hfill
\subfloat[]{\label{fig:ecalsinlayers250break}\includegraphics[width=0.5\textwidth]{OptimisationStudies/Plots/JetEnergyResolutions/JER_vs_SiliconECalNumberofLayers_500GeV_DiJet_Breakdown.pdf}}
\subfloat[]{\label{fig:ecalscnlayers250break}\includegraphics[width=0.5\textwidth]{OptimisationStudies/Plots/JetEnergyResolutions/JER_vs_ScintillatorECalNumberofLayers_500GeV_DiJet_Breakdown.pdf}}
\caption[Contributions to the jet energy resolution shown as function of number of layers in the ECal using the nominal ILD detector model for \protect\subref{fig:ecalsinlayers45break} the silicon ECal option and 45~GeV jets, \protect\subref{fig:ecalscnlayers45break} the scintillator ECal option and 45~GeV jets, \protect\subref{fig:ecalsinlayers250break} the silicon ECal option and 250~GeV jets and \protect\subref{fig:ecalscnlayers250break} the scintillator ECal option and 250~GeV jets.  The black curves correspond to the standard reconstruction, the blue curves to the intrinsic energy resolution contribution to the jet energy resolution, the red curves to the confusion contribution to the jet energy resolution and the magenta curves to the confusion contribution to the jet energy resolution related solely to photon reconstruction.]{Contributions to the jet energy resolution shown as function of number of layers in the ECal using the nominal ILD detector model for \protect\subref{fig:ecalsinlayers45break} the silicon ECal option and 45~GeV jets, \protect\subref{fig:ecalscnlayers45break} the scintillator ECal option and 45~GeV jets, \protect\subref{fig:ecalsinlayers250break} the silicon ECal option and 250~GeV jets and \protect\subref{fig:ecalscnlayers250break} the scintillator ECal option and 250~GeV jets.  The black curves correspond to the standard reconstruction, the blue curves to the intrinsic energy resolution contribution to the jet energy resolution, the red curves to the confusion contribution to the jet energy resolution and the magenta curves to the confusion contribution to the jet energy resolution related solely to photon reconstruction.}
\label{fig:ecalnlayersbreak}
\end{figure}

%========================================================================================

\subsection{ECal Active Material}
In sections \ref{sec:ecalcells} and \ref{sec:ecalnlayers} the performance of the ECal was reported for both the silicon and scintillator options and to a large extent the performance of the two options was similar, but not identical:

\begin{itemize}
\item The intrinsic energy resolution of the silicon ECal option is better than that of the scintillator option at very high energies.  For 500~GeV photons the intrinsic energy resolution is $\sim 25\%$ better for the silicon option.  Section \ref{sec:nominaldetectorperformance} contains a comparison between the photon energy resolution for the two ECal options, which clearly illustrates this.  The most likely origin of the differing energy resolutions is the implementation of Birks' law \cite{Birks:1951boa} for scintillator active materials, which states
%
\begin{equation}
d\mathcal{L}/dx \propto \frac{dE/dx}{1+k_{B}dE/dx}\text{ ,}
\end{equation}
%
\noindent where $d\mathcal{L}/dx$ is the scintillation light yield per unit path length, $dE/dx$ is the energy deposited per unit path length and $k_{B}$ is a material property constant.  For large energy deposits per unit length, such as those found in high energy photons, the light yield saturates causing a degradation in the energy resolution.  When comparing the photon energy resolution for the silicon and scintillator ILD ECal options, which can be found in section \ref{sec:nominaldetectorperformance}, the saturation effect starts to degrade the energy resolution for the scintillator option around 50~GeV.  However, the degradation in energy resolution is relatively small event up to 100~GeV.
\item The "dead" region due to the presence of the MPPC in the simulation of the scintillator ECal option degrades performance of the detector for small transverse granularities, see figure \ref{fig:ecalcellsizegamma}.
\end{itemize}

In summary, the performance of the two options, in terms of energy and jet energy resolution, at ILC-like energies is comparable.  However, the silicon option is preferred when manufacture and implementation of the two models is compared.  While constructing silicon wafers to fit a $5 \times 5 \text{ mm}^{2}$ square cell size is achievable, this would be extremely challenging for scintillator tiles.  To resolve this in actuality, the scintillator ECal option would have to use $5 \times 45 \text{ mm}^{2}$ scintillator strips that are arranged in alternating directions in each ECal layer \cite{Behnke:2013lya}.  By combining information from neighbouring layers it becomes possible to approach an effective $5 \times 5 \text{ mm}^{2}$ square cell size.  

%========================================================================================
%========================================================================================

\section{Hadronic Calorimeter Optimisation}
\label{sec:hcal}
The purpose of an hadroinc calorimeter (HCal) is to measures the energy deposits from hadronic showers.  The HCal in the default ILD detector model, summarised in table \ref{table:defaultildhcal}, is approximately 6 nuclear interaction lengths ($\lambda_{I}$) deep.  The ECal contributes approximately one $\lambda_{I}$ giving a total of $\approx 7 \lambda_{I}$, which is sufficient to contain jets at ILC like energies.  The longitudinal structure of this model consists of 48 readout layers each containing a 3~mm active layer of scintillator and a 20~mm absorber layer of iron.  

\begin{table}[h!]
\centering
\begin{tabular}{ l l}
\hline
Parameter & Default Value \\
\hline
Cell Size & $30 \times 30 \text{ mm}^{2}$ square cells \\
Number of Layers & 48 readout layers \\
Active Material Choice & Scintillator \\
Active Material Thickness & 3 mm  \\
Absorber Material Choice & Steel \\
Absorber Material Thickness & 20 mm \\
\hline
\end{tabular}
\caption[The configuration of the HCal in the nominal ILD detector model \cite{Behnke:2013lya}.]{The configuration of the HCal in the nominal ILD detector model \cite{Behnke:2013lya}.}
\label{table:defaultildhcal}
\end{table}
% Nuclear interaction length iron 167.7mm
% Nuclear interaction length tungsten 99.46mm 
% Nuclear interaction length silicon 465.2mm 
% Nuclear interaction length polystyrene 770.7mm

There are several readout approaches under consideration for the HCal including fully analogue, fully digital and semi-digital.  Analogue readout reports the energy within each HCal cell using a continuous variable, while digital readout only produces a response if the energy deposited within a calorimeter cell is above a given threshold.  The semi-digital approach mirrors that of the digital approach, but has three responses each with a different energy threshold.  While the energy resolution for digital calorimeters is not as good as that of analogue calorimeters, it is possible to construct smaller cell sizes using a digital readout.  In traditional calorimetry, a digital calorimeter would give a worse jet energy resolution than the analogue equivalent, however, that is not necessarily the case in particle flow calorimetry.  If a digital calorimeter could be realised with a much small cell size than the analogue equivalent, then the affect of confusion in the digital calorimeter may be reduced such that it compensates for any loss to intrinsic energy resolution.  In the following studies only the optimisation of the analogue HCal is presented as this is the readout approach used in the nominal ILD detector model.  

A number of options were simulated where the following parameters in the HCal were varied:
\begin{itemize}
\item Cell size:  This is crucial for successful application particle flow calorimetry for making associations between clusters of calorimeter hits and charged particle tracks.  It is expected that the intrinsic energy resolution be invariant to changes in the HCal cell size.  
\item Number of readout layers:  The number of layers in the HCal are varied, however, the thickness of those layers match those of the nominal ILD HCal design.  This means the total depth of the HCal in $\lambda_{I}$ is changing.  It is expected that this study will determine the effect of leakage of energy out of the back of the HCal.
\item Longitudinal sampling frequency:   This involves changing the number of readout layers in the HCal while simultaneously changing the thicknesses of the active and absorber layers to keep the total number of $\lambda_{I}$ in the HCal constant.  As this modifies the sampling of particle showers in the HCal, it will affect the intrinsic energy resolution of the HCal.
\item Sampling fraction:  This is the ratio of the active medium thickness to the absorber medium thickness.  This controls how particle showers within the calorimeter are sampled.  In this study the total depth of the HCal in $\lambda_{I}$ is held constant between detector models.  
\item Absorber material choice:  Two options have been considered: steel and tungsten.  This choice affects the growth and propagation of hadronic showers.  
\end{itemize}

%========================================================================================

\subsection{HCal Cell Size}
\label{sec:hcalcells}
The HCal cell size is an important detector parameter in the application of particle flow calorimetry.  Smaller HCal cell sizes will lead to a finer spatial resolution that can be used to better separate charged and neutral particle calorimetric energy deposits.  On the other hand, this will also lead to an increase in the number of readout channels that will raise the cost of the calorimeter.  Therefore, it is highly desirable to achieve the optimal physics performance using the largest cell size possible.  The nominal ILD HCal has a 30~mm square cell size and in this study the following cell sizes were considered; $10 \times 10 \text{ mm}^{2}$, $20 \times 20 \text{ mm}^{2}$, $30 \times 30 \text{ mm}^{2}$, $40 \times 40 \text{ mm}^{2}$, $50 \times 50 \text{ mm}^{2}$ and $100 \times 100 \text{ mm}^{2}$.  

In the nominal ILD detector, 50~GeV long-lived neutral kaons ($\text{K}^{0}_{L}$s) will deposit $\sim 65\%$ of their energy in the HCal and $\sim 35\%$ in the ECal.  As 50~GeV $\text{K}^{0}_{L}$s deposit the bulk of their energy in the HCal, they are appropriate to use when determining the performance of the HCal.  However, it should be emphasised that the $\text{K}^{0}_{L}$ energy resolutions represent the intrinsic energy resolution of the whole ILD detector and not purely that of the HCal.  

Figure \ref{fig:hcalcellser} shows the energy resolution for 50~GeV $\text{K}^{0}_{L}$s as a function of cell size.  As expected, the hadronic energy resolution does not strongly depend on the HCal cell size.  The only statistically significant variation in energy resolution is observed for the $100 \times 100 \text{ mm}^{2}$ HCal cell size.  For this model the energy resolution gets gets worse by $\sim8\%$ in comparison to the other models considered.  The most likely cause of this is a reduction in the effectiveness of the HCal hit energy truncation, which is described in section \ref{sec:hcalcelltruncation}.  The reduced effectiveness is expected because the precision used when obtaining the optimal energy truncation becomes worse as HCal cell size diverges from the nominal value.  

\begin{figure}[h!]
\centering
\includegraphics[width=0.5\textwidth]{OptimisationStudies/Plots/EnergyResolution/ER_vs_HCalCellSize_50GeVKaon0L.pdf}
\caption[The energy resolution as a function of HCal cell size for 50~GeV $\text{K}^{0}_{L}$ events using the nominal ILD detector model.]{The energy resolution as a function of HCal cell size for 50~GeV $\text{K}^{0}_{L}$ events using the nominal ILD detector model.}
\label{fig:hcalcellser}
\end{figure}

A smaller HCal cell size will lead to better separation of charged and neutral hadron calorimetric energy deposits, therefore, it is expected that the confusion contribution to the jet energy resolution will be reduced by using smaller HCal cell sizes.  Figure \ref{fig:hcalcellsize} shows the jet energy resolution as a function of cell size in the HCal.  At low jet energies there is no strong dependency of the jet energy resolution on the HCal cell size, which is as expected from the $\text{K}^{0}_{L}$ energy resolution study.  For high energy jets there is a clear dependence, with lower HCal cell sizes leading to better jet energy resolutions; the jet energy resolution for 250~GeV jets goes from $\sim 2.7\%$ to $\sim 3.5\%$ when the HCal cell size is increased from $10 \times 10 \text{ mm}^{2}$ to $100 \times 100 \text{ mm}^{2}$.  Examining the different contributions to the jet energy resolution, shown in figure \ref{fig:hcalcellsizebreak} it can be seen that the intrinsic energy resolution contribution does not depend on the HCal cell size; it is the confusion contribution that drives the overall trend in the jet energy resolution.  This is particularly clear at high jet energies where the confusion contribution to the jet energy resolution dominates that of the intrinsic energy resolution contribution.  At high jet energies smaller HCal cell sizes leads to a reduction in the effect of confusion; the confusion contribution to the jet energy resolution is reduced by $\sim 25\%$ when reducing the HCal cell size from $100 \times 100 \text{ mm}^{2}$ to $10 \times 10 \text{ mm}^{2}$.  At low jet energies the trend is less clear, as the confusion contribution is less dominant.  Nevertheless, a reduction in the effect of confusion with decreasing cell size is still visible for all but the smallest HCal cell size.  The most likely cause of the increase in confusion for the smallest HCal cell size at low energies is the tuning of the PandoraPFA algorithms to the nominal ILD HCal cell size.  For both the 45 and 250~GeV jets, the photon confusion does not depend on the HCal cell size.  This indicates that changes to the confusion term seen when varying the HCal cell size are related solely to the reconstruction of hadrons.  

\begin{figure}[h!]
\centering
\includegraphics[width=0.5\textwidth]{OptimisationStudies/Plots/JetEnergyResolutions/JER_vs_HCalCellSize.pdf}
\caption[The jet energy resolution as a function of HCal cell size for various jet energies using the nominal ILD detector model.]{The jet energy resolution as a function of HCal cell size for various jet energies using the nominal ILD detector model.}
\label{fig:hcalcellsize}
\end{figure}

\begin{figure}[h!]
\centering
\subfloat[]{\label{fig:hcalcellsize45break}\includegraphics[width=0.5\textwidth]{OptimisationStudies/Plots/JetEnergyResolutions/JER_vs_HCalCellSize_91GeV_DiJet_Breakdown.pdf}}
\subfloat[]{\label{fig:hcalcellsize250break}\includegraphics[width=0.5\textwidth]{OptimisationStudies/Plots/JetEnergyResolutions/JER_vs_HCalCellSize_500GeV_DiJet_Breakdown.pdf}}
\caption[Contributions to the jet energy resolution shown as function of HCal cell size using the nominal ILD detector model for \protect\subref{fig:hcalcellsize45break} 45~GeV jets and \protect\subref{fig:hcalcellsize250break} 250~GeV jets.  The black curves correspond to the standard reconstruction, the blue curves to the intrinsic energy resolution contribution to the jet energy resolution, the red curves to the confusion contribution to the jet energy resolution and the magenta curves to the confusion contribution to the jet energy resolution related solely to photon reconstruction.]{Contributions to the jet energy resolution shown as function of HCal cell size using the nominal ILD detector model for \protect\subref{fig:hcalcellsize45break} 45~GeV jets and \protect\subref{fig:hcalcellsize250break} 250~GeV jets.  The black curves correspond to the standard reconstruction, the blue curves to the intrinsic energy resolution contribution to the jet energy resolution, the red curves to the confusion contribution to the jet energy resolution and the magenta curves to the confusion contribution to the jet energy resolution related solely to photon reconstruction.}
\label{fig:hcalcellsizebreak}
\end{figure}

A comparison of the results from the ECal and HCal cell size optimisation studies shows that the jet energy resolution has a stronger dependency on the ECal cell size than on the HCal cell size; increasing the nominal ECal cell size by a factor of three makes the jet energy resolution for 250~GeV jets worse by $\sim 20\%$, while increasing the nominal HCal cell size by the same factor makes the jet energy resolution worse by $\sim 12\%$.  This is to be expected as in the particle flow paradigm $\approx 30\%$ of jet energy is recorded in the ECal, while only $\approx 10\%$ is recorded in the HCal.  Consequently, the potential effect of double counting and omitting energy deposits, i.e. confusion, is greater in the ECal than the HCal.  Therefore, minimising confusion in the ECal is expected to be more crucial for the overall jet energy resolution, which is what is observed.  Furthermore, as PandoraPFA groups calorimeter hits together using a cone clustering approach, identifying the start of a particle shower is key for determining how calorimeter hits are grouped together deeper into the calorimeters.  In effect, this means the grouping of calorimeter hits in the HCal depends upon information gathered in the ECal.  Therefore, if the ECal performance is sufficiently good, even with coarse HCal cell sizes, excellent performance can be achieved.  

In summary, the confusion contribution to the jet energy resolution falls as the HCal cell size is reduced, while the intrinsic energy resolution of the detector is largely unaffected.  As this dependancy is relatively weak, even the use of $100 \times 100 \text{ mm}^{2}$ HCal cell sizes would be enough to allow for separation of the hadronic decays of W and Z bosons, i.e. $\sigma_{E}/E \lesssim 3.8\%$ \cite{arXiv:0907.3577}, at ILC like energies.  However, there are benefits to having smaller HCal cell size; the jet energy resolution is reduced from $\sim3.5\%$ to $\sim2.8\%$ for 250~GeV jets when decreasing the HCal cell size from $100 \times 100 \text{ mm}^{2}$ to $10 \times 10 \text{ mm}^{2}$.  

%========================================================================================

\subsection{HCal Number of Layers}
\label{sec:hcalnlayers}
In this study, the total number of layers in the HCal was varied.  In contrast to the longitudinal sampling frequency study, the active and absorber layer thicknesses in the HCal were not altered.  Changing the number of layers in this way leads to a change in the total thickness of the calorimeter.  This study is sensitive to the effects, if any, of leakage of energy out of the back of the calorimeters.  The manufacturing cost of the HCal is proportional to the number of readout channels and layers.  Therefore, minimising the number of layers, while retaining excellent physics performance is important.  Here detector models were simulated with a HCal containing 36, 42, 48 (nominal), 54 and 60 layers. 

It is expected that the energy resolution of the detector will improve when the number of layers in the HCal is increased since fewer events should suffer from the effects of leakage.  Any improvements seen by increasing the number of layers in the HCal is expected only up to the point where the majority of hadronic showers are fully contained by the calorimeters.  The energy resolution as a function of number of layers in the HCal for 50~GeV $\text{K}^{0}_{L}$ is shown in figure \ref{fig:hcalnfixedlayerser}.  The energy resolution becomes worse as the number of layers in the HCal is reduced below 48 layers, while above this point additional layers do not change the energy resolution.  This indicates that the majority of hadronic showers at this energy are fully contained by a 48 layer HCal.  As reducing the number of HCal layers to 36 only causes a small degradation, $\sim 10\%$, in the neutral hadron energy resolution, it is feasible to consider reducing the number of layers in the ILD HCal.  

\begin{figure}[h!]
\centering
\includegraphics[width=0.5\textwidth]{OptimisationStudies/Plots/EnergyResolution/ER_vs_HCalNFixedLayers_50GeVKaon0L.pdf}
\caption[The energy resolution as a function of number of layers in the HCal for 50~GeV $\text{K}^{0}_{L}$ events using the nominal ILD detector model.]{The energy resolution as a function of number of layers in the HCal for 50~GeV $\text{K}^{0}_{L}$ events using the nominal ILD detector model.}
\label{fig:hcalnfixedlayerser}
\end{figure}

Figure \ref{fig:hcalnfixedlayers} shows the jet energy resolution as a function of the number of layers in the HCal.  For low energy jets, where intrinsic energy resolution dominates, the jet energy resolution is does not depend on the number of layers in the HCal.  At high jet energies, where confusion dominates, increasing the number of layers in the HCal improves the jet energy resolution; the jet energy resolution is goes from $\sim3.4\%$ to $\sim3.0\%$ for 250~GeV jets when increasing the number of HCal layers from 36 to 48.  The origin of these trends is leakage of energy out of the back of the calorimeters, which becomes more problematic as the number of layers in the HCal is reduced and the jet energy increases.  

Figure \ref{fig:hcalnfixedlayersbreak} shows the jet energy resolution contributions as a function of the number of layers in the HCal.  These results appear somewhat counterintuitive in that the intrinsic energy resolution of the detector does not seem to depend on the number of layers in the HCal even for high energy jets.  However, this is expected given only 10\% of jet energy is carried in the form of neutral hadrons and the neutral hadron energy resolution, for 50~GeV hadrons, is only weakly dependent on the number of HCal layers.  Leakage does have an effect on the intrinsic energy resolution, however, the use of $\text{RMS}_{90}$ obscures part of this by excluding events where leakage is significant.  The fractional decrease in $\text{RMS}_{90}$ for the intrinsic energy distribution when increasing the number of HCal layers from 36 to 60 is $\sim4\%$, however, the change in the full RMS is $\sim23\%$.  Figure \ref{fig:hcalnfixedlayersbreak} also shows that the confusion contribution is far more sensitive to the number of layers in the HCal than the intrinsic energy resolution.  This sensitivity originates from the reclustering stage of the reconstruction in events where leakage has occurred.  In these events, when PandoraPFA compares the momentum of a charged particle track to the cluster of calorimeter hits that it produces, there will be a disparity.  To resolve the disparity, PandoraPFA will associate other calorimeter energy deposits that were not produced by the charged particle to the track to compensate for the leaked energy, which produces confusion.  As photons are largely contained within the ECal at these energies, the photon confusion contribution to the jet energy resolution has no dependence on the number of layers in the HCal. 

\begin{figure}[h!]
\centering
\includegraphics[width=0.5\textwidth]{OptimisationStudies/Plots/JetEnergyResolutions/JER_vs_NumberOfHCalLayersOfFixedDepth.pdf}
\caption[The jet energy resolution as a function of number of layers in the HCal for various jet energies using the nominal ILD detector model.]{The jet energy resolution as a function of number of layers in the HCal for various jet energies using the nominal ILD detector model.}
\label{fig:hcalnfixedlayers}
\end{figure}

\begin{figure}[h!]
\centering
\subfloat[]{\label{fig:hcalnfixedlayers45break}\includegraphics[width=0.5\textwidth]{OptimisationStudies/Plots/JetEnergyResolutions/JER_vs_NumberOfHCalLayersOfFixedDepth_91GeV_DiJet_Breakdown.pdf}}
\subfloat[]{\label{fig:hcalnfixedlayers250break}\includegraphics[width=0.5\textwidth]{OptimisationStudies/Plots/JetEnergyResolutions/JER_vs_NumberOfHCalLayersOfFixedDepth_500GeV_DiJet_Breakdown.pdf}}
\caption[Contributions to the jet energy resolution shown as function of number of layers in the HCal using the nominal ILD detector model for \protect\subref{fig:hcalnfixedlayers45break} 45~GeV jets and \protect\subref{fig:hcalnfixedlayers250break} 250~GeV jets.  The black curves correspond to the standard reconstruction, the blue curves to the intrinsic energy resolution contribution to the jet energy resolution, the red curves to the confusion contribution to the jet energy resolution and the magenta curves to the confusion contribution to the jet energy resolution related solely to photon reconstruction.]{Contributions to the jet energy resolution shown as function of number of layers in the HCal using the nominal ILD detector model for \protect\subref{fig:hcalnfixedlayers45break} 45~GeV jets and \protect\subref{fig:hcalnfixedlayers250break} 250~GeV jets.  The black curves correspond to the standard reconstruction, the blue curves to the intrinsic energy resolution contribution to the jet energy resolution, the red curves to the confusion contribution to the jet energy resolution and the magenta curves to the confusion contribution to the jet energy resolution related solely to photon reconstruction.}
\label{fig:hcalnfixedlayersbreak}
\end{figure}

In summary, even if the number of layers in the HCal were reduced by 25\%, the jet energy resolution would be sufficient for separating the hadronic decays of the W and Z bosons at ILC energies, i.e. $\sigma_{E}/E \lesssim 3.8\%$ \cite{arXiv:0907.3577}.  Although, the effects of leakage do make the jet energy resolution worse for ILC like energies, once the number of layers in the HCal is reduced from 48 layers, therefore, it is desirable to have a minimum of 48 layers in the ILD HCal.
  
%========================================================================================

\subsection{HCal Longitudinal Sampling Frequency}
\label{sec:hcalsamplingfrequency}
Several detector models were simulated where the longitudinal sampling frequency in the HCal was modified.  The longitudinal sampling frequency was altered by changing the number of layers in the HCal, while simultaneously changing the active and absorber layer thicknesses, to maintain the total number of nuclear interaction lengths.  For each model considered, the absorber material was steel, containing a total of 5.72~$\lambda_{I}$, and the active material was scintillator, containing a total of 0.19~$\lambda_{I}$.  The ratio of the active to absorber layers thicknesses (the sampling fraction) in these models is the same as in the nominal ILD HCal.  A summary of the detector models considered is given in table \ref{table:nlayershcaloption}.  

\begin{table}[h!]
\centering
\begin{tabular}{ l l l }
\hline
Number $N_{\text{Layers HCal}}$ & Absorber Thickness & Active Thickness \\
 & [mm] & [mm] \\
\hline
60 & 16.00 & 2.40 \\ 
54 & 17.78 & 2.67 \\
48 & 20.00 & 3.00 \\
42 & 22.86 & 3.43 \\
36 & 26.67 & 4.00 \\
30 & 32.00 & 4.80 \\
24 & 40.00 & 6.00 \\
18 & 53.33 & 8.00 \\
\hline
\end{tabular}
\caption[Longitudinal configuration of the HCal in the detector models considered.]{Longitudinal configuration of the HCal in the detector models considered.}
\label{table:nlayershcaloption}
\end{table}

Figure \ref{fig:hcalnlayerser} shows the energy resolution for 50~GeV $\text{K}^{0}_{L}$ as a function of number of layers in the HCal.  As the number of layers in the HCal is increased, the energy resolution improves.  This is because increasing the number of layers in a sampling calorimeter, while leaving the total material budget unchanged, will lead to greater sampling of particles showering within it and a reduction the stochastic contribution to the energy resolution.  The energy resolution is less pronounced than the naive expectation of 1/$N_{\text{HCal}}$, where $N_{\text{HCal}}$ is the number of layers in the HCal, because this relationship only holds for the energy resolution of a single sampling calorimeter and these results are for the full ILD detector, including the $\approx 1 \lambda_{I}$ in the ECal.  Furthermore, the 1/$N_{\text{HCal}}$ functional form neglects a number of effects, such as instrumentation defects and electrical noise, that should be included when parameterising the energy resolution \cite{Fabjan:2003aq}.  

\begin{figure}[h!]
\centering
\includegraphics[width=0.5\textwidth]{OptimisationStudies/Plots/EnergyResolution/ER_vs_NHCalVariableLayers_50GeVKaon0L.pdf}
\caption[The energy resolution as a function of the longitudinal sampling frequency in the HCal for 50~GeV $\text{K}^{0}_{L}$ events using the nominal ILD detector model.]{The energy resolution as a function of the longitudinal sampling frequency in the HCal for 50~GeV $\text{K}^{0}_{L}$ events using the nominal ILD detector model.}
\label{fig:hcalnlayerser}
\end{figure}  

Figure \ref{fig:hcalnlayers} shows the jet energy resolution as a function of the longitudinal sampling frequency in the HCal.  Increasing the number of layers in the HCal leads to an improvement in the HCal; when the number of layers in the HCal is increased from 18 to 60 the jet energy resolution for 250~GeV jets improves by $\sim17\%$.  

Figure \ref{fig:hcalnlayersbreak} shows that both the intrinsic energy resolution and confusion improve with increasing longitudinal sampling frequency.  For 250~GeV jets, when increasing the number of HCal layers from 18 to 60 the intrinsic energy resolution contribution goes from $\sim 1.9\%$ to $\sim 1.6\%$ and the confusion contribution goes from $\sim 3.0\%$ to $\sim 2.4\%$.  The twofold improvement is expected because increasing the longitudinal sampling frequency improves the intrinsic energy resolution of a sampling calorimeter, which has the knock-on effect of lowering the confusion.  The resulting reduction in confusion is due to the improved precision obtained when comparing the momenta of charged particle tracks and the energy of clusters of calorimeter hits.  These comparisons are used to guide event reconstruction in PandoraPFA, therefore, if the precision of these comparisons is improved, the confusion is reduced as described in section \ref{sec:jerdecomposition}.   

\begin{figure}[h!]
\centering
\includegraphics[width=0.5\textwidth]{OptimisationStudies/Plots/JetEnergyResolutions/JER_vs_NumberOfLayersInTheHCal.pdf}
\caption[The jet energy resolution as a function of longitudinal sampling frequency in the HCal for various jet energies using the nominal ILD detector model.]{The jet energy resolution as a function of longitudinal sampling frequency in the HCal for various jet energies using the nominal ILD detector model.}
\label{fig:hcalnlayers}
\end{figure}

\begin{figure}[h!]
\centering
\subfloat[]{\label{fig:hcalnlayers45break}\includegraphics[width=0.5\textwidth]{OptimisationStudies/Plots/JetEnergyResolutions/JER_vs_NumberOfLayersInTheHCal_91GeV_DiJet_Breakdown.pdf}}
\subfloat[]{\label{fig:hcalnlayers250break}\includegraphics[width=0.5\textwidth]{OptimisationStudies/Plots/JetEnergyResolutions/JER_vs_NumberOfLayersInTheHCal_500GeV_DiJet_Breakdown.pdf}}
\caption[Contributions to the jet energy resolution shown as function of the longitudinal sampling frequency in the HCal using the nominal ILD detector model for \protect\subref{fig:hcalnlayers45break} 45~GeV jets and \protect\subref{fig:hcalnlayers250break} 250~GeV jets.  The black curves correspond to the standard reconstruction, the blue curves to the intrinsic energy resolution contribution to the jet energy resolution, the red curves to the confusion contribution to the jet energy resolution and the magenta curves to the confusion contribution to the jet energy resolution related solely to photon reconstruction.]{Contributions to the jet energy resolution shown as function of the longitudinal sampling frequency in the HCal using the nominal ILD detector model for \protect\subref{fig:hcalnlayers45break} 45~GeV jets and \protect\subref{fig:hcalnlayers250break} 250~GeV jets.  The black curves correspond to the standard reconstruction, the blue curves to the intrinsic energy resolution contribution to the jet energy resolution, the red curves to the confusion contribution to the jet energy resolution and the magenta curves to the confusion contribution to the jet energy resolution related solely to photon reconstruction.}
\label{fig:hcalnlayersbreak}
\end{figure}

It is clear that a larger number of layers in the HCal benefits both the intrinsic energy resolution of the ILD detector as well as reducing the confusion contribution to the jet energy resolution.  As there are few physics analyses that rely on the identification and categorisation of individual neutral hadrons, but there are many that rely on identification and categorisation of photons, the intrinsic energy resolution of the HCal is less crucial from a physics perspective than that of the ECal.  However, these studies show the HCal has a crucial role to play in jet reconstruction in the particle flow paradigm.  To achieve a jet energy resolution of $\sigma_{E}/E \lesssim 3.8\%$ \cite{arXiv:0907.3577}, which is required to separate the W and Z hadronic decays, the ILD detector will require a minimum of 42 layers in the HCal.  This longitudinal sampling frequency is required particularly for low energy jets where the energy resolution is dominated by the intrinsic energy resolution of the detector.

%========================================================================================

\subsection{HCal Sampling Fraction}
\label{sec:hcalsamplingfraction}
The performance of the ILD detector was studied for different ratios of active to absorber later thicknesses in the HCal.  In the nominal detector model, the active scintillator layer thickness is 3~mm, while the absorber layer thickness is 20~mm giving a sampling fraction of 0.15.  HCal models were simulated where this ratio was changed from 0.05 to 0.25 in steps of 0.05, while retaining the same number of interaction lengths.  

No performance changes in the energy resolution for 50~GeV $\text{K}^{0}_{L}$s or the jet energy resolution for 91, 200, 360 and 500~GeV Z$\rightarrow$uds di-jet events were observed when varying the ratio of active to absorber later thicknesses.  Based on these simulations, there is no suggestion that varying this ratio has any statistically significant effect on the physics performance.  Although this study indicates that thinning the active layer thickness would not change performance, hardware effects must also be considered to determine whether these conclusions hold true in a real detector.  A study into the effects of the readout electronics is required before changing the active layer thicknesses to determine whether a MIP signal can be clearly distinguished when changing the sampling fraction.  

%========================================================================================

\subsection{HCal Absorber Material}
\label{sec:hcalabsorbermaterial}
The nominal choice of HCal absorber material is steel with tungsten providing a feasible alternative \cite{Blaising:2015nla}.  Although tungsten is more expensive than steel, it contains a larger number of nuclear interaction lengths per unit length.  Therefore, using tungsten as the absorber material would allow for a reduction in the size of the HCal, while retaining the same number of nuclear interaction lengths.  Reducing the depth of the calorimeter would decrease the size of the solenoid required, which would offset some of the additional cost of tungsten. 

Table \ref{table:hcalabsmaterial} shows the configuration for the steel and tungsten HCal options that were used in the full ILD simulation.  To isolate the effects of changing the absorber material, the total depth, in nuclear interaction lengths, was kept constant when comparing the two options.  Furthermore, the sampling fraction was also held constant.  A number of different physics lists exist within GEANT4 for the modelling of hadronic showers.  The default model for high energy physics calorimetry is the QGSP\_BERT physics list.  This uses the quark-gluon string model \cite{Folger:2003sb} with the precompound model of nuclear evaporation \cite{geantStringModel} (QGSP) for high energy interactions and the Bertini (BERT) cascade model \cite{Guthrie:1968ue} for intermediate energy interactions.  For the study of absorber materials both the QGSP\_BERT and the QGSP\_BERT\_HP physics lists were used.  The QGSP\_BERT\_HP list uses the high precision neutron package (NeutronHP) to deal with the transportation of neutrons from below 20 MeV to thermal energies.  This added detail is necessary for accurate modelling of hadronic showers in tungsten \cite{Adloff:2014rya}.  

\begin{table}[h!]
\centering
\begin{tabular}{ l l l }
\hline
Parameter & Steel HCal Option & Tungsten HCal Option \\
\hline
Cell Size & $30 \times 30 \text{ mm}^{2}$ square cells & $30 \times 30 \text{ mm}^{2}$ square cells\\
Number of Layers & 48 readout layers & 48 readout layers\\
Absorber Material Thickness [mm] & 20.0 & 12.0 \\
Active Material Choice & Scintillator & Scintillator \\
Active Material Thickness [mm] & 3.0 & 1.8 \\
\hline
\end{tabular}
\caption[The configuration of the steel and tungsten HCal options \cite{Behnke:2013lya}.]{The configuration of the steel and tungsten HCal options \cite{Behnke:2013lya}.}
\label{table:hcalabsmaterial}
\end{table}

One of the dominant processes governing the energy deposition of hadronic showers in calorimeters is spallation \cite{Wigmans:2000vf}.  Spallation begins with the collision of a high energy incident particle with nucleons in the calorimeter absorber material.  This collision creates a cascade of high energy hadronic particles, e.g. protons, neutrons and pions, are produced within the nucleus.  If these energies are large enough, some of these particles may escape the nucleus and form secondary particles in the hadronic shower.  After this initial collision, the nuclei of the absorbing material are left in an excited state.  Assuming the excited nuclei are sufficiently stable that they will not undergo fission, they will return to a stable state by ejecting energy in the form of particles in a process called evaporation.  Evaporation of neutrons, which is the dominant form of evaporation, significantly delays the growth of hadronic showers as after the evaporation process some of these neutrons participate in neutron capture \cite{Adloff:2014rya}.  Neutron capture involves an absorber nuclei capturing a neutron and then emitting a photon as it returns to a stable state.  The time taken for the neutron capture mechanism to proceed is limited by the lifetime of the unstable nuclei \cite{Caldwell:1992te}, which typically makes neutron capture one of the slowest mechanisms by which hadronic showers can propagate.  The number of evaporation neutrons released in a hadronic shower increases with the atomic number, Z, of the absorber material of a calorimeter increases.  This is because of the increase in neutron content of the absorber material nuclei \cite{Adloff:2014rya}.  As the number of evaporation neutrons increases, more neutron capture processes are initiated, which results in a longer hadronic shower development time.  Figure \ref{fig:hcalabsmaterialtiming} shows the shower development times for hadronic showers in the tungsten (Z=74) and steel (iron, Z=26) HCal options and, as expected, the shower development time is greater for tungsten.  

\begin{figure}[h!]
\centering
\includegraphics[width=0.5\textwidth]{OptimisationStudies/Plots/Description/HCalAbsorberMaterialTimings.pdf}
\caption[The fraction of the total calorimetric energy deposited in the HCal as a function of time for 25~GeV $\text{K}^{0}_{L}$ events using the steel and tungsten HCal options.  Results are shown for both the QGSP\_BERT and QGSP\_BERT\_HP physics lists.  The calorimeter hit times have been corrected for straight line time of flight to the impact point.]{The fraction of the total calorimetric energy deposited in the HCal as a function of time for 25~GeV $\text{K}^{0}_{L}$ events using the steel and tungsten HCal options.  Results are shown for both the QGSP\_BERT and QGSP\_BERT\_HP physics lists.  The calorimeter hit times have been corrected for straight line time of flight to the impact point.}
\label{fig:hcalabsmaterialtiming}
\end{figure} 

Table \ref{table:erhcalabsmaterial} shows the energy resolution for 50~GeV $\text{K}^{0}_{L}$s obtained using the nominal ILD detector model with various HCal absorber materials and GEANT4 physics lists.  In comparison to steel, tungsten option offers a $\sim 8\%$ improvement in the energy resolution for 50~GeV neutral hadrons (using the QGSP\_BERT\_HP physics list).  This can be attributed to differences in the nuclear structure of the two materials, which will lead to different developments of the hadronic showers within them. For example, the energy losses to nuclear binding energies are smaller in tungsten than steel, as the target nucleons are less stable than in iron, therefore, less energy is needed to liberate them.  This will lead to a larger signal for tungsten and a reduction in the energy resolution in comparison to steel.  The results of table \ref{table:erhcalabsmaterial} also indicate that the addition of the high precision neutron package was not important for this study.  

\begin{table}[h!]
\centering
\begin{tabular}{ l c }
\hline
HCal Option & Energy Resolution [\%] \\
\hline
Steel, QGSP\_BERT & $8.8\pm0.2$ \\
Steel, QGSP\_BERT\_HP & $9.0\pm0.3$ \\
Tungsten, QGSP\_BERT & $8.3\pm0.2$ \\
Tungsten, QGSP\_BERT\_HP & $8.3\pm0.2$ \\
\hline
\end{tabular}
\caption[The energy resolution for 50~GeV $\text{K}^{0}_{L}$s obtained using the nominal ILD detector with various HCal absorber materials and GEANT4 physics lists.  A 100~ns timing cut was applied to the steel and tungsten HCal options in these simulations.]{The energy resolution for 50~GeV $\text{K}^{0}_{L}$s obtained using the nominal ILD detector with various HCal absorber materials and GEANT4 physics lists.  A 100~ns timing cut was applied to the steel and tungsten HCal options in these simulations.}
\label{table:erhcalabsmaterial}
\end{table}

It should be emphasised that the HCal hit energy truncation, as described in chapter \ref{chap:energyestimators}, used for the tungsten and steel HCal options differs because tungsten contains a larger number of radiation lengths per nuclear interaction length than steel does.  As the HCal primarily measures hadronic showers, one may naively expect the number of radiation lengths in the HCal to be irrelevant, given both options have the same number of nuclear interaction lengths.  However, this is not the case because all hadronic showers have an electromagnetic component generated by the decays of hadrons to photons, e.g  $\pi^{0} \rightarrow \gamma \gamma$ and $\eta \rightarrow \gamma \gamma$.  This leads to hadronic showers depositing more energy per calorimeter hit in tungsten than in steel and makes retuning the HCal hit energy truncation a necessity.  As expected, the truncation used for tungsten, 5~GeV, is larger than for steel, 1~GeV, because of the increased average hit energy.

Table \ref{table:jerhcalabsmaterial} shows the jet energy resolutions for selected jet energies obtained using the nominal ILD detector with various HCal absorber materials and GEANT4 physics lists.  These results indicate that steel outperforms tungsten as the HCal absorber material.  The magnitude of the improvement offered using steel grows as the jet energy increases; the jet energy resolution is $\sim 3\%$ better for the steel option for 45~GeV jets, while for 250~GeV jets the improvement is $\sim 11\%$.  The intrinsic energy resolution and confusion contributions to the jet energy resolution for 45 and 250~GeV jets are shown in table \ref{table:jerbdhcalabsmaterial}.  The intrinsic energy resolution contribution to the jet energy resolution is almost identical for the two HCal options, which is expected because the $\text{K}^{0}_{L}$ energy resolution was only slightly better for the tungsten option.  The tungsten option is unlikely to give a significantly better intrinsic energy resolution because only the small fraction of jet energy associated with neutral hadrons is measured in the HCal. The confusion contribution to the jet energy resolution is larger for tungsten than for steel; for 250~GeV jets the confusion contribution is $\sim 3.4\%$ in tungsten and only $\sim 3.0\%$ in steel.  The larger confusion contribution is expected for the tungsten option because hadronic showers are generally wider in tungsten.  The transverse profile of hadronic showers in the two HCal options is illustrated in figure \ref{fig:transversedistanceshower}, which shows the normalised distribution of the energy weighted transverse distance from the shower axis to the calorimeter hits for 50~GeV hadronic showers for both the steel and tungsten HCal options.  Increasing the average hadronic shower width makes resolving individual particle showers in a dense jet environment more challenging, which means more calorimetric energy deposits will be incorrectly clustered together.  This in turn results in incorrect associations being made between calorimetric energy deposits and charged particle tracks i.e. an increased confusion contribution.  Again, the use of the QGSP\_BERT\_HP physics list, as opposed to QGSP\_BERT, made a minimal impact on these results.

\begin{table}[h!]
\centering
\begin{tabular}{ l c c c c }
\hline
 & \multicolumn{4}{c}{Jet Energy Resolution [\%]} \\
HCal Option & 45~GeV & 100~GeV & 180~GeV & 250~GeV \\
\hline
Steel, QGSP\_BERT & $3.65 \pm 0.05$ &$2.88 \pm 0.04$ &$2.85 \pm 0.04$ &$2.97 \pm 0.05$ \\
Steel, QGSP\_BERT\_HP & $3.67 \pm 0.05$ &$2.92 \pm 0.04$ &$2.86 \pm 0.04$ &$3.03 \pm 0.04$ \\
Tungsten, QGSP\_BERT & $3.78 \pm 0.05$ & $3.12 \pm 0.04$ & $3.15 \pm 0.04$ & $3.43 \pm 0.04 |$ \\
Tungsten, QGSP\_BERT\_HP & $3.80 \pm 0.05$ & $3.08 \pm 0.04$ & $3.24 \pm 0.04$ & $3.41 \pm 0.04$ \\
%Tungsten, QGSP\_BERT & $3.67 \pm 0.05$ &$3.12 \pm 0.04$ &$3.36 \pm 0.04$ &$3.76 \pm 0.05$ \\ 1~GeV Truncation 
%Tungsten, QGSP\_BERT\_HP & $3.69 \pm 0.05$ &$3.03 \pm 0.04$ &$3.38 \pm 0.04$ &$3.80 \pm 0.05$ \\ 1~GeV Truncation
\hline
\end{tabular}
\caption[The jet energy resolution for selected jet energies obtained using the nominal ILD detector with various HCal absorber materials and GEANT4 physics lists.  A 100~ns timing cut was applied to the steel and tungsten HCal options in these simulations.]{The jet energy resolution for selected jet energies obtained using the nominal ILD detector with various HCal absorber materials and GEANT4 physics lists.  A 100~ns timing cut was applied to the steel and tungsten HCal options in these simulations.}
\label{table:jerhcalabsmaterial}
\end{table}

\begin{table}[h!]
\centering
\begin{tabular}{ l c c c c }
\hline
 & \multicolumn{4}{c}{Jet Energy Resolution [\%]} \\
 & \multicolumn{2}{c}{45~GeV} & \multicolumn{2}{c}{250~GeV} \\
HCal Option & Intrinsic & Confusion & Intrinsic & Confusion \\
\hline
Steel, QGSP\_BERT & $2.93 \pm 0.04$ & $2.16 \pm 0.06$ & $1.69 \pm 0.02$ &$2.45 \pm 0.05$ \\
Steel, QGSP\_BERT\_HP & $2.98 \pm 0.04$ &$2.15 \pm 0.06$ &$1.65 \pm 0.02$ &$2.53 \pm 0.04$ \\
Tungsten, QGSP\_BERT & $2.97 \pm 0.04$ & $2.34 \pm 0.06$ & $1.65 \pm 0.02$ & $3.01 \pm 0.05$ \\
Tungsten, QGSP\_BERT\_HP & $2.92 \pm 0.04$ & $2.42 \pm 0.06$ & $1.65 \pm 0.02$ & $2.99 \pm 0.05$ \\
\hline
\end{tabular}
\caption[The contributions to the jet energy resolution obtained using the nominal ILD detector with various HCal absorber materials and GEANT4 physics lists.  A 100~ns timing cut was applied to the steel and tungsten HCal options in these simulations.]{The contributions to the jet energy resolution obtained using the nominal ILD detector with various HCal absorber materials and GEANT4 physics lists.  A 100~ns timing cut was applied to the steel and tungsten HCal options in these simulations.}
\label{table:jerbdhcalabsmaterial}
\end{table}

\begin{figure}[h!]
\includegraphics[width=0.5\textwidth]{OptimisationStudies/Plots/EnergyResolution/ShowerShape/Profile2.pdf}
\caption[The normalised distribution of the energy weighted transverse distance of the calorimeter hits from a 50~GeV hadronic shower to the shower axis.  The blue and red lines show the energy weighted transverse distance obtained using a steel and tungsten HCal absorber material in the ILD detector respectively.  The simulations used the QGSP\_BERT\_HP physics list.]{The normalised distribution of the energy weighted transverse distance of the calorimeter hits from a 50~GeV hadronic shower to the shower axis.  The blue and red lines show the energy weighted transverse distance obtained using a steel and tungsten HCal absorber material in the ILD detector respectively.  The simulations used the QGSP\_BERT\_HP physics list.}
\label{fig:transversedistanceshower}
\end{figure}

The impact of the choice of HCal absorber material is small on both the neutral hadron energy resolution and intrinsic energy resolution, however,  the steel HCal option outperforms the tungsten option in terms of pattern recognition confusion.  When examining the mechanical properties of steel and tungsten, it is clear that steel has a significant advantage over tungsten in terms of rigidity \cite{Linssen:2012hp}.  This means that fewer support structures would be required for the calorimeter leading to less dead material and better performance, which makes steel the preferred option.

%========================================================================================
%========================================================================================

\section{Global Detector Parameters}
The overall detector size and the magnetic field strength are major cost drivers for the ILD detector.  Both will affect the jet energy resolution and studies showing their impact on detector performance are presented here.

%========================================================================================

\subsection{The Magnetic Field Strength}
\label{sec:bfield}
In the particle flow paradigm the momentum of charged particles is obtained through the curvature of their trajectory as they bend in the magnetic field.  Therefore, the magnetic field is an integral element for the successful application of particle flow calorimetry.  Furthermore, the magnetic field deflects charged particles away from neutral particles in jets.  The stronger the magnetic field, the larger the average separation between the calorimetric energy deposits made by charged and neutral particles in jets, which reduces the effect of confusion.  Therefore, it is expected that a stronger magnetic field will lead to better jet energy resolutions through a reduction of the confusion contribution to the jet energy resolution.  

Detector models were simulated where the magnetic field was varied from 1.0 to 5.0~T in steps of 0.5~T and the resulting jet energy resolutions  are shown in figure \ref{fig:bfield}.  The larger the magnetic field strength, the better the jet energy resolution.  Increasing the magnetic field strength from 1.0 to 5.0~T improves the jet energy resolution for 250~GeV jets by $\sim 25 \%$.  The higher the jet energy, the stronger the dependence of the jet energy resolution on the magnetic field strength.  

\begin{figure}[h!]
\includegraphics[width=0.5\textwidth]{OptimisationStudies/Plots/JetEnergyResolutions/JER_vs_MagneticFieldStrength.pdf}
\caption[The jet energy resolution using the nominal ILD detector as a function of the magnetic field strength for various jet energies.]{The jet energy resolution using the nominal ILD detector as a function of the magnetic field strength for various jet energies.}
\label{fig:bfield}
\end{figure}

Figure \ref{fig:bfieldbreak} shows the breakdown of the jet energy resolution into the various contributions.  As expected, there is a reduction in the confusion contribution with increasing magnetic field strength.  Furthermore, there is a reduction in intrinsic energy resolution with increasing magnetic field strength for low energy jets.  This is most likely due to particles being directed into the forward region of the detector.  When a charged particle passes through a magnetic field it will, assuming no energy losses, traverse a helix.  The radius of curvature, $R$, of that helix is given by 
%
\begin{equation}
R = \frac{p_{\text{T}}}{qB} \text{ ,}
\end{equation}
%
where $p_{\text{T}}$ is the transverse momentum of the charged particle with respect to the magnetic field, $q$ is the electric charge of the particle and $B$ is the magnetic field strength.  When the magnetic field strength increases, the radius of curvature for charged particles will decrease and more low charged particles will be directed toward the forward regions of the detector.  As the tracking coverage in the forward region of the detector is worse in the central region \cite{Behnke:2013lya}, increasing the magnetic field strength leads to fewer charged particles being reconstructed, which is illustrated in figure \ref{fig:bfieldchargedparticles}, and a degradation in the energy resolution.  For high jet energies, low transverse momentum charged particles will still get directed to the forward regions of the detector, however, these contribute fractionally less energy to the total reconstructed energy.  Therefore, the trend of worsening intrinsic energy resolution with increasing magnetic field strength is less pronounced as the jet energy grows.  

At high jet energies, reducing the magnetic field strength appears to degrade the intrinsic energy resolution; the intrinsic energy resolution for 250~GeV jets goes from $\sim1.8\%$ to $\sim2.1\%$ when reducing the magnetic field strength from 5.0~T to 1.0~T.  This trend is due to an artefact in the definition of the intrinsic energy resolution for jets meaning it is not a genuine effect.  The intrinsic energy resolution is highly non-trivial to determine; Monte-Carlo (MC) information is used to make all associations between charged particle tracks and clusters of calorimeter hits as follows:

\begin{enumerate}
\item Each calorimeter hit is associated to the MC particle that deposits the largest amount of energy in that hit;
\item Clusters of calorimeter hits formed by the same MC particle are clustered together;
\item Each charged particle track is associated to the MC particle that produced it;
\item Clusters of calorimeter hits are associated to charged particle tracks if they are made by the same MC particle.
\end{enumerate}

This procedure assumes that only one MC particle deposits significant energy per calorimeter hit.  If multiple MC particles deposit significant energy in the same calorimeter hit, this assumption breaks down and errors are made when associating charged particle tracks to calorimetric energy deposits.  These errors cause the same double counting and omission of energy deposits as confusion does, however, they have a smaller effect because multiple MC particles deposit significant energy in the same calorimeter hit is rare in finely segmented calorimeters.  As the overlap of particle showers within the calorimeter grows, as it does for high energy jets when reducing the magnetic field, the intrinsic energy resolution appears to get worse because of this confusion-like effect.  Because this effect is small in comparison to changes in the confusion contribution, the overall dependence of the detector performance on the magnetic field strength can be confidently quantified.  

\begin{figure}[h!]
\subfloat[]{\label{fig:bfield45break}\includegraphics[width=0.5\textwidth]{OptimisationStudies/Plots/JetEnergyResolutions/JER_vs_MagneticFieldStrength_91GeV_DiJet_Breakdown.pdf}}
\subfloat[]{\label{fig:bfield250break}\includegraphics[width=0.5\textwidth]{OptimisationStudies/Plots/JetEnergyResolutions/JER_vs_MagneticFieldStrength_500GeV_DiJet_Breakdown.pdf}}
\caption[Contributions to the jet energy resolution shown as function of the magnetic field strength using the nominal ILD detector model for \protect\subref{fig:bfield45break} 45~GeV jets and \protect\subref{fig:bfield250break} 250~GeV jets.  The black curves correspond to the standard reconstruction, the blue curves to the intrinsic energy resolution contribution to the jet energy resolution, the red curves to the confusion contribution to the jet energy resolution and the magenta curves to the confusion contribution to the jet energy resolution related solely to photon reconstruction.]{Contributions to the jet energy resolution shown as function of the magnetic field strength using the nominal ILD detector model for \protect\subref{fig:bfield45break} 45~GeV jets and \protect\subref{fig:bfield250break} 250~GeV jets.  The black curves correspond to the standard reconstruction, the blue curves to the intrinsic energy resolution contribution to the jet energy resolution, the red curves to the confusion contribution to the jet energy resolution and the magenta curves to the confusion contribution to the jet energy resolution related solely to photon reconstruction.}
\label{fig:bfieldbreak}
\end{figure}

\begin{figure}[h!]
\includegraphics[width=0.5\textwidth]{OptimisationStudies/Plots/Description/BField/BFieldNumbers_91GeV_Z_uds.pdf}
\caption[The mean number of reconstructed charged particles as a function of the magnetic field strength for 91~GeV Z$\rightarrow$uds di-jet events.  The nominal ILD detector model was used and the pattern recognition has been fully cheated using the MC information.]{The mean number of reconstructed charged particles as a function of the magnetic field strength for 91~GeV Z$\rightarrow$uds di-jet events.  The nominal ILD detector model was used and the pattern recognition has been fully cheated using the MC information.}
\label{fig:bfieldchargedparticles}
\end{figure}

In summary, increasing the magnetic field strength is beneficial to the jet energy resolution because it reduces confusion from associating tracks to calorimetric energy deposits from charged particles.  The intrinsic energy resolution is also dependent upon the magnetic field strength, however, the effect is small in comparison to the confusion.  Although a magnetic field of 1.5~T would give good enough performance to be able to separate the hadronic decays of W and Z bosons, i.e. $\sigma_{E}/E \lesssim 3.8\%$ \cite{arXiv:0907.3577}, at the energies considered, increasing the field strength further significantly improves the detector performance.

%========================================================================================

\subsection{Inner ECal Radius}
The impact on the jet energy resolution of the overall size of the detector was studied by simulating detector models where the ECal inner radius was altered.  The ECal inner radii considered were 1208, 1408, 1608, 1808 (nominal) and 2008~mm.  

Figure \ref{fig:ecalinnerr} shows the dependence of the jet energy resolution on the ECal inner radius.  Increasing the ECal inner radius increases the separation between particles as they enter the calorimeters, which reduced the effect of confusion and improves the jet energy resolution.  As confusion is more dominant at higher energies, the benefits to using a larger ECal radius grow with increasing jet energy; increasing the ECal inner radius from 1208~mm to 2008~mm improves the jet energy resolution by $\sim 9\%$ for 45~GeV jets, but by $\sim 25\%$ for 250~GeV jets.  Figure \ref{fig:ecalinnerrbreak} shows the decomposition of the jet energy resolution into its different component.  These results explicitly show a reduction in confusion with increasing ECal inner radius; the confusion contribution goes from $\sim 3.4\%$ to $\sim 2.4\%$ when increasing the ECal inner radius from 1208~mm to 2008~mm.  The intrinsic energy resolution of the detectors shows no strong dependence on the inner ECal radius.  The apparent degradation in intrinsic energy resolution at low ECal inner radii is an artefact of the association of a single MC particle per calorimeter cell when running the cheated pattern recognition as explained in section \ref{sec:bfield}.  The dominant effect driving the jet energy resolution is, as expected, the confusion.  

\begin{figure}[h!]
\includegraphics[width=0.5\textwidth]{OptimisationStudies/Plots/JetEnergyResolutions/JER_vs_ECalInnerRadius.pdf}
\caption[The jet energy resolution using the nominal ILD detector as a function of the ECal inner radius for various jet energies.]{The jet energy resolution using the nominal ILD detector as a function of the ECal inner radius for various jet energies.}
\label{fig:ecalinnerr}
\end{figure}

\begin{figure}[h!]
\subfloat[]{\label{fig:ecalinnerr45break}\includegraphics[width=0.5\textwidth]{OptimisationStudies/Plots/JetEnergyResolutions/JER_vs_ECalInnerRadius_91GeV_DiJet_Breakdown.pdf}}
\subfloat[]{\label{fig:ecalinnerr250break}\includegraphics[width=0.5\textwidth]{OptimisationStudies/Plots/JetEnergyResolutions/JER_vs_ECalInnerRadius_500GeV_DiJet_Breakdown.pdf}}
\caption[Contributions to the jet energy resolution shown as function of the ECal inner radius using the nominal ILD detector model for \protect\subref{fig:ecalinnerr45break} 45~GeV jets and \protect\subref{fig:ecalinnerr250break} 250~GeV jets.  The black curves correspond to the standard reconstruction, the blue curves to the intrinsic energy resolution contribution to the jet energy resolution, the red curves to the confusion contribution to the jet energy resolution and the magenta curves to the confusion contribution to the jet energy resolution related solely to photon reconstruction.]{Contributions to the jet energy resolution shown as function of the ECal inner radius using the nominal ILD detector model for \protect\subref{fig:ecalinnerr45break} 45~GeV jets and \protect\subref{fig:ecalinnerr250break} 250~GeV jets.  The black curves correspond to the standard reconstruction, the blue curves to the intrinsic energy resolution contribution to the jet energy resolution, the red curves to the confusion contribution to the jet energy resolution and the magenta curves to the confusion contribution to the jet energy resolution related solely to photon reconstruction.}
\label{fig:ecalinnerrbreak}
\end{figure}

In conclusion, increasing the ECal inner radius benefits the jet energy resolution because it increases the separation between particles as they enter the calorimeter, which reduces confusion.

%========================================================================================
%========================================================================================

\section{Summary}
The effect of varying the configuration of the calorimeters, the magnetic field strength and the overall detector size on the single particle and jet energy resolutions were presented in this chapter.  For both the ECal and the HCal, the dominant factor determining the intrinsic energy resolution was the longitudinal sampling frequency.  However, the jet energy resolution had the strongest sensitivity to the ECal cell size, which shows that spatial recognition is more important when using particle flow calorimetry than intrinsic energy resolution.  The HCal cell size was found to be less significant than the ECal cell size for determining the jet energy resolution because separation of nearby particle showers in the HCal uses the spatial information gathered in the ECal.  In the particle flow paradigm, fine segmentation in the ECal can compensate for the coarser HCal granularities.  The jet energy resolution also showed a strong dependence on the magnetic field strength and the overall detector size.  Increasing both the magnetic field and overall detector size leads to greater separation of nearby particle showers in the calorimeters, which reduces the effect of confusion.  

%========================================================================================
%========================================================================================


  %% Physics Analysis
% \chapter{The Sensitivity of CLIC to Anomalous Gauge Couplings through Vector Boson Scattering}
\label{chap:PhysicsAnalysis}

%% Restart the numbering to make sure that this is definitely page #1!
\pagenumbering{arabic}

\chapterquote{Kids, you tried your best, and you failed miserably.  The lesson is, never try.}%
{Homer Simpson}

\section{Background}
A process that will show sensitivity to the $\alpha_{4}$ and $\alpha_{5}$ anomalous gauge couplings in the CLIC experiemnt is vector boson scattering.  There are several channels that will be affected by these anomalos couplings at CLIC and these are summarised in figures \ref{fig:vbsw}, \ref{fig:vbsz}, \ref{fig:vbswz} and \ref{fig:vbszw} where $q = \text{u, d, s, b, c}$ and $l = \text{e, } \mu \text{, } \tau \text{, } \nu_{e} \text{, } \nu_{\nu} \text{, } \nu_{\tau}$.

\begin{figure}
\begin{tikzpicture}[]
\begin{feynman}
\vertex (a1);
\vertex[above left=2cm of a1] (a2);
\vertex[above right=1cm and 2cm of a1] (a3) {\(W^{\pm},Z\)};
\vertex[below left=2cm of a1] (a4);
\vertex[below right=1cm and 2cm of a1] (a5) {\(W^{\mp},Z\)};
\vertex[above left=1cm of a2] (i1) {\(e^{-}\)};
\vertex[below left=1cm of a4] (i2) {\(e^{+}\)};
\vertex[above right=1cm and 3cm of a3] (i3) {\(q,l\)};
\vertex[below right=1cm and 3cm of a3] (i4) {\(\bar{q},\bar{l}\)};
\vertex[above right=1cm and 3cm of a5] (i5) {\(q,l\)};
\vertex[below right=1cm and 3cm of a5] (i6) {\(\bar{q},\bar{l}\)};
\vertex[above=1cm of a3] (v1) {\(\nu_{e}\)};
\vertex[below=1cm of a5] (v2) {\(\bar{\nu_{e}}\)};
\diagram* {
   (a1) -- [boson, edge label'=\(W^{-}\)] (a2) 
   (a1) -- [boson] (a3) 
   (a1) -- [boson, edge label=\(W^{+}\)] (a4) 
   (a1) -- [boson] (a5) 
   (i1) -- [fermion] (a2) -- [fermion] (v1)
   (v2) -- [fermion] (a4) -- [fermion] (i2)
   (i4) -- [fermion] (a3) -- [fermion] (i3)
   (i6) -- [fermion] (a5) -- [fermion] (i5)
};
\end{feynman}
\end{tikzpicture}
\caption[Feynman diagram of vector boson scattering at CLIC involving radiation of W bosons.]{Feynman diagram of vector boson scattering at CLIC involving radiation of W bosons.}
\label{fig:vbsw}
\end{figure}

\begin{figure}
\begin{tikzpicture}[]
\begin{feynman}
\vertex (a1);
\vertex[above left=2cm of a1] (a2);
\vertex[above right=1cm and 2cm of a1] (a3) {\(W^{\pm},Z\)};
\vertex[below left=2cm of a1] (a4);
\vertex[below right=1cm and 2cm of a1] (a5) {\(W^{\mp},Z\)};
\vertex[above left=1cm of a2] (i1) {\(e^{-}\)};
\vertex[below left=1cm of a4] (i2) {\(e^{+}\)};
\vertex[above right=1cm and 3cm of a3] (i3) {\(q,l\)};
\vertex[below right=1cm and 3cm of a3] (i4) {\(\bar{q},\bar{l}\)};
\vertex[above right=1cm and 3cm of a5] (i5) {\(q,l\)};
\vertex[below right=1cm and 3cm of a5] (i6) {\(\bar{q},\bar{l}\)};
\vertex[above=1cm of a3] (v1) {\(e^{-}\)};
\vertex[below=1cm of a5] (v2) {\(e^{+}\)};
\diagram* {
   (a1) -- [boson, edge label'=\(Z\)] (a2)
   (a1) -- [boson] (a3)
   (a1) -- [boson, edge label=\(Z\)] (a4)
   (a1) -- [boson] (a5)
   (i1) -- [fermion] (a2) -- [fermion] (v1)
   (v2) -- [fermion] (a4) -- [fermion] (i2)
   (i4) -- [fermion] (a3) -- [fermion] (i3)
   (i6) -- [fermion] (a5) -- [fermion] (i5)
};
\end{feynman}
\end{tikzpicture}
\caption[Feynman diagram of vector boson scattering at CLIC involving radiation of Z bosons.]{Feynman diagram of vector boson scattering at CLIC involving radiation of Z bosons.}
\label{fig:vbsz}
\end{figure}

\begin{figure}
\begin{tikzpicture}[]
\begin{feynman}
\vertex (a1);
\vertex[above left=2cm of a1] (a2);
\vertex[above right=1cm and 2cm of a1] (a3) {\(W^{-}\)};
\vertex[below left=2cm of a1] (a4);
\vertex[below right=1cm and 2cm of a1] (a5) {\(Z\)};
\vertex[above left=1cm of a2] (i1) {\(e^{-}\)};
\vertex[below left=1cm of a4] (i2) {\(e^{+}\)};
\vertex[above right=1cm and 3cm of a3] (i3) {\(q,l\)};
\vertex[below right=1cm and 3cm of a3] (i4) {\(\bar{q},\bar{l}\)};
\vertex[above right=1cm and 3cm of a5] (i5) {\(q,l\)};
\vertex[below right=1cm and 3cm of a5] (i6) {\(\bar{q},\bar{l}\)};
\vertex[above=1cm of a3] (v1) {\(\nu_{e}\)};
\vertex[below=1cm of a5] (v2) {\(e^{+}\)};
\diagram* {
   (a1) -- [boson, edge label'=\(W^{-}\)] (a2)
   (a1) -- [boson] (a3)
   (a1) -- [boson, edge label=\(Z\)] (a4)
   (a1) -- [boson] (a5)
   (i1) -- [fermion] (a2) -- [fermion] (v1)
   (v2) -- [fermion] (a4) -- [fermion] (i2)
   (i4) -- [fermion] (a3) -- [fermion] (i3)
   (i6) -- [fermion] (a5) -- [fermion] (i5)
};
\end{feynman}
\end{tikzpicture}
\caption[Feynman diagram of vector boson scattering at CLIC involving radiation of one Z and one W boson.]{Feynman diagram of vector boson scattering at CLIC involving radiation of one Z and one W boson.}
\label{fig:vbswz}
\end{figure}

\begin{figure}
\begin{tikzpicture}[]
\begin{feynman}
\vertex (a1);
\vertex[above left=2cm of a1] (a2);
\vertex[above right=1cm and 2cm of a1] (a3) {\(Z\)};
\vertex[below left=2cm of a1] (a4);
\vertex[below right=1cm and 2cm of a1] (a5) {\(W^{+}\)};
\vertex[above left=1cm of a2] (i1) {\(e^{-}\)};
\vertex[below left=1cm of a4] (i2) {\(e^{+}\)};
\vertex[above right=1cm and 3cm of a3] (i3) {\(q,l\)};
\vertex[below right=1cm and 3cm of a3] (i4) {\(\bar{q},\bar{l}\)};
\vertex[above right=1cm and 3cm of a5] (i5) {\(q,l\)};
\vertex[below right=1cm and 3cm of a5] (i6) {\(\bar{q},\bar{l}\)};
\vertex[above=1cm of a3] (v1) {\(e^{-}\)};
\vertex[below=1cm of a5] (v2) {\(\bar{\nu_{e}}\)};
\diagram* {
   (a1) -- [boson, edge label'=\(Z\)] (a2)
   (a1) -- [boson] (a3)
   (a1) -- [boson, edge label=\(W^{+}\)] (a4)
   (a1) -- [boson] (a5)
   (i1) -- [fermion] (a2) -- [fermion] (v1)
   (v2) -- [fermion] (a4) -- [fermion] (i2)
   (i4) -- [fermion] (a3) -- [fermion] (i3)
   (i6) -- [fermion] (a5) -- [fermion] (i5)
};
\end{feynman}
\end{tikzpicture}
\caption[Feynman diagram of vector boson scattering at CLIC involving radiation of one Z and one W boson.]{Feynman diagram of vector boson scattering at CLIC involving radiation of one Z and one W boson.}
\label{fig:vbszw}
\end{figure}

To determine whether an event is sensitive to $\alpha_{4}$ and $\alpha_{5}$ it will be necessary to determine whether the visible final states have been produced from the decay of W and Z bosons.  A key descriminator in this procedure will be the invariant mass of the W and Z candidates.  In light of this the hadronic decays of the W and Z bosons are only considered as the leptonic decays may contain neutrinos.

As the W and Z bosons in vector boson scattering are intermediate states in the Feynman diagrams, they will not be directly observed in the detector and will instead contribute to processes with the final states containing possible decay products of the bosons \nu{\nu}qqqq, l{\nu}qqqq and llqqqq.  In theory all processes will be affected by non zero $\alpha_{4}$ and $\alpha_{5}$, however, the effects may be extremely small as they contribute to very high order expansions of the Hamiltonian.  Event generation software does not calculate the expansions of the Hamiltonian to all orders, but instead tuncates the expansion to leave the dominant terms.  In the case of anomalous couplings this corresponds to certain final state cross sections being invariant to changes in $\alpha_{4}$ and $\alpha_{5}$.  

\section{Event Generation}

The event generation software used by the CLIC experiment is Whizard. 

To find out which states show sensitivity to the anomalous couplings two cross section calculations were made using different values of $\alpha_{4}$ and $\alpha_{5}$ for relevant processes involving the hadronic decays of the W and Z bosons from vector boson scattering, which can be found in table \ref{table:xstest}.  In the standard model the values of $\alpha_{4}$ and $\alpha_{5}$ are zero.  The only final states showing sensitivty to the anomalous couplings are \nu{\nu}qqqq, l{\nu}qqqq and llqqqq, which correspond to the final states from the Feynman diagrams shown above (\ref{fig:vbsw}, \ref{fig:vbsz}, \ref{fig:vbswz} and \ref{fig:vbszw}).  As this analysis focuses on the hadronic decays of the bosons involved in vector boson scattering the final states involving leptonic decays of the bosons e.g. \nu{\nu}llqq were not included in this cross check.  These leptonic dominated final states were also removed from the background samples used in this study as isolated lepton finding would largely veto all such events from selection.

The sensitvity of an individual event to the anomalous gauge couplings is determined through an event weight.  This weight corresponds to the ratio using non-zero $\alpha_{4}$ and $\alpha_{5}$ and using zero $\alpha_{4}$ and $\alpha_{5}$ of the square of the matrix element used in the cross section calculation.  This reweighting procedure has many advantages over the alternative procedure of generating new samples with fixed $\alpha_{4}$ and $\alpha_{5}$ most notably the absence of systematic errors that may appear in new event generation.  

The cross check shows that the most sensitive channel to the anomalous gauge couplings is the \nu{\nu}qqqq indicating that the best sensitivity measurement should focus upon this channel, which is the aim of this analysis.  

The CLIC experiment has a repository of simulated and reconstructed samples that can be used for physics analyses, however, for the relevant final states there is no way to calculate the event weights for these samples.  Therefore, new samples for which reweighting is possible were created and processed through the CLIC reconstruction chain.  New samples were created only for the \nu \nu qqqq final state as the l{\nu}qqqq and llqqqq final states have a significantly lower sensitivity.  As will be shown in subsequent chapters, the application of an isolated lepton finder in the selection processor will largely veto the l{\nu}qqqq and llqqqq final states, therefore, the absence of weight information for these final states will not significantly affect the sensitivity measurement based on the \nu{\nu}qqqq final state.

\begin{table}[h]
\begin{tabular}{*4l}    
\toprule
Final State & Cross Section [fb] & Cross Section [fb] & Percentage Change [\%] & CLIC Cross Section\\
 & ($\alpha_{4} = 0.00$,  $\alpha_{5} = 0.00$) & ($\alpha_{4} = 0.05$,  $\alpha_{5} = 0.05$) & &\\
\midrule
ee $\rightarrow$ \nu \nu qqqq  &2.08E+01       &3.46E+01 & +66.3 & 24.7\\
ee $\rightarrow$ l \nu qqqq  &1.12E+02       &1.13E+02 & +0.9 & 115.3\\
ee $\rightarrow$ \nu \nu qqqq  &5.97E+01       &6.86E+01 & +14.9 & 71.7\\
\bottomrule
\hline
\end{tabular}
\caption[Cross section for selected processes for given value of $\alpha_{4}$ and $\alpha_{5}$.]{Cross section for selected processes for given value of $\alpha_{4}$ and $\alpha_{5}$.  Channels considered where there were no changes to the cross section measurment when varying $\alpha_{4}$ and $\alpha_{5}$ were ee $\rightarrow$ qqqq, ee $\rightarrow$ \nu \nu qq, ee $\rightarrow$ l \nu qq, ee $\rightarrow$ llqq, ee $\rightarrow$ qq, e$\gamma_{BS}$ $\rightarrow$ qqqqe, $\gamma_{BS}$ e $\rightarrow$ qqqqe, e$\gamma_{EPA}$ $\rightarrow$ qqqqe, $\gamma_{EPA}$ e $\rightarrow$ qqqqe, $\gamma_{BS}$ $\rightarrow$ qqqq\nu, $\gamma_{BS}$ e $\rightarrow$ qqqq\nu, e$\gamma_{EPA}$ $\rightarrow$ qqqq\nu, $\gamma_{EPA}$ e $\rightarrow$ qqqq\nu, $\gamma_{BS}\gamma_{BS}$ $\rightarrow$ qqqq, $\gamma_{BS}\gamma_{EPA}$ $\rightarrow$ qqqq, $\gamma_{EPA}\gamma_{BS}$ $\rightarrow$ qqqq and $\gamma_{EPA}\gamma_{EPA}$ $\rightarrow$ qqqq}
\label{table:xstest}
\end{table}

\section{Validation Of New Samples}

An ideantical setup to that used for the official CLIC sample was used for the event generation and reconstructin.  Several reconstructed level distributions were compared to the official CLIC samples to ensure the samples were behaving appropriately.  These are shown below.


\section{Reconstruction}



\section{Analysis Processor and Jet Pairing}

\section{Event Selection}

\section{Fit}

  %% Summary
% \chapter{Summary}
\label{chap:summary}

\chapterquote{There, sir! that is the perfection of vessels!}
{Jules Verne, 1828--1905}

%========================================================================================

% Validation new technology for CLIC vertex detector
% Pushed boundaries of energy resolution
% Validated calibration of detector simulation
% Optimisation studies of calorimeters
% VBS + AGC Analysis shows good results compared to LHC

The work presented in this thesis has made significant contributions to the future linear collider in terms of both detector design, event reconstruction and demonstration of physics potential.  

A capacitively coupled pixel was prototyped and tested using both lab and test beam measurements to determine whether the devices were viable for use at the CLIC vertex detector.  The performance of these prototyped devices was extremely good, even with significant offsets between the sensor and readout ASICs that could appear in the manufacturing process.  Although modifications would be required for the final design of the sensor and readout ASICs, the technique of capacitively coupling is viable for use at the future linear collider.  

Studies into the calorimeter design have helped to clarify the detector parameters that are crucial for achieving outstanding performance when using particle flow calorimetry.  This allows for informed decisions to be made that minimise the cost of the detector, while retaining exceptional jet energy resolutions.  Reliability in the conclusions of this study could only be achieved by employing the calibration procedure that was developed for the linear collider simulation.  

Development of novel software techniques, which make full use of the segmentation of the linear collider calorimeters, led to a significant improvement in the energy resolution of the linear collider detector.  This improvement in energy resolution would be extremely expensive if it were achieved by modifying the design of the calorimeters, therefore, as well as extending the physics reach of the detector a significant cost saving has been made.   

This final study presented determined the sensitivity of the CLIC experiment to the anomalous gauge couplings $\alpha_{4}$ and $\alpha_{5}$ using the vector boson scattering process.  The signal final state ${\nu}{\nu}$qqqq was selected for this analysis based on the relative sensitivities of final states showing sensitivity to these couplings.  Background processes were then selected based on whether they could be confused with the signal.  An event selection procedure was applied to separate the signal and backgrounds.  The significance obtained from this event selection was 52.7 (90.6) for CLIC running at 1.4 (3) TeV.  Finally, a $\chi^{2}$ fit was applied to the distribution of the invariant mass of the system to determine the sensitivity of the CLIC experiment to the anomalous gauge couplings.  The sensitivity manifested itself in the form of event weights for the signal final state.  Using this procedure the one $\sigma$ confidence limits on the couplings, assuming the corresponding coupling is zero, were found to be:
%
\begin{equation}
-0.0082 < \alpha_{4} < 0.0116 \text{,} \\
-0.0055 < \alpha_{5} < 0.0078 \text{,}
\end{equation}
%
\noindent at 1.4 TeV and:
%
\begin{equation}
-0.0010 < \alpha_{4} < 0.0011 \text{,} \\
-0.0007 < \alpha_{5} < 0.0007 \text{,}
\end{equation}
%
\noindent at 3 TeV.  These limits significantly improve on the measurements made at the LHC, Run 1, by a factor of approximately 10 (100) at 1.4 (3) TeV \cite{Green:2016trm}.  This is a significant improvement indicating just one aspect of the physics capabilities of the linear collider.  This study adds further weight to the argument for the construction of a linear collider.   

%========================================================================================
\end{mainmatter}

%\begin{appendices}
%  %% The "\appendix" call has already been made in the declaration
%% of the "appendices" environment (see thesis.tex).
\chapter{Pointless extras}
\label{app:Pointless}

\chapterquote{%
Le savant n'\'etudie pas la nature parce que cela est utile; \\
\indent il l'\'etudie parce qu'il y prend plaisir, \\
\indent et il y prend plaisir parce qu'elle est belle.}%
{Henri Poincar\'e, 1854--1912}

Appendixes (or should that be ``appendices''?) make you look really clever, 'cos
it's like you had more clever stuff to say than could be fitted into the main
bit of your thesis. Yeah. So everyone should have at least three of them\dots

\section{Anomalous Gauge Coupling Quartic Vertices Of Relevance in Vector Boson Scattering}
\label{sec:expansionalpha4alpha5}

The anomalous gauge couplings involving $\alpha_{4}$ and $\alpha_{5}$ arise in EFT through the addition of the following terms to the Lagrangian.

\begin{equation}
\text{Tr}(V^{\mu}V_{\nu})] \text{Tr}(V^{\nu}V_{\mu})] \text{ and } [\text{Tr}(V^{\mu}V_{\mu})]^{2} 
\end{equation}

Where $V_{\mu}$ is defined in the following way.

\begin{equation}
V_{\mu} = \Sigma(D_{\mu}\Sigma)^{\dagger}
\end{equation}
%V_{\mu} = ig\frac{\sigma^{i}}{2}W^{i}_{\mu} + i\frac{g'}{2}B_{\mu}

and $\Sigma$, the Higgs field matrix, is defined as. 

\begin{equation}
\Sigma = \text{exp}(-\frac{i}{v}\textbf{w})
\end{equation}

Where $\textbf{w} = w^{a} \sigma^{a}$.  $w^{a}$ are the ... and $\sigma^{a}$ are the Pauli spin matrices.  The covariant derivative of the Higgs field matrix is

\begin{equation}
D_{\mu}\Sigma = (\partial_{\mu} + \frac{ig}{2}W_{\mu} - \frac{ig'}{2}B_{\mu}\sigma^{3})\Sigma
\end{equation}

For clarity consider the unitarity gauge where $\textbf{w} = 0$, which implies $\Sigma = 1$.  In this gauge $V_{\mu}$ takes the following form.

\begin{equation*}
V_{\mu} = \frac{i}{2}(gW_{\mu}^{i}\sigma^{i} - g'B_{\mu}\sigma^{3}) = \frac{i}{2}
  \begin{pmatrix}
    gW_{\mu}^{3} - g'B_{\mu} & g(W_{\mu}^{1} - iW_{\mu}^{2}) \\
    g(W_{\mu}^{1} + iW_{\mu}^{2}) & -gW_{\mu}^{3} + g'B_{\mu}
  \end{pmatrix} \\
= \frac{i}{2}
  \begin{pmatrix}
    \sqrt{g^{2} + g'^{2}} Z_{\mu} & g\sqrt{2}W^{+}_{\mu} \\
    g\sqrt{2}W^{-}_{\mu} & \sqrt{g^{2} + g'^{2}} Z_{\mu}
  \end{pmatrix}
\end{equation*}

Where the relationship between the mass and gauge symmetry basis are as follows.

\begin{equation}
  W^{+}_{\mu} = \frac{1}{\sqrt{2}}(W^{1}_{\mu} - i W^{2}_{\mu}) \\
  W^{-}_{\mu} = \frac{1}{\sqrt{2}}(W^{1}_{\mu} + i W^{2}_{\mu}) \\
  Z_{\mu} = c_{w}W^{3}_{\mu} - s_{w}B_{\mu} \\
  A_{\mu} = s_{w}W^{3}_{\mu} + c_{w}B_{\mu}
\end{equation}

With $c_{w} = \frac{g}{\sqrt{g^{2} + g'^{2}}}$ and $s_{w} = \frac{g'}{\sqrt{g^{2} + g'^{2}}}$.  Consider the expansion of the terms to be incldued in the Lagrangian.

\begin{equation}
V^{\mu}V_{\nu} = \frac{-1}{4}
  \begin{pmatrix}
    \sqrt{g^{2} + g'^{2}} Z^{\mu} & g\sqrt{2}W^{+\mu} \\
    g\sqrt{2}W^{-\mu} & \sqrt{g^{2} + g'^{2}} Z^{\mu}
  \end{pmatrix}
  \begin{pmatrix}
    \sqrt{g^{2} + g'^{2}} Z_{\nu} & g\sqrt{2}W^{+}_{\nu} \\
    g\sqrt{2}W^{-}_{\nu} & \sqrt{g^{2} + g'^{2}} Z_{\nu}
  \end{pmatrix}
\end{equation}

\begin{equation}
  \text{Tr}[V^{\mu}V_{\nu}] = \frac{-1}{2} ((g^{2} + g'^{2})Z^{\mu}Z_{\nu} + g^{2}W^{+\mu}W^{-}_{\nu} + g^{2}W^{-\mu}W^{+}_{\nu}) 
\end{equation}

\begin{equation}
  \text{Tr}[V^{\mu}V_{\nu}]\text{Tr}[V_{\mu}V^{\nu}] = \frac{(g^{2} + g'^{2})^2}{4}(Z^{\mu}Z_{\mu})^{2} + g^{2}(g^{2} + g'^{2})(Z^{\mu}Z^{\nu}W^{-}_{\mu}W^{+}_{\nu}) \\
+ \frac{g^{4}}{2}(W^{-\mu}W^{+}_{\mu})^{2} + \frac{g^{4}}{2}(W^{-\mu}W^{+\nu}W^{-}_{\mu}W^{+}_{\nu})
\end{equation}

\begin{equation}
  \text{Tr}[V^{\mu}V_{\mu}]^{2} = \frac{(g^{2} + g'^{2})^2}{4}(Z^{\mu}Z_{\mu})^{2} + g^{2}(g^{2} + g'^{2})(Z^{\mu}Z^{\nu}W^{-}_{\mu}W^{+}_{\nu}) \\
+ g^{4}(W^{-\mu}W^{+}_{\mu})^{2}
\end{equation}

These two terms change the cross section for the vector boson scattering processes at CLIC that involve $ZZ \rightarrow ZZ$, $W^{+}W^{-} \rightarrow ZZ$, $ZZ \rightarrow W^{+}W^{-}$ and $W^{+}W^{-} \rightarrow W^{+}W^{-}$.  

\begin{equation}
  \begin{tikzpicture}[]
  \begin{feynman}
    \vertex (a1);
    \vertex[above left=1cm of a1] (a2) {\(Z\)};
    \vertex[above right=1cm of a1] (a3) {\(Z\)};
    \vertex[below left=1cm of a1] (a4) {\(Z\)};
    \vertex[below right=1cm of a1] (a5) {\(Z\)};
    \diagram* {
       (a1) -- [boson] (a2) 
       (a1) -- [boson] (a3) 
       (a1) -- [boson] (a4) 
       (a1) -- [boson] (a5) 
    };
  \end{feynman}
  \end{tikzpicture}
  \subset (\alpha_{4} + \alpha_{5}) \frac{(g^{2} + g'^{2})^2}{4}
\end{equation}

\begin{equation}
  \begin{tikzpicture}[]
  \begin{feynman}
    \vertex (a1);
    \vertex[above left=1cm of a1] (a2) {\(W\)};
    \vertex[above right=1cm of a1] (a3) {\(W\)};
    \vertex[below left=1cm of a1] (a4) {\(Z\)};
    \vertex[below right=1cm of a1] (a5) {\(Z\)};
    \diagram* {
       (a1) -- [boson] (a2) 
       (a1) -- [boson] (a3) 
       (a1) -- [boson] (a4) 
       (a1) -- [boson] (a5) 
    };
  \end{feynman}
  \end{tikzpicture}
  \subset (\alpha_{4} + \alpha_{5}) g^{2}(g^{2} + g'^{2})
\end{equation}

\begin{equation}
  \begin{tikzpicture}[]
  \begin{feynman}
    \vertex (a1);
    \vertex[above left=1cm of a1] (a2) {\(W\)};
    \vertex[above right=1cm of a1] (a3) {\(W\)};
    \vertex[below left=1cm of a1] (a4) {\(W\)};
    \vertex[below right=1cm of a1] (a5) {\(W\)};
    \diagram* {
       (a1) -- [boson] (a2) 
       (a1) -- [boson] (a3) 
       (a1) -- [boson] (a4) 
       (a1) -- [boson] (a5) 
    };
  \end{feynman}
  \end{tikzpicture}
  \subset (\alpha_{4} + 2\alpha_{5}) \frac{g^{4}}{2} \text{ and } \frac{g^{4}}{2}\alpha_{4}
\end{equation}

\section{$\chi^{2}$ Contour Plots for Jet Algorithm Optimisation}

\begin{figure}
\subfloat[][Longitudinally Invariant Kt Algorithm, R = 0.7, Loose Selected PFOs, 1.4 TeV Events]{\label{fig:chi2jetalgoptkt0p70lpfos1400GeV} \includegraphics[width=0.3\textwidth]{PhysicsAnalysis/Plots/Chi2ContoursOptimisation/1400GeV/KtLPFOsR0p70.pdf}}
\subfloat[][Longitudinally Invariant Kt Algorithm, R = 0.9, Loose Selected PFOs, 1.4 TeV Events]{\label{fig:chi2jetalgoptkt0p90lpfos1400GeV} \includegraphics[width=0.3\textwidth]{PhysicsAnalysis/Plots/Chi2ContoursOptimisation/1400GeV/KtLPFOsR0p90.pdf}}
\subfloat[][Longitudinally Invariant Kt Algorithm, R = 1.1, Loose Selected PFOs, 1.4 TeV Events]{\label{fig:chi2jetalgoptkt1p10lpfos1400GeV} \includegraphics[width=0.3\textwidth]{PhysicsAnalysis/Plots/Chi2ContoursOptimisation/1400GeV/KtLPFOsR1p10.pdf}}\hfill
\subfloat[][Longitudinally Invariant Kt Algorithm, R = 0.7, Selected PFOs, 1.4 TeV Events]{\label{fig:chi2jetalgoptkt0p70spfos1400GeV} \includegraphics[width=0.3\textwidth]{PhysicsAnalysis/Plots/Chi2ContoursOptimisation/1400GeV/KtSPFOsR0p70.pdf}}
\subfloat[][Longitudinally Invariant Kt Algorithm, R = 0.9, Selected PFOs, 1.4 TeV Events]{\label{fig:chi2jetalgoptkt0p90spfos1400GeV} \includegraphics[width=0.3\textwidth]{PhysicsAnalysis/Plots/Chi2ContoursOptimisation/1400GeV/KtSPFOsR0p90.pdf}}
\subfloat[][Longitudinally Invariant Kt Algorithm, R = 1.1, Selected PFOs, 1.4 TeV Events]{\label{fig:chi2jetalgoptkt1p10spfos1400GeV} \includegraphics[width=0.3\textwidth]{PhysicsAnalysis/Plots/Chi2ContoursOptimisation/1400GeV/KtSPFOsR1p10.pdf}}\hfill
\subfloat[][Longitudinally Invariant Kt Algorithm, R = 0.7, Tight Selected PFOs, 1.4 TeV Events]{\label{fig:chi2jetalgoptkt0p70tpfos1400GeV} \includegraphics[width=0.3\textwidth]{PhysicsAnalysis/Plots/Chi2ContoursOptimisation/1400GeV/KtTPFOsR0p70.pdf}}
\subfloat[][Longitudinally Invariant Kt Algorithm, R = 0.9, Tight Selected PFOs, 1.4 TeV Events]{\label{fig:chi2jetalgoptkt0p90tpfos1400GeV} \includegraphics[width=0.3\textwidth]{PhysicsAnalysis/Plots/Chi2ContoursOptimisation/1400GeV/KtTPFOsR0p90.pdf}}
\subfloat[][Longitudinally Invariant Kt Algorithm, R = 1.1, Tight Selected PFOs, 1.4 TeV Events]{\label{fig:chi2jetalgoptkt1p10tpfos1400GeV} \includegraphics[width=0.3\textwidth]{PhysicsAnalysis/Plots/Chi2ContoursOptimisation/1400GeV/KtTPFOsR1p10.pdf}}\hfill
\caption[$\chi^{2}$ Sensitivity contours for the $\text{qqqq}\nu\nu$ final state arising from a fit to $\text{cos}\theta^{*}_{\text{Jets}}$ at 1.4 TeV for different values of jet reconstruction parameters.]{$\chi^{2}$ Sensitivity contours for the $\text{qqqq}\nu\nu$ final state arising from a fit to $\text{cos}\theta^{*}_{\text{Jets}}$ at 1.4 TeV for different values of jet reconstruction parameters.}
\label{fig:chi2jetalgopt1400GeV}
\end{figure}

\begin{figure}
\subfloat[][Longitudinally Invariant Kt Algorithm, R = 0.7, Loose Selected PFOs, 1.4 TeV Events]{\label{fig:a4chi2jetalgoptkt0p70lpfos1400GeV} \includegraphics[width=0.3\textwidth]{PhysicsAnalysis/Plots/Chi2ContoursOptimisation/1400GeV/KtLPFOsR0p70_alpha4.pdf}}
\subfloat[][Longitudinally Invariant Kt Algorithm, R = 0.9, Loose Selected PFOs, 1.4 TeV Events]{\label{fig:a4chi2jetalgoptkt0p90lpfos1400GeV} \includegraphics[width=0.3\textwidth]{PhysicsAnalysis/Plots/Chi2ContoursOptimisation/1400GeV/KtLPFOsR0p90_alpha4.pdf}}
\subfloat[][Longitudinally Invariant Kt Algorithm, R = 1.1, Loose Selected PFOs, 1.4 TeV Events]{\label{fig:a4chi2jetalgoptkt1p10lpfos1400GeV} \includegraphics[width=0.3\textwidth]{PhysicsAnalysis/Plots/Chi2ContoursOptimisation/1400GeV/KtLPFOsR1p10_alpha4.pdf}}\hfill
\subfloat[][Longitudinally Invariant Kt Algorithm, R = 0.7, Selected PFOs, 1.4 TeV Events]{\label{fig:a4chi2jetalgoptkt0p70spfos1400GeV} \includegraphics[width=0.3\textwidth]{PhysicsAnalysis/Plots/Chi2ContoursOptimisation/1400GeV/KtSPFOsR0p70_alpha4.pdf}}
\subfloat[][Longitudinally Invariant Kt Algorithm, R = 0.9, Selected PFOs, 1.4 TeV Events]{\label{fig:a4chi2jetalgoptkt0p90spfos1400GeV} \includegraphics[width=0.3\textwidth]{PhysicsAnalysis/Plots/Chi2ContoursOptimisation/1400GeV/KtSPFOsR0p90_alpha4.pdf}}
\subfloat[][Longitudinally Invariant Kt Algorithm, R = 1.1, Selected PFOs, 1.4 TeV Events]{\label{fig:a4chi2jetalgoptkt1p10spfos1400GeV} \includegraphics[width=0.3\textwidth]{PhysicsAnalysis/Plots/Chi2ContoursOptimisation/1400GeV/KtSPFOsR1p10_alpha4.pdf}}\hfill
\subfloat[][Longitudinally Invariant Kt Algorithm, R = 0.7, Tight Selected PFOs, 1.4 TeV Events]{\label{fig:a4chi2jetalgoptkt0p70tpfos1400GeV} \includegraphics[width=0.3\textwidth]{PhysicsAnalysis/Plots/Chi2ContoursOptimisation/1400GeV/KtTPFOsR0p70_alpha4.pdf}}
\subfloat[][Longitudinally Invariant Kt Algorithm, R = 0.9, Tight Selected PFOs, 1.4 TeV Events]{\label{fig:a4chi2jetalgoptkt0p90tpfos1400GeV} \includegraphics[width=0.3\textwidth]{PhysicsAnalysis/Plots/Chi2ContoursOptimisation/1400GeV/KtTPFOsR0p90_alpha4.pdf}}
\subfloat[][Longitudinally Invariant Kt Algorithm, R = 1.1, Tight Selected PFOs, 1.4 TeV Events]{\label{fig:a4chi2jetalgoptkt1p10tpfos1400GeV} \includegraphics[width=0.3\textwidth]{PhysicsAnalysis/Plots/Chi2ContoursOptimisation/1400GeV/KtTPFOsR1p10_alpha4.pdf}}\hfill
\caption[$\chi^{2}$ as a function of $\alpha_{4}$ assuming $\alpha_{5} = 0$ for the $\text{qqqq}\nu\nu$ final state arising from a fit to $\text{cos}\theta^{*}_{\text{Jets}}$ at 1.4 TeV for different values of jet reconstruction parameters.]{$\chi^{2}$ as a function of $\alpha_{4}$ assuming $\alpha_{5} = 0$ for the $\text{qqqq}\nu\nu$ final state arising from a fit to $\text{cos}\theta^{*}_{\text{Jets}}$ at 1.4 TeV for different values of jet reconstruction parameters.}
\label{fig:a4chi2jetalgopt1400GeV}
\end{figure}

\begin{figure}
\subfloat[][Longitudinally Invariant Kt Algorithm, R = 0.7, Loose Selected PFOs, 1.4 TeV Events]{\label{fig:a5chi2jetalgoptkt0p70lpfos1400GeV} \includegraphics[width=0.3\textwidth]{PhysicsAnalysis/Plots/Chi2ContoursOptimisation/1400GeV/KtLPFOsR0p70_alpha5.pdf}}
\subfloat[][Longitudinally Invariant Kt Algorithm, R = 0.9, Loose Selected PFOs, 1.4 TeV Events]{\label{fig:a5chi2jetalgoptkt0p90lpfos1400GeV} \includegraphics[width=0.3\textwidth]{PhysicsAnalysis/Plots/Chi2ContoursOptimisation/1400GeV/KtLPFOsR0p90_alpha5.pdf}}
\subfloat[][Longitudinally Invariant Kt Algorithm, R = 1.1, Loose Selected PFOs, 1.4 TeV Events]{\label{fig:a5chi2jetalgoptkt1p10lpfos1400GeV} \includegraphics[width=0.3\textwidth]{PhysicsAnalysis/Plots/Chi2ContoursOptimisation/1400GeV/KtLPFOsR1p10_alpha5.pdf}}\hfill
\subfloat[][Longitudinally Invariant Kt Algorithm, R = 0.7, Selected PFOs, 1.4 TeV Events]{\label{fig:a5chi2jetalgoptkt0p70spfos1400GeV} \includegraphics[width=0.3\textwidth]{PhysicsAnalysis/Plots/Chi2ContoursOptimisation/1400GeV/KtSPFOsR0p70_alpha5.pdf}}
\subfloat[][Longitudinally Invariant Kt Algorithm, R = 0.9, Selected PFOs, 1.4 TeV Events]{\label{fig:a5chi2jetalgoptkt0p90spfos1400GeV} \includegraphics[width=0.3\textwidth]{PhysicsAnalysis/Plots/Chi2ContoursOptimisation/1400GeV/KtSPFOsR0p90_alpha5.pdf}}
\subfloat[][Longitudinally Invariant Kt Algorithm, R = 1.1, Selected PFOs, 1.4 TeV Events]{\label{fig:a5chi2jetalgoptkt1p10spfos1400GeV} \includegraphics[width=0.3\textwidth]{PhysicsAnalysis/Plots/Chi2ContoursOptimisation/1400GeV/KtSPFOsR1p10_alpha5.pdf}}\hfill
\subfloat[][Longitudinally Invariant Kt Algorithm, R = 0.7, Tight Selected PFOs, 1.4 TeV Events]{\label{fig:a5chi2jetalgoptkt0p70tpfos1400GeV} \includegraphics[width=0.3\textwidth]{PhysicsAnalysis/Plots/Chi2ContoursOptimisation/1400GeV/KtTPFOsR0p70_alpha5.pdf}}
\subfloat[][Longitudinally Invariant Kt Algorithm, R = 0.9, Tight Selected PFOs, 1.4 TeV Events]{\label{fig:a5chi2jetalgoptkt0p90tpfos1400GeV} \includegraphics[width=0.3\textwidth]{PhysicsAnalysis/Plots/Chi2ContoursOptimisation/1400GeV/KtTPFOsR0p90_alpha5.pdf}}
\subfloat[][Longitudinally Invariant Kt Algorithm, R = 1.1, Tight Selected PFOs, 1.4 TeV Events]{\label{fig:a5chi2jetalgoptkt1p10tpfos1400GeV} \includegraphics[width=0.3\textwidth]{PhysicsAnalysis/Plots/Chi2ContoursOptimisation/1400GeV/KtTPFOsR1p10_alpha5.pdf}}\hfill
\caption[$\chi^{2}$ as a function of $\alpha_{5}$ assuming $\alpha_{4} = 0$ for the $\text{qqqq}\nu\nu$ final state arising from a fit to $\text{cos}\theta^{*}_{\text{Jets}}$ at 1.4 TeV for different values of jet reconstruction parameters.]{$\chi^{2}$ as a function of $\alpha_{5}$ assuming $\alpha_{4} = 0$ for the $\text{qqqq}\nu\nu$ final state arising from a fit to $\text{cos}\theta^{*}_{\text{Jets}}$ at 1.4 TeV for different values of jet reconstruction parameters.}
\label{fig:a5chi2jetalgopt1400GeV}
\end{figure}

\begin{figure}
\subfloat[][Longitudinally Invariant Kt Algorithm, R = 0.7, Loose Selected PFOs, 3 TeV Events]{\label{fig:chi2jetalgoptkt0p70lpfos3000GeV} \includegraphics[width=0.3\textwidth]{PhysicsAnalysis/Plots/Chi2ContoursOptimisation/3000GeV/KtLPFOsR0p70.pdf}}
\subfloat[][Longitudinally Invariant Kt Algorithm, R = 0.9, Loose Selected PFOs, 3 TeV Events]{\label{fig:chi2jetalgoptkt0p90lpfos3000GeV} \includegraphics[width=0.3\textwidth]{PhysicsAnalysis/Plots/Chi2ContoursOptimisation/3000GeV/KtLPFOsR0p90.pdf}}
\subfloat[][Longitudinally Invariant Kt Algorithm, R = 1.1, Loose Selected PFOs, 3 TeV Events]{\label{fig:chi2jetalgoptkt1p10lpfos3000GeV} \includegraphics[width=0.3\textwidth]{PhysicsAnalysis/Plots/Chi2ContoursOptimisation/3000GeV/KtLPFOsR1p10.pdf}}\hfill
\subfloat[][Longitudinally Invariant Kt Algorithm, R = 0.7, Selected PFOs, 3 TeV Events]{\label{fig:chi2jetalgoptkt0p70spfos3000GeV} \includegraphics[width=0.3\textwidth]{PhysicsAnalysis/Plots/Chi2ContoursOptimisation/3000GeV/KtSPFOsR0p70.pdf}}
\subfloat[][Longitudinally Invariant Kt Algorithm, R = 0.9, Selected PFOs, 3 TeV Events]{\label{fig:chi2jetalgoptkt0p90spfos3000GeV} \includegraphics[width=0.3\textwidth]{PhysicsAnalysis/Plots/Chi2ContoursOptimisation/3000GeV/KtSPFOsR0p90.pdf}}
\subfloat[][Longitudinally Invariant Kt Algorithm, R = 1.1, Selected PFOs, 3 TeV Events]{\label{fig:chi2jetalgoptkt1p10spfos3000GeV} \includegraphics[width=0.3\textwidth]{PhysicsAnalysis/Plots/Chi2ContoursOptimisation/3000GeV/KtSPFOsR1p10.pdf}}\hfill
\subfloat[][Longitudinally Invariant Kt Algorithm, R = 0.7, Tight Selected PFOs, 3 TeV Events]{\label{fig:chi2jetalgoptkt0p70tpfos3000GeV} \includegraphics[width=0.3\textwidth]{PhysicsAnalysis/Plots/Chi2ContoursOptimisation/3000GeV/KtTPFOsR0p70.pdf}}
\subfloat[][Longitudinally Invariant Kt Algorithm, R = 0.9, Tight Selected PFOs, 3 TeV Events]{\label{fig:chi2jetalgoptkt0p90tpfos3000GeV} \includegraphics[width=0.3\textwidth]{PhysicsAnalysis/Plots/Chi2ContoursOptimisation/3000GeV/KtTPFOsR0p90.pdf}}
\subfloat[][Longitudinally Invariant Kt Algorithm, R = 1.1, Tight Selected PFOs, 3 TeV Events]{\label{fig:chi2jetalgoptkt1p10tpfos3000GeV} \includegraphics[width=0.3\textwidth]{PhysicsAnalysis/Plots/Chi2ContoursOptimisation/3000GeV/KtTPFOsR1p10.pdf}}\hfill
\caption[$\chi^{2}$ Sensitivity contours for the $\text{qqqq}\nu\nu$ final state arising from a fit to $\text{cos}\theta^{*}_{\text{Jets}}$ at 1.4 TeV for different values of jet reconstruction parameters.]{$\chi^{2}$ Sensitivity contours for the $\text{qqqq}\nu\nu$ final state arising from a fit to $\text{cos}\theta^{*}_{\text{Jets}}$ at 1.4 TeV for different values of jet reconstruction parameters.}
\label{fig:chi2jetalgopt1400GeV}
\end{figure}

\begin{figure}
\subfloat[][Longitudinally Invariant Kt Algorithm, R = 0.7, Loose Selected PFOs, 3 TeV Events]{\label{fig:a4chi2jetalgoptkt0p70lpfos3000GeV} \includegraphics[width=0.3\textwidth]{PhysicsAnalysis/Plots/Chi2ContoursOptimisation/3000GeV/KtLPFOsR0p70_alpha4.pdf}}
\subfloat[][Longitudinally Invariant Kt Algorithm, R = 0.9, Loose Selected PFOs, 3 TeV Events]{\label{fig:a4chi2jetalgoptkt0p90lpfos3000GeV} \includegraphics[width=0.3\textwidth]{PhysicsAnalysis/Plots/Chi2ContoursOptimisation/3000GeV/KtLPFOsR0p90_alpha4.pdf}}
\subfloat[][Longitudinally Invariant Kt Algorithm, R = 1.1, Loose Selected PFOs, 3 TeV Events]{\label{fig:a4chi2jetalgoptkt1p10lpfos3000GeV} \includegraphics[width=0.3\textwidth]{PhysicsAnalysis/Plots/Chi2ContoursOptimisation/3000GeV/KtLPFOsR1p10_alpha4.pdf}}\hfill
\subfloat[][Longitudinally Invariant Kt Algorithm, R = 0.7, Selected PFOs, 3 TeV Events]{\label{fig:a4chi2jetalgoptkt0p70spfos3000GeV} \includegraphics[width=0.3\textwidth]{PhysicsAnalysis/Plots/Chi2ContoursOptimisation/3000GeV/KtSPFOsR0p70_alpha4.pdf}}
\subfloat[][Longitudinally Invariant Kt Algorithm, R = 0.9, Selected PFOs, 3 TeV Events]{\label{fig:a4chi2jetalgoptkt0p90spfos3000GeV} \includegraphics[width=0.3\textwidth]{PhysicsAnalysis/Plots/Chi2ContoursOptimisation/3000GeV/KtSPFOsR0p90_alpha4.pdf}}
\subfloat[][Longitudinally Invariant Kt Algorithm, R = 1.1, Selected PFOs, 3 TeV Events]{\label{fig:a4chi2jetalgoptkt1p10spfos3000GeV} \includegraphics[width=0.3\textwidth]{PhysicsAnalysis/Plots/Chi2ContoursOptimisation/3000GeV/KtSPFOsR1p10_alpha4.pdf}}\hfill
\subfloat[][Longitudinally Invariant Kt Algorithm, R = 0.7, Tight Selected PFOs, 3 TeV Events]{\label{fig:a4chi2jetalgoptkt0p70tpfos3000GeV} \includegraphics[width=0.3\textwidth]{PhysicsAnalysis/Plots/Chi2ContoursOptimisation/3000GeV/KtTPFOsR0p70_alpha4.pdf}}
\subfloat[][Longitudinally Invariant Kt Algorithm, R = 0.9, Tight Selected PFOs, 3 TeV Events]{\label{fig:a4chi2jetalgoptkt0p90tpfos3000GeV} \includegraphics[width=0.3\textwidth]{PhysicsAnalysis/Plots/Chi2ContoursOptimisation/3000GeV/KtTPFOsR0p90_alpha4.pdf}}
\subfloat[][Longitudinally Invariant Kt Algorithm, R = 1.1, Tight Selected PFOs, 3 TeV Events]{\label{fig:a4chi2jetalgoptkt1p10tpfos3000GeV} \includegraphics[width=0.3\textwidth]{PhysicsAnalysis/Plots/Chi2ContoursOptimisation/3000GeV/KtTPFOsR1p10_alpha4.pdf}}\hfill
\caption[$\chi^{2}$ as a function of $\alpha_{4}$ assuming $\alpha_{5} = 0$ for the $\text{qqqq}\nu\nu$ final state arising from a fit to $\text{cos}\theta^{*}_{\text{Jets}}$ at 3 TeV for different values of jet reconstruction parameters.]{$\chi^{2}$ as a function of $\alpha_{4}$ assuming $\alpha_{5} = 0$ for the $\text{qqqq}\nu\nu$ final state arising from a fit to $\text{cos}\theta^{*}_{\text{Jets}}$ at 3 TeV for different values of jet reconstruction parameters.}
\label{fig:a4chi2jetalgopt3000GeV}
\end{figure}

\begin{figure}
\subfloat[][Longitudinally Invariant Kt Algorithm, R = 0.7, Loose Selected PFOs, 3 TeV Events]{\label{fig:a5chi2jetalgoptkt0p70lpfos3000GeV} \includegraphics[width=0.3\textwidth]{PhysicsAnalysis/Plots/Chi2ContoursOptimisation/3000GeV/KtLPFOsR0p70_alpha5.pdf}}
\subfloat[][Longitudinally Invariant Kt Algorithm, R = 0.9, Loose Selected PFOs, 3 TeV Events]{\label{fig:a5chi2jetalgoptkt0p90lpfos3000GeV} \includegraphics[width=0.3\textwidth]{PhysicsAnalysis/Plots/Chi2ContoursOptimisation/3000GeV/KtLPFOsR0p90_alpha5.pdf}}
\subfloat[][Longitudinally Invariant Kt Algorithm, R = 1.1, Loose Selected PFOs, 3 TeV Events]{\label{fig:a5chi2jetalgoptkt1p10lpfos3000GeV} \includegraphics[width=0.3\textwidth]{PhysicsAnalysis/Plots/Chi2ContoursOptimisation/3000GeV/KtLPFOsR1p10_alpha5.pdf}}\hfill
\subfloat[][Longitudinally Invariant Kt Algorithm, R = 0.7, Selected PFOs, 3 TeV Events]{\label{fig:a5chi2jetalgoptkt0p70spfos3000GeV} \includegraphics[width=0.3\textwidth]{PhysicsAnalysis/Plots/Chi2ContoursOptimisation/3000GeV/KtSPFOsR0p70_alpha5.pdf}}
\subfloat[][Longitudinally Invariant Kt Algorithm, R = 0.9, Selected PFOs, 3 TeV Events]{\label{fig:a5chi2jetalgoptkt0p90spfos3000GeV} \includegraphics[width=0.3\textwidth]{PhysicsAnalysis/Plots/Chi2ContoursOptimisation/3000GeV/KtSPFOsR0p90_alpha5.pdf}}
\subfloat[][Longitudinally Invariant Kt Algorithm, R = 1.1, Selected PFOs, 3 TeV Events]{\label{fig:a5chi2jetalgoptkt1p10spfos3000GeV} \includegraphics[width=0.3\textwidth]{PhysicsAnalysis/Plots/Chi2ContoursOptimisation/3000GeV/KtSPFOsR1p10_alpha5.pdf}}\hfill
\subfloat[][Longitudinally Invariant Kt Algorithm, R = 0.7, Tight Selected PFOs, 3 TeV Events]{\label{fig:a5chi2jetalgoptkt0p70tpfos3000GeV} \includegraphics[width=0.3\textwidth]{PhysicsAnalysis/Plots/Chi2ContoursOptimisation/3000GeV/KtTPFOsR0p70_alpha5.pdf}}
\subfloat[][Longitudinally Invariant Kt Algorithm, R = 0.9, Tight Selected PFOs, 3 TeV Events]{\label{fig:a5chi2jetalgoptkt0p90tpfos3000GeV} \includegraphics[width=0.3\textwidth]{PhysicsAnalysis/Plots/Chi2ContoursOptimisation/3000GeV/KtTPFOsR0p90_alpha5.pdf}}
\subfloat[][Longitudinally Invariant Kt Algorithm, R = 1.1, Tight Selected PFOs, 3 TeV Events]{\label{fig:a5chi2jetalgoptkt1p10tpfos3000GeV} \includegraphics[width=0.3\textwidth]{PhysicsAnalysis/Plots/Chi2ContoursOptimisation/3000GeV/KtTPFOsR1p10_alpha5.pdf}}\hfill
\caption[$\chi^{2}$ as a function of $\alpha_{5}$ assuming $\alpha_{4} = 0$ for the $\text{qqqq}\nu\nu$ final state arising from a fit to $\text{cos}\theta^{*}_{\text{Jets}}$ at 3 TeV for different values of jet reconstruction parameters.]{$\chi^{2}$ as a function of $\alpha_{5}$ assuming $\alpha_{4} = 0$ for the $\text{qqqq}\nu\nu$ final state arising from a fit to $\text{cos}\theta^{*}_{\text{Jets}}$ at 3 TeV for different values of jet reconstruction parameters.}
\label{fig:a5chi2jetalgopt3000GeV}
\end{figure}

%% Big appendixes should be split off into separate files, just like chapters
%\input{app-myreallybigappendix}

%\end{appendices}

\begin{backmatter}
  \input{backmatter}
\end{backmatter}

%% Close
\end{document}

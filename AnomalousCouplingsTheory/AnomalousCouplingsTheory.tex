\chapter{Anomalous Gauge Coupling Theory}
\label{chap:anomalousgaugecouplingtheory}

\chapterquote{There, sir! that is the perfection of vessels!}
{Jules Verne, 1828--1905}

\section{Physics Theory}
Quantum field theory is the best model that currently exists for the behaviour of fundamental particles in the universe.  It is based upon fundamental symmetries that are obeyed by the Lagrangian, which dictates how particles behave.  The standard model is a non-abelian gauge theory of the $\text{SU}(3) \times \text{SU}(2_{\text{L}}) \times \text{U}(1)$ symmetry.  This symmetry acts to describe the electromagnetic, weak nuclear and strong nuclear forces observed in the universe.  This occurs via force mediating boson particles, 12 gauge bosons in total,1 photon, 3 weak bosons and 8 gluons.  

The gauge symmetry forbids a mass term for any of these bosons, however, the electroweak gauge bosons have been measured to be massive indicating a piece is missing from this theory.  This problem is solved via the introduction of the Higgs field that undergoes spontaneous symmetry breaking.

\subsection{Spontaneous Symmetry Breaking}
To begin with consider a complex scalar field \phi with the Klein-Gordon Lagrangian

\begin{equation}
\mathcal{L} = \partial^{\mu} \phi^{*} \partial_{\mu} \phi - \text{m}^{2} |\phi|^{2} = \partial^{\mu} \phi^{*} \partial_{\mu} \phi - \text{V}(\phi)
\end{equation}

$\partial^{\mu}$ is the partial derivate of the scalar field $\phi$ with respect to the position 4-vector.  $\mathcal{L}$ is Lorentz invariant.

This Lagrangian is invariant under the global symmetry $\phi \rightarrow e^{i\alpha} \phi$.  By adding extra terms to the Lagrangian that retain the invariance to the global symmetry it is possible to modify interactions of this scalar field.  This can be interpreted as modifying the potential, all terms in the Lagrangian without derivatives, of the scalar field.  For example consider a fourth order potential of the following form.

\begin{equation}
\text{V}(\phi) = \text{m}^{2}|\phi|^{2} + \lambda |\phi|^{4}
\end{equation}

The potential has a minima at zero, however, if $\text{m}^{2} < 0$ then the minima exists on a circle in the complex $\phi$ plane.  This complex circle is centred at $(0,0)$ and has radius $\sqrt{\frac{-\text{m}^{2}}{\lambda}}$.  To quantise this theory it is necessary to expand about the minima of the potential, however, in the case of $\text{m}^{2} < 0$ there are an infinite number of choices of minima to expand about.  Irrespective of the choice of minima for this configuration the symmetry $\phi \rightarrow e^{i\alpha} \phi$ is broken.  Fluctuations about the minima along the degenerate direction leave the potential unchanged, which is a consequence of the breaking of the $\phi \rightarrow e^{i\alpha} \phi$ symmetry; this is known as spontaneous symmetry breaking.

Goldstone's theorem implies that for Lorentz-invariant theories spontaneous symmetry breaking always leads to the existence a massless particle and this can be seen when expanding the complex scalar theory example about the minima.

\begin{equation}
\phi = \frac{1}{\sqrt{2}}(v + \phi_{1} + i \phi_{2})
\end{equation}

Where $\phi_{1/2}$ are real fields.  Applying this parameterisation to the Lagrangian shows yields a mass term of $\sqrt{-\text{m}^{2}}$ for the $\phi_{1}$ field while the mass of the $\phi_{2}$ field yields a massless particle.  

\begin{equation}
\mathcal{L} = \frac{1}{2}\partial^{\mu} \phi_{1} \partial_{\mu} \phi_{1} + \frac{1}{2}\partial^{\mu} \phi_{2} \partial_{\mu} \phi_{2} - \text{m}^{2}|\phi_{1}|^{2} + 0.|\phi_{2}|^{2} + ...
\end{equation}

This procedure is the origin of the gauge boson mass terms when considering local symmetries instead of global ones.  This can be see in the example under consideration by promoting the global to local symmetry let $\alpha \rightarrow \alpha(x)$ and $\partial^{\mu} \rightarrow D^{\mu} = \partial^{\mu} + ieA^{\mu}$ where $A^{\mu}$ a the gauge field that transforms as $A^{\mu} \rightarrow A^{\mu} - \partial^{\mu}\alpha(x)$.  The new Lagrangian becomes

\begin{equation}
\mathcal{L} = (D^{\mu} \phi)^{*} (D_{\mu} \phi) - \text{m}^{2} |\phi|^{2} - \lambda |\phi|^{4}
\end{equation}

If there is a non zero minima in the potential, v, then a gauge boson mass term appears of the form $+\frac{e^{2}v^{2}}{2} A^{\mu} A_{\mu}$.

\subsection{Electroweak Interactions}
The electroweak sector of the standard model is that related to the $\text{SU}(2_{\text{L}}) \times \text{U}1$ symmetry.  In this sector spontaneous symmetry breaking must occur in such a way as to leave three massive gauge bosons,  $\text{W}^{\pm}$ and Z, and one massless gauge bosons, the photon.  This can be achieved by considering a Higgs field H that is a doublet of the $\text{SU}(2_{\text{L}})$ symmetry, with weak hypercharge $\frac{1}{2}$, in the following potential.

\begin{equation}
\text{V}(\phi) = -\text{\mu}^{2}\text{H}^{\dagger}\text{H} + \lambda (\text{H}^{\dagger}\text{H})^{2}
\end{equation}

The minima of this field is at 

\begin{equation}
\sqrt{\text{H}^{\dagger}\text{H}} = \frac{v}{\sqrt{2}} = \sqrt{\frac{\mu^{2}}{2\lambda}}
\end{equation}

without loss of generality we may choose to expand this field around the point

\begin{equation}
\langle \text{H} \rangle = \binom{0}{\frac{v}{\sqrt{2}}}
\end{equation}

where v is real.  Consider the kinematic term in the Lagrangian, $\partial^{\mu} \text{H}^{\dagger} \partial_{\mu} \text{H}$.  The covariant derivative of this Higgs field must satisfy the $\text{SU}(2_{\text{L}}) \times \text{U}1$ gauge symmetry and so takes the form

\begin{equation}
D_{\mu} \text{H} = (\partial_{\mu} + ig\frac{\sigma^{i}}{2}W^{i}_{\mu} + i\frac{g'}{2}B_{\mu})H
\end{equation}

If there is mixing of the $\text{SU}(2_{\text{L}})$ and U(1) fields of the form $W^{3}_{\mu} = \text{cos}{\theta_{W}}Z_{\mu} + \text{sin}{\theta_{W}}Z_{\mu}$ and $B_{\mu} = -\text{sin}{\theta_{W}}Z_{\mu} + \text{cos}{\theta_{W}}Z_{\mu}$ and the unmixed $\text{SU}(2_{\text{L}})$ fields transform in the following way $W^{\pm}_{\mu} = \frac{1}{\sqrt{2}}(W^{1}_{\mu} \mp i W^{2}_{\mu})$, then non zero mass terms for the electroweak bosons arise.  $\theta_{W}$ is known as the Weinberg angle.

\begin{equation}
\frac{(gv)^{2}}{4} W^{+}_{\mu} W^{-\mu} + \frac{(g^{2} + g'^{2})v^{2}}{8} Z_{\mu} Z^{\mu}
\end{equation}

The boson mass terms are as follows
\begin{equation}
m_{W} = \frac{gv}{2}, m_{Z} = \frac{v\sqrt{g^{2} + g'^{2}}}{2} = \frac{m_{W}}{\text{cos}{\theta_{W}}}, m_{A} = 0
\end{equation}

This model yields a massless photon, $m_{A} = 0$, as well as producing massive electroweak gauge bosons in a ratio that match the measured values of $m_{W} = 80.385 \pm 0.015$ and $m_{Z} = 91.1876 \pm 0.0021$ \cite{Beringer:1900zz}.

\subsection{Effective Field Theory}
There are a number of features in the observable universe that cannot be accounted for using the standard model of particle physics.  However, the standard model is a very good description of the interactions between particles over the energy range being probed at modern particle collider experiments.  Therefore, any underlying theory governing the interactions of particles must behave like the standard model over the energy range, or distance scale, that modern particle collider experiments have covered and then will deviate to account for the true underlying theory.  Effective field theories (EFTs) work from this premise by assuming that an underlying theory has a momentum scale, $\Lambda$, below which standard model behaviour is replicated.  

Quantum field theories must be renormalizable to ensure that non-infinite predictions of the coefficients in the Lagrangian can be made and tested.  Infinities arise from non-renormalizable theories due to divergent integrals from loop diagrams that assume the theory being applied is valid at all energy and length scales.  Effective field theories act to avoid such problems by only integrating up to the momentum scale $\Lambda$ and not above it.  This avoids assumptions that the theory is applicable across all energy scales and allows the theory to make measurable predictions.  

In the EFT framework it is no longer necessary to enforce renormalization in the choice of operators appearing in the Lagrangian.  As the standard model should be replicated at the low energy scale it is appropriate when creating the EFT Lagrangian to append new operators, which may violate some or all the gauge symmetries of the standard model, to the standard model Lagrangian.

\begin{equation}
\mathcal{L}_{EFT} = \mathcal{L}_{SM} + \sum_{\text{dimension d}} \sum_{i} \frac{c_{i}^{(d)}}{\Lambda^{d-4}} \mathcal{O}_{i}^{(d)}
\end{equation}

Where the sums runs over d, all the dimensions of the operator, and i, any operator of dimension d.  The additional operator coefficients are chosen such that dimensionally they are consistent with the standard model Lagrangian.  

\subsection{$\alpha_{4}$ and $\alpha_{5}$}

In subsequent chapters a discussion of the sensitivity of the CLIC experiment to the $\alpha_{4}$ and $\alpha_{5}$ parameters takes place.  The origin of those parameters is discussed here.

$\alpha_{4}$ and $\alpha_{5}$ are coefficients that arise from an effective field theory.  They arise by considering the following operators in an EFT framework.

\begin{equation}
\text{Tr}[V^{\mu}V_{\mu}]^{2} \text{ and } \text{Tr}[V^{\mu}V_{\nu}]\text{Tr}[V^{\nu}V_{\mu}]
\end{equation}

Where 

\begin{equation}
V^{\mu} = 
 \end{equation}

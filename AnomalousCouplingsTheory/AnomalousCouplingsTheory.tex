\chapter{Anomalous Gauge Coupling Theory}
\label{chap:anomalousgaugecouplingtheory}

\chapterquote{There, sir! that is the perfection of vessels!}
{Jules Verne, 1828--1905}
%
%========================================================================================
%========================================================================================
%
\section{The Standard Model}
The Standard Model is a non-abelian gauge theory of the $\text{SU}(3) \times \text{SU}(2)_{\text{L}} \times \text{U}(1)$ symmetry group.  It provides a description of three of the four fundamental forces of nature: the electromagnetic, weak and strong nuclear forces.  CITE.  The Standard Model contains a total of 24 fermion fields: six flavours of quark, each with three colours, and six leptons.  A summary of the properties of these particles is given in table \ref{table:smleptons} and \ref{table:smquarks}.  As these fields are spin-$\frac{1}{2}$, the fields obey the Dirac equation:
%
\begin{equation}
\mathcal{L} = \overline{\psi}(i \slashed{\partial} - m)\psi \text{ ,}
\end{equation}
%
\noindent where $\mathcal{L}$ is the Lagrangian density and $m$ is a mass term.  The derivative term, $\slashed{\partial} = \gamma^{\mu}\partial_{\mu}$, represents a summation over the partial derivate, $\partial^{\mu} = (\frac{\partial}{\partial{t}},\frac{\partial}{\partial{x}},\frac{\partial}{\partial{y}},\frac{\partial}{\partial{z}})$, of the scalar field $\psi$ with respect to the position 4-vector, $x^{\mu} = (t,x,y,z)$, and the gamma matrices, $\gamma^{\mu}$.  Each of the gauge transformations of the Standard Model act are defined by a unitary operator \textrm{U}, which acts to transform the vector space, $\Psi$, formed from a combination of fermion fields, $\psi$, in the following way:
%
\begin{equation}
\Psi \rightarrow \Psi' = \textrm{U}\Psi \text{ .}
\end{equation}
%
The Lagrangian density describing the fermion fields in the Standard Model is invariant under a $\text{SU}(3)$, a $\text{SU}(2)_{\text{L}}$ and a U(1) gauge transformations .  The $\text{SU}(2)_{L}$ gauge symmetry acts on doublets formed, in the fundamental representation, of pairs of left handed chiral components, $\psi_{L} = \frac{1}{2}(1-\gamma_{5})\psi$.  The fundamental representation of a SU(N) theory allows the symmetry to be expressed using $N \times N$ matrices acting on a $N$-dimensional vector.  The right handed components, $\psi_{R} = \frac{1}{2}(1+\gamma_{5})\psi$, transform trivially as a singlet.  Similarly, the SU(3) symmetry acts on triplets formed of the fermion fields for each flavour of quark.  All fields transform under the fundamental representation of U(1).  The Standard Model is a non-abelian theory as gauge transformations from a given symmetry group, with the exception of the U(1) symmetry, do not commute with each other.  The invariance of the Standard Model Lagrangian to these gauge transformations is established by introducing 12 gauge fields, which are summarised in table \ref{table:smbosons}, via the covariant derivate of the fermion fields:
%
\begin{equation}
\partial^{\mu} \rightarrow D^{\mu} = \partial^{\mu} + ig_{1}YB^{\mu} + ig_{2} \textbf{T} \cdot \textbf{W}^{\mu} + ig_{3}\textbf{X} \cdot \textbf{G}^{\mu} \text{ ,}
\end{equation}
%
\noindent where $B^{\mu}$ is the gauge field for the U(1) symmetry, $\textbf{W}^{\mu}$ ($\text{W}^{\mu}_{j}, j =1,2,3$) are the fields of the $\text{SU}(2)_{\text{L}}$ symmetry and $\textbf{G}^{\mu}$ ($\text{G}^{\mu}_{j}, j =1,..,8$) are the fields of the SU(3).  $Y$ is the weak hypercharge, which is related to the chirality and flavour of the fermion it relates to.  $g_{1}$, $g_{2}$ and $g_{3}$ are three coupling constants related to the three symmetry groups of the Standard Model.  Mixing of the gauge fields for the U(1) and SU(2) symmetry of the form:
%
\begin{equation}
\text{Z}_{\mu} = \text{cos}{\theta_{W}} \text{W}^{3}_{\mu} - \text{sin}{\theta_{W}} \text{B}_{\mu} \text{ ,}\\
\text{A}_{\mu} = \text{sin}{\theta_{W}} \text{W}^{3}_{\mu} + \text{cos}{\theta_{W}} \text{B}_{\mu} \text{ ,}\\
\text{W}^{\pm}_{\mu} = \frac{1}{\sqrt{2}}(\text{W}^{1}_{\mu} \mp i \text{W}^{2}_{\mu}) \text{ ,}
\end{equation}
%
\noindent where:
%
\begin{equation}
\text{cos}{\theta_{W}} = \frac{g_{2}}{g_{1}+g_{2}} \text{ and } \text{sin}{\theta_{W}} = \frac{g_{1}}{g_{1}+g_{2}} \text{ ,}
\end{equation}
%
\noindent produce the $\text{W}^{\pm}$, Z and $\gamma$ gauge bosons.  This mixing ensures that the $\text{W}^{\pm}$ and Z bosons are massive, while the $\gamma$ remains massless.  The $\text{G}^{\mu}_{j}$ fields are the eight massless gluons of the strong force.   $\textbf{T}$ and $\textbf{X}$ are the generators for the SU(2) and SU(3) symmetries, which are typically chosen as:
%
\begin{equation}
T_{i} = \frac{1}{2}\tau_{i} \text{ ,} \\
X_{i} = \frac{1}{2}\lambda_{i} \text{ ,} \\
\end{equation}
%
\noindent where $\tau$ and $\lambda$ are the Pauli and the Gell-Mann matrices respectively.  The gauge fields of the Standard Model, $B_{\mu}$, $\textbf{W}_{\mu}$ and $\textbf{G}_{\mu}$, transform under the gauge transformations as:
%
\begin{equation}
K_{\mu} \rightarrow K'_{\mu} = UK_{\mu}U^{\dagger} + \frac{i}{g}(\partial^{\mu}U)U^{\dagger} \text{ ,} 
\end{equation}
%
\noindent where $K_{\mu}$ is any of $B_{\mu}$, $\textbf{W}_{\mu}$ and $\textbf{G}_{\mu}$ and $g$ is the coupling constants associated to the relevant gauge field.  As the $B_{\mu}$, $\textbf{W}_{\mu}$ and $\textbf{G}_{\mu}$ gauge fields are spin-1, they are described by the Proca action:
%
\begin{equation}
\mathcal{L} = -\frac{1}{4}F^{{\mu}{\nu}}_{i}F_{{\mu}{\nu}{i}} + \frac{1}{2}m_{K}^{2}K_{i\mu}K^{\mu}_{i} \text{ ,} 
\end{equation}
%
\noindent where:
%
\begin{equation}
F^{{\mu}{\nu}}_{i} = \partial^{\mu}K^{\nu}_{i} - \partial^{\nu}K^{\mu}_{i} - gf_{ijk}K^{\mu}_{j}K^{\nu}_{k} \text{ ,} 
\end{equation}
%
\noindent where $f_{ijk}$ are the fully anti-symmetric structure constants of the symmetry group, $K^{\mu}_{i}$ is the $i^{th}$ gauge field of the group and $m_{K}$ is a mass term for the gauge bosons.  The structure constants are defined from the commutation relations between generators of the symmetry group:
%
\begin{equation}
[T_{i},T_{j}] = if_{ijk}T_{k} \text{.} 
\end{equation}
%
\noindent These structure constants that govern the self-interactions for the gauge bosons.  There is only one structure constant for the U(1) symmetry, which is zero as the U(1) symmetry is abelian.  The SU(2) symmetry structure constants are $f_{ijk} = \epsilon_{ijk}$, where $\epsilon_{ijk}$ is the Levi-Civita tensor.  Due to the symmetries that are enforced in the Standard Model $m_{K} = 0$ for all the gauge fields present, however, it is clear that is is not the case in nature.  The gauge boson mass terms are introduced through the Higgs field, as described in section \ref{sec:higgsphysics}. 

\begin{table}[h!]
\centering
\begin{tabular}{l l r r r}
\hline
Generation & Particle & Mass [MeV] & Spin & Q/e \\
\hline
1 & $e^{-}$ & $548.579909070\pm0.000000016$ & 1/2 & -1 \\
& $\nu_{e}$ & - & 1/2 & 0 \\
\hline
2 & $\mu^{-}$ & $105.6583745\pm0.0000024$ & 1/2 & -1 \\
& $\nu_{\mu}$ & - & 1/2 & 0 \\
\hline
3 & $\tau^{-}$ & $1776.86\pm0.12$ & 1/2 & -1 \\
& $\nu_{\tau}$ & - & 1/2 & 0 \\
\end{tabular}
\caption[The mass, spin and electric charge (Q) of the leptons found in the Standard Model \cite{Beringer:1900zz}.  Neutrino masses have not been included in the above table as precise measurements are yet to be made.  However, oscillations between different neutrino flavour states have been observed, which indicates that the flavour and mass eigenstates differ and that the neutrinos have a non-zero mass.  The current upper bound on neutrino mass measurements is 2 eV.]{The mass, spin and electric charge (Q) of the leptons found in the Standard Model \cite{Beringer:1900zz}.  Neutrino masses have not been included in the above table as precise measurements are yet to be made.  However, oscillations between different neutrino flavour states have been observed, which indicates that the flavour and mass eigenstates differ and that the neutrinos have a non-zero mass.  The current upper bound on neutrino mass measurements is 2 eV.}
\label{table:smleptons}
\end{table}

\begin{table}[h!]
\centering
\begin{tabular}{l l r r r}
\hline
Generation & Particle & Mass [MeV] & Spin & Q/e \\
\hline
1 & $u$ & $2.2^{+0.6}_{-0.4}$ & 1/2 & +2/3 \\
 & $d$ & $4.7^{+0.5}_{-0.4}$ & 1/2 & -1/3 \\
\hline
2 & $c$ & $1270\pm30$ & 1/2 & +2/3 \\
 & $s$ & $98^{+8}_{-4}$ & 1/2 & +2/3 \\
\hline
3 & $t$ & $173210 \pm 510 \pm 710$ & 1/2 & +2/3 \\
 & $b$ & $4180^{+40}_{-30}$ & 1/2 & -1/3 \\
\end{tabular}
\caption[The mass, spin and electric charge (Q) of the quarks found in the Standard Model \cite{Beringer:1900zz}.  Each of the particles in the above table corresponds to three fermion fields, one for each of the three colours of the SU(3) symmetry.]{The mass, spin and electric charge (Q) of the quarks found in the Standard Model \cite{Beringer:1900zz}.  Each of the particles in the above table corresponds to three fermion fields, one for each of the three colours of the SU(3) symmetry.} 
\label{table:smquarks}
\end{table}

\begin{table}[h!]
\centering
\begin{tabular}{l l r r r}
\hline
Force & Particle & Mass [GeV] & Spin & Q/e \\
\hline
Electromagnetic & $\gamma$ & 0 & 1 & 0 \\
\hline
Weak Nuclear & $\text{W}^{\pm}$ & $80.385 \pm 0.015$ & 1 & $\pm$ 1 \\
& Z & $91.1876 \pm 0.0021$ & 1 & 0 \\
\hline
Strong Nuclear & $g$ ($\times 8$ colours) & 0 & 1 & 0 \\
\hline
Higgs & H & $125.1 \pm 0.3$ & 0 & 0 \\
\end{tabular}
\caption[The mass, spin and electric charge (Q) of the gauge bosons found in the Standard Model \cite{Beringer:1900zz}.  The $\gamma$ and $g$s theoretically have zero mass, which is consistent with measurements.  The upper bound on the $\gamma$ mass has been measured at $10^{-18}$ eV, while gluon masses of up to a few MeV have not been precluded.  The upper bound on the magnitude of the charge of the $\gamma$ is measured at $10^{-35}$.]{The mass, spin and electric charge (Q) of the gauge bosons found in the Standard Model \cite{Beringer:1900zz}.  The $\gamma$ and $g$s theoretically have zero mass, which is consistent with measurements.  The upper bound on the $\gamma$ mass has been measured at $10^{-18}$ eV, while gluon masses of up to a few MeV have not been precluded.  The upper bound on the magnitude of the charge of the $\gamma$ is measured at $10^{-35}$.}
\label{table:smbosons}
\end{table}

%========================================================================================
%========================================================================================

\section{Higgs Physics}
\label{sec:higgsphysics}
The gauge symmetries present in the Standard Model forbid a mass term for the gauge bosons, however, the electroweak gauge bosons are known to have mass, which indicates the theory as it stands is incomplete.  Similarly, mass terms, $m\overline{\psi}\psi$, for the leptons and quarks are forbidden as they violate the $\text{SU}(2)_{\text{L}}$ symmetry as the symmetry applied only to left hand chiral components.  The mass terms are generated in the Standard Model by introducing a Higgs field, which undergoes spontaneous symmetry breaking.

%========================================================================================

\subsection{Spontaneous Symmetry Breaking}
\label{sec:ssb}
To illustrate spontaneous symmetry breaking, consider a complex scalar field \psi with the Klein-Gordon Lagrangian:
%
\begin{equation}
\mathcal{L} = \partial^{\mu} \psi^{*} \partial_{\mu} \psi -m^{2} |\psi|^{2} = \partial^{\mu} \psi^{*} \partial_{\mu} \psi - V(\psi) \text{ ,}
\label{equ:kleingordon}
\end{equation}
%
\noindent where $\mathcal{L}$ the is Lorentz invariant Lagrangian density, $\partial^{\mu} = (\frac{\partial}{\partial{t}},\frac{\partial}{\partial{x}},\frac{\partial}{\partial{y}},\frac{\partial}{\partial{z}})$ is the partial derivative of the scalar field $\psi$ with respect to the position 4-vector $x^{\mu} = (t,x,y,z)$, $m$ is a mass term and $V(\psi)$ is the potential the field $\psi$.  This Lagrangian density is invariant under the global symmetry $\psi \rightarrow e^{i\alpha} \psi$.  By adding extra terms to the Lagrangian, which retain the invariance to the global symmetry, it is possible to modify the interactions of this scalar field.  For example consider modifying the potential of the scalar field to the following:
%
\begin{equation}
\text{V}(\psi) = m^{2}|\psi|^{2} + \lambda |\psi|^{4} \text{ ,}
\end{equation}
%
\noindent If $m^{2} > 0$, the potential has a minima at zero, however, if $m^{2} < 0$ then the minima exists on a circle in the complex $\psi$ plane.  This complex circle is centred at $(0,0)$ and has radius $v = \sqrt{-m^{2}/\lambda}$.  To quantise this theory it is necessary to expand about the minima of the potential.  However, in the case of $m^{2} < 0$ there are an infinite number of choices of minima to expand about and irrespective of the choice of minima for this configuration the symmetry $\psi \rightarrow e^{i\alpha} \psi$ is broken.  Fluctuations about the minima along the degenerate direction leave the potential unchanged, which is a consequence of the breaking of the $\psi \rightarrow e^{i\alpha} \psi$ symmetry; this is known as spontaneous symmetry breaking.

Goldstone's theorem \cite{Goldstone:1962es} implies that for Lorentz-invariant theories spontaneous symmetry breaking always leads to the existence a massless particle.  This can be seen in this example when expanding the complex scalar theory example about the minima.
%
\begin{equation}
\psi = \frac{1}{\sqrt{2}}(v + \psi_{1} + i \psi_{2}) \text{ ,}
\end{equation}
%
\noindent where $\psi_{1}$ and $\psi_{2}$ are real fields and $v = \sqrt{-m^{2}/\lambda}$.  Applying this parameterisation to the Lagrangian yields a mass term of $\sqrt{-m^{2}}$ for the $\psi_{1}$ field, however, there is no corresponding mass term for the $\psi_{2}$ field.  This indicates the $\psi_{2}$ field describes massless particles:
%
\begin{equation}
\mathcal{L} = \frac{1}{2}\partial^{\mu} \psi_{1} \partial_{\mu} \psi_{1} + \frac{1}{2}\partial^{\mu} \psi_{2} \partial_{\mu} \psi_{2} - m^{2}|\psi_{1}|^{2} + ... \text{ ,}
\end{equation}
%
\noindent This procedure is the origin of the gauge boson mass terms when applied to local symmetries instead of global ones.  For example consider the global symmetry, $\psi \rightarrow e^{i\alpha} \psi$, that exists in equation \ref{equ:kleingordon}.  Consider promoting this global symmetry to a local symmetry by letting $\alpha \rightarrow \alpha(x)$ and $\partial^{\mu} \rightarrow D^{\mu} = \partial^{\mu} + iA^{\mu}$, where $A^{\mu}$ a the gauge field that transforms as $A^{\mu} \rightarrow A^{\mu} - \partial^{\mu}\alpha(x)$.  In that case the new Lagrangian becomes:
%
\begin{equation}
\mathcal{L} = (D^{\mu} \psi)^{*} (D_{\mu} \psi) - \text{m}^{2} |\psi|^{2} - \lambda |\psi|^{4} \text{ .}
\end{equation}
%
\noindent If there is a non-zero minima in the potential, i.e. $m^{2} < 0$ and $v = \sqrt{-m^{2}/\lambda}$, then a gauge boson mass term appears of the form $+\frac{v^{2}}{2} A^{\mu} A_{\mu}$.

%========================================================================================

\subsection{Electroweak Interactions}
The electroweak sector of the Standard Model is that related to the $\text{SU}(2)_{\text{L}} \times \text{U}(1)$ symmetry.  In this sector, spontaneous symmetry breaking must occur in such a way as to give three massive gauge bosons,  $\text{W}^{\pm}$ and Z, and one massless gauge boson, the photon.  This can be achieved through a Higgs field, H, that transforms as a doublet under the $\text{SU}(2)_{\text{L}}$ symmetry.  The potential of the Higgs field, $\text{V}(\text{H})$, is:
%
\begin{equation}
\text{V}(\text{H}) = -\mu^{2}\text{H}^{\dagger}\text{H} + \lambda (\text{H}^{\dagger}\text{H})^{2} \text{ ,}
\end{equation}
%
\noindent where $\mu$ and $\lambda$ are constants.  The minima of this field is at:
%
\begin{equation}
\sqrt{\text{H}^{\dagger}\text{H}} = \frac{v}{\sqrt{2}} = \sqrt{\frac{\mu^{2}}{2\lambda}} \text{ .}
\end{equation}
%
\noindent Without loss of generality we may choose to expand this field around the point:
%
\begin{equation}
\langle \text{H} \rangle = \binom{0}{\frac{v}{\sqrt{2}}} \text{ ,}
\end{equation}
%
\noindent where $v$ is real.  In this case, consider the kinematic term in the Lagrangian, $\partial^{\mu} \text{H}^{\dagger} \partial_{\mu} \text{H}$.  The covariant derivative of this Higgs field must satisfy the $\text{SU}(2)_{\text{L}} \times \text{U}(1)$ gauge symmetry meaning it takes the form:
%
\begin{equation}
D_{\mu} \text{H} = (\partial_{\mu} + ig_{1}YB_{\mu} + ig_{2}\frac{\tau^{i}}{2}W^{i}_{\mu})H \text{ ,}
\end{equation}
%
\noindent where $g_{1}$ and $g_{2}$ are coupling constants for the relevant gauge symmetries, $Y = \frac{1}{2}$ is the weak hypercharge of the Higgs, $\tau^{i}$ are the Pauli matrices and $B_{\mu}$ is the gauge field related to the U(1) symmetry and $W^{i}_{\mu}$ are the gauge fields related to the $\text{SU}(2)_{\text{L}}$ symmetry.  If there is mixing of the $\text{SU}(2)_{\text{L}}$ and U(1) fields of the form:
%
\begin{equation}
\text{Z}_{\mu} = \text{cos}{\theta_{W}} \text{W}^{3}_{\mu} - \text{sin}{\theta_{W}} \text{B}_{\mu} \text{ ,}\\
\text{A}_{\mu} = \text{sin}{\theta_{W}} \text{W}^{3}_{\mu} + \text{cos}{\theta_{W}} \text{B}_{\mu} \text{ ,}\\
\text{W}^{\pm}_{\mu} = \frac{1}{\sqrt{2}}(\text{W}^{1}_{\mu} \mp i \text{W}^{2}_{\mu}) \text{ ,}
\end{equation}
%
\noindent then non-zero mass terms for the electroweak bosons arise:
%
\begin{equation}
\frac{(gv)^{2}}{4} \text{W}^{+}_{\mu} \text{W}^{-\mu} + \frac{(g^{2} + g'^{2})v^{2}}{8} \text{Z}_{\mu} \text{Z}^{\mu} \text{ .}\\
\end{equation}
%
\noindent This leads to the following boson masses:
%
\begin{equation}
\begin{aligned}
m_{\text{W}} = & \frac{gv}{2} \text{ ,} \\
m_{\text{Z}} = & \frac{v\sqrt{g^{2} + g'^{2}}}{2} = \frac{m_{W}}{\text{cos}{\theta_{W}}} \text{ ,} \\
m_{\text{A}} = & 0 \text{ ,}
\end{aligned}
\end{equation}
%
\noindent where $\theta_{W}$ is known as the Weinberg angle.  This model yields a massless photon, $m_{\text{A}} = 0$, as well as producing massive electroweak gauge bosons in a ratio that match the measured values of $m_{\text{W}} = 80.385 \pm 0.015$ and $m_{\text{Z}} = 91.1876 \pm 0.0021$ \cite{Beringer:1900zz}.  These masses predicate that $\rho = 1$ where $\rho$ is defined as:
%
\begin{equation}
\rho = \frac{m_{\text{W}}^{2}}{m_{\text{Z}}^{2}\text{cos}{\theta_{W}}^{2}} \text{ ,}
\label{equ:custodialsymmetry}
\end{equation}
%
\noindent This is ratio is constant, which is due to the custodial symmetry that exists after the $\text{SU}(2)_{\text{L}} \times \text{U}(1)$ symmetry has been broken to $\text{U}_{Q}(1)$.

%========================================================================================

\section{Effective Field Theory}
There are a number of features in the observable universe that cannot be accounted for using the Standard Model of particle physics.  However, the Standard Model is a very good description of the interactions between particles at the energies being probed at modern particle collider experiments.  Any underlying theory governing the interactions of particles must, therefore, behave like the Standard Model over these energies, or distance scales.  Above such energies the theory will deviate from the Standard Model to account for the full underlying theory.  Effective field theories (EFTs) work from this premise by assuming that the complete theory  has a momentum scale, $\Lambda$, below which Standard Model behaviour is replicated.  

Quantum field theories must be renormalizable to ensure that non-infinite predictions of the coefficients in the Lagrangian can be made and tested.  Infinities arise from non-renormalizable theories due to divergent integrals from loop diagrams that assume the theory being applied is valid at all energy and length scales.  Effective field theories act to avoid such problems by only integrating up to the momentum (mass) scale $\Lambda$ ($\Delta$) and not above it.  At the energy scale being considered, any infinities arising from the loop calculations in the EFT can be absorbed into a finite number of parameters.  This methodology avoids the assumption that the theory in question is applicable to all energy scales and allows measurable predictions to be made.  

In the EFT framework, it is no longer necessary to enforce renormalization in the choice of operators appearing in the Lagrangian.  As the Standard Model should be replicated at the low energy scale it is appropriate when creating the EFT Lagrangian to append new operators to the Standard Model Lagrangian that account for new physics.  These new terms do not necessarily have to obey the same gauge symmetries as those found in the Standard Model.  This gives the general form for an EFT Lagrangian as:
%
\begin{equation}
\mathcal{L}_{EFT} = \mathcal{L}_{SM} + \sum_{\text{dimension d}} \sum_{i} \frac{c_{i}^{(d)}}{\Lambda^{d-4}} \mathcal{O}_{i}^{(d)} \text{ ,}
\end{equation}
%
\noindent where $c_{i}^{(d)}$ are the coefficient of the operators $\mathcal{O}_{i}^{(d)}$ and the summations run over all the dimensions, $d$, of the operator and all unique operators, $i$, of dimension, $d$.  The presence of the $\Lambda^{d-4}$ in the denominator is needed to ensure correct dimensionality of the new terms being added to the Lagrangian, but it also sets the scale, $\Lambda$, of new physics being added.  At energies below this scale it is possible to find the dominant terms for the EFT and consider these as corrections to the Standard Model, while above this scale the EFT breaks down as each term in the Lagrangian has a non-negligible coefficient.  In the extremal limit, $\Lambda \rightarrow \infty$, the Standard Model is recovered as new physics is too far out of reach to have any impact on observables.

%========================================================================================

\section{Electroweak Chiral Lagrangian}
The introduction of a Higgs field transforming appropriately under the relevant gauge transformations is able to produce mass terms in the Lagrangian for the $\text{W}^{\pm}$ and Z bosons.  However, it is possible to introduce these terms via an EFT approach.  In such an approach the EFT considered is that of the electroweak chiral Lagrangian, which takes the place of the Higgs field in the Standard Model.  In this model the symmetry breaking is achieved by introducing a field $\Sigma(x)$ that transforms under the $\text{SU}(2)_{\text{L}}$ transformations U($x$) and the $\text{U}(1)$ transformations V($x$) as
%
\begin{equation}
\Sigma(x) \rightarrow U(x) \Sigma(x) V^{\dagger}(x) \text{ ,}
\end{equation}
%
\noindent $\Sigma(x)$ is parameterised as follows:
%
\begin{equation}
\Sigma(x) = \text{exp} \bigg(\frac{-i}{v} \Sigma^{3}_{a=1} \text{W}^{a}\tau^{a}\bigg)\text{ ,}
\end{equation}
%
\noindent where $\text{W}^{a}$ are the Goldstone bosons generated from the spontaneous symmetry breaking of the electroweak symmetry.  The following term if added to the Lagrangian generate the mass terms for the $\text{W}^{\pm}$ and Z gauge bosons:
%
\begin{equation}
\mathcal{L}_{M} = \frac{v^{2}}{4} \text{Tr} (\mathcal{D}^{\mu} \Sigma^{\dagger} \mathcal{D}_{\mu} \Sigma) = -\frac{v^{2}}{4}\text{Tr}(V_{\mu} V^{\mu}) \text{ ,} 
\end{equation}
%
\noindent where $V_{\mu} = (\mathcal{D}_{\mu}\Sigma) \Sigma^{\dagger}$ and $m_{W} = \frac{gv}{2}$.  The covariant derivate of the $\Sigma(x)$ is defined as:
%
\begin{equation}
\mathcal{D}_{\mu} \Sigma(x) = \partial_{\mu} \Sigma(x) + \frac{ig}{2}\text{W}_{\mu}^{a}\tau^{a}\Sigma(x) - \frac{ig'}{2}\text{B}_{\mu}\tau^{3}\Sigma(x) \text{ ,}
\end{equation}
%
\noindent where $\mathcal{D}_{\mu}$ is the covariant derivative of the $\Sigma$ field, $g$ and $g'$ are coupling constants for the U(1) and SU$(2)_\text{L}$ symmetries respectively and $\tau^{a}$ are the Pauli spin matrices, which are the generators for the SU(2) symmetry.  $\mathcal{L}_{M}$ contains mass terms for the gauge bosons, which match those produced from the introduction of a Higgs field as shown in section \ref{sec:ssb}.  The mass terms are:
%
\begin{equation}
\frac{v^{2}}{4}\text{Tr}[V^{\mu}V_{\mu}] = - \frac{(gv)^{2}}{4} \text{W}^{+}_{\mu} \text{W}^{-\mu} - \frac{(g^{2} + g'^{2})v^{2}}{8} \text{Z}_{\mu} \text{Z}^{\mu} \\
\begin{aligned}
m_{\text{W}} = & \frac{gv}{2} \text{ ,} \\
m_{\text{Z}} = & \frac{v\sqrt{g^{2} + g'^{2}}}{2} = \frac{m_{\text{W}}}{\text{cos}{\theta_{W}}} \text{ ,}
\end{aligned}
\end{equation}
%
\noindent The full parameterisation of the gauge boson interactions is found in \cite{Herrero:1994tj}.  The origin of these terms is the parameterisation of the Higgs field through $\Sigma$ that replicates the low energy behaviour of the Standard Model.  It was shown by Longhitano \cite{Longhitano:1980tm} that there are several relevant operators that are $\text{SU}(2)_{\text{L}} \times \text{U}(1)$ and CP invariant up to dimension 4 that should be considered in this theory.  Of those operators only two involve quartic massive gauge boson vertices and that preserve the custodial symmetry \cite{Belyaev:354051} shown in equation \ref{equ:custodialsymmetry}, which are:
%
\begin{equation}
\alpha_{4}\text{Tr}[V^{\mu}V_{\nu}]\text{Tr}[V^{\nu}V_{\mu}] \text{ and } \alpha_{5}\text{Tr}[V^{\mu}V_{\mu}]^{2} \text{ ,}
\end{equation}
%
\noindent The full expansion of these terms in terms of the massive gauge bosons $\text{W}^{\pm}$ and Z is given in the appendices \ref{sec:expansionalpha4alpha5}.  These terms contribute to the vector boson scattering processes involving $\text{W}^{\pm}$ and Z bosons.  A study into the sensitivity of the CLIC experiment to the anomalous gauge couplings $\alpha_{4}$ and $\alpha_{5}$ is presented in section \ref{chap:PhysicsAnalysis}.

%========================================================================================

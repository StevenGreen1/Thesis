\chapter{Introduction}
\label{chap:introduction}

\chapterquote{There, sir! that is the perfection of vessels!}
{Jules Verne, 1828--1905}

%========================================================================================

The Standard Model has proven to be one of the greatest accomplishments of modern day particle physics.  It provides precise predictions of particle interactions over a wide range of energies that have been experimental verified.  The final piece of the Standard Model to be discovered was the Higgs boson, which was found by the ATLAS \cite{Aad:2012tfa} and CMS \cite{Chatrchyan:2012xdj} experiments at the Large Hadron Collider (LHC) in 2012.  

Despite the remarkable descriptive power of the Standard Model, there are a number of features in the universe that it does not provide a description for.  How does gravity fit into the Standard Model?  Why is there an excess of matter over antimatter in the observable universe?  How does the "dark matter" predicted by astronomers couple with the particles in the Standard Model?  What are the properties of the Higgs field in the Standard Model?  All of these questions demand answering and despite the great successes of the LHC and previous generations of particle physics experiments, there is much more work to be done. 

The linear collider experiments aim to answer these questions.  One of the primary goals of the linear collider experiments is to study the Higgs field of the Standard Model.  A detailed description of the properties of the Higgs field is likely to help in the description of "dark matter" as many extensions of the Standard Model Higgs field contain particles that fit the properties of "dark matter".  The strongest coupling of the Higgs field to other Standard Model fields is, due to its mass, that of the top quark.  Therefore, the linear collider experiment will, alongside the description of the Higgs field, provide detailed description of the properties of the top quark.  Another goal is to provide high precision measurements of the electroweak sector in the Standard Model.  As the electroweak sector of the Standard Model is the only place where CP violation can occur, a detailed description will help determine why there is an excess of matter over antimatter in the universe.  Furthermore, the linear collider will expand the descriptive reach for many Standard Model extensions such as supersymmetry (SUSY).

The linear collider experiments place emphasis on precision measurements that a view to them guiding our understanding of the future particle physics.  For example, LEP electroweak data gave indirect information about the lightness of the Higgs years before its discovery.  The precision measurements are achieved through the application of Particle Flow Calorimetry, which is a revolutionary technique in detector design that offers exceptional energy resolution for jets.  This paradigm shift means the linear collider detectors are significantly different from those found in previous generations of particle colliders.  Furthermore, the design of the detectors is continually evolving meaning that the work being performed now will have permanent impact on the linear collider throughout its lifetime.

The thesis is organised as follows.  Chapter \ref{chap:anomalousgaugecouplingtheory} contains a summary of the Standard Model as well as an outline of the physics of interest related to the analysis presented in chapter \ref{chap:PhysicsAnalysis}.  Chapter \ref{chap:clicvertex} presents a study into a novel technology option for the Compact LInear Collider (CLIC) vertex detector.  Chapter \ref{chap:energyestimators} contains multiple studies related to the treatment of energy deposits in the linear collider simulation.  This begins with an outline of the calibration procedure for the linear collider detector simulation and is followed by a number of novel software based techniques for fully exploiting the linear collider detector design.  Finally, the chapter concludes with a study of the timing requirements applied in the software trigger that will be used at the linear collider experiment.  Chapter \ref{chap:detopt} presents an optimisation study of the linear collider calorimeters.  The starkest contract in detector design when comparing particle flow calorimeter and tradition calorimeter is the design of the calorimeters.  As particle flow calorimetry has never been fully used in a particle collider experiment before, these studies are of particular interest for guiding the detector design at the linear collider.  Chapter \ref{chap:PhysicsAnalysis} contains a study into anomalous gauge couplings that are sensitive to massive gauge boson quartic vertices at the CLIC experiment.  This study is of particular interest to the CLIC experiment as it provides a detailed probe of the electroweak symmetry breaking sector of the Standard Model as well as showing CLICs sensitivity to an extension to the Standard Model.  The thesis concludes with a summary in chapter \ref{chap:summary}.

%========================================================================================
%Physics introduction leading to chapter description.
%Max 2-3 pages
%Standard model extremely successful, missing gravity though
%Higgs discovery added crucial piece for mass generation
%Properties missing:
%Dark matter coupling
%CP violation -> Matter > antimatter
%ILC designed to study Higgs at 125 GeV, e+e-->Zh peak cross section at 250 GeV allows decay of Higgs to be measured by recoil of Z
%Top quark mass also studies.  Heaviest particle so coupling to H will be strong.
%Ultra-Precision for EW sector, which is only known CP violation 
%CLIC EW symmetry breaking at TeV scale
%and SUSY searches
%Improvement to LHC
%Precision Quantitative improvement of what is know , Jump in physics e.g. GUT in SUSY only proposed from precision EW  Lightness of Higgs from LEP electroweak data
%Discovery reach from processes with low production cross section at LHC
%Precision from PFlow
%Chapter Vertex
%Chapter Energy Est
%Chapter Calo Opt
%Chapter Physics Analysis
%========================================================================================
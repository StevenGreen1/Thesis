\chapter{Capacitively Coupled Pixel Detectors for the CLIC Vertex Detector}
\label{chap:theory}

\chapterquote{There, sir! that is the perfection of vessels!}
{Jules Verne, 1828--1905}

%========================================================================================
%========================================================================================

\section{Introduction}

Identification of heavy-flavour quarks and tau-leptons relies at the linear collider experiment will rely upon precise reconstruction of secondary displaced vertices that are produced when these particles decay.  Furthermore, the ability to accurately associate any daughter tracks, which are produced from such decays, to the secondary vertices is essential.  At the CLIC experiment this can only be realised using a vertex detector with a very high spatial resolution, of approximately 3 {\mu}m, and good geometric coverage, extending to low $\theta$ values.  The vertex detector must also have a low material budget (less than 0.2 $\text{X}_{0}$ per layer) in order to prevent additional decay vertices from material interactions (CONFIRM!), and maintain a low occupancy, despite the high presence of beam-induced background particles.  Low occupancy for the vertex detector will be achieved through the use of time-tagging, to an accuracy of 10~ns, to identify particles produced from the physics event of interest.  

As there are currently no technology options that fulfil all of the criteria for the CLIC vertex detector, the CLIC experiment has developed an extensive R\&D program where new technologies for the vertex detector are considered.  High-voltage complementary metal-oxide-semiconductor (HV-CMOS) sensors, which are capacitively coupled to a separate readout application-specific integrated circuit (ASIC) are one such option. The performance of prototype detectors based upon this technology and the impact of mechanical tolerances present in their manufacture are presented.  

%========================================================================================

\subsection{HV-CMOS}

Pixel detectors can be broadly classified in two distinct groups: hybrid detectors, where a separate sensor and readout chip are bonded together; and fully-integrated devices, where the collection diode is implanted in the same piece of silicon as the readout circuitry.  Fully-integrated devices have traditionally not been suitable for applications with tight timing requirements, due to relatively slow charge collection time and limited on-pixel functionality.  However, recent developments in CMOS technologies, a method for constructing integrated circuits, have led to new assembly designs that may overcome some of these issues. 

HV-CMOS is a processing technology whereby the n-MOS and p-MOS transistors forming the on-pixel electronics, are placed entirely within a deep n-well, as shown in \ref{fig:hvcmos}.  By varying the voltage applied at the gate terminal, n-MOS and p-MOS transistors are able to control the currently flowing between the source and drain terminals.  The gate voltage produces an inversion layer between the source and drain terminals that acts as a conduit allowing current to flow between the source and drain, as shown in figure \ref{fig:nmos}.  The magnitude of the voltage at the gate, with respect to the body, controls the width of the inversion layer and henceforth the magnitude of this current.  Logic operations can be performed directly on-pixel using various configurations of the n-MOS and p-MOS transistors.

\begin{figure}[h!]
\centering
\includegraphics[width=0.75\textwidth]{CLICdpVertex/Plots/HV-CMOSDiagram.png}
\caption[Schematic cross section of an HV-CMOS sensor: the deep n-well is the charge-collecting electrode and also contains additional CMOS circuits such as a preamplifier.  Image taken from \cite{Benoit:2016vup}.]{Schematic cross section of an HV-CMOS sensor: the deep n-well is the charge-collecting electrode and also contains additional CMOS circuits such as a preamplifier.  Image taken from \cite{Benoit:2016vup}.}
\label{fig:hvcmos}
\end{figure}

\begin{figure}[h!]
\centering
\includegraphics[width=0.375\textwidth]{CLICdpVertex/Plots/FETDiagram.pdf}
\caption[Schematic cross section of an n-MOS transistor.  p-MOS transistors have a similar cross section where the n and p doped regions are switched.]{Schematic cross section of an n-MOS transistor.  p-MOS transistors have a similar cross section where the n and p doped regions are switched.}
\label{fig:nmos}
\end{figure}

For the HV-CMOS, the deep n-well housing the on-pixel electronics acts as the charge collection diode as well as shielding the circuitry from the p-substrate.  This shielding allows for the application of a moderate bias voltage to the sensor bulk that produces a depletion region, which facilitates fast charge collection via a drift current.  In contrast, traditional monolithic active pixel sensors (MAPS) have a much smaller depletion region meaning charge collection occurs primarily through the slower mechanism of diffusion.  Furthermore, in conventional MAPS there is potential for competition for charge collection between the n-well collecting diode and the p-MOS transistors used to perform logic operations as the p-MOS transistors are embedded within an n-well.  The HV-CMOS technology does not suffer from this effect as the deep n-well collecting diode houses the p-MOS transistors.

The HV-CMOS technology offers the possibility of fast charge collection with integrated on-pixel functionality, however, several limitations still exist.  As the on-pixel electronics have to be placed inside the deep n-well and the n-wells have to be separated from each other, there is a limited physical area of the pixel that can be used for the transistor layout, which limits the available on-pixel functionality.  In addition to this, it is not possible to implement full CMOS logic inside the deep n-well as coupling between p-MOS transistors and the collection diode will lead to noise injection.  While it is possible to embed p-MOS transistors within a p-well to shield them from the deep n-well, quadruple-well technology, and give access to full CMOS logic this option is not readily available for prototyping.  By restricting the complexity of on-pixel electronics and using a separate readout ASIC, it is possible to overcome many of these issues.  When coupled with the fast charge collection time and removal of competition in charge collection this makes the HV-CMOS technology option highly desirable for use in the CLIC vertex detector.

%========================================================================================

\subsection{CLIC ASICs}
As the HV-CMOS technology is such a promising option for use at the CLIC vertex detector, prototype devices based on this technology have been developed for testing.  Two ASICs have been developed: the charge coupled pixel device version 3 (CCPDv3), a sensor chip based on the HV-CMOS technology, and the CLICpix, a readout chip providing additional on-pixel logic operations.  The pitch of the sensors, both the CCPDv3 and the CLICpix, produced for this study is 25 {\mu}m, which should be sufficient to meet the requirements for the CLIC vertex detector and the matrix size is 64$\times$64.  A schematic of these devices can be found in figure \ref{fig:ccpdandclicpix}.
 
\begin{figure}[h!]
\centering
\includegraphics[width=1.0\textwidth]{CLICdpVertex/Plots/schematic.pdf}
\caption[Schematic of CCPDv3 and CLICpix pixels.]{Schematic of CCPDv3 and CLICpix pixels.}
\label{fig:ccpdandclicpix}
\end{figure}

%========================================================================================

\subsubsection{CCPDv3}

The construction of the CCPDv3 was done using a 65nm CMOS process, which is the smallest size building blocks that can be used for creating the integrated circuits on a silicon wafer.  This pushes the boundaries of semiconductor device fabrication for particle collider pixel detectors as current LHC experiments typically use 130 and/or 250~nm CMOS processes in the pixel detectors \cite{Aaij:2244311, Dominguez:1481838}.  This makes it possible to have more complex on-pixel circuitry incorporated into the CLIC vertex detector than would be possible in previous generations of pixel detectors.  

%========================================================================================

\subsubsection{CLICpix}

The CLICpix is a hybrid pixel readout chip that has been developed for the CLIC vertex detector.  Each CLICpix pixel contains a charge-integrating amplifier connected to a discriminator, as shown in figure \ref{fig:ccpdandclicpix}.  The discriminator fires for as long as the input signal is over a given threshold and this output is then used as the input for further logic operations.  The additional logic operations record the time of arrival and magnitude of the collected charge, using a Time over Threshold (ToT) measurement.  The ToT is stored in a 4-bit on-pixel counter.  The CLICpix operates using a shutter-based readout, where the entire matrix is kept active while the shutter is open and when closed the matrix is readout in its entirety.  This is designed in order to match the expected beam structure for the CLIC experiment, as the accelerator will deliver bunch trains of $e^{+}$ and $e^{-}$ that are separated by 20~ms.  Each bunch train contains 312~bunches with a spacing of 0.5~ns, giving a total train length of 156~ns.  Furthermore, the shutter-based readout is well suited to power-pulsing, whereby, the power to front-end electronics is turned off between bunch crossings.  This helps to significantly reduce the power consumption of the pixel detector.

The threshold voltage seen by each CLICpix pixel is slightly different, due to variations in the manufacturing process.  If these variations are not accounted for then the behaviour of the device across the matrix will not be uniform.  To minimise the impact of these fluctuations each CLICpix pixel contains a 4-bit local adjustment to the threshold voltage, which is calibrated to unify the response across the matrix.  The threshold 'equalisation' is achieved by performing two threshold scans across the matrix, once with all four bits set to 0 (no local threshold adjustment), and a second time with all four bits set to 1 (maximum local threshold adjustment).  For each scan, the baseline voltage of each pixel is determined. By applying a linear interpolation between the 0000 and 1111 cases, each pixel can be tuned to a common point, such that all pixels respond at the same global threshold.

%========================================================================================

\subsection{Capacitive Coupling}

Solder bump-bonding is the typical method used to connect sensor and readout ASICs in hybrid pixel detectors, however, it is expensive and sets limits on the thickness of both ASICs that is required for mechanical stability.  An alternative procedure for connecting the ASICs involves using a thin layer of glue to form a capacitive connection between the two.  This procedure reduces the cost and material budget with respect to bump-bonding making it highly desirable for use at the CLIC vertex detector.  However, as this procedure has only recently been used to produce prototype devices extensive tests have to be performed to determine whether this option is feasible.  In order to make this procedure viable it is necessary to implement an amplifier in the CCPDv3 on-pixel logic, shown in figure \ref{fig:ccpdandclicpix}, to boost the signal to overcome the intrinsically small capacitance of the gluing layer.

%========================================================================================
%========================================================================================

\section{Device Fabrication}

When using a capacitively coupled glue layer to connect the sensor and readout ASIC there are two main issues that arise from the device manufacturing procedure: the uniformity of the gluing layer and the physical alignment of the sensor and readout pads.  The former has been investigated in [CERN FABRICATION NOTE CITE] and the latter is the focus of this study.  A number of pixel detectors have been constructed that purposefully contained misalignments between the CCPDv3 and the CLICpix pads, as shown in figure \ref{fig:alignment}, so that the impact of a misalignment on detector performance can be characterised.  Table \ref{table:alignment} contains a summary of the samples studied.

\begin{figure}[h!]
\centering
\includegraphics[width=1.0\textwidth]{CLICdpVertex/Plots/misalignedPads.pdf}
\caption[Schematic of alignment of CCPDv3 and CLICpix sensors studied in this analysis.]{Alignment schematic of the CCPDv3 + CLICpix detectors studied.  The red dotted line represents the CCPDv3 pad and the solid black line represents the CLICpix top metal layer.  From left to right; centred pixels, 1/4 offset (6.25 {\mu}m) and 1/2 offset (12.5 {\mu}m).}
\label{fig:alignment}
\end{figure}

\begin{table}[h!]
\centering
\begin{tabular}{ l l }
\hline
Assembly & Alignment \\ 
\hline
SET9 & Centred \\
SET10 & $\frac{1}{4}$ Offset \\
%SET11 & $\frac{1}{2}$ Offset \\
SET12 & Centred \\
SET13 & Centred \\
%SET14 & $\frac{1}{2}$ Offset REMOVED, NO?\\
SET15 & Centred \\
SET16 & $\frac{1}{2}$ Offset \\
\hline
\end{tabular}
\caption[A list of the devices considered in this study, showing the misalignment of the CCPDv3 and CLICpix coupling pads.]{A list of the devices considered in this study, showing the misalignment of the CCPDv3 and CLICpix coupling pads.}
\label{table:alignment}
\end{table}

The full details of the glueing procedure can be found in [CERN FABRICATION NOTE CITE], along with a study of the absolute precision of the manufacturing procedure.  It was found that for devices constructed in an identical fashion to those considered here, the glue layer thicknesses were less than 1~{\mu}m and the precision on the pad positioning was less than 0.5~{\mu}m.  

%========================================================================================
%========================================================================================

\section{Device Characterisation}
The devices produced for this analysis were characterised using a series of lab experiments and in realistic experimental conditions using the CERN SPS test beam.  Due to the complexities of testing devices in a test beam, extensive lab test were performed first to characterise as many properties of the assemblies as possible.  The lab experiments performed were as follows:

\begin{itemize}
\item \textbf{Radioactive source measurements}.  The goal of this measurement is to examine both the output of the CCPDv3 voltage and the response of the CLICpix readout ASIC when a radioactive source is used to deposit charge within the CCPDv3 sensor.  
\item \textbf{Test pulse calibration of the CLICpix chip}.  The goal of this measurement is to calibrate the response of the CLICpix sensor.  This is achieved by injecting a voltage pulse of fixed height directly into the input of the ASIC and examining the output response.  
\end{itemize} 

%========================================================================================

\subsection{Source Measurements}
For this measurement a radioactive source was used to deposit charge within the CCPDv3 sensor and used to characterise the response of the whole assembly.  The CCPDv3 sensor converts the deposited charge into a voltage, which in turn passes through the capacitively coupled glue layer and into the CLICpix chip.  Measurements were made of the output voltage produced by the CCPDv3 and the response of the CLICpix readout chip, in units of ToT.  As the exact amount of charge deposited by the radioactive sources is unknown, this experiment focuses on examining the form of the voltage signal produced by the CCPDv3 and determining the response of the CLICpix chip as a function of this voltage.  As the CCPDv3 signal passes through the capacitively coupled glue layer before being measured by the CLICpix chip, this experiment characterises the behaviour of the gluing layer as well as the sensor and readout chips.

%========================================================================================

\subsubsection{Experimental Setup}
The radioactive material used in this experiment was a $\text{Sr}^{90}$.  $\text{Sr}^{90}$ undergoes $\beta^{-}$ decay to form $\text{Y}^{90}$, which in turn undergoes $\beta^{-}$ decay to form the stable isotope $\text{Z}^{90}$.  Each $\beta^{-}$ decay produces an $\text{e}^{-}$ and a $\bar{\nu_{e}}$, and the $\text{e}^{-}$ goes on to deposit charge in the CCPDv3 sensor.  The $\text{Sr}^{90}$ source used for this experiment had an activity of 29.6 MBq.  

The radioactive source was positioned directly above the back-side of the CCPDv3 sensor, and measurements were made of both the ToT output from the CLICpix and the CCPDv3 analogue signal for individual pixels on the sensor.  The CCPDv3 pulse shape was recorded on a fast sampling oscilloscope that was also used to trigger the CLICpix readout.  The on-pixel event counter, which is located in the CLICpix ASIC, was used to veto events where multiple hits occurred within the active shutter period.  The CCPDv3 sensor was biased to 60~V during this experiment.  The analogue output has a baseline voltage of $\approx 1.15$~V, with signal saturation occurring around a height of 700~mV.  Examples of CCPDv3 output voltage pulses when using the $\text{Sr}^{90}$ source can be seen in figure \ref{fig:pulseshapes}.

\begin{figure}[h!]
\centering
\subfloat[]{\label{fig:pulseshape1}\includegraphics[width=0.5\textwidth]{CLICdpVertex/Plots/HV-CMOS/Frames/PulseShape01000NoOffset.pdf}}
\subfloat[]{\label{fig:pulseshape2}\includegraphics[width=0.5\textwidth]{CLICdpVertex/Plots/HV-CMOS/Frames/PulseShape01005NoOffset.pdf}}\hfill
\subfloat[]{\label{fig:pulseshape3}\includegraphics[width=0.5\textwidth]{CLICdpVertex/Plots/HV-CMOS/Frames/PulseShape01006NoOffset.pdf}}
\subfloat[]{\label{fig:pulseshape4}\includegraphics[width=0.5\textwidth]{CLICdpVertex/Plots/HV-CMOS/Frames/PulseShape01008NoOffset.pdf}}
\caption[CCPDv3 voltage pulses produced by radioactive $\text{Sr}^{90}$ source.]{CCPDv3 voltage pulses produced by radioactive $\text{Sr}^{90}$ source.}
\label{fig:pulseshapes}
\end{figure}

%========================================================================================

\subsubsection{Analysis}
The quantities of interest related to the CCPDv3 output voltage are the pulse height, the peak height of the voltage pulse, and the rise time, the time it takes for the CCPDv3 to reach the pulse height.  For ease of analysis the baseline voltage is subtracted from the CCPDv3 output voltage and the pulse shape inverted before the following analysis is applied to extract the variables of interest.  The pulse height is defined using a Gaussian fit to the peak of the voltage pulse.  This method is used to minimise the dependency of the pulse height on small fluctuations in the output voltage.  The peak of the voltage pulse is defined as the region where the change in the CCPDv3 voltage output is greater than 90\% of the maximum change in the CCPDv3 voltage output.  The rise time us defined as the time taken for the signal to go from 10\% to 90\% of the maximum change in the CCPDv3 voltage output.  This definition makes the rise time metric more robust against fluctuations in the CCPDv3 voltage output.  Examples of the calculation of these metrics for a representative pulse are shown in figure \ref{fig:pulseshapeanalysis}.  Due to the design of the CCPDv3 matrix, it is only possible to record the CCPDv3 voltage output for 15 pixels running along one edge of the 64 $\times$ 64 matrix.  Therefore, in the subsequent analysis data was taken for each of these accessible pixels and combined.

\begin{figure}[h!]
\centering
\subfloat[]{\label{fig:pulseshapeanalysisvoltage}\includegraphics[width=0.5\textwidth]{CLICdpVertex/Plots/HV-CMOS/Frames/PulseShape01000FittingVoltage.pdf}}
\subfloat[]{\label{fig:pulseshapeanalysistime}\includegraphics[width=0.5\textwidth]{CLICdpVertex/Plots/HV-CMOS/Frames/PulseShape01000FittingRiseTime.pdf}}
\caption[An example calculation of the pulse height and rise time for the CCPDv3 output voltage.  In this example the black line show the CCPDv3 output voltage having removed the baseline voltage and inverted the pulse shape as a function of time.  This output voltage is produced when charge is deposited in the CCPDv3 from the decay products of a radioactive source, $\text{Sr}^{90}$.  \protect\subref{fig:pulseshapeanalysisvoltage} The definition of the pulse height.  Pulse height is defined as the amplitude of a Gaussian function fitted across the peak of the voltage pulse.  The peak of the voltage pulse is defined as the region where the voltage in excess of 90\% of the raw pulse height and is indicated in the figure by the red arrow.  The red dotted line shows the Gaussian fit used to extract the pulse height.  \protect\subref{fig:pulseshapeanalysistime} The definition of rise time.  Rise time is defined as the time taken for the CCPDv3 voltage to rise from 10\% to 90\% of the raw pulse height.  The rise time, and change in CCPDv3 output voltage over this time, are shown in the figure by the blue arrows.]{An example calculation of the pulse height and rise time for the CCPDv3 output voltage.  In this example the black line show the CCPDv3 output voltage having removed the baseline voltage and inverted the pulse shape as a function of time.  This output voltage is produced when charge is deposited in the CCPDv3 from the decay products of a radioactive source, $\text{Sr}^{90}$.  \protect\subref{fig:pulseshapeanalysisvoltage} The definition of the pulse height.  Pulse height is defined as the amplitude of a Gaussian function fitted across the peak of the voltage pulse.  The peak of the voltage pulse is defined as the region where the voltage in excess of 90\% of the raw pulse height and is indicated in the figure by the red arrow.  The red dotted line shows the Gaussian fit used to extract the pulse height.  \protect\subref{fig:pulseshapeanalysistime} The definition of rise time.  Rise time is defined as the time taken for the CCPDv3 voltage to rise from 10\% to 90\% of the raw pulse height.  The rise time, and change in CCPDv3 output voltage over this time, are shown in the figure by the blue arrows.}
\label{fig:pulseshapeanalysis}
\end{figure}

%========================================================================================

\subsubsection{Results - Rise Time Vs Pulse Height}
\label{sec:resultsrisetimepulseheight}
The mean rise time as function of pulse height for the CCPDv3 output voltage is shown in figure \ref{fig:risetime}.  This was determined by binning the measurements in terms of pulse height and determining the mean rise time for measurements in each bin.  The pulse height was binned using a bin width of 4 mV ranging from 0 to 700 mV.  At least 10 measurements per pulse height bin were used for the calculation of the average rise time.  The error bars on this figure show the standard error in the mean rise time.  Data was only included in this analysis if the on-pixel event counter registered a single hit in the time window used to take data.

\begin{figure}[h!]
\centering
\includegraphics[width=1.0\textwidth]{CLICdpVertex/Plots/RadSourceAnalysis/AllSETs_RiseTime_PulseHeight.pdf}
\caption[The CCPDv3 output voltage rise time as a function of pulse height.]{The CCPDv3 output voltage rise time as a function of pulse height.}
\label{fig:risetime}
\end{figure}
 
The data in figure \ref{fig:risetime} shows that the rise time for the CCPDv3 front-end is approximately 300~ns across all samples.  The rise time is largely independent of pulse height for all but the smallest signals.  For very small pulse heights ($< 100$ mV) rise times are significantly larger, which suggests that the deposited charge takes a longer time to be collected.  This may be due to charge transport occurring via diffusion rather than drift, which is a much slower mechanism.  A gradual reduction in the rise time is observed as the total of deposited charge, which is proportional to the pulse height, increases.  This is expected for higher charge deposits as it will take less time for the CCPDv3 to collect enough charge to reach saturation if there is more charge deposited in the sensor bulk.  These results show that the intrinsic performance of the CCPDv3 sensors in the devices tested is very similar, which will make comparisons of the misaligned samples more straightforward.  Furthermore, almost identical intrinsic performance was observed for the CLICpix readout chips, as shown in section \ref{sec:testpulsecalibration}.  This indicates that any performance differences observed between these devices is entirely due to the capacitively coupled glue layer and the pad alignment.    

%========================================================================================

\subsubsection{Results - ToT Vs Pulse Height}
\label{sec:resultstotpulseheight}
The mean ToT measured in the CLICpix as a function of the CCPDv3 output voltage pulse height is shown in figure \ref{fig:tot}.  Determination of the mean and error bars for the ToT as a function of pulse height measurement is identical to that described above for the rise time as a function of pulse height measurement.   

\begin{figure}[h!]
\centering
\includegraphics[width=1.0\textwidth]{CLICdpVertex/Plots/RadSourceAnalysis/AllSETs_TargetTot_PulseHeight.pdf}
\caption[The mean ToT measured on the CLICpix ASIC as a function of CCPDv3 voltage pulse height.]{The mean ToT measured on the CLICpix ASIC as a function of CCPDv3 voltage pulse height.}
\label{fig:tot}
\end{figure}

The distribution of the mean ToT against the pulse height shows that, for samples where the CCPDv3 and CLICpix are centred, the ToT increases with pulse height up to values of approximately 300~mV while at larger pulse heights the mean ToT saturates at $\approx 13$.  It is expected that the $\frac{1}{4}-$ and $\frac{1}{2}-$offset samples will have a lower ToT than the centred samples due to the lower effective capacitance between the CCPDv3 and CLICpix pads.  The greater the offset is, the smaller the effective capacitance to the target CLICpix pad will be, and so the lower the recorded ToT.  This is can be seen when comparing the centred samples to SET 10, the $\frac{1}{4}-$offset sample, and SET 16, the $\frac{1}{2}-$offset sample.  

%========================================================================================

\subsubsection{Results - Cross Couplings}
Capacitive coupling of the sensor to the readout ASIC in a pixel detector can lead to the unwanted effect of cross-coupling.  Cross-coupling is the transfer of signal from a sensor pad to the readout ASIC on adjacent pads, which will occur if there is a non-negligible capacitance between adjacent pads.  In this case the signal is still transferred between the aligned sensor and readout pads, however, the cross-capacitance is large enough unwanted additional hits in neighbouring pads will be created.  This issue is particularly relevant for this study as any misalignment of the sensor and readout pads will result in an increase in the cross-capacitance along the direction of the misalignment.   

Any effects of cross-coupling can be studied using the same setup as was used in section \ref{sec:resultstotpulseheight} for the ToT against pulse height analysis, but by considering the ToT on the adjacent CLICpix pixel along the direction of the misalignment.  The mean ToT on the adjacent pixel is shown as a function of the pulse height for all devices where the CCPDv3 and CLICpix are aligned in figure \ref{fig:totcrosscoupling1} and for the misaligned samples in figure \ref{fig:totcrosscoupling2}.

\begin{figure}[h!]
\centering
\subfloat[]{\label{fig:totcrosscoupling1}\includegraphics[width=1.0\textwidth]{CLICdpVertex/Plots/RadSourceAnalysis/NoCrossCouplingSETs_Tot_X_PulseHeight.pdf}}\hfill
\subfloat[]{\label{fig:totcrosscoupling2}\includegraphics[width=1.0\textwidth]{CLICdpVertex/Plots/RadSourceAnalysis/CrossCouplingSETs_Tot_X_PulseHeight.pdf}}
\label{fig:totcrosscoupling}
\caption[The mean ToT measured on the adjacent CLICpix pixel, along the direction of the offset, as a function of CCPDv3 voltage pulse height.]{The mean ToT measured on the adjacent CLICpix pixel, along the direction of the offset, as a function of CCPDv3 voltage pulse height.}
\end{figure}

% HERE

% Centred sample chat
The distribution of the mean ToT on the adjacent CLICpix pixel shows little correlation with the pulse height on the target pixel for majority of pulse heights considered.  There is a weak correlation for small, $<$200~mV, pulse heights, which will most likely occur when charge occurs across two neighbouring CCPDv3 pixels.  There is also a correlation for large pulse heights and this is most likely due to cross-coupling.  Even if the cross-capacitance between a CCPDv3 pad and the neighbouring CLICpix pad is small, if the CCPDv3 output voltage pulse is large enough it will overcome the small capacitance and inject some charge into the neighbouring CLICpix and cause the observed correlation.  

For the $\frac{1}{4}-$offset sample there is no correlation between the adjacent CLICpix mean ToT and pulse height for all but the largest pulse heights.  As expected there is a clear correlation between adjacent ToT and pulse height for large pulse heights due to the cross-capacitance.  This correlation is stronger for this sample as the misalignment acts to increase the cross-capacitance in the direction of the offset indicating that cross-coupling will be more problematic for offset samples.  There is also a correlation present for low pulse heights, however, as the capacitance of the offset samples is lower the gradient of the adjacent ToT against the pulse height is much flatter.

% Cross coupling

The mean ToT on the adjacent pad is correlated with the pulse height up to $\sim$200~m, while above this the correlation 

For the $\frac{1}{4}-$offset sample 

it is clear that there the effects of cross-coupling are more prominent as there is a strong correlation between the  

SO: this part should be rewritten a little. Most likely you have a threshold effect at low pulse heights (not enough charge injected to make a hit, followed by enough to make a hit but just with a "default" low ToT). Then at high pulse heights you start to see a linear rise, which suggests you are now injecting enough charge to have a good correlation. But these are both second order to the point above, of showing that the number of hits in the main peak is N, and in the side peak is M << N (for low coupling).

No correlation between the adjacent pixel ToT and the CCPDv3 pulse height is observed for the samples shown in figure \ref{fig:totcrosscoupling1} for all but the lowest values of pulse height.  The correlation observed at low pulse heights may arise due to the signal $\text{e}^{-}$, which is primarily recorded in the target pixel, depositing a small amount of charge in the adjacent adjacent pixel.  This can happen as the $\text{e}^{-}$ may not be traveling normal to the pixel surface.  However, as this feature is not present in all samples it could indicate a small misalignment exists in the samples showing the correlation.  

There is, however, a strong correlation between adjacent pixel ToT and the CCPDv3 pulse height, shown in figure \ref{fig:totcrosscoupling2}, for SET 16, which is one of the $\frac{1}{2}$ offset samples.  This distribution is almost identical to the the target pixel ToT distribution as a function of CCPDv3 pulse height, which is what would be expected given an equal signal charge sharing between the two readout ASICs.  This indicates the charge sharing is well understood for this $\frac{1}{2}$ offset sample.   

For the $\frac{1}{4}$ offset sample correlation is present only for low pulse heights as was the case for the centred samples.  However, the mean ToT within the uncorrelated region is centred around $\approx$ 5 units of ToT, which is lower than was observed for the centred samples.  This is due to the offset reducing the total capacitance between the CCPDv3 and CLICpix in comparison to the centred samples and thus reducing the ToT recorded.

Cross coupling was observed in one of the $\frac{1}{2}$ offset samples and, assuming that the other $\frac{1}{2}$ offset sample was manufactured incorrectly, then charge sharing was well understood for these $\frac{1}{2}$ offset samples.  No cross coupling was observed for any of the other samples considered in this analysis.  

THE RESULTS OF THIS SECTION SHOULD BE DISCUSSED AND I SHOULD TAKE A LOOK ONCE YOU HAVE CHANGED THE TEXT AROUND AGAIN.



%========================================================================================

\subsection{Test Pulse Calibration}
\label{sec:testpulsecalibration}
%~~~

This measurement focuses upon determining the response of the CLICpix readout chip as a function of the input voltage entering the chip.  The CLICpix readout chip allows for a fixed height voltage pulse to be directly applied at the input of the chip meaning a full calibration of the chip can be performed.  This experiment extends the characterisation of the CLICpix chip beyond what was found using the radioactive source measurements as applying the voltage directly to the CLICpix fully isolates the response of the chip from effects relating to the glue layer.  

%~~~

I WOULD PRESENT A LITTLE MORE "STORY" STYLE, AND NOT SIMPLY AS A LIST OF UNRELATED EXPERIMENTS

In order to understand the charge transfer to the CLICpix, it is important to perform a calibration of the CLICpix front-end response. This was achieved by directly injecting a voltage pulse of fixed height directly into a capacitor held in each pixel, to inject a known quantity of charge. This gives a measure of the performance of the CLICpix independently of the CCPDv3 sensor.  %Due to the construction of the sensor it was not possible to access the CCPDv3 to perform a similar test to isolate its performance.  

%========================================================================================

\subsubsection{Experimental Setup}

In order to prevent any influence from neighbouring pixels during the testpulse measurements the matrix was pulsed in stages, with change injected into 1 out of every 16 pixels while masking the others.  This was repeated 15 more times using different mask configurations until the entire matrix had been sampled. This procedure was repeated 100 times so that the average ToT on a per-pixel level could be recorded.  The pulse height injected into the CLICpix varied from 2 to 180 mV in steps of 2 mV;  an example of the mean ToT plotted against the injected pulse height is shown in figure \ref{fig:testpulseexamplenofit}.  

\begin{figure}
\centering
\includegraphics[width=1.0\textwidth]{CLICdpVertex/Plots/TestPulseCalibration/Fits/Set9/ToT_PulseHeight_Set_9_ChipID_001ec0db94b1_Pixel_x0_y0_NoFit.pdf}
\caption[CLICpix ToT as a function of injected pulse height.]{CLICpix ToT as a function of injected pulse height for a single pixel.  The black markers are the mean ToT and the error bars are the standard error on the mean.}
\label{fig:testpulseexamplenofit}
\end{figure}

%========================================================================================

\subsubsection{Analysis}

The functional form of the ToT against pulse height plot is described using a surrogate function as in \cite{AlipourTehrani:2054922}, which is defined as

\begin{equation}
y  = ax + b  - \frac{c}{x-t}
\end{equation}

\noindent where $y$ is the ToT, $x$ is the pulse height and $a$, $b$, $c$ and $t$ are fit parameters.  For large pulse heights the linear relationship dominates while for low pulse heights the inversely proportional term dominates.  $c$ describes the curvature of the graph, while $t$ determines the asymptote below which no signal is detected.  Figure \ref{fig:testpulseexamplefit} shows an example of the application of this fit.  As this function does not describe saturation of the ToT or the region below threshold, the fit is only applied on data points where the mean ToT is greater than 1 and less than 14.75.  

\begin{figure}
\centering
\includegraphics[width=1.0\textwidth]{CLICdpVertex/Plots/TestPulseCalibration/Fits/Set9/ToT_PulseHeight_Set_9_ChipID_001ec0db94b1_Pixel_x0_y0_Fit.pdf}
\caption[CLICpix ToT as a function of injected pulse height.]{CLICpix ToT as a function of injected pulse height for a single pixel.  The black markers are the mean ToT and the error bars are the standard error on the mean.  The solid red line shows the surrogate function fit and the dotted red lines show the range where the fit was applied.}
\label{fig:testpulseexamplefit}
\end{figure}

The application of this fit condenses the information for individual pixels to four parameters.  These parameters can be averaged to categorise the CLICpix response across the matrix.  

%========================================================================================

\subsubsection{Results}
\label{sec:testpulsecalibrationresults}

A known issue withe the design of the CLICpix ASIC is an unwanted feedback capacitance between the discriminator output and amplifier input, leading to a fixed injected charge for each measured hit and operation of the chip at a higher-than-expected threshold. The magnitude of this effect is additionally different for odd and even columns due to the slightly differing physical layouts. This feature can be observed by examining the distribution of the fit parameters;  these are shown for assembly SET9 in figure \ref{fig:fitparams}.  The peak at zero in the distribution of the $a$ and $b$ parameters, $\approx$ 150 in total, correspond to masked pixels in the detector. (CONFIRM THIS!) While the $a$ and $b$ parameters are centred around a single value, indicating a similar response in the linear region of the surrogate function, the $c$ and $t$ parameters are centred around one of two values.  When examining the distribution of these parameters as a function of position on the matrix, shown in figure \ref{fig:fitparams2d} for the same device, it can be seen that the structure is indeed related to the column a given pixel is in.  This feature is present in all devices considered, and the underlying cause will be remedied in the next generation of the CLICpix ASIC.

\begin{figure}
\centering
\subfloat[$a$ parameter.]{\label{fig:fitparams1}\includegraphics[width=0.5\textwidth]{CLICdpVertex/Plots/TestPulseCalibration/FitParam/OneDHistFitParamA_Set9.pdf}}
\subfloat[$b$ parameter.]{\label{fig:fitparams2}\includegraphics[width=0.5\textwidth]{CLICdpVertex/Plots/TestPulseCalibration/FitParam/OneDHistFitParamB_Set9.pdf}}\hfill
\subfloat[$c$ parameter.]{\label{fig:fitparams3}\includegraphics[width=0.5\textwidth]{CLICdpVertex/Plots/TestPulseCalibration/FitParam/OneDHistFitParamC_Set9.pdf}}
\subfloat[$t$ parameter.]{\label{fig:fitparams4}\includegraphics[width=0.5\textwidth]{CLICdpVertex/Plots/TestPulseCalibration/FitParam/OneDHistFitParamT_Set9.pdf}}
\label{fig:fitparams}
\caption[Distribution of surrogate function fit parameters for SET 9.]{Distribution of surrogate function fit parameters for device SET 9.}
\end{figure}

\begin{figure}
\centering
\subfloat[$c$ parameter.]{\label{fig:fitparams2dc}\includegraphics[width=0.5\textwidth]{CLICdpVertex/Plots/TestPulseCalibration/FitParam/FitParamC_Set9.pdf}}
\subfloat[$t$ parameter.]{\label{fig:fitparams2dt}\includegraphics[width=0.5\textwidth]{CLICdpVertex/Plots/TestPulseCalibration/FitParam/FitParamT_Set9.pdf}}
\label{fig:fitparams2d}
\caption[Distribution as a function of matrix position of surrogate function fit parameter $c$ for SET 9.]{Distribution of surrogate function fit parameters $c$ and $t$ for device SET 9.}
\end{figure}

I DONT KNOW WHAT THE FIRST PART OF THE CAPTION IS, BUT IF I HAVE CORRECTED THE SECOND PART (THAT APPEARS IN THE TEXT) PLEASE UPDATE THE FIRST PART IF IT IS RELEVANT AND SHOWN SOMEWHERE!

The matrix-averaged surrogate function fit parameters for all devices can be found in tables \ref{table:clicpixfitparamseven} and  \ref{table:clicpixfitparamsodd}, for the even and odd columns respectively.  The surrogate function using these average parameters as input is shown in figure \ref{fig:testpulsemeanfit}.

REMOVE SET14

\begin{table}[h!]
\centering
\begin{tabular}{ l r r r r}
\hline
Assembly & $a$ & $b$ & $c$ & $t$ \\ 
\hline
SET 9   & $0.0875 \pm 0.0005$ & $2.41 \pm 0.03$ & $5.1 \pm 0.1$ & $12.79 \pm 0.15$ \\
SET 10 & $0.0769 \pm 0.0005$ & $2.58 \pm 0.03$ & $7.5 \pm 0.2$ & $8.02 \pm 0.14$ \\
SET 12 & $0.0725 \pm 0.0005$ & $2.87 \pm 0.04$ & $12.1 \pm 0.3$ & $7.86 \pm 0.22$  \\
SET 13 & $0.0708 \pm 0.0005$ & $2.69 \pm 0.03$ & $16.2 \pm 0.3$ & $6.65 \pm 0.18$ \\
SET 15 & $0.0856 \pm 0.0005$ & $2.34 \pm 0.03$ & $5.1 \pm 0.2$ & $12.51 \pm 0.13$ \\
SET 16 & $0.0746 \pm 0.0004$ & $2.32 \pm 0.02$ & $13.7 \pm 0.3$ & $6.65\pm 0.16$ \\
\hline
\end{tabular}
\caption[Average fit parameters for even columns of CLICpix sensor.]{Average fit parameters for even columns of the different CLICpix assemblies.}
\label{table:clicpixfitparamseven}
\end{table}

\begin{table}[h!]
\centering
\begin{tabular}{ l r r r r}
\hline
Assembly & $a$ & $b$ & $c$ & $t$ \\ 
\hline
SET 9   & $0.0834 \pm 0.0003$ & $1.72 \pm 0.01$ & $61.0 \pm 0.3$ & $0.25 \pm 0.09$ \\
SET 10 & $0.0759 \pm 0.0002$ & $1.63 \pm 0.01$ & $43.2 \pm 0.2$ & $0.10 \pm 0.02$ \\
SET 12 & $0.0731 \pm 0.0003$ & $1.92 \pm 0.02$ & $51.5 \pm 0.3$ & $0.36 \pm 0.12$ \\
SET 13 & $0.0713 \pm 0.0002$ & $1.72 \pm 0.01$ & $52.5 \pm 0.3$ & $0.18 \pm 0.07$ \\
SET 15 & $0.0836 \pm 0.0003$ & $1.52 \pm 0.02$ & $52.7 \pm 0.3$ & $0.42 \pm 0.08$ \\
SET 16  & $0.0727 \pm 0.0002$ & $1.49 \pm 0.01$ & $50.7 \pm 0.2$ & $0.10 \pm 0.03$ \\
\hline
\end{tabular}
\caption[Average fit parameters for odd columns of CLICpix sensor.]{Average fit parameters for odd columns of the different CLICpix assemblies.}
\label{table:clicpixfitparamsodd}
\end{table}

\begin{figure}
\centering
\includegraphics[width=1.0\textwidth]{CLICdpVertex/Plots/TestPulseCalibration/FitParam/AverageToT_vs_InjectedPulseHeight.pdf}
\caption[Fitted CLICpix ToT as a function of injected pulse height averaged across the device matrix.]{Fitted ToT as a function of injected pulse height for all samples, averaged across the device matrix.}
\label{fig:testpulsemeanfit}
\end{figure}

As figure \ref{fig:testpulsemeanfit} shows, the response of the CLICpix to the injected pulse height is rather uniform across all samples.  In general, the turn-on pulse height is $\approx$ 10~mV and saturation (i.e. ToT of 15 units) occurs at $\approx$ 150~mV.  For even-numbered columns there is a sharper rise in ToT than for odd-numbered columns due to the different quantity of (unwanted) injected charge. This effect was observed across all devices considered. The uniformity of the sample response ensures that differing effects between the assemblies produced with different misalignments are not due to the response of the CLICpix front end.

%========================================================================================
%========================================================================================

\section{Test Beam Analysis}
\label{sec:testbeam}

%~~~

The lab based measurements have helped to describe many of the characteristics of the devices considered here, however, their use is limited as little information can be deduced about the incoming particles using lab based measurements.  Therefore, to record device efficiencies it is necessary to place the device in a telescope to track the incoming and outgoing particles passing through the device.  Furthermore, the radioactive source calibration cannot be used here as the particles produced by the source would not be of high enough energy to pass through the entire telescope.  This means that it is necessary to use a test beam in conjunction with the telescope to be able to full quantify the device characteristics.

%~~~

AGAIN, BETTER INTRO? Describe a bit why we have to do testbeams, efficiency measurements, etc. Try not to present it in this detached "we have performed an experiment" - there is some reasoning behind this! 

This sections describes the performance of the devices when placed under real experimental conditions at the CERN test beam area.  

%========================================================================================

\subsection{Test Beam Setup}

COMBINED WITH ABOVE COMMENT
The overall goal of this experiment was to determine the tracking performance of the capacitively coupled pixel sensors and to see whether the misalignment of the CCPDv3 and the CLICpix changes the performance.  To that end the samples were mounted on a telescope and placed in a test beam to determine the efficiency, defined as the ratio of number of recorded tracks passing through device to the actual number of tracks passing through device, of the samples.  

These test beam experiments were carried out in August and September 2015 on the H6 beam line in the CERN SPS North Area.  The beam consisted of positively charged hadrons of momenta 120 GeV/c.  Mean particle rates of 500 kHz/cm$^{2}$ were observed during the 4.8 s spills at intervals of 25 s.  \textcolor{red}{Is this data correct?}.

AGAIN, MAYBE EXPAND THIS. You don't really explain why you need a telescope, or what it really is/is used for. Something like "in order to reconstruct the particle trajectory, several planes of detectors are used to..." etc.

During this experiment the samples mounted on an EUDET/AIDA telescope \cite{Rubinskiy:2000287}, which consists of six planes sensors using the Mimosa pixel technology.  This telescope provides a resolution of 1.6 $\mu$m on the intercept position between tracks passing through the device and the device under test (DUT) mounted on it.  

%========================================================================================

\subsection{Analysis}
The track position on the DUT is calculated using the measured particle trajectory through the telescope planes.  This is followed by a search around the intercept position on the DUT to find an associated cluster; a region of 75 $\mu$m, or 3 pixels, about the intercept position is used.  For multi-pixel clusters, the cluster position is calculated as the ToT-weighted centre-of-gravity.  As tracks may undergo non-negligible multiple scattering, a $\chi^{2}$ cut is used to remove less precisely reconstructed particles. Pixels identified on the DUT deemed to be noisy were removed from the analysis, along with any tracks intercepting within half a pixel.  A pixel was deemed noisy if it responded at a mean rate greater than 5 $\sigma$ in comparison to the average rate.  Finally, all tracks occurring within 125 $\mu$m of each other were vetoed, in order to reduce the possibility of mis-association of clusters to tracks. 

MAYBE SAY WHY ALIGNMENT IS NECESSARY? IN GENERAL THIS SECTION COULD DO WITH A LITTLE MORE CONTEXT

An alignment procedure was applied to both the telescope planes and the DUT, in order to account for the physical layout of the setup.  The six telescope planes were aligned by producing rough tracks, and then varying the global alignment parameters of each plane in turn, in order to minimise the track $\chi^{2}$. This was performed iteratively until no further gain was observed. After the telescope planes were aligned, the DUT was aligned by varying its alignment parameters in order to minimise the summed square of track residuals over many events.

%========================================================================================

AGAIN A BIT OF CONTEXT.

\subsection{Single Hit Efficiency}
The single hit efficiency, $\epsilon$, is defined as the number of tracks with associated clusters in the CLICpix assembly, $n$, divided by the number of tracks reconstructed through the detector using the telescope, $m$. The errors shown on the efficiency measurements are given by $\sqrt{\frac{\epsilon (1 - \epsilon)}{m}}$, which follows from the variance of $n$ given binomial statistics with mean $\epsilon$.  

CONTEXT

\begin{figure}
\centering
\includegraphics[width=1.0\textwidth]{CLICdpVertex/Plots/TestBeamData/EfficiencyThresholdPlot.pdf}
\caption[Efficiency as a function of threshold.]{Efficiency as a function of threshold.}
\label{fig:efficiency}
\end{figure}

CONTEXT (just in case you missed that ;-))

The efficiency as a function of threshold is shown in figure \ref{fig:efficiency}.  The data indicates that, as expected, for all assemblies the single hit efficiency of the detector decreases when a higher amount of charge is required to generate a signal.  However, it is clear that for the $\frac{1}{2}$ offset sample, SET 16, the efficiency is significantly lower in comparison to the other samples. There is a minor degradation in performance when considering the $\frac{1}{4}$ offset sample, but these results are still comparable to several of the centred samples.  Overall, it can be concluded that manufacturing tolerances up to $\frac{1}{4}$ of a pixel width would not significantly affect performance.    

AGAIN THIS NEEDS EXPANSION. You spent so much time on the earlier section, but now on the results you just say "see figure 1. end.". You can talk a bit more about the lower signal expected by reduced capacitance, etc.

%========================================================================================

\section{Conclusions}


%========================================================================================

  

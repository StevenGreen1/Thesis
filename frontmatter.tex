%% Title
\titlepage[of Emmanuel College]{%
  A dissertation submitted to the University of Cambridge\\ for the degree of Doctor of Philosophy}

%% Abstract
\begin{abstract}%[\smaller \thetitle\\ \vspace*{1cm} \smaller {\theauthor}]
  %\thispagestyle{empty}
This thesis describes the optimisation of the calorimeter design for collider experiments
at the future Compact LInear Collider (CLIC) and the International Linear Collider (ILC). 
The detector design of these experiments is built around high-granularity Particle Flow Calorimetry
that, in contrast to traditional calorimetry, uses the energy measurements for charged particles from 
the tracking detectors. This can only be realised if calorimetric energy deposits from
charged particles can be separated from those of neutral particles. This is made possible with fine
granularity calorimeters and sophisticated pattern recognition software, which is provided by the
PandoraPFA algorithm. This thesis presents results on Particle Flow calorimetry performance for a number
of detector configurations. To obtain these results a new calibration procedure was developed 
and applied to the detector simulation and reconstruction to ensure optimal performance 
was achieved for each detector configuration considered.

This thesis also describes the development of a software compensation technique that vastly improves the
intrinsic energy resolution of a Particle Flow Calorimetry detector. This technique is implemented
within the PandoraPFA framework and demonstrates the gains that can be made by fully exploiting
the information provided by the fine granularity calorimeters envisaged at a future linear collider.

A study of the sensitivity of the CLIC experiment to anomalous gauge couplings that
effect vector boson scattering processes is presented. These anomalous couplings provide insight
into possible beyond standard model physics. This study, which utilises the excellent jet energy resolution from
Particle Flow Calorimetry, was performed at centre-of-mass energies of 1.4 TeV and 3 TeV with integrated 
luminosities of 1.5$\text{ab}^{-1}$ and 2$\text{ab}^{-1}$ respectively. The precision achievable at CLIC is shown to be approximately 
one to two orders of magnitude better than that currently offered by the LHC.

In addition, a study into various technology options for the CLIC vertex detector is described.
\end{abstract}


%% Declaration
\begin{declaration}
  This dissertation is the result of my own work, except where explicit
  reference is made to the work of others, and has not been submitted
  for another qualification to this or any other university. This
  dissertation does not exceed the word limit for the respective Degree
  Committee.
  \vspace*{1cm}
  \begin{flushright}
    Steven Green
  \end{flushright}
\end{declaration}


%% Acknowledgements
\begin{acknowledgements}
  Of the many people who deserve thanks, some are particularly prominent,
  such as my supervisor\dots
\end{acknowledgements}


%% Preface
%\begin{preface}
%  This thesis describes my research on various aspects of the \LHCb
%  particle physics program, centred around the \LHCb detector and \LHC
%  accelerator at \CERN in Geneva.

%  \noindent
%  For this example, I'll just mention \ChapterRef{chap:SomeStuff}
%  and \ChapterRef{chap:MoreStuff}.
%\end{preface}

%% ToC
\tableofcontents


%% Strictly optional!
\frontquote{%
  Writing in English is the most ingenious torture\\
  ever devised for sins committed in previous lives.}%
  {James Joyce}
%% I don't want a page number on the following blank page either.
\thispagestyle{empty}
